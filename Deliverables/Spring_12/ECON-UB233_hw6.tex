\documentclass[11pt]{exam}

\oddsidemargin=0.25truein \evensidemargin=0.25truein
\topmargin=-0.5truein \textwidth=6.0truein \textheight=8.75truein

%\RequirePackage{graphicx}
\usepackage{comment}
\usepackage{verbatim}
\usepackage{booktabs}
\usepackage{graphicx}
\usepackage{hyperref}
\urlstyle{rm}   % change fonts for url's (from Chad Jones)
\hypersetup{
    colorlinks=true,        % kills boxes
    allcolors=blue,
    pdfsubject={ECON-UB233, Macroeconomic foundations for asset pricing},
    pdfauthor={Dave Backus @ NYU},
    pdfstartview={FitH},
    pdfpagemode={UseNone},
%    pdfnewwindow=true,      % links in new window
%    linkcolor=blue,         % color of internal links
%    citecolor=blue,         % color of links to bibliography
%    filecolor=blue,         % color of file links
%    urlcolor=blue           % color of external links
% see:  http://www.tug.org/applications/hyperref/manual.html
}

\renewcommand{\thefootnote}{\fnsymbol{footnote}}
\newcommand{\var}{\mbox{\it Var\/}}

%\printanswers

% document starts here
\begin{document}
\parskip=\bigskipamount
\parindent=0.0in
\thispagestyle{empty}
{\large ECON-UB 233 \hfill Dave Backus @ NYU}

\bigskip\bigskip
\centerline{\Large \bf Lab Report \#6: Sums \& Mixtures}
\centerline{(Started: March 26, 2012; Revised: \today)}

\bigskip
{\it Due at the start of class.
You may speak to others, but whatever you hand in should be your own work.}

\bigskip
We'll look at a Bernoulli mixture and show that it
can generate signficant departures from the lognormal model
and the Black-Scholes-Merton formula.
The good news is that it's simpler than the Poisson mixture.
The bad news is that it doesn't scale easily
to different time intervals, so we'll stick with a time interval of one year.

\begin{questions}

\begin{solution}
Answers follow.  See Matlab code at the end for calculations.
\end{solution}

%-----------------------------------------------------------------------
\question (sums and mixtures)
Let us say that the log-price of the underlying has two components:
\begin{eqnarray*}
    \log s_{t+1} &=& y_{t+1} \;\;=\;\; x_{1t+1} + x_{2t+1},
\end{eqnarray*}
with $( x_{1t+1}, x_{2t+1})$ independent.
The first component is normal:
$ x_{1t+1} \sim \mathcal{N}(\mu,\sigma^2)$.
The second component, the ``jump,'' is a mixture:
with probability $1-\omega$, $ x_{2t+1} = 0$,
and with probability $\omega$,
$ x_{2t+1} \sim \mathcal{N}(\theta,\delta^2)$.

With these inputs, the pdf for $y$ is a weighted average of normals:
\begin{eqnarray}
    p(y) &=& (1-\omega) \cdot ( 2 \pi \sigma^2)^{-1/2} \exp[ - (y-\mu)^2/2\sigma^2]
            \nonumber \\
    && + \; \omega \cdot [ 2 \pi (\sigma^2+\delta^2)]^{-1/2}
            \exp[ - [y-(\mu+\theta)]^2/2(\sigma^2 + \delta^2)] .
            \label{eq:bernoulli-mixture}
\end{eqnarray}
If $\omega = 0$, the second component drops out.
Otherwise, we have a weighted average of two normal densities.

\begin{parts}
\part Show that the cumulant generating function for $x_{1t+1}$ is
\begin{eqnarray*}
    k (s; x_1) &=&  \mu s + \sigma^2 s^2 / 2 .
\end{eqnarray*}

\part Show that the cumulant generating function for $x_{2t+1}$ is
\begin{eqnarray*}
    k (s; x_2) &=& \log \left[ (1-\omega) + \omega e^{ \theta s + \delta^2 s^2 / 2 }
            \right] .
\end{eqnarray*}
\part What is the cgf for $y_{t+1}$?
What are its mean, variance, skewness, and excess kurtosis?
What parameters determine the sign of skewness?

\part Suppose we started with (\ref{eq:bernoulli-mixture}).
How do we know it integrates to one?
What is its cgf?
\end{parts}

\begin{solution}
The overall idea here is that this construction
can give us useful departures from a normal distribution.
Details follow from applying the usual methods.
\begin{parts}
\part  The usual normal cgf.
\part This is a Bernoulli mixture with one twist:
the ``$\omega$ branch'' has a normal mgf in it.
If $\delta = 0$, it's just like the Bernoulli we looked at in
Lab Report \#1.
Otherwise, we get some additional terms.

\part The cgf of the sum is the sum of the cgfs:
\begin{eqnarray*}
        k (s; y) &=& k (s; x_1) + k (s; x_2) \\
             &=& ( \mu s + \sigma^2 s^2 / 2 ) +
              \log \left[ (1-\omega) + \omega e^{ \theta s + \delta^2 s^2 / 2 }
            \right] .
\end{eqnarray*}
We find it cumulants from derivatives, which are conveniently
computed by Matlab.
The mean and variance are
\begin{eqnarray*}
    \kappa_1 &=& \mu + \omega \theta \\
    \kappa_2 &=& \sigma^2 + \omega (1-\omega) \theta^2 + \omega \delta^2 .
\end{eqnarray*}
Each has terms from each component.
The third and fourth cumulants come only from $x_2$, because the normal
component has zero cumulants beyond the first two.
The third one is
\begin{eqnarray*}
    \kappa_3 &=& \omega (1-\omega) \theta [ 3 \delta^2 + (1-2\omega) \theta^2]  \\
    \kappa_4 &=& \omega (1-\omega)
        \{  \theta^4 [1 - 6 \omega(1-\omega)] + 3 \delta^4 + (1-2 \omega) 6 \delta^2 \theta^2
        \}
\end{eqnarray*}
This is a bit of a mess, but for $\omega$ small,
$\kappa_3$ depends on the sign of $\theta$
and $\kappa_4 >0 $,
so the mixture introduces skewness and excess kurtosis,
both of which are absent from normal random variables.

\part We know normal pfd's integrate to one.
The is a weighted average of normal pdf's, so it must
integrate to one as well.
\end{parts}
\end{solution}


%-----------------------------------------------------------------------
\question (Merton-like option pricing)
With the same setup, we can illustrate the value of mixtures in
generating nonnormal distributions and
their impact on option prices and volatility smiles.

\begin{parts}
\part Risk-neutral asset pricing tells us, in general, that
\begin{eqnarray*}
    s_t &=& q^1_t E^* (s_{t+1} )
            \;\;=\;\; q^1_t E^* \left( e^{y_{t+1}}\right)
            \;\;=\;\; q^1_t  e^{k(1; y)}  .
\end{eqnarray*}
We refer to this as the no-arbitrage condition.
What is the no-arbitrage condition for our example?

We'll use this condition to set $\mu$:
given values for everything else, we'll choose $\mu$ to satisfy this condition.

\part Recall that if the risk-neutral distribution is
 $\log s_{t+1} = y_{t+1} \sim \mathcal{N}(\kappa_1,\kappa_2)$,
 then the put price at strike $k$ is
\begin{eqnarray*}
    f(k; \kappa_1, \kappa_2) &=& q^1_t k N(d) - q^1_t e^{\kappa_1 + \kappa_2/2}
            N(d-\kappa_2^{1/2}) \\
            d&=& (\log k - \kappa_1)/\kappa_2^{1/2} .
\end{eqnarray*}
(Note:  this isn't quite the usual $d$.)
What is the call price?

\part
Use (b) to show that the put price in the mixture model is a weighted average:
\begin{eqnarray*}
    q^p_t &=& (1-\omega) \cdot f(k; \mu, \sigma^2) +
        \omega \cdot f(k; \mu+\theta, \sigma^2 + \delta^2) .
\end{eqnarray*}

\part Consider these inputs:
$\sigma = 0.04$, $\omega = 0.01$,
$\theta = -0.3$, $\delta = 0.15$,
$s_t = 100$, and $q^1_t = 1$.
What is $\mu$?
What are the prices of put options with strikes
$ k = (80, 90, 100, 110, 120)$?
(Use a finer grid if you have this automated.)
What are the implied volatilities?

\part What happens to the volatility smile when you set
\begin{itemize}
\item $\theta = 0$?
\item $\theta = + 0.3$?
\item $\delta = 0.25$?
\end{itemize}
Make sure you adjust $\mu$ in each case.
\end{parts}

\begin{solution}
\begin{parts}
\part The condition is
\begin{eqnarray*}
    s_t &=& q^1_t \exp( \mu + \sigma^2 / 2 )
            \left[ (1-\omega) + \omega e^{ \theta + \delta^2 / 2 } \right] .
\end{eqnarray*}
It's not pretty, but given values of the other inputs we can use it to set $\mu$.

\part This is a question of integrating over the distribution,
as we did in class.
We find the call price from put-call parity.

\part If the pdf is a sum, then when we integrate we can integrate
over each term separately:  the integral of a sum is the sum of the integrals.
Each integral gives us a BSM-like formula, but with different mean and variance.
The put price is the weighted average of the two formulas, as stated.

\part Now we can put all this to work.
With these inputs, the arbitrage condition gives us $ \mu = 4.6069$.
The put prices are (I chose a slightly different range):

\begin{center}
\tabcolsep=0.15in
\begin{tabular}{rr}
\toprule
Strike    &  Put Price  \\
\midrule
  90   & 0.1606   \\
   94  &  0.2806  \\
   98  &  0.9285  \\
  102  &  2.8801  \\
  106  &  6.1523  \\
  110  & 10.0147 \\
\bottomrule
\end{tabular}
\end{center}

It may be asking too much to compute the implied vols, too,
but they give us a clearer picture.
The Matlab code is a little quirky, you need to play with the
starting values to get it to work.
But you'll find that there's a distinct downward slope to the smile.
That disappears if we set $\theta = 0$, since $\theta$ is the source
of skewness in the model.
Instead we get a u-shape.
When we increase $\delta$, the u is more pronounced.

In short, the Bernoulli mixture is capable of giving us a wide range of shapes
for the smile.

\end{parts}
\end{solution}

\end{questions}

\vfill \centerline{\it \copyright \ \number\year \
NYU Stern School of Business}

%\end{document}

\pagebreak
Matlab program:
\verbatiminput{../Matlab/hw6_s12.m}

\end{document}


%  EXTRA STUFF


