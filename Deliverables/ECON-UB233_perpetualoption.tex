\documentclass[11pt]{exam}

\oddsidemargin=0.25truein \evensidemargin=0.25truein
\topmargin=-0.5truein \textwidth=6.0truein \textheight=8.75truein

\usepackage{comment}
\usepackage{enumerate}
\usepackage{hyperref}
\urlstyle{rm}   % change fonts for url's (from Chad Jones)
\hypersetup{
    colorlinks=true,        % kills boxes
    allcolors=blue,
    pdfsubject={ECON-UB233, Macroeconomic foundations for asset pricing},
    pdfauthor={Dave Backus @ NYU},
    pdfstartview={FitH},
    pdfpagemode={UseNone},
%    pdfnewwindow=true,      % links in new window
%    linkcolor=blue,         % color of internal links
%    citecolor=blue,         % color of links to bibliography
%    filecolor=blue,         % color of file links
%    urlcolor=blue           % color of external links
% see:  http://www.tug.org/applications/hyperref/manual.html
}

%\usepackage{amssymb}
%\usepackage{amsfonts}
%\usepackage{booktabs}

\renewcommand{\thefootnote}{\fnsymbol{footnote}}
%\newcommand{\bv}{\begin{verbatim}}
%\newcommand{\ev}{\end{verbatim}}

% document starts here
\begin{document}
\parskip=\bigskipamount
\parindent=0.0in
\thispagestyle{empty}
{\large ECON-UB 233 \hfill Dave Backus @ NYU}

\bigskip\bigskip
\centerline{\Large \bf Practice Problem:  Perpetual Options}
\centerline{Revised: \today}

\bigskip
%{\it This will not be collected or graded,
%but should be completed before the second class.
%You may be called upon in class to explain one of your answers.}

Consider an American call option with no expiration date.
The problem underlies the extensive literature on so-called {\it real options\/}, 
in which a firm considers, for example, the decision to extract oil 
from land it owns.  
They can do it any time, what time is best?  

A general statement of the problem might be expressed this way.  
We have a Markov process for some state variable $x_t$.
The ``underlying'' (the asset on which the option is written)
pays dividend $d(x_t)$ at date $t$ and
satisfies the (ex-dividend) valuation relation
\begin{eqnarray}
    V(x_t) &=&  E_t \left\{ m(x_t,x_{t+1}) \left[ d(x_{t+1}) + V(x_{t+1}) \right]
                \right\} .
    \label{eq:perpetual-underlying}
\end{eqnarray}
%We'll use this one below.
The process for $x_t$ gives us one for $V(x_t)$. 

Now consider a perpetual American call option with strike $k$.
Its value solves
\begin{eqnarray}
    J(x_t) &=& \max \left\{
            E_t m(x_t,x_{t+1}) J(x_{t+1}) ,
            V(x_t) - k
            \right\} .
        \label{eq:perpetual-option}
\end{eqnarray}
This is a Bellman equation for the value function $J$.
The first element on the right is what we get if we wait ---
the current value of the same option next period.
The second is what we get if we exercise ---
the value now of the underlying minus the strike.
Many such problems have a threshold property:
we exercise the option when $V_t \geq V^*$ for some threshold $V^*$.

There's a nice example that we can solve by guess and verify. 
It has what we might call the Black-Scholes-Merton lognormal structure.  
Its ingredients include:  
(i)~We skip the state $x_t$ and deal directly with $V_t$.
(ii)~Risk-neutral pricing with $m = e^{-r}$ and lognormal risk-neutral distribution 
of $V$:  
$\log V_{t+1} - \log V_t \sim \mathcal{N}(\kappa_1,\kappa_2)$.
(iii)~Constant dividend rate $\delta$:  $d_{t+1} = \delta V_{t+1}$.  


Your mission is to work through this example and find the value function $J$ 
and threshold $V^*$.  
%
\begin{enumerate}[(a)]
\item What is the Bellman equation for this example? 

\item Apply the valuation equation (\ref{eq:perpetual-underlying}) to our assumptions
about the distribution of $V_{t+1}$.  
What restriction on $(r,\delta,\kappa_1,\kappa_2)$ is implied?  

\item Guess that the value function has the form 
$J(V) = A V^a$ for parameters $(A,a)$ to be determined. 
With this function, what is the value of the ``wait'' branch of the option?  

\item At the threshold $V^*$, we are indifferent between waiting and exercising.
Use this to establish two conditions on $(A,a,V^*)$.  

\item We need one more condition.  Use the envelope condition, 
evaluated at $V_t = V^*$, to produce it.

\item Now use all three conditions to find the unknowns 
$(A,a,V^*)$.  

\item Verify that $J$ is increasing and convex.   
Why does it make sense for the option value to be convex?  

\item Choose numbers for the inputs $(r,\delta,\kappa_1,\kappa_2, k)$ and
graph the value function.  


\end{enumerate}


\vfill \centerline{\it \copyright \ \number\year \
NYU Stern School of Business}

\end{document}


{\it Example 1 (BSM structure).\/}
This seems to be a standard example, with structure taken from the BSM option model.
I found it in Gerber and Shiu (1994, Astin Bulletin) and Huang (2007, ms).
Structure includes:
(i)~Risk-neutral pricing with $m = e^{-r}$ and risk-neutral
$\log V_{t+1} - \log V_t \sim \mathcal{N}(\kappa_1,\kappa_2)$.
(ii)~Constant dividend rate $\delta$:  $d_{t+1} + V_{t+1} = e^\delta V_{t+1} $.
The valuation equation (\ref{eq:perpetual-underlying}) becomes
\begin{eqnarray*}
    V_t &=&  e^{-r} E_t \left( e^\delta V_{t+1} \right)
    \;\;\;\Leftrightarrow \;\;\;
    V_t \;\;=\;\; e^{-r} e^{\delta} \ V_t \ e^{\kappa_1 + \kappa_2/2} .
\end{eqnarray*}
This implies the no-arbitrage condition,
\begin{eqnarray*}
    0 &=& \delta - r + \kappa_1 + \kappa_2/2 ,
\end{eqnarray*}
which we'll use to nail down $\kappa_1$.

We do guess and verify on the value function.
The option valuation equation (\ref{eq:perpetual-option}) is now
\begin{eqnarray*}
    J(V_t) &=& \max \left\{
            e^{-r} E_t J(V_{t+1}), V_t - k
            \right\} .
\end{eqnarray*}
Guess $J(V) = A V^a$ for parameters $(A,a)$ to be determined.
That gives us three unknowns:  $A$, $a$, and the boundary value $V^*$.
If $V = V^*$ we're indifferent between exercising and not,
so we have
\begin{eqnarray*}
    J(V^*) &=& A \ V^{*a}
            \;\;=\;\; e^{-r} A \ V^{*a} \ e^{a \kappa_1 + a^2 \kappa_2/2}
            \;\;=\;\;  V^* - k .
\end{eqnarray*}
That's two equations.
The third is the envelope condition,
\begin{eqnarray*}
    J_V(V^*) &=& a A V^{*a-1} \;\;=\;\; 1.
\end{eqnarray*}
The first equation gives us $a$:
\begin{eqnarray}
    r &=& a \kappa_1 + a^2 \kappa_2/2
    \;\;\;\Rightarrow\;\;\;
    a \;\;=\;\; \frac{-\kappa_1 + (\kappa_1^2 + 2 \kappa_2 r)^{1/2}}{\kappa_2} .
    \label{eq:quadratic}
\end{eqnarray}
We take the positive root $a \geq 1$ because a call option price is increasing in the underlying.
(Show this?)
The other two equations give us $ V^* = [a/(a-1)] k$.
We'll leave $A$ for another time.

One feature of this example is that $J$ is convex.
That's true because $a \geq 1$.
Another is $ V^* > k$.
We know that in general, but it's clear here.
Yet another is that $V^*$ is increasing in risk $\kappa_2$.
If we increase $\kappa_2$, then (\ref{eq:quadratic}) tells us $a$ must fall.
(This is subtle, because when we increase $\kappa_2$, $\kappa_1$ must fall
to satisfy the no-arb condition. But it falls less than proportionately.)
Then $V^* = [a/(a-1)] k $ rises.
So uncertainty raises $V^*$.
[Longer to wait?  Not clear, because we've changed the process for $V$.]

