\documentclass[11pt]{exam}

\oddsidemargin=0.25truein \evensidemargin=0.25truein
\topmargin=-0.5truein \textwidth=6.0truein \textheight=8.75truein

%\RequirePackage{graphicx}
\usepackage{comment}
\usepackage{verbatim}
\usepackage{booktabs}
\usepackage{hyperref}
\urlstyle{rm}   % change fonts for url's (from Chad Jones)
\hypersetup{
    colorlinks=true,        % kills boxes
    allcolors=blue,
    pdfsubject={ECON-UB233, Macroeconomic foundations for asset pricing},
    pdfauthor={Dave Backus @ NYU},
    pdfstartview={FitH},
    pdfpagemode={UseNone},
%    pdfnewwindow=true,      % links in new window
%    linkcolor=blue,         % color of internal links
%    citecolor=blue,         % color of links to bibliography
%    filecolor=blue,         % color of file links
%    urlcolor=blue           % color of external links
% see:  http://www.tug.org/applications/hyperref/manual.html
}

\renewcommand{\thefootnote}{\fnsymbol{footnote}}
\newcommand{\var}{\mbox{\it Var\/}}

\printanswers

% document starts here
\begin{document}
\parskip=\bigskipamount
\parindent=0.0in
\thispagestyle{empty}
{\large ECON-UB 233 \hfill Dave Backus @ NYU}

\bigskip\bigskip
\centerline{\Large \bf Lab Report \#4: Asset Pricing Fundamentals}
\centerline{Revised: \today}

\bigskip
{\it Due at the start of class.
You may speak to others, but whatever you hand in should be your own work.
Please include your Matlab code.}

\begin{questions}
%-----------------------------------------------------------------------
\question {\it State prices, pricing kernels, and risk-neutral probabilities.\/}
Consider a 3-state distribution in which the state $z$ takes on the values
$\{-1, 0, 1\}$ with
probabilities $\{\omega, 1-2\omega, \omega \}$.
A random variable $x$ is defined by $x(z) = \mu + \delta z$.
Statisticians would say that $x$ has a ``categorical distribution.''

We'll build a representative-agent economy on this distribution
by setting log consumption growth equal to $x$.
Utility has the usual additive form,
\begin{eqnarray*}
    u(c_0) + \beta \sum_z p(z) u[c_1(z)] ,
\end{eqnarray*}
with $u(c) = c^{1-\alpha}/(1-\alpha)$ (power utility).

\begin{parts}
\part What are the mean and variance of $x$?
\part What are the traditional measures of skewness and excess kurtosis,
$\gamma_1$ and $\gamma_2$?
Under what conditions is $\gamma_2$ large?
\part What is the pricing kernel?
\part What are the state prices?
\part What are the risk-neutral probabilities?
How do they differ from the true probabilities?
%How do they vary with risk aversion $\alpha$?
\part Which is more valuable, a claim to one unit of the good in state $z=-1$
or one unit in state $z=+1$? Why? \\
\end{parts}

\begin{solution}
The idea here is to go through a concrete example
and compute state prices, the pricing kernel,
and risk-neutral probabilities, and think about what each
one does.
\begin{parts}
\part The easiest way to do this is with the cgf.
The mgf is
\begin{eqnarray*}
    h(s) &=& \omega e^{\mu - \delta} + (1-2\omega) e^\mu + \omega e^{\mu+\delta}
\end{eqnarray*}
and the cgf is $k(s) = \log h(s)$.
The first four cumulants are
\begin{eqnarray*}
    \kappa_1 &=& \mu \\
    \kappa_2 &=& 2\delta^2 \omega \\
    \kappa_3 &=& 0 \\
    \kappa_4 &=& 2 \delta^4 \omega (1-6 \omega).
\end{eqnarray*}
The first one is the mean, the second one is the variance.

\part Skewness and excess kurtosis are
\begin{eqnarray*}
    \gamma_1  &=& \kappa_3/\kappa_2^{3/2} \;\;=\;\; 0 \\
    \gamma_2  &=& \kappa_4/\kappa_2^{2} \;\;=\;\; 1/(2 \omega) - 3 .
\end{eqnarray*}
The first is clear, because the distribution is symmetric.
Evidently $\gamma_2$ is large when $\omega$ is small.
It's zero when $\omega = 1/6$
\part The pricing kernel is
\begin{eqnarray*}
    m[x(z)] &=& \beta e^{-\alpha x}
            \;\;=\;\;  \beta e^{-\alpha (\mu + \delta z)} .
\end{eqnarray*}
It's decreasing in $x$ and $z$.
In other words, $m$ is higher in states where $z$ is lower (``bad states'').
\part The state prices are
\begin{eqnarray*}
    Q(z) &=& p(z) m(z)
            \;\;=\;\;  p(z) \beta e^{-\alpha (\mu + \delta z)} ,
\end{eqnarray*}
with $p(z)$ given in the previous question.

\item The risk-neutral probabilities are
\begin{eqnarray*}
    p^*(z) &=& p(z) m(z) /q^1
            \;\;=\;\;
            \left\{
            \begin{array}{ll}
            \omega \beta e^{-\alpha (\mu-\delta)}/q^1 & \mbox{ for } z=-1 \\
            (1-2\omega) \beta  e^{-\alpha \mu} /q^1          & \mbox{ for } z=0    \\
            \omega \beta e^{-\alpha (\mu +\delta)}/q^1 & \mbox{ for } z=1 \\
            \end{array}
            \right.
\end{eqnarray*}
where
\begin{eqnarray*}
    q^1 &=& \sum_z p(z) m(z)
        \;\;=\;\; \beta e^{-\alpha \mu} \left[ \omega (e^{\alpha \delta} + e^{-\alpha \delta}) + (1-2 \omega) \right] .
\end{eqnarray*}
If risk aversion $\alpha$ is zero,
then $m(z) = \beta$ in all states.
Otherwise, higher values of $\alpha$ raise the value
of the bad state and lower it in the good state.

\part The only difference is the term $e^{\pm \alpha \delta}$.
As long as $\delta > 0$, then $e^{\alpha \delta} > e^{-\alpha \delta}$
and we have $Q(-1) > Q(1)$:
state $z=-1$ is more valuable than state $z=1$.
That's the usual result, that bad states are more valuable than good states.
If $\delta < 0$ the answer's the same, but
$z=1$ is the bad state.
\end{parts}
\end{solution}

%-----------------------------------------------------------------------
\question {\it Excess kurtosis and the equity premium.\/}
Continue with the same environment.
We're going to vary $\omega$ and see how that affects the equity premium.
The idea is to explore the role of excess kurtosis, which we've seen
is controlled by $\omega$.
In all cases, set $\alpha = 10$ and $\beta = 0.99$.

\begin{parts}
\part We observe log consumption growth has a mean in US data of roughly 0.02
and a standard deviation of 0.035.
Given a value of $\omega$, what values of $\mu$ and $\delta$ reproduce
these values?

\part Suppose $\omega = 1/6$.  Why is this value a useful benchmark?
What values of $(\mu,\delta)$ reproduce the mean and variance of log consumption growth?

\part Suppose equity is a claim to the growth rate of consumption $e^x$.
What is the equity premium with these values?

\part Suppose $\omega = 1/20$.
How do your choices of $(\mu,\delta)$ adjust to keep the mean and variance
of log consumption growth at their observed values?
How does the equity premium change?
\end{parts}

\begin{solution}
Here the plan is to explore the possible role of changes in the distribution, 
specifically changes in excess kurtosis.  
\begin{parts}
\part The idea here is to choose parameters to reproduce the mean and variance
of log consumption growth in US data.
The mean is $\mu$, so we set $\mu = 0.02$.
The variance is $2\delta^2 \omega$.
Given a value of $\omega > 0$,
we set $2\delta^2 \omega = 0.035^2$ or
$\delta = 0.035/(2\omega)^{1/2} $.

\part With $\omega = 1/6$, excess kurtosis is zero,
the same value as the normal distribution.
To match the variance, we then need $\delta = 0.0606$.
$\mu$ remains 0.02.

\part
The equity premium is
$ E(r^e) - r^1 = 0.0143$ or 1.43\%.
See the Matlab code for the calculation.

\part If we reduce $\omega$ to 1/20, excess kurtosis goes up to 7.
The equity premium rises to 0.0160.
The impact is modest, but it illustrates the potential role
of adding more kurtosis to the distribution.


\end{parts}
\end{solution}

\end{questions}
%\end{document}

%\vfill \centerline{\it \copyright \ \number\year \ NYU Stern School of Business}


%\pagebreak
{\bf Matlab program:}
\verbatiminput{../Matlab/hw4_s13.m}
\end{document}


%  EXTRA STUFF


%-----------------------------------------------------------------------
\question {\it Disaster risk and the equity premium.\/}
We'll add a third ``disaster'' state to our analysis of the equity premium
and see how it changes our view of it.
The key input is the distribution of log consumption growth,
\begin{eqnarray*}
    \log g &=& \left\{
                \begin{array}{ll}
                \mu + \sigma & \mbox{with probability } (1-\omega)/2 \\
                \mu - \sigma & \mbox{with probability } (1-\omega)/2 \\
                \mu - \delta & \mbox{with probability } \omega
                \end{array}
                \right.
\end{eqnarray*}
What's the idea?
If $\omega = 0$, we're back to our symmetric two-state distribution.
But if we introduce a small positive value of $\omega$ and a ``largish'' $\delta>0$,
we have a ``disaster'' state that changes the distribution dramatically.

The question is what this does to the equity premium.
We define the equity premium in logs,
\begin{eqnarray*}
    E ( \log r^e - \log r^1 ) ,
\end{eqnarray*}
and aim at a target value of 0.0400 (4\%).
As before, we'll define equity in this model as a claim to consumption growth $g$.

The structure is roughly similar to that in the Matlab program we used in class,
but I encourage you to write your own.
(Why?   I usually find it harder to adapt someone else's code than write my own.
You're welcome to do either, but don't say I didn't warn you.)

\begin{parts}
\part If $\omega = 0$, what values of $\mu$ and $\sigma$ deliver
the observed mean and variance of log consumption growth, namely
0.0200 and $0.0350^2$?
\part Suppose $\beta = 0.99$ and $\alpha = 5$.
What are $r^1$, $\log r^1$, and
the equity premium,  $E \log r^e - \log r^1 $?

\part What is the equity premium if $\alpha$
equals 10?  20?

\part Now consider $\omega = 0.01$ and $\delta = 0.30$.
(These numbers are based on a series of studies by Robert Barro and his coauthors.)
With these numbers, what values of $\mu$ and $\sigma$ reproduce
the observed mean and variance of log consumption growth?

\part With (again) $\beta = 0.99$ and $\alpha = 5$,
what are $r^1$, $\log r^1$, $E \log r^e$, and
the equity premium,  $E \log r^e - \log r^1 $?

\part What is the equity premium if $\alpha$
equals 10?  20?
How does it compare to your previous calculations?

\part (extra credit)
How does the equity premium change if $\delta = - 0.30$,
so that the extreme state is good news rather than bad?
Why?

\part (extra credit) How does entropy differ between the
disaster and no-disaster cases?
\end{parts}

\begin{solution}
\begin{parts}
\part The expressions for the mean and variance of $\log g$ are
\begin{eqnarray*}
    E (\log g)   &=&  \mu + \omega \delta \;\;=\;\; 0.0200 \\
    \mbox{Var}(\log g) &=& (1-\omega) \sigma^2 + \omega (1-\omega) \delta^2
                \;\;=\;\; 0.0350^2 .
\end{eqnarray*}
When $\omega = 0$, the mean is $\mu = 0.0200$ and the standard
deviation is $\sigma = 0.0350$.

\part With these values, we have
$r^1 = 1.0995$, $\log r^1 = 0.0948$, and
$E \log r^e - \log r^1 = 0.0055 $.

\part The equity premium is 0.0112 when $\alpha = 10$ and 0.0208 when $\alpha = 20$.
Both are well below our target value of 0.0400.

\part When $\omega = 0.01$,
we need to set $\mu = 0.0230$ and $\sigma = 0.0183$ to maintain the mean and
variance at their sample values.
See (a).

\part With these values, we have
$r^1 = 1.0907$, $\log r^1 = 0.0868$, and
$E \log r^e - \log r^1 = 0.0097 $.
That's partial success: the equity premium went up,
even though we held the mean and variance of
log consumption growth constant.
In that respect, the disaster state is a useful innovation,
but we still need risk aversion above 5 to hit the equity premium.

\part The equity premium is 0.0438 when $\alpha = 10$
and 0.2285 when $\alpha = 20$.
Evidently a much smaller value of $\alpha$ suffices when
we have a disaster state.

\part If we switch the sign of $\delta$, the equity premium
is below even the two-state version:  0.0037.
Apparently positive skewness in consumption and dividend growth
isn't helpful.
It's not part of the question,
but this is connected to the discussion of skewness and entropy:
positive skewness in $\log m$ increases entropy.
With power utility, that requires negative skewness in $\log g$.

\part We compute entropy directly from $m$:
\begin{eqnarray*}
    H(m) &=& \log E (m) - E \log m .
\end{eqnarray*}
With $\alpha = 5$, entropy is 0.0232 with the disaster state,
0.0152 without.  So the disaster increases entropy.
The difference between the two increases with risk aversion.
When $\alpha = 20$,
entropy is 1.5675 with the disaster state,
0.2273 without.
Yaron's bazooka!

It's not part of the question, but changing the sign of
$\delta$ reverses all this:
we need disasters, not booms.
\end{parts}
\end{solution}
