\documentclass[11pt]{exam}

\oddsidemargin=0.25truein \evensidemargin=0.25truein
\topmargin=-0.5truein \textwidth=6.0truein \textheight=8.75truein

%\RequirePackage{graphicx}
\usepackage{comment}
\usepackage{verbatim}
\usepackage{booktabs}
\usepackage{hyperref}
\urlstyle{rm}   % change fonts for url's (from Chad Jones)
\hypersetup{
    colorlinks=true,        % kills boxes
    allcolors=blue,
    pdfsubject={ECON-UB233, Macroeconomic foundations for asset pricing},
    pdfauthor={Dave Backus @ NYU},
    pdfstartview={FitH},
    pdfpagemode={UseNone},
%    pdfnewwindow=true,      % links in new window
%    linkcolor=blue,         % color of internal links
%    citecolor=blue,         % color of links to bibliography
%    filecolor=blue,         % color of file links
%    urlcolor=blue           % color of external links
% see:  http://www.tug.org/applications/hyperref/manual.html
}

\renewcommand{\thefootnote}{\fnsymbol{footnote}}
\newcommand{\var}{\mbox{\it Var\/}}

\printanswers

% document starts here
\begin{document}
\parskip=\bigskipamount
\parindent=0.0in
\thispagestyle{empty}
{\large ECON-UB 233 \hfill Dave Backus @ NYU}

\bigskip\bigskip
\centerline{\Large \bf Lab Report \#3: Consumption, Risk, and Portfolio Choice}
\centerline{Revised: \today}

\bigskip
{\it Due at the start of class.
You may speak to others, but whatever you hand in should be your own work.
Please include your Matlab code.}

\begin{questions}

%-----------------------------------------------------------------------
\question {\it Exponential risks.\/}
Consider risk when when $x = - \log c $ is exponential,
a tractable nonnormal distribution.
The minus sign gives us negative skewness, which power utility agents
dislike.
The goal is to see how this shows up in the risk penalty.

The connection between $c$ and $x$ implies $c = e^{\log c} = e^{-x}$.
The pdf is
\begin{eqnarray*}
    p(x)  &=& \lambda e^{-\lambda x}
\end{eqnarray*}
for $x \geq 0$ and $\lambda > 0$.
The cgf is
\begin{eqnarray*}
    k(s)  &=& - \log \left( 1 - s/\lambda \right) .
\end{eqnarray*}
In terms of $x$, the mean of $c$ is $\bar{c} = E(c) = E (e^{-x})$.

\begin{parts}
\part Use the cgf to compute the variance, skewness, and excess kurtosis
of $x$.  How do they compare to those of the normal distribution?
\part Use the cgf to compute $\bar{c}$.
\part Choose the value of $\lambda$ that sets
the variance equal to $0.02^2$, the number we used earlier.
\part For extra credit:  how does this differ from the lognormal
example we did last week?  Why?
\end{parts}

\begin{solution}
Same approach as the previous problem.
\begin{parts}
\part The first four cumulants are
$\kappa_1 = 1/\lambda$,
$\kappa_2 = 1/\lambda^2$,
$\kappa_3 = 2/\lambda^3$,
and $\kappa_4 = 6/\lambda^4$.
Skewness and excess kurtosis are therefore
$\gamma_1 = 2$ and $\gamma_2 = 6$.

\part If $k(s)$ is the cgf of $x$,
then $\log \bar{c} = k(-1)$ ($k$ evaluated at $s=-1$):
$k(-1) = - \log (1 + 1/\lambda)$.

\part We solve $0.02^2 = 1/\lambda^2$ or $\lambda = 1/0.02 = 50$.

\part The risk penalties are
%
\begin{center}
\begin{tabular}{lccc}
        & $\alpha=2$  & $\alpha=10$  & $\alpha=20$ \\
\midrule
lognormal    &  0.0004 & 0.0020 & 0.0040 \\
exponential  &  0.0004 & 0.0022 & 0.0054
\end{tabular}
\end{center}
%
The point is that departures from normality matter, although they're not
especially large here.
The numbers suggest that the impact increases with risk aversion $\alpha$.
This may seem mysterious now, 
but we could elaborate on this using the power series expansion of the cgf.
That would show us that the $j$th cumulant of $\log g$ is multiplied
by $(-\alpha)^j/j!$, so as we increase $\alpha$ some of these terms can increase
dramatically.
Amir Yaron (Wharton prof) refers to this as a ``bazooka,''
since it allows you to take a little nonnormality
and generate enormous risk premiums.
More on this kind of thing later in the course.
\end{parts}
\end{solution}

%-----------------------------------------------------------------------
\question {\it Consumption, saving, and the interest rate.\/}
Here's a variant of the consumption-saving problem we did in class.
Let the agent have the problem
\begin{eqnarray*}
   \max_{c_0, c_1} &&  u(c_0) + \beta u(c_1) \\
        s.t. &&  c_0 + q c_1 \;\;=\;\; y_0 + q y_1
\end{eqnarray*}
with $u(c) = c^{1-\alpha}/(1-\alpha)$.
The idea of the budget constraint is that the present
value of consumption equals the present value of income.

\begin{parts}
\part How is $q$ related to the interest rate?
\part What is saving in this setting?
How is saving connected to second-period consumption $c_1$?
\part What are the first-order conditions for the problem?
\part What are the optimal choices of $c_0$ and $c_1$ given
the price $q$?
\end{parts}

\begin{solution}
We did all of this in class but the last part:  
solve explicitly for $(c_0,c_1)$. 
If $\lambda$ is the Lagrange multiplier on the constraint, 
the first-order conditions are 
\begin{eqnarray*}
    c_0:&&  u'(c_0) \;\;=\;\; \lambda \\
    c_1:&&  \beta u'(c_1) \;\;=\;\; \lambda q.
\end{eqnarray*}
With power utility, we have $u'(c) = c^{-\alpha}$.

We have three equations (the two foc's plus the budget constraint)
in the three unknowns $(c_0,c_1,\lambda)$.  
The goal here is to find $c_0$ and $c_1$ as functions of $q$.
Here's a line of attack that works.  
Solve the foc's for consumption:  
\begin{eqnarray*}
    c_0 &=& (1/\lambda)^{1/\alpha} \\
    c_1 &=& (1/\lambda)^{1/\alpha} (q/\beta)^{1/\alpha} .
\end{eqnarray*}
Then we have 
\begin{eqnarray*}
    c_0 + q c_1 &=& (1/\lambda)^{1/\alpha} \left[ 1 + (q/\beta)^{1/\alpha} \right] .
\end{eqnarray*}
From the budget constraint, this is $y_0 + q y_1 = Y$ (say), 
so we have 
\begin{eqnarray*}
    (1/\lambda)^{1/\alpha} &=&  {Y}/[1 + (q/\beta)^{1/\alpha} ].
\end{eqnarray*}
That gives us 
\begin{eqnarray*}
    c_0 &=& {Y}/[1 + (q/\beta)^{1/\alpha} ] \;\;=\;\; {Y}\beta^{1/\alpha} /(\beta^{1/\alpha} + q^{1/\alpha}  ) \\
    c_1 &=& {Y}q^{1/\alpha}/(\beta^{1/\alpha} + q^{1/\alpha} ) .
\end{eqnarray*}
You'll note that both are proportional to $Y$.
These preferences have the convenient property that they scale nicely:
if we double income, we double both consumptions.  
The technical term is ``homothetic.''



\end{solution}




%-----------------------------------------------------------------------
\question {\it Two-state portfolio choice.\/}
Portfolio choice problems are notoriously unfriendly,
but we can get a sense of their properties with an example.
The agent's consumption-saving-portfolio-choice problem is
\begin{eqnarray*}
   \max_{c_0, a} &&  u(c_0) + \beta \sum_z p(z) u[c_1(z)] \\
        s.t. &&  c_1(z)\;\;=\;\; (y_0-c_0)[(1-a) r^1 + a r^e(z)] .
\end{eqnarray*}
The idea is that we have two assets,
a riskfree asset with gross return $r^1$ (``$1$'' for one-period bond)
and a risky asset with gross return $r^e$  (``$e$'' for equity).
We invest a fraction $a$ of our saving in equity and the complementary
fraction $1-a$ in the riskfree bond.
If $a>1$, the agent has a levered position.
Note that this version of the problem is slightly different from the one we used in class.

We'll deal with a special case:
two states, $z=1$ and $z=2$;
equally likely, $p(1) = p(2) = 1/2$;
and power utility, $u(c) = c^{1-\alpha}/(1-\alpha)$.
Parameter values include
$\beta = 1/1.1$,
$r^1 = 1.1$,
$r^e(1) = 1.0$, $r^e(2) = 1.4$,
$ y_0 = 1$.

\begin{parts}
\part What are the mean and variance of the return on equity?
\part What are the implied prices of Arrow securities?
Hint:  Recall that Arrow securities pay off in one state only.
Our assets are combinations of Arrow securities,
so it's a question of unbundling them.
\part What are the first-order conditions for $c_0$ and $a$?
Show that with power utility,
the latter can be determined without knowing $c_0$.
\part Solve these conditions numerically for $\alpha = 5$.
What values do you get for $a$, $c_0$, $c_1(1)$, and $c_1(2)$?
Comment:  I did this
by varying $a$ until its first-order condition was satisfied.
You could also compute the first-order condition for a grid of values for $a$,
and choose the one that works best.
\part How does your answer change if you use $\alpha = 2$?
Does the difference make sense to you?
\end{parts}

\begin{solution}
\begin{parts}
\part The mean is 1.2 (the average of 1.0 and 1.4)
and the variance is $0.2^2 = 0.04$.

\part
Let $Q(z)$ be the price of the Arrow security that pays one
in state $z$.
A one-period bond pays one in each state, so its price is the sum:
\begin{eqnarray*}
    q^1 \; (=1/r^1)  &=&  Q(1) + Q(2)  .
\end{eqnarray*}
Equity is a little more complicated, because we have to agree on
units.  Let's choose units to that the price of a share is one
unit of date-0 consumption.
Then it pays off 1.0 units of date-1 consumption in state $z=1$
and 1.4 units in state $z=2$:
\begin{eqnarray*}
    q^e \; (=1)  &=&  1.0 Q(1) + 1.4 Q(2)  .
\end{eqnarray*}
Now it's a matter of algebra to find
$q^A(1) = 0.6818$ and $q^A(2) = 0.2273$.


\part The first-order conditions are
\begin{eqnarray*}
    c_0:  &&  u'(c_0) \;\;=\;\; \beta \sum_z p(z)
%            u' \left\{  (y_0-c_0) [ (1-a)r^1 + a r^e(z)] \right\}
            u'[c_1(z)] [ (1-a)r^1 + a r^e(z)] \\
    a:    && 0 \;\;=\;\; \beta \sum_z p(z) u'[c_1(z)] [r^e(z) - r^1 ]  .
\end{eqnarray*}
Written out more completely, the latter becomes
\begin{eqnarray*}
    0 &=& \beta \sum_z p(z) (y_0-c_0)^{1-\alpha} [ (1-a)r^1 + a r^e(z)]^{-\alpha} [r^e(z) - r^1 ] \\ 
    &=&  \sum_z p(z) [ (1-a)r^1 + a r^e(z)]^{-\alpha} [r^e(z) - r^1 ] .
\end{eqnarray*}
which depends on $a$ but not $c_0$.

\part I did this numerically.
We'll talk about numerical methods for solving equations shortly (``root-finding''),
but the simplest way is to compute the right side of the previous equation
for a grid of values for $a$.
We take the value that produces the first-order condition closest to zero.
A grid of 0.01 gives us $a=1.70$ when $\alpha=2$.

Now that we know $a$, we can find $c_0$ from the first first-order condition:
\begin{eqnarray*}
    c_0^{-\alpha}  &=&  \beta (y_0-c_0)^{-\alpha}
            \sum_z p(z) [ (1-a)r^1 + a r^e(z)]^{1-\alpha}  .
\end{eqnarray*}
We compute the sum on the rhs first --- denote it by $S$ (for sum).
Then we solve for $c_0$:
\begin{eqnarray*}
    (y_0-c_0)/c_0 &=& (\beta S)^{-1/\alpha}
        \;\;\Rightarrow\;\;
        c_0  \;\;=\;\; y_0 /[ 1 + (\beta S)^{-1/\alpha}] .
\end{eqnarray*}
Second-period consumption follows from the budget constraint:
$c_1(z)=  (y_0-c_0)[(1-a) r^1 + a r^e(z)] $.

The numbers are listed below:
\begin{center}
\begin{tabular}{lccccc}
                &  $a$  &  $S$    & $c_0$   &  $c_1(1)$ &  $c_1(2)$  \\
\midrule
$\alpha = 2$    & 1.702 & 0.8482  & 0.4676  & 0.4951    & 0.8576 \\
$\alpha = 5$    & 0.637 & 0.6135  & 0.4708  & 0.5484    & 0.6832
\end{tabular}
\end{center}

\part When $\alpha=5$, we get $a=0.64$ by the same method.
The punchline:  when we increase risk aversion, we hold less of the risky asset
and less risky second-period consumption as a result.
\end{parts}
\end{solution}


\end{questions}


\vfill \centerline{\it \copyright \ \number\year \
NYU Stern School of Business}

\pagebreak
{\bf Matlab program:}
\verbatiminput{../Matlab/hw3_s13.m}

\end{document}



%  EXTRA STUFF
%-----------------------------------------------------------------------
\question {\it Consumption, investment, and state prices.\/}
Consider a two-period economy with a linear technology.
What is equilibrium consumption growth?
What are the state prices?
What is a claim to one unit of capital worth?

To address these questions,
we use a variant of our two-period economy,
with dates 0 and 1 and states $z$ at date 1 that occur with probability $p(z)$.
The representative agent has utility function
\begin{eqnarray*}
    u(c_0) + \beta \sum_z p(z) u[c_1(z)] ,
\end{eqnarray*}
with $u(c) = c^{1-\alpha}/(1-\alpha)$ for $\alpha > 0$
(power utility).
She is endowed with $y_0$ units of the date-0 good, nothing at date 1.
The technology is linear:  $k$ units of the date-0 good invested in capital
generate $z k$ units of the good in state $z$ at date 1.
The resource constraints are therefore
\begin{eqnarray*}
    c_0 + k &=& y_0  \\
    c_1(z) &=& z k ,
\end{eqnarray*}
with one of the latter for each state $z$.
Think of the productivity factor $z$ as $a(z)$ with $a(z) = z$.

\begin{parts}
\part What are the classical ``ingredients'' of this economy?
\part What is the associated Pareto problem?
What are its first-order conditions?
\part Suppose $z$ is lognormal:  that is,
$\log z \sim \mathcal{N}(\kappa_1,\kappa_2)$.
Use the properties of lognormal random variables to show
that $E(z^a) = \exp( a \kappa_1 + a^2 \kappa_2/2)$
for any real number $a$.
\part Use this result to find the optimal values of $c_0$ and $k$.
Given these values, what is saving?
\part  (extra credit) What is $c_1(z)$?
% ?? next part a little tough
What are the state prices?
\part (extra credit) What is the value of one unit of capital,
that is, a claim to $z$ units of output in each state $z$?
%What are the effects of the parameters $(\alpha, \kappa_1,\kappa_2)$?
\end{parts}

\begin{solution}
\begin{parts}
\part Commodities:  the good at date 0 plus the good in all states at date 1 \\
Agents:  one \\
Preferences, endowment, and technologies:  given above \\
Resource constraints:  ditto
\part Pareto problem based on the Lagrangean:
\begin{eqnarray*}
    \mathcal{L}  &=&  \log c_0  + \beta \sum_z p(z) \log c_1(z) \\
         && + \; q_0 (y_0 - c_0 - k)
            + \sum_z q_1(z) [z k - c_1(z)] .
\end{eqnarray*}
We choose $c_0$, $k$, and $c_1(z)$ to maximize this.
The first-order conditions are
\begin{eqnarray*}
    c_0: &&  c_0^{-\alpha} \;\;=\;\; q_0 \\
    k:   &&  q_0 \;\;=\;\; \sum_z q_1(z) z \\
    c_1(z): &&  \beta p(z) c_1(z)^{-\alpha} \;\;=\;\; q_1(z)
\end{eqnarray*}

\part Define $x=\log z$.
Its mgf is $e^{sx} = E(e^{s\log z}) = E(z^s) = e^{s\kappa_1 + s^2 \kappa_2/2}$.
Setting $s=a$ gives us the answer.

\part This is moderately demanding, but here's how it works.
With apologies for mixing sums and integrals,
we use the first and third first-order conditions to substitute
for $q_0$ and $q_1(z)$ in the second:
\begin{eqnarray*}
        1 &=& \sum_z \frac{q_1(z)}{q_0} z
            \;\;=\;\; \sum_z \frac{\beta p(z) c_1(z)^{-\alpha}}{c_0^{-\alpha}} \; z
            \;\;=\;\; \sum_z \beta p(z) \frac{(zk)^{-\alpha}}{(y_0-k)^{-\alpha}} \; z \\
          &=& \beta (y_0-k)^\alpha k^{-\alpha} \sum_z  p(z)  z^{1-\alpha}
             \;\;=\;\;  \beta (y_0-k)^\alpha k^{-\alpha} E (z^{1-\alpha} ) .
\end{eqnarray*}
To simplify, denote $ E(z^{1-\alpha}) = Z$.
Then consumption and capital are
\begin{eqnarray*}
    k &=& \frac{(\beta Z)^{1/\alpha}}{1+ (\beta Z)^{1/\alpha}} \; y_0,
        \;\;\;  c_0 \;\;=\;\; \frac{1}{1+ (\beta Z)^{1/\alpha}} \; y_0
\end{eqnarray*}
Using the lognormal result, we have
\begin{eqnarray*}
    Z &=& E(z^{1-\alpha}) \;\;=\;\; e^{(1-\alpha)\kappa_1 + (1-\alpha)^2 \kappa_2/2}
\end{eqnarray*}

\part Evidently
\begin{eqnarray*}
    c_1(z) &=& z k \;\;=\;\; z y_0 \frac{(\beta Z)^{1/\alpha}}{1+ (\beta Z)^{1/\alpha}} .
\end{eqnarray*}
The big part at the end is a constant, so it's ugly but innocuous.
The state prices are (more tedious substitution)
\begin{eqnarray*}
    q(z) &=& p(z) \beta [c_1(z)/c_0]^{-\alpha}
            \;\;=\;\; p(z) z^{-\alpha}/Z .
\end{eqnarray*}
As usual, the state price is the product of
a pricing kernel [here $m(z) = z^{-\alpha}/Z$]
and a probability [the normal density for $\log z$,
which we haven't bothered to write out].

\part
This is claim to $z$ next period,
with value at date 0 of
\begin{eqnarray*}
    q^e &=& E (z^{1-\alpha}/Z \;\;=\;\; 1.
\end{eqnarray*}
Hmmmm...
Why does this make sense?
Because one unit of capital is one unit of the good
at date 0, whose price is one since we're valuing
assets in units of the date-0 good.


\end{parts}
\end{solution}


%-----------------------------------------------------------------------
\question (labor supply and demand)
One of the central variables in macroeconomics is employment, so it's helpful
to have something like that in our models.
We do that here with a modest extension of our two-period economy,
with dates 0 and 1 and date-1 states $z$ that occur with probability $p(z)$.

Suppose the representative agent has one unit of labor each period,
which she can use to work or enjoy as leisure (everything but work).
Her utility function might be written
\begin{eqnarray*}
    u(c_0,1-n_0) + \beta \sum_z p(z) u[c_1(z),1-n_1(z)] ,
\end{eqnarray*}
with $n$ being the quantity of time spent working and $1-n$ the quantity of leisure.
There are no endowments other than the one unit of labor in each period.

Production uses labor as follows.  At date-0, each unit of labor used in production
generates on unit of consumption.
The resource constraint for date-0 consumption is therefore
\begin{eqnarray*}
    c_0 + k &=& n_0 .
\end{eqnarray*}
Production at date 1 in state $z$ uses both capital and labor,
giving us the date-1 resource constraints
\begin{eqnarray*}
    c_1(z) &=& z f[k,n(z)] ,
\end{eqnarray*}
one for each state $z$.
There are a finite number of states $z$,
each occurring with probability $p(z)$.
Productivity is also $z$.

\begin{parts}
\part What are the classical ``ingredients'' of this economy?
\part What is the associated Pareto problem?
\part Solve the Pareto problem for the special case
$u(c,1-n) = \log c + \lambda \log (1-n)$ and $f(k,n) = k^\theta n^{1-\theta} $.
Comment:  If you get stuck, try the deterministic version.
\part How does $c_1$ vary with $z$?  Why?  Does that seem realistic to you?
\part How does $n$ vary with $z$?  Why?  Does that seem realistic to you?
\part Wage = mpn ...  How does that vary with $z$?
%\part Equity:  claim to mpk times k.
\end{parts}

\begin{solution}
\begin{parts}
\part Commodities:  the good at date 0, labor/leisure at date 0,
and the same in all states at date 1  \\
Agents:  one \\
Preferences, endowment, and technologies:  given above \\
Resource constraints:  ditto
\part Pareto problem based on the Lagrangean:
\begin{eqnarray*}
    \mathcal{L}  &=&  \left[ \log c_0 + \lambda \log (1-n_0) \right]
            + \beta \sum_z p(z) \left[ \log c_1(z) + \lambda \log [1-n_1(z)] \right]  \\
         && + \; q_0 (n_0 - c_0 - k)
            + \sum_z q_1(z) [z k^\theta n_1(z)^{1-\theta} - c_1(z)] .
\end{eqnarray*}
We choose consumption, leisure, and capital to maximize this.
\part The first-order conditions are

\end{parts}
\end{solution}

