\documentclass[11pt]{exam}

\oddsidemargin=0.25truein \evensidemargin=0.25truein
\topmargin=-0.5truein \textwidth=6.0truein \textheight=8.75truein

%\RequirePackage{graphicx}
\usepackage{comment}
\usepackage{verbatim}
\usepackage{booktabs}
\usepackage{hyperref}
\urlstyle{rm}   % change fonts for url's (from Chad Jones)
\hypersetup{
    colorlinks=true,        % kills boxes
    allcolors=blue,
    pdfsubject={ECON-UB233, Macroeconomic foundations for asset pricing},
    pdfauthor={Dave Backus @ NYU},
    pdfstartview={FitH},
    pdfpagemode={UseNone},
%    pdfnewwindow=true,      % links in new window
%    linkcolor=blue,         % color of internal links
%    citecolor=blue,         % color of links to bibliography
%    filecolor=blue,         % color of file links
%    urlcolor=blue           % color of external links
% see:  http://www.tug.org/applications/hyperref/manual.html
}

\renewcommand{\thefootnote}{\fnsymbol{footnote}}
\newcommand{\var}{\mbox{\it Var\/}}

%\printanswers

% document starts here
\begin{document}
\parskip=\bigskipamount
\parindent=0.0in
\thispagestyle{empty}
{\large ECON-UB 233 \hfill Dave Backus @ NYU}

\bigskip\bigskip
\centerline{\Large \bf Lab Report \#2: Certainty Equivalents \& Portfolio Choice}
\centerline{Revised: \today}

\bigskip
{\it Due at the start of class.
You may speak to others, but whatever you hand in should be your own work.
Please include your Matlab code.}

\begin{questions}

\begin{solution}
Most of the answers are computed in the Matlab
program listed at the end and posted
\href{http://pages.stern.nyu.edu/~dbackus/233/hw2_s12.m}{here}.
\end{solution}

%-----------------------------------------------------------------------
\question {\it The sum of normals is normal.\/}
The idea here is to use cumulant generating functions (cgfs)
to show that the sum of independent normal random variables is also normal.
It's helpful to break the problem into manageable pieces, like this:
\begin{parts}
\part Consider two independent random variables $x_1$ and $x_2$,
    not necessarily normal.
    Show that the cgf of the sum $y = x_1 + x_2$ is the sum of their cgfs:
\begin{eqnarray*}
	k(s; y) &=& k(s; x_1) + k(s;x_2) .
\end{eqnarray*}
Hint:  note the form of the pdf and apply the definition of the cgf.

\part Suppose $x_i \sim \mathcal{N}(\kappa_{i1}, \kappa_{i2})$.
[This bit of notation means:   $x_i$ is normally distributed with
mean $\kappa_{i1}$ and variance $\kappa_{i2}$.]
What is $x_i$'s cgf?
Hint:  we did this in class.

\part Use (a) to find the cgf of $y = x_1 + x_2$,
with $(x_1,x_2)$ as described in (b) (namely, normal with given means and variances).
How do you know that $y$ is also normal?  What are its mean and variance?

\part Extend this result to $y = a x_1 + b x_2$ for any real numbers $(a,b)$.
\end{parts}

\begin{solution}
\begin{parts}
\part If $x_1$ and $x_2$ are independent, they're pdf factors:
$ p_{12}(x_1,x_2) = p_1(x_1) p_2(x_2) $.
That means the mgf of $x_1+x_2$ is the product of their individual mgf's
and their cgf is the sum.
\part $  k(s; x_i) = s \kappa_{1i} + s^2 \kappa_{2i}/2 $.
(If this isn't burned into your memory, please burn it now.)
\part Sum the cgf's:
\begin{eqnarray*}
    k(s; y) &=& k(s,x_1) + k(s,x_2) \\
            &=& (s \kappa_{11} + s^2 \kappa_{21}/2) + (s\kappa_{12} + s^2 \kappa_{22}/2) \\
            &=& s (\kappa_{11} + \kappa_{12} )
                + s^2 (\kappa_{21} + \kappa_{22}) .
\end{eqnarray*}
It's normal because it's mgf has the form of a normal random variable.
\part Still normal, but with a slight change in mean and variance:
\begin{eqnarray*}
    k(s; y) &=& s (a \kappa_{11} + b \kappa_{12} )
                + s^2 (a^2 \kappa_{21} + b^2 \kappa_{22})/2  .
\end{eqnarray*}
\end{parts}
\end{solution}

%-----------------------------------------------------------------------
\question {\it Lognormal risks.\/}
Our mission is to compute the {\it risk penalty\/}
associated with lognormal risks and power utility.
What risks matter?  What role does the risk aversion parameter play?

To be concrete, let $x = \log c \sim \mathcal{N}(\kappa_1, \kappa_2)$
and $u(c) = c^{1-\alpha}/(1-\alpha)$ for $\alpha> 0$.
The risk penalty is
\begin{eqnarray*}
    \mbox{\it rp\/} &=& \log (\bar{c}/\mu)
                \;\;=\;\; \log \bar{c} - \log \mu ,
\end{eqnarray*}
where $\bar{c} = E (c) = E( e^{x})$ and $\mu$ is the certainty equivalent.
We'll compute the terms one by one.

\begin{parts}
\part What is the cumulant generating function (cgf) of $x = \log c$?
\part Use the cgf to compute $\bar{c}$.
\part Use the cgf to compute the certainty equivalent $\mu$.
\part What is the risk penalty?  How does it depend on risk?  Risk aversion?
\part Suppose $\kappa_2 = 0.02^2$.  What is the risk penalty if $\alpha = 2$?
$\alpha = 10$?  $\alpha = 20$?
\end{parts}

\begin{solution}
Here risk is the variance $\kappa_2$ and risk aversion is the preference
parameter $\alpha$.
The risk penalty is $\alpha\kappa_2/2$, so both risk and risk aversion matter.
The cgf is used to compute all of the components:
$\bar{c}$, $\mu$, and $\mbox{\it rp\/}$.
See the Matlab program for details.
\end{solution}

%-----------------------------------------------------------------------
\question {\it Exponential risks.\/}
Now consider the same problem when $x = - \log c $ is exponential,
a tractable nonnormal distribution.
The connection between $c$ and $x$ implies $c = e^{\log c} = e^{-x}$.
The pdf is
\begin{eqnarray*}
    p(x)  &=& \lambda e^{-\lambda x}
\end{eqnarray*}
for $x \geq 0$ and $\lambda > 0$.
The cgf is
\begin{eqnarray*}
    k(s)  &=& - \log \left( 1 - s/\lambda \right) .
\end{eqnarray*}
In terms of $x$, the mean of $c$ is $\bar{c} = E(c) = E (e^{-x})$.

\begin{parts}
\part Use the cgf to compute the variance, skewness, and excess kurtosis
of $x$.  How do they compare to those of the normal distribution?
\part Use the cgf to compute $\bar{c}$.
\part Choose the value of $\lambda$ that sets
the variance equal to $0.02^2$, the number we used earlier.
\part
What is the risk penalty if $\alpha = 2$? $\alpha = 10$?
$\alpha = 20$?
How does your answer differ from the previous one?  Why?
\end{parts}

\begin{solution}
Same approach as the previous problem.
\begin{parts}
\part The first four cumulants are
$\kappa_1 = 1/\lambda$,
$\kappa_2 = 1/\lambda^2$,
$\kappa_3 = 2/\lambda^3$,
and $\kappa_4 = 6/\lambda^4$.
Skewness and excess kurtosis are therefore
$\gamma_1 = 2$ and $\gamma_2 6$.

\part If $k(s)$ is the cgf of $x$,
then $\log \bar{c} = k(-1)$ ($k$ evaluated at $s=-1$):
$k(-1) = - \log (1 + 1/\lambda)$.

\part We solve $0.02^2 = 1/\lambda^2$ or $\lambda = 1/0.02 = 50$.


\part The risk penalties are
%
\begin{center}
\begin{tabular}{lccc}
        & $\alpha=2$  & $\alpha=10$  & $\alpha=20$ \\
\midrule
lognormal    &  0.0004 & 0.0020 & 0.0040 \\
exponential  &  0.0004 & 0.0022 & 0.0054
\end{tabular}
\end{center}
%
The point is that departures from normality matter, although they're not
especially large here.
The numbers suggest that the impact increases with risk aversion $\alpha$.
We could elaborate on this using the power series expansion of the cgf,
which would show you that the $j$th cumulant is multiplied
by $(-\alpha)^j/j!$, so as we increase $\alpha$ some of these terms can increase
dramatically.
Amir Yaron (Wharton prof) refers to this as a ``bazooka,''
since it allows you to take a little nonnormality
and generate enormous risk premiums.
More on this kind of thing later in the course.
\end{parts}
\end{solution}


%-----------------------------------------------------------------------
\question {\it Two-state portfolio choice.\/}
Portfolio choice problems are notoriously unfriendly,
but we can get a sense of their properties with an example.
The agent's consumption-saving-portfolio-choice problem is
\begin{eqnarray*}
   \max_{c_0, a} &&  u(c_0) + \beta \sum_z p(z) u[c_1(z)] \\
        s.t. &&  c_1(z)\;\;=\;\; (y_0-c_0)[(1-a) r^1 + a r^e(z)] .
\end{eqnarray*}
The idea is that we have two assets,
a riskfree asset with gross return $r^1$ (``$1$'' for one-period bond)
and a risky asset with gross return $r^e$  (``$e$'' for equity).
We invest a fraction $a$ of our saving in equity and the complementary
fraction $1-a$ in the riskfree bond.
If $a>1$, the agent has a levered position.
Note that this is slightly different from the version we used in class.

For most of this question, we'll deal with a special case:
two states, $z=1$ and $z=2$;
equally likely, $p(1) = p(2) = 1/2$;
and power utility, $u(c) = c^{1-\alpha}/(1-\alpha)$.
Parameter values include
$\beta = 1/1.1$,
$r^1 = 1.1$,
$r^e(1) = 1.0$, $r^e(2) = 1.4$,
$ y_0 = 1$.

\begin{parts}
\part What are the mean and variance of the return on equity?
\part What are the implied prices of Arrow securities?
Hint:  Recall that Arrow securities pay off in one state only.
Our assets are combinations of Arrow securities,
so it's a question of unbundling them.
\part What are the first-order conditions for $c_0$ and $a$?
Show that with power utility,
the latter can be determined without knowing $c_0$.
\part Solve these conditions for $\alpha = 5$.
What values do you get for $a$, $c_0$, $c_1(1)$, and $c_1(2)$?
Comment:  You can do this by hand, but I did it numerically
by varying $a$ until its first-order condition was satisfied.
\part How does your answer change if you use $\alpha = 2$?
Does the difference make sense to you?
\end{parts}

\begin{solution}
\begin{parts}
\part The mean is 1.2 (the average of 1.0 and 1.4)
and the variance is $0.2^2 = 0.04$.

\part
Let $q^A(z)$ be the price of the Arrow security that pays one
in state $z$.
A one-period bond pays one in each state, so its price is the sum:
\begin{eqnarray*}
    q^1 \; (=1/r^1)  &=&  q^A(1) + q^A(2)  .
\end{eqnarray*}
Equity is a little more complicated, because we have to agree on
units.  Let's choose units to that the price of a share is one
unit of date-0 consumption.
Then it pays off 1.0 units of date-1 consumption in state $z=1$
and 1.4 units in state $z=2$:
\begin{eqnarray*}
    q^e \; (=1)  &=&  1.0 q^A(1) + 1.4 q^A(2)  .
\end{eqnarray*}
Now it's a matter of algebra to find
$q^A(1) = 0.6818$ and $q^A(2) = 0.2273$.


\part The first-order conditions are
\begin{eqnarray*}
    c_0:  &&  u'(c_0) \;\;=\;\; \beta \sum_z p(z)
%            u' \left\{  (y_0-c_0) [ (1-a)r^1 + a r^e(z)] \right\}
            u'[c_1(z)] [ (1-a)r^1 + a r^e(z)] \\
    a:    && 0 \;\;=\;\; \beta \sum_z p(z) u'[c_1(z)] [r^e(z) - r^1 ]  .
\end{eqnarray*}
Written out more completely, the latter becomes
\begin{eqnarray*}
    0 &=& \beta \sum_z p(z) (y_0-c_0)^{1-\alpha} [ (1-a)r^1 + a r^e(z)]^{-\alpha} [r^e(z) - r^1 ] \\ &=&  \sum_z p(z) [ (1-a)r^1 + a r^e(z)]^{-\alpha} [r^e(z) - r^1 ] .
\end{eqnarray*}
which depends on $a$ but not $c_0$.

\part I did this numerically.
We'll talk about numerical methods for solving equations shortly (``root-finding''),
but the simplest way is to compute the right side of the previous equation
for a grid of values for $a$.
We take the value that produces the first-order condition closest to zero.
A grid of 0.01 gives us $a=1.70$ when $\alpha=2$.

Now that we know $a$, we can find $c_0$ from the first first-order condition:
\begin{eqnarray*}
    c_0^{-\alpha}  &=&  \beta (y_0-c_0)^{-\alpha}
            \sum_z p(z) [ (1-a)r^1 + a r^e(z)]^{1-\alpha}  .
\end{eqnarray*}
We compute the sum on the rhs first --- denote it by $S$ (for sum).
Then we solve for $c_0$:
\begin{eqnarray*}
    (y_0-c_0)/c_0 &=& (\beta S)^{-1/\alpha}
        \;\;\Rightarrow\;\;
        c_0  \;\;=\;\; y_0 /[ 1 + (\beta S)^{-1/\alpha}] .
\end{eqnarray*}
Second-period consumption follows from the budget constraint:
$c_1(z)=  (y_0-c_0)[(1-a) r^1 + a r^e(z)] $.

The numbers are listed below:
\begin{center}
\begin{tabular}{lccccc}
                &  $a$  &  $S$    & $c_0$   &  $c_1(1)$ &  $c_1(2)$  \\
\midrule
$\alpha = 2$    & 1.702 & 0.8482  & 0.4676  & 0.4951    & 0.8576 \\
$\alpha = 5$    & 0.637 & 0.6135  & 0.4708  & 0.5484    & 0.6832
\end{tabular}
\end{center}


\part When $\alpha=5$, we get $a=0.64$ by the same method.
The punchline:  when we increase risk aversion, we hold less of the risky asset
and less risky second-period consumption as a result.

\end{parts}
\end{solution}
\end{questions}

\vfill \centerline{\it \copyright \ \number\year \
NYU Stern School of Business}

\end{document}


\pagebreak
Matlab program:
\verbatiminput{../Matlab/hw2_s12.m}

\end{document}



