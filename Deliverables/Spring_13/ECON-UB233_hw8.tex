\documentclass[11pt]{exam}

\oddsidemargin=0.25truein \evensidemargin=0.25truein
\topmargin=-0.5truein \textwidth=6.0truein \textheight=8.75truein

\usepackage{comment}
\usepackage{verbatim}
\usepackage{booktabs}
\usepackage{graphicx}
\usepackage{hyperref}
\urlstyle{rm}   % change fonts for url's (from Chad Jones)
\hypersetup{
    colorlinks=true,        % kills boxes
    allcolors=blue,
    pdfsubject={ECON-UB233, Macroeconomic foundations for asset pricing},
    pdfauthor={Dave Backus @ NYU},
    pdfstartview={FitH},
    pdfpagemode={UseNone},
%    pdfnewwindow=true,      % links in new window
%    linkcolor=blue,         % color of internal links
%    citecolor=blue,         % color of links to bibliography
%    filecolor=blue,         % color of file links
%    urlcolor=blue           % color of external links
% see:  http://www.tug.org/applications/hyperref/manual.html
}

\renewcommand{\thefootnote}{\fnsymbol{footnote}}
\newcommand{\var}{\mbox{\it Var\/}}

%\printanswers

% document starts here
\begin{document}
\parskip=\bigskipamount
\parindent=0.0in
\thispagestyle{empty}
{\large ECON-UB 233 \hfill Dave Backus @ NYU}

\bigskip\bigskip
\centerline{\Large \bf Lab Report \#8:  Bond Prices \& Predictable Returns}
\centerline{Revised: \today}

\bigskip
{\it Due at the start of class.
You may speak to others, but whatever you hand in should be your own work.
Please include your Matlab code.}

\begin{questions}

%\begin{solution}
%Answers follow.  See Matlab code at the end for calculations.
%\end{solution}

%-----------------------------------------------------------------------
\question {\it Bond basics.\/}
Consider the bond prices at some date $t$:

\medskip
\begin{center}
\tabcolsep=0.1in
\begin{tabular}{lrrr}
\toprule
Maturity $n$   & \phantom{xx} Price $q^n$ \\ %& \phantom{xx} Yield $y^n$ & Forward $f^{n-1}$ \\
\midrule
0           &    1.0000   \\
1 year      &    0.9512   \\
2 years     &    0.8958   \\
3 years     &    0.8353   \\
4 years     &    0.7788   \\
5 years     &    0.7261   \\
\bottomrule
\end{tabular}
\end{center}
\medskip

\begin{parts}
\part What are the yields $y^n$?
\part What are the forward rates $f^{n-1}$?
\part Suppose $f^1_t$ rises by 1 percent.
How does $y^1_t$ change?
How does $y^2_t$ change?
\end{parts}

\begin{solution}
\begin{center}
\tabcolsep=0.1in
\begin{tabular}{lrrr}
\toprule
Maturity $n$   & \phantom{xx} Price $q^n$ & \phantom{xx} Yield $y^n$ & Forward $f^{n-1}$ \\
\midrule
0           &    1.0000  \\
1 year      &    0.9512  & 0.0500 & 0.0500 \\
2 years     &    0.8958  & 0.0550 & 0.0600 \\
3 years     &    0.8353  & 0.0600 & 0.0700 \\
4 years     &    0.7788  & 0.0625 & 0.0700 \\
5 years     &    0.7261  & 0.0640 & 0.0700 \\
\bottomrule
\end{tabular}
\end{center}
\medskip

\begin{parts}
\part See above.
\part See above.
\part If $f^1$ rises by 1\%, then $y^1= f^0$ doesn't change
and $y^2 = (f^0+f^1)/2 $ rises by 0.5\%.
\end{parts}
\end{solution}


%-----------------------------------------------------------------------
\question {\it Exponential-affine equity valuation.\/}
We explore variation in expected excess returns
(``risk premiums'') in an environment like the one we used for bond pricing.
We say a return or expected return is predictable if it depends on
the state.
If we know the state, we can use it to ``predict'' the return.
In the asset management business, you might imagine strategies to exploit such
information.
For example, you might invest more in assets when their expected returns are high,
and less when they're low.

Here's the model.  The pricing kernel is
\begin{eqnarray*}
    \log m_{t+1} &=& - (\lambda_0 + \lambda_1 x_t)^2/2 - x_t
                + (\lambda_0 + \lambda_1 x_t) w_{t+1} \\
        x_{t+1} &=& (1-\varphi) \delta + \varphi x_t + \sigma w_{t+1} ,
\end{eqnarray*}
where $\{ w_t \}$ is (as usual) a sequence of independent standard normal random variables.
This model differs from Vasicek in having a price of risk $\lambda$ that
depends on the state $x_t$.
The dependence is carefully designed to deliver loglinear (``exponential-affine'')
bond prices.

\begin{parts}
\part What is the (stationary) mean of $x_t$?  The variance?
\part What is the price $q^1_t$ of a one-period bond?
What is its (gross) return $r^1_{t+1}$?
\part What is the mean log return $E (\log r^1_{t+1})$?
What parameter(s) govern its value in the model?
\part Now consider the price of ``equity,'' a claim to the dividend
\begin{eqnarray*}
    \log d_{t+1} &=&  \alpha + \beta w_{t+1} .
\end{eqnarray*}
What is its price $q^e_t$?
Its (gross) return $r^e_{t+1}$?
\part What is the expected excess return $ E_t (r^e_{t+1} - r^1_{t+1})$ in state $x_t$?
In what states is it high?  In what states low?
\part Which equation is responsible for the predictability of the expected return?
\end{parts}
Comment:  most of these calculations are simpler in logs.
Eg, compute $\log q^1_t$ rather than $q^1_t$.


\begin{solution}
The point is that we can generate predictable excess
returns for equity just as we did for bonds.
\begin{parts}
\part $x_t$ is an AR(1).
The mean is $\delta$ and the variance is $\sigma^2/(1-\varphi^2)$.
\part  The model is set up to deliver
\begin{eqnarray*}
    \log q^1_t &=& \log E_t (m_{t+1}) \;\;=\;\; - x_t \\
    \log r^1_{t+1} &=& - \log q^1_t \;\;=\;\; x_t .
\end{eqnarray*}
\part The mean log return is the mean of $x_t$, namely $\delta$.
\part The price is
\begin{eqnarray*}
    \log q^e_t &=& \log E_t (m_{t+1} d_{t+1})
            \;\;=\;\; \alpha - x_t + \beta^2/2 + \beta ( \lambda_0+ \lambda_1 x_t) .
\end{eqnarray*}
The return is
\begin{eqnarray*}
    \log r^e_{t+1} &=& \log d_{t+1} - \log q^e_t
            \;\;=\;\;  x_t - \beta^2/2  -
                \beta (\lambda_0 + \lambda_1 x_t) + \beta w_{t+1}.
\end{eqnarray*}
\part The expected excess return is
\begin{eqnarray*}
   E_t ( \log r^e_{t+1}- r^1_{t+1}) &=&
             - \beta^2/2  -  \beta (\lambda_0 + \lambda_1 x_t)
\end{eqnarray*}
The second term is predictable through $x_t$.
\end{parts}
\end{solution}


\end{questions}

%\vfill \centerline{\it \copyright \ \number\year \ NYU Stern School of Business}

%\end{document}

%\pagebreak
{\bf Matlab program:}
\verbatiminput{../Matlab/hw8_s13.m}

\vfill \centerline{\it \copyright \ \number\year \
NYU Stern School of Business}


\end{document}


%  EXTRA STUFF


