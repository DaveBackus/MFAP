\documentclass[11pt]{article}

\oddsidemargin=0.25truein \evensidemargin=0.25truein
\topmargin=-0.5truein \textwidth=6.0truein \textheight=8.75truein

%\RequirePackage{graphicx}
\usepackage{comment}
\usepackage[dvipdfm]{hyperref}
\urlstyle{rm}   % change fonts for url's (from Chad Jones)
\hypersetup{
    colorlinks=true,        % kills boxes
    allcolors=blue,
    pdfsubject={ECON-UB233, Macroeconomic foundations for asset pricing},
    pdfauthor={Dave Backus @ NYU},
    pdfstartview={FitH},
    pdfpagemode={UseNone},
%    pdfnewwindow=true,      % links in new window
%    linkcolor=blue,         % color of internal links
%    citecolor=blue,         % color of links to bibliography
%    filecolor=blue,         % color of file links
%    urlcolor=blue           % color of external links
% see:  http://www.tug.org/applications/hyperref/manual.html
}

%\renewcommand{\thefootnote}{\fnsymbol{footnote}}

% document starts here
\begin{document}
\parskip=\bigskipamount
\parindent=0.0in
\thispagestyle{empty}
{\large ECON-UB 233 \hfill Dave Backus @ NYU}

\bigskip\bigskip
\centerline{\Large \bf Syllabus:  Macroeconomic Foundations for Asset Prices}
%\bigskip\bigskip
%\centerline{\Large \bf Syllabus}
\centerline{(Started: July 12, 2011; Revised: \today)}

\subsection*{Overview}

This course is about links between business cycles
and asset prices ---
between the ups and downs of the economy as a whole
and returns on bonds, equity indexes, and related assets.
It's also about the tools used to study these links:
mathematical tools, software tools, and economic tools.
This kind of systematic analysis of economic data is common in
investment management and research, consulting,
and the business world more generally.
I expect it to be a demanding course, but a useful one,
whether your plans call for Wall Street, Main Street,
graduate school, or something else entirely.


The skills we'll develop to collect, manipulate,
and interpret data
are among the most valuable you can have in modern life,
and they're not easy to learn on your own.
You will learn to think about data from the
perspective of quantitative models,
to use professional software
(Matlab, not Excel!) to collect and manipulate data,
and to use data to apply and develop models.
Our applications are to macroeconomics and finance,
but the skills you acquire here are general ones.


We will touch on:
business cycle indicators,
the relation between economic growth and asset returns,
``arbitrage-free'' asset pricing,
equity index options as bets on economic growth,
the implied volatility smile,
leading and lagging indicators,
real (inflation-indexed) and nominal bonds,
and possibly monetary policy and inflation.
If there are other topics of burning interest, let me know
and I'll see if I can work them in.
Each of these topics is preceded by a class or two on
the ``math tools'' needed to do justice to it.
A complete set of topics and materials is posted on
the course web site:

\bigskip
\centerline{\url{https://sites.google.com/site/nyusternmacrofoundations/}}

The course is part of our evolving ``frontiers of economics''
sequence.
Some knowledge of calculus, probability theory,
and linear algebra is recommended,
but more important are your willingness to engage in quantitative thinking
and courage to learn things you don't yet know.


\subsection*{Materials}

There is no textbook.  Everything you need will be
posted on the course website.
The notes are likely to be terse and dense
(what can I say, that's what passes for my style),
but I expect us to breathe life into the material in class.

If you would like additional reading on a particular subject,
let me know.
Despite its reputation,
Wikipedia is often good for mathematics.

You may choose to buy Matlab.  See Lab Report \#0 for instructions.


\subsection*{Requirements}

The course is a mixture of economic ideas and
the mathematical and software tools needed to put those ideas to practical use.
The best way to learn both is by using them.
We do that with lots of ``lab reports'' (short assignments, almost one a week).
Three in-class quizzes provide opportunities to consolidate your knowledge
and show what you have learned.
The theory here is to do a little work all the time rather than lots of
work once in a while.

Your grade will be computed from
\begin{center}
\begin{tabular}{ll}
Quiz \#1    &   25\% \\
Quiz \#2    &   25\% \\
Quiz \#3    &   25\% \\
Lab Reports  \hspace*{0.25in}    &   25\%  (best 6 of 8)
\end{tabular}
\end{center}
I will drop the two lowest grades on lab reports,
but doing them all will be a useful learning experience
and an indication of your dedication.
Final grades are not subject to any fixed distribution.
The number of A grades, for example,
will depend only on your performance in the course.
If you make a good-faith effort,
I expect it to be hard to get less than a B.


\subsection*{Important dates}

Lab Reports are due January 30, February 6, February 13, February 22,
March 21, March 28, April 4, April 23, and April 30.
%
They must be submitted at the start of class --- late reports will not be accepted.

Quizzes are scheduled for February 27, April 9, and May 7.

All of these dates are listed on the course calendar,
available on the course website.
You can also combine it with your own calendar, but that takes a little more work.
Ask me if you're interested.
%If you're in Google Calendar, here are the instructions:
%Click on the arrow next to ``other calendars,''
%choose ``add by url,''
%and paste in this link:

%\url{https://www.google.com/calendar/ical/ggqlihrhph8tp6368225id811c%40group.calendar.google.com/public/basic.ics}


\subsection*{Policies}

Ethics, disabilities, and many other things are governed by NYU
and Stern policies.
If you have any questions about them, please ask me.

\end{document}
