\documentclass[11pt]{exam}

\oddsidemargin=0.25truein \evensidemargin=0.25truein
\topmargin=-0.5truein \textwidth=6.0truein \textheight=8.75truein

%\RequirePackage{graphicx}
\usepackage{verbatim}
\usepackage{booktabs}
\usepackage{comment}
\usepackage{hyperref}
\urlstyle{rm}   % change fonts for url's (from Chad Jones)
\hypersetup{
    colorlinks=true,        % kills boxes
    allcolors=blue,
    pdfsubject={ECON-UB233, Macroeconomic foundations for asset pricing},
    pdfauthor={Dave Backus @ NYU},
    pdfstartview={FitH},
    pdfpagemode={UseNone},
%    pdfnewwindow=true,      % links in new window
%    linkcolor=blue,         % color of internal links
%    citecolor=blue,         % color of links to bibliography
%    filecolor=blue,         % color of file links
%    urlcolor=blue           % color of external links
% see:  http://www.tug.org/applications/hyperref/manual.html
}

\renewcommand{\thefootnote}{\fnsymbol{footnote}}
\newcommand{\var}{\mbox{\it Var\/}}

\printanswers

% document starts here
\begin{document}
\parskip=\bigskipamount
\parindent=0.0in
\thispagestyle{empty}
{\large ECON-UB 233 \hfill Dave Backus @ NYU}

\bigskip\bigskip
\centerline{\Large \bf Quiz \#3}
\centerline{Revised: \today}

\bigskip
{\it Please write your name below.
Then complete the exam in the space provided.
There are THREE questions.
You may refer to one page of notes:
standard paper, both sides, any content you wish.}

\bigskip
\begin{flushleft}
\rule{4in}{0.5pt} \\ (Name and signature)
\end{flushleft}


\begin{questions}
%-----------------------------------------------------------------------
\item  {\it Moving average dynamics.\/}
Consider the stochastic process
\begin{eqnarray*}
    x_t &=& \delta + w_t + \theta w_{t-1}  ,
\end{eqnarray*}
where $\{ w_t \} $ is our usual collection of independent standard normal random variables.
%
\begin{parts}
\item Is $x$ Markov for some definition $z_t$ of the state at date $t$?
 What is the distribution of $x_{t+1}$ conditional on $z_t$?
 (5~points)
\item What is the equilibrium distribution of $x$?
 (5~points)
\item What are the autocovariance and autocorrelation functions?
 (10~points)
\item What is the maximum first autocorrelation $\rho(1)$?
For what value of $\theta$ does it occur?
 (10~points)
\end{parts}

\begin{solution}
\begin{parts}
\item The natural definition of the state is $z_t = w_t$.
Given $z_t$, $x_{t+1}$ is normal with mean $\delta _ \theta w_t$
and variance one.
That's the one-period conditional distribution.
\item The equilibrium distribution is normal with mean $\delta$
and variance $1 + \theta^2$.
\item The autocovariance function is
\begin{eqnarray*}
    \gamma(k) &=& \left\{
                \begin{array}{ll}
                1 + \theta^2    & k=0 \\
                \theta          & k=1 \\
                0               & k\geq 2 .
                \end{array}
                \right.
\end{eqnarray*}
That is:  an MA(1) has a one-period memory.
The autocorrelation function is $\rho(k) = \gamma(k)/\gamma(0)$.
\item The first autocorrelation is
$\rho(1) = \theta/(1+\theta^2$.
This takes its highest value of one-half when $\theta = 1$.
\end{parts}
\end{solution}

%\pagebreak \phantom{xx} \pagebreak
%-----------------------------------------------------------------------
\item  {\it Inflation and government deficits.\/}
Here's a variant of our forward-looking inflation model.
We have, as usual, the quantity theory plus a velocity equation:
\begin{eqnarray*}
    m_t + v_t &=& p_t + y_t \\
    v_t &=& \alpha \left( E_t p_{t+1} - p_t \right) .
\end{eqnarray*}
Then we set $y_t = 0$ (to keep things simple) and connect the money supply $m_t$
to the government deficit $d_t$:
\begin{eqnarray*}
        m_{t} &=& \delta d_t \\
        d_{t+1} &=& \varphi d_t + \sigma w_{t+1} .
\end{eqnarray*}
%
\begin{parts}
\item Express this model as a forward-looking difference equation in
which the price level $p_t$ is a function of its expected future value
and the deficit.
(15~points)
\item How is the price level connected to the current deficit?
(15~points)
\end{parts}

\begin{solution}
\begin{parts}
\item Substitutions give us
\begin{eqnarray*}
    p_t &=& [\alpha/(1+\alpha)] E_t p_{t+1} + [\delta/(1+\alpha)] d_t \\
        &=& \lambda E_t p_{t+1} + \delta^\prime d_t ,
\end{eqnarray*}
where the second line is compact notation for the first.
\item This has the usual forward-looking solution in which $p_t$
is connected to expected future deficits.
We could grind through this, but it's easier to use the method of undetermined
coefficients.
If we guess the solution has the form $ p_t = a d_t$ for some coefficient $a$ to be determined,
then
\begin{eqnarray*}
    p_{t+1} &=& a d_{t+1} \;\;=\;\; a (\varphi d_t + \sigma w_{t+1}) \\
   E_t (  p_{t+1})  &=&  a \varphi d_t .
\end{eqnarray*}
If we substitute into the difference equation above and collect coefficients
of $d_t$, we get
\begin{eqnarray*}
    a &=& \frac{\delta'}{1-\lambda \varphi}  \;\;=\;\; \frac{\delta}{1+\alpha(1-\varphi)} .
\end{eqnarray*}
In words:  the impact of the current deficit on the price is larger if
(i)~money is more strongly tied to deficits (large $\delta$),
(ii)~deficits are more persistent (large $\varphi$),
and (iii)~velocity is less sensitive to expected inflation (small $\alpha$).
[Last one sounds wrong, mistake somewhere??]
\end{parts}
\end{solution}

%\pagebreak \phantom{xx} \pagebreak
%-----------------------------------------------------------------------
\item  {\it Expected returns on bonds.\/}
Consider the bond pricing model
\begin{eqnarray*}
    \log m_{t+1} &=& - (\lambda_0 + \lambda_1 w_t)^2/2 - x_t
            + (\lambda_0 + \lambda_1 w_t) w_{t+1} \\
            x_{t} &=& \delta + \sigma (w_t + \theta w_{t-1}) ,
\end{eqnarray*}
where the $w_t$'s are  independent standard normal random variables.
%
\begin{parts}
\item Suppose bond prices take the form
\begin{eqnarray*}
    \log q^n_t &=& A_n + B_n w_t + C_n w_{t-1} .
\end{eqnarray*}
How is $(A_{n+1},B_{n+1},C_{n+1})$ connected to $(A_{n},B_{n},C_{n})$?
What are the values of $(A_{n},B_{n},C_{n})$ for $n=0,1,2$?
(20~points)
\item Express the expected log excess return on a two-period bond as a function
of the coefficients $(A_{n},B_{n},C_{n})$ for $n=1,2$.
Use your solution for the coefficients to describe how expected log excess returns vary with the
interest rate innovation $w_t$.
(20~points)
\end{parts}

\begin{solution}
\begin{parts}
\item We use the usual formula,
$ q^{n+1}_t = E_t (m_{t+1} q^n_{t+1})$.
Then with substitutions,
\begin{eqnarray*}
    \log (m_{t+1} q^n_{t+1}) &=& A_n - \delta + (C_n-\sigma) w_t - \sigma\theta w_{t-1}
            - (\lambda_0 + \lambda_1 w_t)^2/2 \\
            && + \; [B_n + (\lambda_0 + \lambda_1 w_t)] w_{t+1} .
\end{eqnarray*}
The first line on the rhs gives us the conditional mean, the second line gives us the variance.
Some intensive algebra gives us
\begin{eqnarray*}
    A_{n+1} &=& A_n - \delta + B_n^2/2 + B_n \lambda_0 \\
    B_{n+1} &=& C_n - \sigma + B_n \lambda_1 \\
    C_{n+1} &=& - \sigma \theta .
\end{eqnarray*}
That gives us
\begin{center}
\begin{tabular}{cccc}
$n$ & $A_n$ & $B_n$ & $C_n$ \\
\midrule
0 & 0 & 0 & 0 \\
1 & $-\delta$ & $-\sigma$ & $-\sigma \theta$ \\
2 & $- 2\delta -\lambda_0 \sigma + \sigma^2/2$ & $- \sigma (1+\theta+\lambda_1)$ & $-\sigma\theta$
\end{tabular}
\end{center}

\item The log excess return is
\begin{eqnarray*}
    \log r^2_{t+1} - \log r^1_{t+1} &=& \log q^1_{t+1} - \log q^2_t + \log q^1_t \\
            &=& (2 A_1 - A_2) + (C_1 + B_1 - B_2) w_t + (C_1 - C_2) w_{t-1} \\
            &&        + \; B_1 w_{t+1} .
\end{eqnarray*}
The expected excess return knocks out the last term.

If we substitute our solutions, we have
\begin{eqnarray*}
   E_t \left( \log r^2_{t+1} - \log r^1_{t+1}\right)
            &=& (\sigma \lambda_0 - \sigma^2/2)  + (\sigma\lambda_1) w_t  .
\end{eqnarray*}
So the key parameter is $\lambda_1$:  the sensitivity of the ``price of risk''
to $w_t$.
\end{parts}
\end{solution}

\end{questions}
%\pagebreak \phantom{xx}

\vfill \centerline{\it \copyright \ \number\year \
NYU Stern School of Business}
\end{document}



