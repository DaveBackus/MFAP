\documentclass[11pt]{exam}

\oddsidemargin=0.25truein \evensidemargin=0.25truein
\topmargin=-0.5truein \textwidth=6.0truein \textheight=8.75truein

%\RequirePackage{graphicx}
\usepackage{comment}
\usepackage{hyperref}
\usepackage{booktabs}
\urlstyle{rm}   % change fonts for url's (from Chad Jones)
\hypersetup{
    colorlinks=true,        % kills boxes
    allcolors=blue,
    pdfsubject={ECON-UB233, Macroeconomic foundations for asset pricing},
    pdfauthor={Dave Backus @ NYU},
    pdfstartview={FitH},
    pdfpagemode={UseNone},
%    pdfnewwindow=true,      % links in new window
%    linkcolor=blue,         % color of internal links
%    citecolor=blue,         % color of links to bibliography
%    filecolor=blue,         % color of file links
%    urlcolor=blue           % color of external links
% see:  http://www.tug.org/applications/hyperref/manual.html
}

\renewcommand{\thefootnote}{\fnsymbol{footnote}}
\newcommand{\var}{\mbox{\it Var\/}}

\printanswers

% document starts here
\begin{document}
\parskip=\bigskipamount
\parindent=0.0in
\thispagestyle{empty}
{\large ECON-UB 233 \hfill Dave Backus @ NYU}

\bigskip\bigskip
\centerline{\Large \bf Quiz \#1}
\centerline{Spring 2013}

\bigskip
{\it Please write your name below,
then complete the exam in the space provided.
There are FOUR questions.
You may refer to one page of notes:
standard paper, both sides, any content you wish.}

\bigskip
\begin{flushleft}
\rule{4in}{0.5pt} \\ (Name and signature)
\end{flushleft}


\begin{questions}
%-----------------------------------------------------------------------
\question {\it Moments, cumulants, and generating functions (20 points).\/}
Consider an arbitrary random variable $x$.
\begin{parts}
\part Define the moment generating function of $x$.
How is the cumulant generating function related to it?
(5~points)
\part How is the cumulant generating function of $y = \alpha + \beta x$
related to the cgf of $x$?
(5~points)
\part What is the second central moment $\mu_2$ of $x$?
How is it connected to the moment generating function?
(5~points)
\part What is the third cumulant $\kappa_3$ of $x$?
How is it connected to the cumulant generating function?
(5~points)
%\part Show that $\kappa_2 = \mu_2$.  (5~points)
\end{parts}

\begin{solution}
\begin{parts}
\part The mgf $h$ is defined by:  $h(s) = E(e^{sx})$.
The cgf is its log:  $k(s) = \log h(s)$.
\part The mgf of $y$ is $h_y(s) = E(e^{sy}) = E(e^{s(\alpha + \beta x}) = e^{s\alpha} h_x(\beta s)$.
The cgf is its log:  $k_y(s) = s \alpha + k_x (\beta s)$.
\part The second central moment is the variance.  If the mean is $\bar{x}$,
it's $\mbox{Var}(x) = E (x-\bar{x})^2 = E(x^2) - \bar{x}^2$.
$ E(x^2)$ is the second raw or noncentral moments, and $\bar{x}$ is the first.
They are the second and first derivatives, resp, of the mgf, evaluated at $s=0$.
\part The third cumulant is the third derivative of the cgf, also evaluated at $s=0$.
It's also the third central moment.
\end{parts}
\end{solution}

%\pagebreak \phantom{xx} \pagebreak
%-----------------------------------------------------------------------
\question {\it Risk and return (30 points).}
Consider an agent with utility
\begin{eqnarray*}
    U &=&  E [u(c)]    %\sum_z p(z) u[c(z)] ,
\end{eqnarray*}
where $u(c) = c^{1-\alpha}/(1-\alpha)$ for some $\alpha > 0$.
She invests one and consumes the gross return $r$.

\begin{parts}
\part What is her expected utility if she invests everything in a riskfree asset
whose (gross) return is 1.1?
(Her consumption is therefore 1.1 in every state.)
What is the certainty equivalent of this outcome?
(10~points)
\part What is her expected utility if she invests in an asset whose return is lognormal:
$ \log r \sim \mathcal{N}(\kappa_1,\kappa_2)$?
What is her certainty equivalent?
(10~points)
\part For what values of $\kappa_1$ and $\kappa_2$ is the risky asset preferred?
(10~points)
\end{parts}

\begin{solution}
\begin{parts}
\part The certainty equivalent of a sure thing $\bar{c}$ is $\bar{c}$.
More formally, if consumption is $\bar{c}$ in all states, then the certainty
equivalent $\mu$ solves
\begin{eqnarray*}
    U(\bar{c}, \bar{c}, \ldots, \bar{c}) &=& U(\mu, \mu, \ldots, \mu) ,
\end{eqnarray*}
so $\mu = \bar{c}$.
So the certainty equivalent here is 1.1.

\part We're using properties of lognormal random variables here.
We know $ E(r^{1-\alpha}) = \exp[ (1-\alpha)\kappa_1 + (1-\alpha)^2\kappa_2/2]$.
The certainty equivalent is
$ \mu = E(r^{1-\alpha})^{1/(1-\alpha)} = \exp[\kappa_1 + (1-\alpha)\kappa_2/2]$.

\part Evidently we need
$ \kappa_1 + (1-\alpha)\kappa_2/2 > \log 1.1 $.
So large $\kappa_1 $ helps.  If $\alpha > 1$, small $\kappa_2$ helps, too,
otherwise the reverse.
\end{parts}
\end{solution}

%\pagebreak \phantom{xx} \pagebreak
%-----------------------------------------------------------------------
\question {\it Securities and returns (30 points).\/}
Consider an economy with two assets and two equally likely states.
The assets have dividends

\medskip
\begin{center}
\begin{tabular}{lcc}
\toprule
Asset &  State 1 & State 2 \\
\midrule
1 (``bond'')    &  1    &   1  \\
2 (``equity'')  &  2    &   5  \\
\bottomrule
\end{tabular}
\end{center}

\medskip
The prices of the two assets are $q^1 = 0.7$ and $q^e = 2$.

\begin{parts}
\part What is the mean return on Asset 1?  Asset 2?
The risk premium on Asset 2?
(15~points)
\part How can you decompose each asset into Arrow securities?
What are the implied prices of Arrow securities?
(15~points)
%\part If risk aversion is $\alpha = 2$,
%what fraction of your wealth does Merton's formula tell you to invest in equity?
%(10~points)
\end{parts}

\begin{solution}
\begin{parts}
\part The bond has a sure return $ r^1 = 1/0.7 = 1.43 $.
Equity has returns
\begin{eqnarray*}
    r^e(z) &=&
        \left\{
        \begin{array}{ll}
        2/2 = 1 &  \mbox{for } z=1 \\
        5/2 = 2.5 &  \mbox{for } z=2
        \end{array}
        \right.
\end{eqnarray*}
The states have probability one-half each (``equally likely''), so its
expected return is 1.75.
Its risk premium is $1.75 - 1.43 = 0.32$.

\part The bond pays one in each state, so it consists of the two Arrow securities.
Equity pays 2 in state 1 and 5 in state 2, so it consists of 2 units of the state-1 Arrow securities
and 5 units of the state-2 Arrow security.
If the prices of Arrow securities are denoted by $Q(z)$,
then we have
\begin{eqnarray*}
    0.7 &=&  Q(1) + Q(2) \\
    2   &=& 2 Q(1) + 5 Q(2) .
\end{eqnarray*}
That gives us $Q(1) = 0.5$ and $Q(2) = 0.2$.

\end{parts}
\end{solution}


%\pagebreak \phantom{xx} \pagebreak
%-----------------------------------------------------------------------
\question {\it Saving and investment (20 points).}
Consider the Pareto problem of choosing $(c_0,k)$ to maximize
\begin{eqnarray*}
    U &=& u(c_0) + \beta \sum_z p(z) u[c_1(z)],
\end{eqnarray*}
subject to the resource constraints
\begin{eqnarray*}
    c_0 + k &\leq& y_0 \\
    c_1(z)  &\leq& z f(k) .
\end{eqnarray*}
There is one of the second constraint for each state $z$.
Here $k$ is capital --- plant and equipment --- produced at date 0 and used
to produce output $z f(k)$ at date 1.
The amount of output is random and depends on the state $z$.

\begin{parts}
\part What is the associated Lagrangian?
(10~points)
\part What are the first-order conditions for $c_0$ and $k$?
(10~points)
\end{parts}

\begin{solution}
\begin{parts}
\part If we use $q_0$ and $q_1(z)$ as multipliers on the constraints,
the Lagrangian is
\begin{eqnarray*}
    \mathcal{L} &=& u(c_0) + \beta \sum_z p(z) u[c_1(z)]
        + q_0 (y_0 - c_0 - k) + \sum_z q_1(z) [z f(k) - c_1(z)] .
\end{eqnarray*}
\part The first-order conditions are
\begin{eqnarray*}
    c_0: &&  u'(c_0) - q_0 \;\;=\;\; 0\\
    k:   &&  - q_0 + \sum_z q_1(z) z f'(k) .
\end{eqnarray*}
\end{parts}
\end{solution}

%\pagebreak \phantom{xx} %\pagebreak
\end{questions}

%\phantom{xx}
\vfill \centerline{\it \copyright \ \number\year \
NYU Stern School of Business}
\end{document}



