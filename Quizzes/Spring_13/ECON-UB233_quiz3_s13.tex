\documentclass[11pt]{exam}

\oddsidemargin=0.25truein \evensidemargin=0.25truein
\topmargin=-0.5truein \textwidth=6.0truein \textheight=8.75truein

%\RequirePackage{graphicx}
\usepackage{verbatim}
\usepackage{booktabs}
\usepackage{comment}
\usepackage{hyperref}
\urlstyle{rm}   % change fonts for url's (from Chad Jones)
\hypersetup{
    colorlinks=true,        % kills boxes
    allcolors=blue,
    pdfsubject={ECON-UB233, Macroeconomic foundations for asset pricing},
    pdfauthor={Dave Backus @ NYU},
    pdfstartview={FitH},
    pdfpagemode={UseNone},
%    pdfnewwindow=true,      % links in new window
%    linkcolor=blue,         % color of internal links
%    citecolor=blue,         % color of links to bibliography
%    filecolor=blue,         % color of file links
%    urlcolor=blue           % color of external links
% see:  http://www.tug.org/applications/hyperref/manual.html
}

\renewcommand{\thefootnote}{\fnsymbol{footnote}}
\newcommand{\var}{\mbox{\it Var\/}}

\printanswers

% document starts here
\begin{document}
\parskip=\bigskipamount
\parindent=0.0in
\thispagestyle{empty}
{\large ECON-UB 233 \hfill Dave Backus @ NYU}

\bigskip\bigskip
\centerline{\Large \bf Quiz \#3}
\centerline{Revised: \today}

\bigskip
{\it Please write your name below.
Then complete the exam in the space provided.
There are TWO questions.
You may refer to one page of notes:
standard paper, both sides, any content you wish.}

\bigskip
\begin{flushleft}
\rule{4in}{0.5pt} \\ (Name and signature)
\end{flushleft}


\begin{questions}
%-----------------------------------------------------------------------
\question {\it Linear models (60 points).\/}
Consider the linear time series model
\begin{eqnarray*}
    x_{t} &=& \varphi x_{t-1} +  w_t ,
\end{eqnarray*}
with $\{ w_t \}$ independent normal random variables with mean zero and variance one.
Now consider a second random variable $y_t$ built from $x_t$ and
the same disturbance $w_t$ by
\begin{eqnarray*}
    y_{t} &=& x_{t} + \theta w_t .
\end{eqnarray*}
The question is how this combination behaves.

\begin{parts}
\part Is there a state variable for which $x_t$ is Markov?
What is the distribution of $x_{t+1}$ conditional on the state at date $t$?
(10~points)
\part Express $x_t$ as a moving average.
What are its coefficients?
(5~points)
\part Is there a state variable for which $y_t$ is Markov?
What is the distribution of $y_{t+1}$ conditional on the state at date $t$?
(10~points)
\part Express $y_t$ as a moving average.
What are its coefficients?
(5~points)
\part Under what conditions is $y_t$ stable?
That is: under what conditions does the distribution of $y_{t+k}$,
conditional on the state at $t$, converge as $k$ gets large?
(10~points)
\part What is the equilibrium or stationary distribution of $y_t$?
(10~points)
\part What is the first autocorrelation of $y_t$?
(10~points)
\end{parts}

\begin{solution}
\begin{parts}
\part It's Markov with $x_t$.
The distribution of $x_{t+1}$:
\begin{eqnarray*}
    x_{t+1} &=& \varphi x_{t} + w_{t+1}
\end{eqnarray*}
is conditionally normal with mean $\varphi x_t $
and variance one.
\part The moving average representation is
\begin{eqnarray*}
    x_{t} &=& w_t + \varphi w_{t-1} + \varphi^2 w_{t-2}  + \cdots .
\end{eqnarray*}
The coefficients are $(1,\varphi, \varphi^2, \ldots )$.
\part Two answers, both work:  the state can be $x_t$ or (more commonly)
the vector $(y_t,w_t)$.
The distribution of $y_{t+1}$:
\begin{eqnarray*}
    y_{t+1} &=& \varphi x_t + (1+\theta) w_{t+1}
                \;\;=\;\; \varphi (y_t - \theta w_t) + (1+\theta) w_{t+1}
\end{eqnarray*}
is (conditionally) normal with mean $\varphi x_t = \varphi (y_t - \theta w_t)$
and variance $(1+\theta)^2)$.
\part If we add $\theta w_t$ to the expression for $x_t$ above, we get
\begin{eqnarray*}
    y_{t} &=& (1+\theta) w_t + \varphi w_{t-1} + \varphi^2 w_{t-2}  + \cdots .
\end{eqnarray*}
\part It's stable if $|\varphi| < 1$:  we need the moving average coefficients
 to approach zero.
 \part Stationary distribution:  $x_t$ is normal with mean zero
 and variance
 \begin{eqnarray*}
    \mbox{Var}(x_t) &=& (1+\theta)^2 + \varphi^2 + \varphi^4 + \cdots
            \;\;=\;\; (1+\theta)^2 + \varphi^2/(1-\varphi^2) .
 \end{eqnarray*}
\end{parts}
\end{solution}

%\pagebreak \phantom{xx} \pagebreak
%-----------------------------------------------------------------------
\question {\it Stochastic volatility and equity pricing (40 points).\/}
The Cox-Ingersoll-Ross model of bond pricing consists of the equations
\begin{eqnarray*}
    \log m_{t+1} &=& -(1+\lambda^2/2) x_{t} + \lambda x_t^{1/2} w_{t+1} \\
        x_{t+1}  &=& (1-\varphi)\delta + \varphi x_t + \sigma x_t^{1/2} w_{t+1} ,
\end{eqnarray*}
with the usual independent standard normal disturbances $w_t$.
We add to it an equation governing the dividend $d_t$ paid by a one-period
equity-like claim:
\begin{eqnarray*}
    \log d_{t+1} &=& \alpha + \beta x_t + \gamma x_t^{1/2} w_{t+1} .
\end{eqnarray*}
Here $x_t$ plays the role of the state:  if we know $x_t$, we know the conditional
distributions of $ (m_{t+1}, x_{t+1}, d_{t+1})$.

\begin{parts}
\part Conditional on $x_t$,
what are the mean and variance of $\log m_{t+1}$?
What is its distribution?
(10~points)
\part What is the price $q^1_t$ of a one-period bond?
What is its return $ r^1_{t+1} = 1/q^1_t$?
(10~points)
\part What is the price $q^e_t$ of equity, a claim to the dividend
$d_{t+1}$?
What is its return $r^e_{t+1}$?
(10~points)
\part Conditional on $x_t$, what is the expected log excess return on equity,
$E_t (\log r^e_{t+1} - \log r^1_{t+1})$?
%How does it vary with $x_t$?
(10~points)
\end{parts}

\begin{solution}
\begin{parts}
\part Conditional on $x_t$, $\log m_{t+1}$ is normal with mean and variance
\begin{eqnarray*}
    E_t (\log m_{t+1}) &=& -(1+\lambda^2/2) x_{t} \\
    \mbox{Var}_t (\log m_{t+1}) &=& \lambda^2 x_t .
\end{eqnarray*}

\part The price uses the ``mean plus variance over two'' formula:
\begin{eqnarray*}
    \log q^1_t  &=& - x_{t} .
\end{eqnarray*}
The log return is $\log r^1_{t+1} = x_t$.

\part The price of equity is
\begin{eqnarray*}
    \log q^e_t &=& \log E_t (m_{t+1} d_{t+1} ) \\
            &=& \alpha + [\beta - (1+\lambda^2/2)] x_t
                + [(\gamma+\lambda)^2/2 ]  x_t \\
            &=& \alpha + (\beta - 1 + \gamma^2/2 + \gamma\lambda) x_t .
\end{eqnarray*}
The return is
\begin{eqnarray*}
    \log r^e_{t+1} &=& \log d_{t+1} - \log q^e_t
            \;\;=\;\; (1 - \gamma^2/2 - \gamma\lambda) x_t
                + \gamma x_t^{1/2} w_{t+1} .
\end{eqnarray*}
\part Conditional on $x_t$, the expected excess return is
\begin{eqnarray*}
    E_t (\log r^e_{t+1} - \log r^1_{t+1} )
            &=& - (\gamma^2/2 + \gamma\lambda) x_t .
\end{eqnarray*}
\end{parts}
\end{solution}

\end{questions}
%\pagebreak \phantom{xx}

\vfill \centerline{\it \copyright \ \number\year \
NYU Stern School of Business}
\end{document}



