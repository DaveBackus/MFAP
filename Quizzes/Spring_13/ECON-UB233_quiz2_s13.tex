\documentclass[11pt]{exam}

\oddsidemargin=0.25truein \evensidemargin=0.25truein
\topmargin=-0.5truein \textwidth=6.0truein \textheight=8.75truein

%\RequirePackage{graphicx}
\usepackage{verbatim}
\usepackage{booktabs}
\usepackage{comment}
\usepackage{hyperref}
\urlstyle{rm}   % change fonts for url's (from Chad Jones)
\hypersetup{
    colorlinks=true,        % kills boxes
    allcolors=blue,
    pdfsubject={ECON-UB233, Macroeconomic foundations for asset pricing},
    pdfauthor={Dave Backus @ NYU},
    pdfstartview={FitH},
    pdfpagemode={UseNone},
%    pdfnewwindow=true,      % links in new window
%    linkcolor=blue,         % color of internal links
%    citecolor=blue,         % color of links to bibliography
%    filecolor=blue,         % color of file links
%    urlcolor=blue           % color of external links
% see:  http://www.tug.org/applications/hyperref/manual.html
}

\renewcommand{\thefootnote}{\fnsymbol{footnote}}
\newcommand{\var}{\mbox{\it Var\/}}

\printanswers

% document starts here
\begin{document}
\parskip=\bigskipamount
\parindent=0.0in
\thispagestyle{empty}
{\large ECON-UB 233 \hfill Dave Backus @ NYU}

\bigskip\bigskip
\centerline{\Large \bf Quiz \#2}
\centerline{(Spring 2013)}

\bigskip
{\it Please write your name below.
Then complete the exam in the space provided.
There are THREE questions.
You may refer to one page of notes:
standard paper, both sides, any content you wish.}

\bigskip
\begin{flushleft}
\rule{4in}{0.5pt} \\ (Name and signature)
\end{flushleft}

\begin{questions}
%-----------------------------------------------------------------------
\question {\it Approaches to asset pricing (40~points).\/}
Consider a two-period representative agent economy.
Consumption growth $g$ is Bernoulli:
\begin{eqnarray*}
    g &=&
        \left\{
        \begin{array}{lll}
            1.00    &  \mbox{with probability } 1-\omega & \mbox{(state 1)} \\
            1.10    &  \mbox{with probability } \omega   & \mbox{(state 2)}
        \end{array}
        \right.
\end{eqnarray*}
with $\omega = 1/2$.
The representative agent has power utility with discount factor $\beta = 0.98$
and risk aversion $\alpha = 5$.
Equity is a claim to $g$.

\begin{parts}
\part What are the mean and standard deviation of $\log g$?
(10~points)

\part What are the state prices?
Which state has greater value?  Why?
(10~points)

\part What are the price and return of a one-period riskfree bond?
(10~points)

\part What are the risk-neutral probabilities?
How do they compare to the true probabilities?
Why?
(10~points)

\end{parts}

\begin{solution}
\begin{parts}
\part With $\omega = 1/2$, the mean and standard deviation are both 0.0477.
The easiest way to get this is to write it in the form $\log g = \mu \pm \sigma$.
The Matlab code has a more direct calculation.

\part Here we need the pricing kernel:  $m(z) = \beta g(z)^{-\alpha} $.
We get $ m(1) = 0.9800$ and $m(2) = 0.6085 $.
State prices are therefore $Q(z) = p(z) m(z) $:
$ Q(1) = 0.4900$ and $Q(2) = 0.3043$.
Note that the pricing kernel is lower in state 2:
in the good state, marginal utility is lower ---
you'd pay less for an extra unit of the consumption good when consumption is high.
That carries over to state prices because the probabilities
are the same in the two states.
So the reason the price of state 2 is lower is that the good is less valuable in the good state.

\part The price of a one-period bond is
\begin{eqnarray*}
    q^1 &=& \sum p(z) m(z) \;\;=\;\; 0.7943,
\end{eqnarray*}
which implies $r^1 = 1/q^1 = 1.2590$.

\part Risk-neutral probabilities are defined by $ p^*(z) = p(z)m(z)/q^1 = Q(z)/q^1$.
Here we have $p^*(1) = 0.6169$ and $p^*(2) = 0.3831$.
Why?  Same answer as before:
the state price is lower in the good state.
\end{parts}
Matlab program:
\verbatiminput{../Matlab/quiz2_s13.m}
\end{solution}

%\pagebreak \phantom{xx} \pagebreak
%-----------------------------------------------------------------------
\question {\it The high-return asset (10~points).\/}
Consider a similar representative agent economy.
Log consumption growth $\log g$ is normal with mean $\kappa_1$
and variance $\kappa_2$.
The representative agent has power utility with $\beta =1$ and $\alpha=10$.
Equity is a claim to $g^\lambda$.

\begin{parts}
\part What is the price of a claim to equity?  What is the expected return?
(10~points)
\part Denote the expected log return on equity by $ E\log r^e$.
For what value of $\lambda$ is this return highest?
(10~points)
\part What is the entropy of the pricing kernel in this economy?
(10~points)
\end{parts}

\begin{solution}
\begin{parts}
\part This has several steps, all use the familiar expectation of lognormal
random variables.
The equity price solves
$q^e = E(mg) = E (\beta g^{-\alpha}g^{\lambda}) = E (\beta g^{\lambda-\alpha})$.
With $\beta = 1$, we have
\begin{eqnarray*}
    (\lambda-\alpha) \log g &\sim& \mathcal{N}[(\lambda-\alpha)\kappa_1,
        (\lambda-\alpha)^2 \kappa_2 ] ,
\end{eqnarray*}
so
\begin{eqnarray*}
    \log q^e &=& (\lambda-\alpha)\kappa_1 + (\lambda-\alpha)^2 \kappa_2/2 .
\end{eqnarray*}
The expected return in levels is
\begin{eqnarray*}
    E(r^e) &=& E (g^\lambda)/q^e \;\;=\;\;
        \exp[ \alpha \kappa_1 + \alpha (2 \lambda-\alpha) \kappa_2/2 .
\end{eqnarray*}
The expected return in logs is
\begin{eqnarray*}
    E(\log r^e) &=& E (\lambda \log g) - \log q^e
            \;\;=\;\; \alpha \kappa_1 - (\lambda-\alpha)^2 \kappa_2/2 .
\end{eqnarray*}
Either answer is ok.

\part See above.  You can see from the last term that the maximum value is
when $\lambda = \alpha = 10$.
You may recall that the ``high-return asset'' defined this way is $-\log m$,
which is what we have here.

\end{parts}
\end{solution}


%\pagebreak \phantom{xx} \pagebreak
%-----------------------------------------------------------------------
\question {\it Digital options (30 points).\/}
A digital or binary option either pays some fixed amount or not.
Consider, for example, a digital option on the underlying $s_{t+1}$.
A digital call with strike $k$ pays 100 if $s_{t+1} > k$.
A digital put with strike $k$ pays 100 if $s_{t+1} \leq k$.

Such options can be attacked using the same methods we've used
to price traditional options.
Suppose $\log s_{t+1}$ has a normal risk-neutral distribution
with mean $\kappa_1$ and variance $\kappa_2$.

\begin{parts}
\part What is the no-arbitrage condition for this situation?
(10~points)
\part What is the analog of put-call parity?
(10~points)
\part What is the price of a put with strike $k$?
(10~points)
\end{parts}

\begin{solution}
%\url{http://en.wikipedia.org/wiki/Binary_option}
\begin{parts}
\part This is the same as the one we used for BSM.
The idea in general is that we must choose a risk-neutral distribution
that's consistent with the current price of the asset.
In this case we have
\begin{eqnarray*}
    s_t &=& q^1_t E^* s_{t+1} \;\;=\;\; q^1_t e^{\kappa_1 + \kappa_2/2} ,
\end{eqnarray*}
which should look familiar.

\part The idea behind put-call parity is that if we know the price of a put,
we can use it to find the price of a call.  And vice versa.
Here the payoffs are constant, so the form is different.
But you see that a put gives us 100 if $s_{t+1} \leq k$,
and a call gives us 100 if $s_{t+1} > l$, so if we buy one of each we get
100 for sure.
Therefore we have
\begin{eqnarray*}
    q^p_t + q^c_t &=& q^1_t \cdot 100 .
\end{eqnarray*}

\part The payoff from a put option is
\begin{eqnarray*}
    d_{t+1} &=&
        \left\{
        \begin{array}{rl}
            100    &  \mbox{if } s_{t+1} \leq k  \\
            0      &  \mbox{otherwise} .
        \end{array}
        \right.
\end{eqnarray*}
The price is therefore
\begin{eqnarray*}
    q^p_t &=& q^1_t E^* (d_{t+1})  \\
            &=& q^1_t \mbox{Prob}(s_{t+1}\leq k) 100 \\
            &=& q^1_t \mbox{Prob}(\log s_{t+1}\leq \log k) 100 \\
            &=& q^1_t N [ (\log k-\kappa_1)/\kappa_2^{1/2}] 100 .
\end{eqnarray*}
The last line uses the lognormal (risk-neutral) distribution, but
the others hold in general.
In practice, we would use the no-arbitrage condition to eliminate $\kappa_1$,
as we did when we derived the BSM formula:
\begin{eqnarray*}
    (\log k - \kappa_1)/\kappa_2^{1/2} &=&
            [\log (q^1_t k/s_t) + \kappa_2/2]/\kappa_2^{1/2} .
\end{eqnarray*}
There's a similar formula in BSM.


\end{parts}
\end{solution}

%\pagebreak
\end{questions}

\phantom{xx}
\vfill \centerline{\it \copyright \ \number\year \
NYU Stern School of Business}
\end{document}



