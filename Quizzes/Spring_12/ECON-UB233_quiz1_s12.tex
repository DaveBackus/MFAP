\documentclass[11pt]{exam}

\oddsidemargin=0.25truein \evensidemargin=0.25truein
\topmargin=-0.5truein \textwidth=6.0truein \textheight=8.75truein

%\RequirePackage{graphicx}
\usepackage{comment}
\usepackage{hyperref}
\urlstyle{rm}   % change fonts for url's (from Chad Jones)
\hypersetup{
    colorlinks=true,        % kills boxes
    allcolors=blue,
    pdfsubject={ECON-UB233, Macroeconomic foundations for asset pricing},
    pdfauthor={Dave Backus @ NYU},
    pdfstartview={FitH},
    pdfpagemode={UseNone},
%    pdfnewwindow=true,      % links in new window
%    linkcolor=blue,         % color of internal links
%    citecolor=blue,         % color of links to bibliography
%    filecolor=blue,         % color of file links
%    urlcolor=blue           % color of external links
% see:  http://www.tug.org/applications/hyperref/manual.html
}

\renewcommand{\thefootnote}{\fnsymbol{footnote}}
\newcommand{\var}{\mbox{\it Var\/}}

%\printanswers

% document starts here
\begin{document}
\parskip=\bigskipamount
\parindent=0.0in
\thispagestyle{empty}
{\large ECON-UB 233 \hfill Dave Backus @ NYU}

\bigskip\bigskip
\centerline{\Large \bf Quiz \#1}
\centerline{Spring 2012}

\bigskip
{\it Please write your name below.
Then complete the exam in the space provided.
There are THREE questions.
You may refer to one page of notes:
standard paper, both sides, any content you wish.}

\bigskip
\begin{flushleft}
\rule{4in}{0.5pt} \\ (Name and signature)
\end{flushleft}


\begin{questions}
%-----------------------------------------------------------------------
\question (3-state ``Bernoulli'') (30 points)
Consider a 3-state distribution
in which we can explore the impact of high-order cumulants.
This is sometimes called a ``categorical distribution.''

Let us say, to be concrete, that the state $z$ takes on the values
 $\{-1, 0, 1\}$ with
 probabilities $\{\omega, 1-2\omega, \omega \}$.
 A random variable $x$ is defined by $x(z) = \delta z$.

\begin{parts}
\part Does $x$ have a legitimate probability distribution?
(5~points)
\part What is the moment generating of $x$?  The cumulant generating function?
(10~points)
\part What are the mean and variance of $x$?
(5~points)
\part What are the measures of skewness and excess kurtosis,
$\gamma_1$ and $\gamma_2$?
Under what conditions is $\gamma_2$ large?
(5~points)
\part Is there a value of $\omega$ that reproduces the values
of $\gamma_1$ and $\gamma_2$ of a normal random variable?
(5~points)
\end{parts}

\begin{solution}
\begin{parts}
\part Probabilities are positive (maybe greater than or equal to zero)
and sum to one.
So we're all set if $ 0 < \omega < 1$
(or weak inequalities if you prefer).

\part The mgf is
\begin{eqnarray*}
    h(s) &=& \omega (e^{-\delta s} + e^{\delta s}) + (1-2p)
\end{eqnarray*}
The cgf is $k(s) = \log h(s)$.
The mean and variance are
\begin{eqnarray*}
    \kappa_1  &=& 0 \\
    \kappa_2  &=& 2 \delta^2 p .
\end{eqnarray*}
One way to find them is from the derivatives of the cgf.

\part Skewness and excess kurtosis are
\begin{eqnarray*}
    \gamma_1  &=& 0 \\
    \gamma_2  &=& 1/(2 \omega) - 3 .
\end{eqnarray*}
The first is clear, because the distribution is symmetric.
The second one is a calculation based on cumulants.
Or you could compute the 4th central moment directly,
\begin{eqnarray*}
    \mu_4 &=&  \omega (-\delta)^4 + \omega \delta^4 \;\;=\;\;
        2 \omega \delta^4 ,
\end{eqnarray*}
which implies $\gamma_2 = \mu_4 /(\mu_2)^2 - 3  = 1/2\omega - 3 $.
Evidently $\gamma_2$ is large is $\omega    `$ is small.

\part The normal has $\gamma_1 = \gamma_2 = 0$.
We get the first automatically, and the second if $p = 1/6$.
\end{parts}
\end{solution}

%\pagebreak \phantom{xx} \pagebreak
%-----------------------------------------------------------------------
\question (3-state ``Bernoulli,'' continued)
(20~points)
For a representative agent economy based on the same state and random variable,
suppose $x$ is log consumption growth,
$x(z) = \log c_1(z) - \log c_0 $,
and utility has the additive form,
\begin{eqnarray*}
    u(c_0) + \beta \sum_z p(z) u[c_1(z)] ,
\end{eqnarray*}
with
$u(c) = c^{1-\alpha}/(1-\alpha)$.
\begin{parts}
\part What is the pricing kernel?
(5~points)
\part What are the state prices?
(5~points)
\part What are the risk-neutral probabilities?
How do they vary with risk aversion $\alpha$?
(5~points)
\part Which is more valuable, a claim to one unit of the good in state $z=-1$
or one unit in state $z=+1$?
Why?
(5~points)
\end{parts}

\begin{solution}
\begin{parts}
\part The pricing kernel is
\begin{eqnarray*}
    m[x(z)] &=& \beta e^{-\alpha x}
            \;\;=\;\;  \beta e^{-\alpha \delta z} .
\end{eqnarray*}
It's decreasing in $x$ and $z$.
\part The state prices are
\begin{eqnarray*}
    q(z) &=& p(z) m(z)
            \;\;=\;\;  p(z) \beta e^{-\alpha \delta z} ,
\end{eqnarray*}
with $p(z)$ given in the previous question.

\item Risk-neutral probabilities are
\begin{eqnarray*}
    p^*(z) &=& p(z) m(z) /q^1
            \;\;=\;\;
            \left\{
            \begin{array}{ll}
            \omega \beta e^{\alpha \delta}/q^1 & \mbox{ for } z=-1 \\
            (1-2\omega) \beta/q^1          & \mbox{ for } z=0    \\
            \omega \beta e^{-\alpha \delta}/q^1 & \mbox{ for } z=1 \\
            \end{array}
            \right.
\end{eqnarray*}
where
\begin{eqnarray*}
    q^1 &=& \sum_z p(z) m(z)
        \;\;=\;\; \beta \left[ \omega (e^{\alpha \delta} + e^{-\alpha \delta}) + (1-2 \omega) \right] .
\end{eqnarray*}
If risk aversion $\alpha$ is zero,
then $m(z) = \beta$ in all states.
Otherwise, higher values of $\alpha$ raise the value
of the bad state and lower it in the good state.


\part The only difference is the term $e^{\pm \alpha \delta}$.
If $\delta > 0$, then $e^{\alpha \delta} > e^{-\alpha \delta}$
and we have $p(-1) > p(1)$:
state $z=-1$ is more valuable than state $z=1$.
That's the usual result, that bad states are more valuable than good states.
If $\delta < 0$ the answer's the same, but
$z=1$ is the bad state.
\end{parts}
\end{solution}


%\pagebreak \phantom{xx} \pagebreak
%-----------------------------------------------------------------------
\question (short answers) (40 points)
Give short answers to the following questions:
\begin{parts}
%\part What is this function?
%\begin{eqnarray*}
%    p(x) &=& (2\pi \sigma^2)^{-1/2} \exp[-(x-\mu)^2/2 \sigma^2 ]
%\end{eqnarray*}
%What do $\mu$ and $\sigma$ represent?  (5~points)
\part Suppose $x \sim \mathcal{N}(\kappa_1,\kappa_2)$.
What are $\kappa_1$ and $\kappa_2$?
What is the distribution of $\log \beta - \alpha x$?
(5~points)
\part How would you compute skewness in a sample of data,
$x_t$ for $t=1,2,\ldots,T$?
(5~points)
\part What determines whether skewness raises or lowers expected utility?
Which applies to power utility?
(5~points)
\part What is a resource constraint?  How does it differ from a budget constraint?
(5~points)
\part What are the prices of Arrow securities for this collection of
asset prices and dividends?
\begin{eqnarray*}
    \mbox{Asset 1:}&&  q^1 = 1.0, \;\; d^1(1) = 2, \;\; d^1(2) = 1 \\
    \mbox{Asset 2:}&&  q^2 = 1.5, \;\; d^2(1) = 2, \;\; d^2(2) = 2.
\end{eqnarray*}
*Is this a ``no-arbitrage'' environment? How do you know?
(10~points)
\part *In our usual two-period setup, suppose the representative agent
has power utility and
$x(z) = \log c_1(z) - \log c_0 $ has an exponential distribution:
\begin{eqnarray*}
    p(x) &=& \lambda e^{-\lambda x} ,
\end{eqnarray*}
for $x \geq 0$ and $\lambda > 0$.
What is the risk-neutral distribution of $x$?
(10~points)
\end{parts}

\begin{solution}
\begin{parts}
\part Linear combinations of normals.
If $ x \sim \mathcal{N}(\kappa_1,\kappa_2)$,
then $\log \beta - \alpha x \sim
\mathcal{N}(\log \beta - \alpha \kappa_1, \alpha^2 \kappa_2)$.
We can show this using the mgf or cgf, but that's not part
of the question.
\part The $j$th central sample moment is
\begin{eqnarray*}
    \mu_j &=& T^{-1} \sum_t (x_t - \bar{x})^j ,
\end{eqnarray*}
where $\bar{x} = \sum_t x_t /T $ is the sample mean.
Sample skewness is
\begin{eqnarray*}
    \gamma_1 &=& \mu_3 / (\mu_2)^{3/2} .
\end{eqnarray*}

\part In the Samuelson expansion, the skewness term depends on
the third derivative of utility.
If the third is positive, raising skewness raises expected utility.
If negative, the reverse.
With power utility, the third derivative is positive.

\part Resource constraints concern quantities of goods:
for any particular good, you can't consume more than the endowment
plus (net) production.
Budget constraints concerns values of goods:
the value of consumption (the sum across goods of price times quantity)
can't be more than the value of income (a similar sum).

\part Each asset is a combination of Arrow securities, with weights
corresponding to dividends.
Doing it by hand, so to speak, you might notice that
a long position of one unit in asset 2 and
a short position of one unit in asset 1 has a payoff of
zero in state 1 and one in state 2.
In other words, it's the state-2 Arrow security.
Its cost is 0.5.
The rest of the price of asset 1 must come from the first state,
so the price of the state-1 Arrow security must be 0.25.
You can verify that these prices deliver the asset prices in the problem.

How do we know that the given asset prices and dividends are
``arbitrage free''?
The theorem goes both ways.
Since we've found positive state prices consistent
with observed asset prices and dividends,
the economy must be free of arbitrage opportunities.

\part There are a couple ways to do this.  One is to write
the risk-neutral probability as the product of
the true probability and the pricing kernel,
with $q^1$ serving to make sure they integrate to one.
Here we get
\begin{eqnarray*}
    p^*(x) &=& p(x) m(x)
        \;\;=\;\; \lambda e^{-\lambda x} \beta e^{-\alpha x}
                \;\;=\;\; \mbox{constant} \times e^{-(\lambda + \alpha) x} .
\end{eqnarray*}
The constant must be $\lambda + \alpha$ if the probabilities are
to integrate to one.

You could also use the cgf.
We didn't do this in class, but it's in the notes.
\end{parts}
\end{solution}

%\pagebreak \phantom{xx} \pagebreak
\end{questions}

%\phantom{xx}
\vfill \centerline{\it \copyright \ \number\year \
NYU Stern School of Business}
\end{document}



