\documentclass[11pt]{exam}

\oddsidemargin=0.25truein \evensidemargin=0.25truein
\topmargin=-0.5truein \textwidth=6.0truein \textheight=8.75truein

%\RequirePackage{graphicx}
\usepackage{verbatim}
\usepackage{booktabs}
\usepackage{comment}
\usepackage{hyperref}
\urlstyle{rm}   % change fonts for url's (from Chad Jones)
\hypersetup{
    colorlinks=true,        % kills boxes
    allcolors=blue,
    pdfsubject={ECON-UB233, Macroeconomic foundations for asset pricing},
    pdfauthor={Dave Backus @ NYU},
    pdfstartview={FitH},
    pdfpagemode={UseNone},
%    pdfnewwindow=true,      % links in new window
%    linkcolor=blue,         % color of internal links
%    citecolor=blue,         % color of links to bibliography
%    filecolor=blue,         % color of file links
%    urlcolor=blue           % color of external links
% see:  http://www.tug.org/applications/hyperref/manual.html
}

\renewcommand{\thefootnote}{\fnsymbol{footnote}}
\newcommand{\var}{\mbox{\it Var\/}}

\printanswers

% document starts here
\begin{document}
\parskip=\bigskipamount
\parindent=0.0in
\thispagestyle{empty}
{\large ECON-UB 233 \hfill Dave Backus @ NYU}

\bigskip\bigskip
\centerline{\Large \bf Quiz \#3}
\centerline{April 2012}

\bigskip
{\it Please write your name below.
Then complete the exam in the space provided.
There are TWO questions.
You may refer to one page of notes:
standard paper, both sides, any content you wish.}

\bigskip
\begin{flushleft}
\rule{4in}{0.5pt} \\ (Name and signature)
\end{flushleft}


\begin{questions}
%-----------------------------------------------------------------------
\question (moving averages) (50~points)
Consider the MA(2),
\begin{eqnarray*}
    x_t &=& \delta + w_t + \theta_1 w_{t-1} + \theta_2 w_{t-2},
\end{eqnarray*}
with $\{ w_t \} \sim \mbox{NID}(0,1)$
(the $w$'s are independent normals with mean zero and variance one).
Our mission is to explore its properties.

\begin{parts}
\item What is the mean of $x$?  The variance?
(10~points)
\item What are the conditional means,
$E_t (x_{t+1})$, $E_t (x_{t+2})$, and  $E_t (x_{t+3})$?
(10~points)
\item What are the conditional variances,
$\mbox{Var}_t (x_{t+1})$,
$\mbox{Var}_t (x_{t+2})$,
and $\mbox{Var}_t (x_{t+3})$?
(10~points)
\item What is the autocovariance function,
\begin{eqnarray*}
    \gamma(k)  &=& \mbox{Cov}(x_t,x_{t-k}),
\end{eqnarray*}
for $k=0,1,2,3$?
(10~points)
\item What is the autocorrelation function?
Under what conditions are $\rho(1)$ and $\rho(2)$ positive?
(10~points)
\end{parts}

\begin{solution}
\begin{parts}
\item The mean is $\delta$,
\begin{eqnarray*}
   E( x_t) &=& E (\delta + w_t + \theta_1 w_{t-1} + \theta_2 w_{t-2})
        \;\;=\;\; \delta .
\end{eqnarray*}
The variance is
\begin{eqnarray*}
   \mbox{Var} ( x_t) &=& E (x_t - \delta)^2
        \;\;=\;\;  1 + \theta_1^2 + \theta_2^2 .
\end{eqnarray*}
\item The conditional means are
\begin{eqnarray*}
   E_t ( x_{t+1}) &=& E_t (\delta + w_{t+1} + \theta_1 w_{t} + \theta_2 w_{t-1})
        \;\;=\;\; \delta + \theta_1 w_{t} + \theta_2 w_{t-1} \\
   E_t ( x_{t+2}) &=& E_t (\delta + w_{t+2} + \theta_1 w_{t+1} + \theta_2 w_{t})
        \;\;=\;\; \delta + \theta_2 w_{t} \\
   E_t ( x_{t+3}) &=& E_t (\delta + w_{t+3} + \theta_1 w_{t+2} + \theta_2 w_{t+1})
        \;\;=\;\; \delta .
\end{eqnarray*}
You can see that  as we increase the forecast horizon,
the conditional mean approaches the mean.

\item The conditional variances are
\begin{eqnarray*}
   \mbox{Var}_t ( x_{t+1}) &=& E_t [(w_{t+1})^2]
        \;\;=\;\; 1  \\
   \mbox{Var}_t ( x_{t+2}) &=& E_t [(w_{t+2} + \theta_1 w_{t+1})^2]
        \;\;=\;\; 1  + \theta_1^2 \\
   \mbox{Var}_t ( x_{t+3}) &=& E_t [(w_{t+3} + \theta_1 w_{t+2} + \theta_2 w_{t+1})^2]
        \;\;=\;\; 1  + \theta_1^2 + \theta_2^2 .
\end{eqnarray*}
You see here that as we increase the forecast horizon,
the conditional variance approaches the variance.


\item The autocovariance function is
\begin{eqnarray*}
    \mbox{Cov}(x_t,x_{t-k}) &=&
            \left\{
            \begin{array}{ll}
            1 + \theta_1^2 + \theta_2^2 & k=0 \\
            \theta_1 + \theta_1 \theta_2 & k=1 \\
            \theta_2                    & k=2 \\
            0                           & k \geq 3 .
            \end{array}
            \right.
\end{eqnarray*}
\item Autocorrelations are scaled autocovariances:
$\rho(k) = \gamma(k)/\gamma(0)$.
$\rho(2)$ is positive if $\theta_2$ is.
$\rho(1)$ is positive if $\theta_1 (1+\theta_2)$ is.
Both are therefore positive if $\theta_1$ and $\theta_2$ are
positive.
\end{parts}
\end{solution}


%\pagebreak \phantom{xx} \pagebreak
%-----------------------------------------------------------------------
\question (moving average bond pricing) (50~points)
Consider the bond pricing model
\begin{eqnarray*}
    \log m_{t+1} &=& - \lambda^2/2 - x_t + \lambda w_{t+1} \\
    x_t &=& \delta + \sigma (w_t + \theta w_{t-1}) .
\end{eqnarray*}

\begin{parts}
\item What is the short rate $f^0_t$?
(10~points)
\item Suppose bond prices take the form
\begin{eqnarray*}
    \log q^n_{t} &=& A_n + B_n w_t + C_n w_{t-1} .
\end{eqnarray*}
Use the pricing relation to derive recursions connecting
$(A_{n+1}, B_{n+1}, C_{n+1})$ to $(A_{n}, B_{n}, C_{n})$.
What are $(A_{n}, B_{n}, C_{n})$ for $n=0,1,2,3$?
(20~points)
\item Express forward rates as functions of the state $(w_t,w_{t-1}$).
What are $f^1_t$ and $f^2_t$?
(10~points)
\item What is $E(f^1- f^0)$?  What parameters govern its sign?
(10~points)
\end{parts}

\begin{solution}
\begin{parts}
\item The short rate is
\begin{eqnarray*}
    f^0_t &=& - \log E_t m_{t+1} \;\;=\;\;  \lambda^2/2 + x_t - \lambda^2/2
            \;\;=\;\; x_t .
\end{eqnarray*}
The second equality is the usual ``mean plus variance over two'' with the sign flipped
(as indicated by the first equality).
In other words:  the usual setup.
In what follows, we'll kill off $x_t$ by substituting.
\item
Bond prices follow from the pricing relation,
\begin{eqnarray*}
    q_t^{n+1} &=& E_t (m_{t+1} q_{t+1}^n) ,
\end{eqnarray*}
starting with $n=0$ and  $q^0_t = 1$.
The state in this case is $(w_t, w_{t-1})$, a simple example of a
two-dimensional model, hence the extra term in the form of the bond price.
We need
\begin{eqnarray*}
    \log (m_{t+1} q_{t+1}^n) &=&
            A_n - (\lambda^2/2 + \delta) + (\lambda+B_n) w_{t+1}
            + (C_n-\sigma) w_t - \sigma \theta w_{t-1} .
\end{eqnarray*}
The (conditional) mean and variance are
\begin{eqnarray*}
    E_t [ \log (m_{t+1} q_{t+1}^n)] &=&
            A_n - (\lambda^2/2 + \delta)
            + (C_n-\sigma) w_t - \sigma \theta w_{t-1} \\
    \mbox{Var}_t [\log (m_{t+1} q_{t+1}^n)] &=&
            (\lambda+B_n)^2 .
\end{eqnarray*}
Using ``mean plus variance over two'' and lining up terms gives us
\begin{eqnarray*}
        A_{n+1} &=& A_n - (\lambda^2/2 + \delta) + (\lambda+B_n)^2/2 \\
                &=& A_n - \delta + \lambda B_n + (B_n)^2/2 \\
        B_{n+1} &=& C_n - \sigma \\
        C_{n+1} &=& - \sigma \theta
\end{eqnarray*}
for $n=0,1,2,\ldots$.
That gives us
\begin{center}
\begin{tabular}{cccc}
        $n$  & $A_n$ & $B_n$ & $C_n$ \\
        \midrule
        0   &  0 & 0 & 0 \\
        1   &  $-\delta$ & $-\sigma$ & $-\sigma\theta$  \\
        2   &  $-2 \delta - \lambda\sigma + \sigma^2/2$   &  $-\sigma(1+\theta)$ & $-\sigma\theta$ \\
        3   &  X  &  $-\sigma(1+\theta)$ & $-\sigma\theta$
\end{tabular}
\end{center}
with
$X = - 3 \delta - \lambda (2+\theta)+ [1 + (1+\theta)^2] \sigma^2/2 $.

\item In general, forward rates are
\begin{eqnarray*}
    f^n_t &=& (A_n - A_{n+1}) + (B_n - B_{n+1}) w_t + (C_n - C_{n+1}) w_{t-1} .
\end{eqnarray*}
That gives us
\begin{eqnarray*}
    f^0_t &=& \delta + \sigma w_t + \sigma \theta  w_{t-1} \\
    f^1_t &=& \delta + \lambda\sigma - \sigma^2/2 + \sigma \theta w_t  \\
    f^2_t &=& \delta - (1+\theta)^2 \sigma^2/2 + \lambda \sigma (1+\theta) .
\end{eqnarray*}

\item The means are the same with $w_t = w_{t-1} = 0$, their mean.
Therefore
\begin{eqnarray*}
   E (f^1 - f^0) &=& \lambda \sigma - \sigma^2/2 .
\end{eqnarray*}
Therefore we need $\lambda\sigma > \sigma^2/2 $,
so a necessary condition is that $\lambda$ and $\sigma$ have the same sign.

\end{parts}
\end{solution}

\end{questions}


%\pagebreak \phantom{xx}

\vfill \centerline{\it \copyright \ \number\year \
NYU Stern School of Business}
\end{document}



