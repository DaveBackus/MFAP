\documentclass[11pt]{exam}

\oddsidemargin=0.25truein \evensidemargin=0.25truein
\topmargin=-0.5truein \textwidth=6.0truein \textheight=8.75truein

%\RequirePackage{graphicx}
\usepackage{verbatim}
\usepackage{booktabs}
\usepackage{comment}
\usepackage{hyperref}
\urlstyle{rm}   % change fonts for url's (from Chad Jones)
\hypersetup{
    colorlinks=true,        % kills boxes
    allcolors=blue,
    pdfsubject={ECON-UB233, Macroeconomic foundations for asset pricing},
    pdfauthor={Dave Backus @ NYU},
    pdfstartview={FitH},
    pdfpagemode={UseNone},
%    pdfnewwindow=true,      % links in new window
%    linkcolor=blue,         % color of internal links
%    citecolor=blue,         % color of links to bibliography
%    filecolor=blue,         % color of file links
%    urlcolor=blue           % color of external links
% see:  http://www.tug.org/applications/hyperref/manual.html
}

\renewcommand{\thefootnote}{\fnsymbol{footnote}}
\newcommand{\var}{\mbox{\it Var\/}}

\printanswers

% document starts here
\begin{document}
\parskip=\bigskipamount
\parindent=0.0in
\thispagestyle{empty}
{\large ECON-UB 233 \hfill Dave Backus @ NYU}

\bigskip\bigskip
\centerline{\Large \bf Quiz \#2}
\centerline{(Spring 2012)}

\bigskip
{\it Please write your name below.
Then complete the exam in the space provided.
There are THREE questions.
You may refer to one page of notes:
standard paper, both sides, any content you wish.}

\bigskip
\begin{flushleft}
\rule{4in}{0.5pt} \\ (Name and signature)
\end{flushleft}


\begin{questions}
%-----------------------------------------------------------------------
\question (Sharpe ratios) (40~points)
We'll look at Sharpe ratios in a two-period representative agent economy.
Endowment growth $g$ is Bernoulli:
\begin{eqnarray*}
    g &=&
        \left\{
        \begin{array}{ll}
            1.00    &  \mbox{with probability } 1-\omega \\
            1.10    &  \mbox{with probability } \omega
        \end{array}
        \right.
\end{eqnarray*}
with $\omega = 0.3$.
The representative agent has power utility with discount factor $\beta = 0.98$
and risk aversion $\alpha = 5$.
Equity is a claim to $g$.

\begin{parts}
\part What is the pricing kernel for this economy?
What are the state prices?
(10~points)
\part What are the price and return of a one-period riskfree bond?
(10~points)
\part What is the price of equity?
What are the mean and standard deviation of its excess return?
What is its Sharpe ratio?
(10~points)
\part What is the maximum Sharpe ratio for this economy?
(10~points)
\end{parts}

\begin{solution}
\begin{parts}
\part The pricing kernel is $m(z) = \beta g(z)^{-\alpha} $.
Here we get $ m = [0.9800, 0.6085] $.
State prices are $q(z) = p(z) m(z) $ or
$ q = [ 0.6860, 0.1826]$.
\part The price of the bond is
\begin{eqnarray*}
    q^1 &=& \sum p(z) m(z) \;\;=\;\; 0.8686,
\end{eqnarray*}
which implies $r^1 = 1/q^1 = 1.1513$.
\part The price of equity is
\begin{eqnarray*}
    q^1 &=& \sum p(z) m(z) g(z) \;\;=\;\; 0.8868 .
\end{eqnarray*}
The returns are $r^e = [1.1276, 1.2404]$.
The mean and standard deviation follow either from
a brute-force calculation [$ \var(x) = E(x^2) = E(x)^2$]
or related formulas for Bernoulli random variables.
The mean and standard deviation of the excess return are
0.0102 and 0.0517.
The Sharpe ratio is the ratio of the two:  $0.0102/0.0517 = 0.1960$.
\part This is an application of the Hansen-Jagannathan bound.
The maximum Sharpe ratio for this pricing kernel is the ratio
of the standard deviation of the pricing kernel to its mean.
Here we get $0.1702/0.8686 = 0.1960$.
Our asset therefore hits the bound.
That's something of an accident, it works because of the two-state structure,
which means all returns are linear functions of the pricing kernel.
Don't worry if that seems obscure to you.
\end{parts}
Matlab program:
\verbatiminput{../Matlab/quiz2_s12.m}
\end{solution}


%\pagebreak \phantom{xx} \pagebreak
%-----------------------------------------------------------------------
\question (entropy bound revisited) (30~points)
The idea here is to derive the entropy bound from a maximization problem.
We'll do this in an arbitrary two-period economy with a finite set of states.
Each state $z$ has probability $p(z)$
and pricing kernel $m(z)$.
An asset has returns $r(z)$ that satisfy the pricing relation
\begin{eqnarray}
    \sum_z p(z) m(z) r(z) &=& 1.
    \label{eq:pricing-relation}
\end{eqnarray}
Our mission is to characterize the asset with the highest expected log return,
\begin{eqnarray*}
    \sum_z p(z) \log r(z) .
\end{eqnarray*}
We'll refer to this as the ``high-return asset.''

\begin{parts}
\part What is the entropy of the pricing kernel?
Express it in terms of $m(z)$ and $p(z)$.
(10~points)

\part Use Lagrangian methods to find the returns $r(z)$ (one number for each state)
that maximize the expected log return while satisfying the pricing relation (\ref{eq:pricing-relation}).
How is the return on the high-return asset related to the pricing kernel?
(10~points)

\part Show that the high-return asset attains the entropy bound.
(10~points)
\end{parts}

\begin{solution}
\begin{parts}
\part Entropy is defined by
\begin{eqnarray*}
    H(m) &=& \log E(m) - E \log m
            \;\;=\;\; \log \sum_z p(z) m(z) - \sum_z p(z) \log m(z) .
\end{eqnarray*}

\part The idea is to maximize the expected log return with the pricing
relation as a constraint.
The Lagrangian is
\begin{eqnarray*}
    \mathcal{L} &=&     \sum_z p(z) \log r(z)
            + \lambda \left( 1 - \sum_z p(z) m(z) r(z) \right) .
\end{eqnarray*}
The first-order condition for $r(z)$ is
\begin{eqnarray*}
     p(z) / r(z) &=&  \lambda  p(z) m(z) .
\end{eqnarray*}
You can see here that there's an inverse relation between $r(z)$ and $m(z)$,
but we need to eliminate the multiplier $\lambda$.
If we multiply both sides by $r(z)$ and sum over $z$,
we see that the left side is one and the right side is $\lambda$,
so we must have $\lambda = 1$.
That gives us the maximizing return
\begin{eqnarray*}
      r(z) &=&  1/ m(z) .
\end{eqnarray*}
We did this in class using Jensen's inequality,
but this is more constructive.

\part The entropy bound says
\begin{eqnarray*}
    E (\log r - \log r^1) &\leq& H(m) \;\;=\;\; \log E(m) - E \log m .
\end{eqnarray*}
All we need to do is substitute.
We have $r = 1/m$, so
$ E \log r = - E \log m$.
The one-period rate is $ r^1 = 1/ E(m)$, so
$ \log r^1 = - \log E(m)$.
That gives us
\begin{eqnarray*}
    E (\log r - \log r^1) &=& - E \log m + \log E(m) \;\;=\;\; H(m) .
\end{eqnarray*}
(The $E$ in front of $r^1$ is irrelevant here, because $r^1$ is a constant.)


\end{parts}
\end{solution}


%\pagebreak \phantom{xx} \pagebreak
%-----------------------------------------------------------------------
\question (put and call options)
(30~points)
We'll look, once again, at the prices of one-year options when
the risk-neutral distribution of the underlying is lognormal.
If the future value of the underlying is $s_{t+1}$,
then $\log s_{t+1} \sim \mathcal{N}(\kappa_1,\kappa_2)$.
We've seen, in this case, that the price of a European put option
with strike price $k$ is
\begin{eqnarray*}
    q^p_t &=& q^1_t k N(d) - q^1_t e^{\kappa_1 + \kappa_2/2}
            N(d-\kappa_2^{1/2}) \\
            d&=& (\log k - \kappa_1)/\kappa_2^{1/2} .
\end{eqnarray*}

\begin{parts}
\part What is the price of a call option with the same strike price?
(15~points)
\part What is the no-arbitrage condition for this environment?
Use it to derive the BSM formula for call options from your answer to (a).
(15~points)
\end{parts}

\begin{solution}
Axelle says there's a sign error here somewhere,
but I haven't had the time to fix it.
\begin{parts}
\part The key input here is put-call parity, solved for the call price:
\begin{eqnarray*}
    {q^c_t}
    &=&
    {s_t} -
    {q^1_t k} +
    {q^p_t}
\end{eqnarray*}
That gives us a call price of
\begin{eqnarray*}
    q^c_t &=& q^1_t k [1-N(d)] + s_t - q^1_t e^{\kappa_1 + \kappa_2/2}
            N(d-\kappa_2^{1/2}) \\
          &=& q^1_t k N(-d) + s_t - q^1_t e^{\kappa_1 + \kappa_2/2}
            N(d-\kappa_2^{1/2}) .
\end{eqnarray*}
The second line follows from the symmetry of the normal distribution:
$1-N(d) = N(-d)$.

\part The no-arb condition here is $ s_t = q^1_t e^{\kappa_1 + \kappa_2/2} $ or
\begin{eqnarray*}
    \kappa_1 &=& \log (s_t/q^1_t) - \kappa_2/2 .
\end{eqnarray*}
That allows us to simplify the call price to
\begin{eqnarray*}
    q^c_t &=& q^1_t k N(-d) + s_t N(-d+\kappa_2^{1/2}) ,
\end{eqnarray*}
with
\begin{eqnarray*}
    d &=& \frac{\log (q^1_t k/s_t) - \kappa_2/2 }{\kappa_2^{1/2}} .
\end{eqnarray*}
If we change the sign, we end up with the BSM formula.
\end{parts}
\end{solution}

\pagebreak
\end{questions}

\phantom{xx}
\vfill \centerline{\it \copyright \ \number\year \
NYU Stern School of Business}
\end{document}



