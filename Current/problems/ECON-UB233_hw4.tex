\documentclass[11pt]{exam}

\oddsidemargin=0.25truein \evensidemargin=0.25truein
\topmargin=-0.5truein \textwidth=6.0truein \textheight=8.75truein

%\RequirePackage{graphicx}
\usepackage{comment}
\usepackage{verbatim}
\usepackage{booktabs}
\usepackage{hyperref}
\urlstyle{rm}   % change fonts for url's (from Chad Jones)
\hypersetup{
    colorlinks=true,        % kills boxes
    allcolors=blue,
    pdfsubject={ECON-UB233, Macroeconomic foundations for asset pricing},
    pdfauthor={Dave Backus @ NYU},
    pdfstartview={FitH},
    pdfpagemode={UseNone},
%    pdfnewwindow=true,      % links in new window
%    linkcolor=blue,         % color of internal links
%    citecolor=blue,         % color of links to bibliography
%    filecolor=blue,         % color of file links
%    urlcolor=blue           % color of external links
% see:  http://www.tug.org/applications/hyperref/manual.html
}

\usepackage{attachfile}
    \attachfilesetup{color=0.5 0 0.5}

\renewcommand{\thefootnote}{\fnsymbol{footnote}}
\newcommand{\var}{\mbox{\it Var\/}}

\printanswers

% document starts here
\begin{document}
\parskip=\bigskipamount
\parindent=0.0in
\thispagestyle{empty}
{\large ECON-UB 233 \hfill Dave Backus @ NYU}

\bigskip\bigskip
\centerline{\Large \bf Lab Report \#4: Asset Pricing Fundamentals}
\centerline{Revised: \today}

\bigskip
{\it Due at the start of class.
You may speak to others, but whatever you hand in should be your own work.
Please include your Matlab code.}

\begin{solution}
Brief answers follow,
but see also the attached Matlab program and spreadsheet;
download the pdf, open, click on pushpins:
\attachfile{../Matlab/hw/hw4_f13.m} 
\attachfile{../Matlab/hw/FamaFrench_returns.xlsx} \\
{\it Warning:  If you don't see a pushpin above, my guess is you have a Mac.
The pushpin doesn't appear in Preview,
but you can use the Adobe Reader or the equivalent.}
\end{solution}


\begin{questions}
%-----------------------------------------------------------------------
\question {\it Pricing kernels and risk-neutral probabilities with geometric risk.\/}
Consider a representative agent economy with a power utility agent facing ``geometric risk.''
Utility is $u(c) = c^{1-\alpha}/(1-\alpha) $ with risk aversion
parameter $\alpha > 0$.
Log consumption growth is $ x = \log g = \log c_1 - \log c_0 $
for $ x = 0,1,2, \ldots$ with probabilities
$ p(x) = (1-\omega) \omega^x$ for some parameter $\omega$
satisfying $0< \omega e < 1$.
We say that $x$ has a {\it geometric distribution\/}.
%
\begin{parts}
\item What is the pricing kernel for this model?
\item What are the state prices?
\item What are the risk-neutral probabilities $p^*(x)$?
How do they differ from the true probabilities $p(x)$?
\end{parts}

\begin{solution}
\begin{parts}
\item The pricing kernel is
\begin{eqnarray*}
    m(x) &=& \beta e^{-\alpha x} .
\end{eqnarray*}
\item State prices are
\begin{eqnarray*}
    Q(x) &=& p(x) m(x) \;\;=\;\; (1-\omega) \omega^x \beta e^{-\alpha x} .
\end{eqnarray*}
\item Risk-neutral probabilities are $ p^*(x) = p(x) m(x) / q^1 $.
Here $q^1 = E(m) = (1-\omega) \beta /(1-\omega e^{-\alpha}) $.
Risk-neutral probabilities are therefore
\begin{eqnarray*}
    p^*(x) &=&  \omega^x  e^{-\alpha x}(1-\omega e^{-\alpha}) .
\end{eqnarray*}
If $\alpha = 0$ (zero risk aversion), then $p^*(x) = p(x)$.
Otherwise $p^*$ is lower than $p$ for large $x$,
and the reverse for small $x$.
As usual, ``good states'' have lower value, which here is reflected
in the risk-neutral probability.
In (a) it was reflected in the pricing kernel.
\end{parts}
\end{solution}



%-----------------------------------------------------------------------
\question {\it Risk and return in US equity portfolios.\/}
Modern economies issue a wide range of assets,
whose returns can have wildly different properties.
Here we summarize the properties of
returns on some common equity portfolios.

We'll start with data input.  Ken French is Gene Fama's frequent coauthor
and his website is the go-to place for data on equity returns.
Here's the link:

{\small
\url{http://mba.tuck.dartmouth.edu/pages/faculty/ken.french/data_library.html}}.

Download the files associated with, respectively, the ``Fama-French Factors''
and ``Portfolios Formed on Size'' at the links

{\small
\url{http://mba.tuck.dartmouth.edu/pages/faculty/ken.french/ftp/F-F_Research_Data_Factors.zip} \\
\url{http://mba.tuck.dartmouth.edu/pages/faculty/ken.french/ftp/Portfolios_Formed_on_ME.zip}.
}

In both cases, you'll find a txt file inside a zip file.
Copy the first table in each txt file into a spreadsheet with the dates aligned.
In the first file (``factors'') you want the first column (the date),
the second (the excess return on the market),
and the fifth (the short-term or riskfree interest rate RF).
In the second file (``size'') you want the first column (the date again)
and columns three to five (returns on portfolios of small, medium, and large firms).

When you're done, read the data into Matlab.
At this point you should have the riskfree rate and excess returns
on the market and three size portfolios (small, medium, and large) ---
five variables all together.
The numbers are monthly, July 1926 to (last I looked) August 2013.
I believe they are percentages, but you should check.
%
\begin{parts}
\item
Compute the mean, standard deviation, skewness, and excess kurtosis
for each excess return series.
Which portfolio has the highest excess return?  Lowest?

\item The Sharpe ratio for any asset or portfolio is
the mean of its excess return over its standard deviation.
Which portfolio has the highest Sharpe ratio?  Lowest?

\item Do the same for log excess returns;
that is, compute the mean, standard deviation, skewness, and excess kurtosis
for each excess return series.
(To get this right, you should
add the riskfree rate to each excess return,
divide by 100,
add one, then take logs.
Think about why all this is called for.)

\item Comment on the nature of risk faced by each of these portfolios.
If you were considering an investment of your own money,
which properties of returns would you emphasize?
\end{parts}

\begin{solution}
The idea here is to look at some popular equity portfolios
and see how their returns differ from broad-based equity indexes.
We see (i)~mean excess returns in some cases are larger than the equity premium
and (ii)~skewness and kurtosis are a standard feature.
Skewness differs between returns and log returns, as you might expect:
the log is a concave function, so it moderates large positive returns in levels.
Finally, we remind ourselves that $ E(mr) = 1$ does not imply
any particular relation between mean and standard deviation of excess returns.
In that respect, the model is similar to the CAPM, where the mean is connected to
covariance with the market, not the standard deviation or Sharpe ratio
or other simple risk measure.

\begin{parts}
\item Last year's numbers below (sorry, I didn't have time to redo this).

\medskip
\begin{center}
\begin{tabular}{lcccc}
\multicolumn{4}{l}{Properties of monthly excess returns} \\
\toprule
            & \multicolumn{4}{c}{Portfolio} \\
                         \cmidrule{2-5}
Statistic   & market & small  & medium & big \\
\midrule
Mean &  0.6172 &  0.9738  &  0.8532  & 0.5987 \\
Std dev & 5.4571  &  8.5584  &  6.8520 &  5.2757 \\
Skewness &  0.1685 &   2.1813 &  0.9816  &  0.1860 \\
Excess kurtosis &  7.3997 &  21.8628 &  11.7830 &  7.1738 \\
Sharpe ratio  &  0.1131  &  0.1138  &  0.1245 &   0.1135 \\
\bottomrule
\end{tabular}
\end{center}

\medskip
The highest mean excess return is small firms at about 1\% a month,
the lowest is big firms.

\item Sharpe ratios also reported above:
the ratio of the first row to the second.
There's not much difference here among the portfolios,
but medium-size firms have the highest Sharpe ratio.

\item Similar table for log excess returns below:

\medskip
\begin{center}
\begin{tabular}{lrrrr}
\multicolumn{4}{l}{Properties of monthly log excess returns} \\
\toprule
            & \multicolumn{4}{c}{Portfolio} \\
             \cmidrule{2-5}
Statistic   & market & small  & medium & big \\
\midrule
Mean            & 0.4669 & 0.6329 & 0.6224 & 0.4583 \\
Std dev         & 5.4556 & 8.1132 & 6.7205 & 5.2682 \\
Skewness        & $-0.5298$ & 0.4937 & $-0.0850$ & $-0.4779$  \\
Excess kurtosis & 6.5282 & 9.6576 & 7.5655 & 6.3824\\
\bottomrule
\end{tabular}
\end{center}

\item We care about all of these properties.
One worth noting:  the excess kurtosis in all of them.

\end{parts}
\end{solution}


%-----------------------------------------------------------------------
\question {\it Excess kurtosis and the equity premium.\/}
Consider a 3-state distribution in which the state $z$ takes on the values
$\{-1, 0, 1\}$ with
probabilities $\{\omega, 1-2\omega, \omega \}$.
A random variable $x$ is defined by $x(z) = \mu + \delta z$.
Statisticians would say that $x$ has a ``categorical distribution.''

We'll build a representative-agent economy on this distribution
by setting log consumption growth equal to $x$;
that is, $ \log g(z) = \log [c_1(z)/c_0] = x(z)$.
Utility has the usual additive form,
\begin{eqnarray*}
    u(c_0) + \beta \sum_z p(z) u[c_1(z)] ,
\end{eqnarray*}
with $u(c) = c^{1-\alpha}/(1-\alpha)$ (power utility).


We're going to vary $\omega$ and see how that affects the equity premium.
The idea is to explore the role of excess kurtosis, which you'll see is controlled by $\omega$.
In all cases, set $\alpha = 10$ and $\beta = 0.99$.


\begin{parts}
\item What are the mean and variance of $x$?
\item What are the traditional measures of skewness and excess kurtosis,
$\gamma_1$ and $\gamma_2$?
Under what conditions is $\gamma_2$ large?
\item What is the pricing kernel?
\item Which is more valuable, a claim to one unit of the good in state $z=-1$
or one unit in state $z=+1$? Why?

\item We observe that annual log consumption growth has a mean in US data of roughly 0.02 (2\%)
and a standard deviation of 0.035.
Given a value of $\omega$, what values of $\mu$ and $\delta$ reproduce
these values?

\item Suppose $\omega = 1/6$.  Why is this value a useful benchmark?
What values of $(\mu,\delta)$ reproduce the mean and variance of log consumption growth?

\item Suppose equity is a claim to the growth rate of consumption $e^x$.
What is the equity premium with these values?

\item Suppose $\omega = 1/20$.
How do your choices of $(\mu,\delta)$ adjust to keep the mean and variance
of log consumption growth at their observed values?
How does the equity premium change?
Why?
\end{parts}

\begin{solution}
The idea here is to go through a concrete example
and compute state prices, the pricing kernel,
and risk-neutral probabilities, and think about what each
one does.
\begin{parts}
\item The easiest way to do this is with the cgf.
The mgf is
\begin{eqnarray*}
    h(s) &=& \omega e^{\mu - \delta} + (1-2\omega) e^\mu + \omega e^{\mu+\delta}
\end{eqnarray*}
and the cgf is $k(s) = \log h(s)$.
The first four cumulants are
\begin{eqnarray*}
    \kappa_1 &=& \mu \\
    \kappa_2 &=& 2\delta^2 \omega \\
    \kappa_3 &=& 0 \\
    \kappa_4 &=& 2 \delta^4 \omega (1-6 \omega).
\end{eqnarray*}
The first one is the mean, the second one is the variance.

\item Skewness and excess kurtosis are
\begin{eqnarray*}
    \gamma_1  &=& \kappa_3/\kappa_2^{3/2} \;\;=\;\; 0 \\
    \gamma_2  &=& \kappa_4/\kappa_2^{2} \;\;=\;\; 1/(2 \omega) - 3 .
\end{eqnarray*}
The first is clear, because the distribution is symmetric.
Evidently $\gamma_2$ is large when $\omega$ is small.
It's zero when $\omega = 1/6$
\item The pricing kernel is
\begin{eqnarray*}
    m(z) &=& \beta e^{-\alpha x(z)}
            \;\;=\;\;  \beta e^{-\alpha (\mu + \delta z)} .
\end{eqnarray*}

\item The pricing kernel is decreasing in $x$ and $z$.
In other words, $m$ is higher in states where $z$ is lower (``bad states'').
Specifically
\begin{eqnarray*}
    m(-1) &=& \beta e^{-\alpha (\mu - \delta)}
        \;\;>\;\;     m(1) \;\;=\;\; \beta e^{-\alpha (\mu + \delta)}
\end{eqnarray*}
as long as $\delta > 0$.

\item The state prices are
\begin{eqnarray*}
    Q(z) &=& p(z) m(z)
            \;\;=\;\;  p(z) \beta e^{-\alpha (\mu + \delta z)} ,
\end{eqnarray*}
with $p(z)$ given in the previous question.


\item The idea here is to choose parameters to reproduce the mean and variance
of log consumption growth in US data.
The mean is $\mu$, so we set $\mu = 0.02$.
The variance is $2\delta^2 \omega$.
Given a value of $\omega > 0$,
we set $2\delta^2 \omega = 0.035^2$ or
$\delta = 0.035/(2\omega)^{1/2} $.

\item With $\omega = 1/6$, excess kurtosis is zero,
the same value as the normal distribution.
To match the variance, we then need $\delta = 0.0606$.
$\mu$ remains 0.02.

\item
The equity premium is
$ E(r^e) - r^1 = 0.0143$ or 1.43\%.
See the Matlab code for the calculation.

\item If we reduce $\omega$ to 1/20, excess kurtosis goes up to 7.
The equity premium rises to 0.0160.
The impact is modest, but it illustrates the potential role
of adding more kurtosis to the distribution.


\end{parts}
\end{solution}

\end{questions}
%\end{document}

\vfill \centerline{\it \copyright \ \number\year \ NYU Stern School of Business}

\end{document}


%  EXTRA STUFF


%-----------------------------------------------------------------------
\question {\it Disaster risk and the equity premium.\/}
We'll add a third ``disaster'' state to our analysis of the equity premium
and see how it changes our view of it.
The key input is the distribution of log consumption growth,
\begin{eqnarray*}
    \log g &=& \left\{
                \begin{array}{ll}
                \mu + \sigma & \mbox{with probability } (1-\omega)/2 \\
                \mu - \sigma & \mbox{with probability } (1-\omega)/2 \\
                \mu - \delta & \mbox{with probability } \omega
                \end{array}
                \right.
\end{eqnarray*}
What's the idea?
If $\omega = 0$, we're back to our symmetric two-state distribution.
But if we introduce a small positive value of $\omega$ and a ``largish'' $\delta>0$,
we have a ``disaster'' state that changes the distribution dramatically.

The question is what this does to the equity premium.
We define the equity premium in logs,
\begin{eqnarray*}
    E ( \log r^e - \log r^1 ) ,
\end{eqnarray*}
and aim at a target value of 0.0400 (4\%).
As before, we'll define equity in this model as a claim to consumption growth $g$.

The structure is roughly similar to that in the Matlab program we used in class,
but I encourage you to write your own.
(Why?   I usually find it harder to adapt someone else's code than write my own.
You're welcome to do either, but don't say I didn't warn you.)

\begin{parts}
\item If $\omega = 0$, what values of $\mu$ and $\sigma$ deliver
the observed mean and variance of log consumption growth, namely
0.0200 and $0.0350^2$?
\item Suppose $\beta = 0.99$ and $\alpha = 5$.
What are $r^1$, $\log r^1$, and
the equity premium,  $E \log r^e - \log r^1 $?

\item What is the equity premium if $\alpha$
equals 10?  20?

\item Now consider $\omega = 0.01$ and $\delta = 0.30$.
(These numbers are based on a series of studies by Robert Barro and his coauthors.)
With these numbers, what values of $\mu$ and $\sigma$ reproduce
the observed mean and variance of log consumption growth?

\item With (again) $\beta = 0.99$ and $\alpha = 5$,
what are $r^1$, $\log r^1$, $E \log r^e$, and
the equity premium,  $E \log r^e - \log r^1 $?

\item What is the equity premium if $\alpha$
equals 10?  20?
How does it compare to your previous calculations?

\item (extra credit)
How does the equity premium change if $\delta = - 0.30$,
so that the extreme state is good news rather than bad?
Why?

\item (extra credit) How does entropy differ between the
disaster and no-disaster cases?
\end{parts}

\begin{solution}
\begin{parts}
\item The expressions for the mean and variance of $\log g$ are
\begin{eqnarray*}
    E (\log g)   &=&  \mu + \omega \delta \;\;=\;\; 0.0200 \\
    \mbox{Var}(\log g) &=& (1-\omega) \sigma^2 + \omega (1-\omega) \delta^2
                \;\;=\;\; 0.0350^2 .
\end{eqnarray*}
When $\omega = 0$, the mean is $\mu = 0.0200$ and the standard
deviation is $\sigma = 0.0350$.

\item With these values, we have
$r^1 = 1.0995$, $\log r^1 = 0.0948$, and
$E \log r^e - \log r^1 = 0.0055 $.

\item The equity premium is 0.0112 when $\alpha = 10$ and 0.0208 when $\alpha = 20$.
Both are well below our target value of 0.0400.

\item When $\omega = 0.01$,
we need to set $\mu = 0.0230$ and $\sigma = 0.0183$ to maintain the mean and
variance at their sample values.
See (a).

\item With these values, we have
$r^1 = 1.0907$, $\log r^1 = 0.0868$, and
$E \log r^e - \log r^1 = 0.0097 $.
That's partial success: the equity premium went up,
even though we held the mean and variance of
log consumption growth constant.
In that respect, the disaster state is a useful innovation,
but we still need risk aversion above 5 to hit the equity premium.

\item The equity premium is 0.0438 when $\alpha = 10$
and 0.2285 when $\alpha = 20$.
Evidently a much smaller value of $\alpha$ suffices when
we have a disaster state.

\item If we switch the sign of $\delta$, the equity premium
is below even the two-state version:  0.0037.
Apparently positive skewness in consumption and dividend growth
isn't helpful.
It's not part of the question,
but this is connected to the discussion of skewness and entropy:
positive skewness in $\log m$ increases entropy.
With power utility, that requires negative skewness in $\log g$.

\item We compute entropy directly from $m$:
\begin{eqnarray*}
    H(m) &=& \log E (m) - E \log m .
\end{eqnarray*}
With $\alpha = 5$, entropy is 0.0232 with the disaster state,
0.0152 without.  So the disaster increases entropy.
The difference between the two increases with risk aversion.
When $\alpha = 20$,
entropy is 1.5675 with the disaster state,
0.2273 without.
Yaron's bazooka!

It's not part of the question, but changing the sign of
$\delta$ reverses all this:
we need disasters, not booms.
\end{parts}
\end{solution}
