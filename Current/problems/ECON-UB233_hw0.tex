\documentclass[11pt]{exam}

\oddsidemargin=0.25truein \evensidemargin=0.25truein
\topmargin=-0.5truein \textwidth=6.0truein \textheight=8.75truein

\usepackage{comment}
\usepackage{hyperref}
\urlstyle{rm}   % change fonts for url's (from Chad Jones)
\hypersetup{
    colorlinks=true,        % kills boxes
    allcolors=blue,
    pdfsubject={ECON-UB233, Macroeconomic foundations for asset pricing},
    pdfauthor={Dave Backus @ NYU},
    pdfstartview={FitH},
    pdfpagemode={UseNone},
%    pdfnewwindow=true,      % links in new window
%    linkcolor=blue,         % color of internal links
%    citecolor=blue,         % color of links to bibliography
%    filecolor=blue,         % color of file links
%    urlcolor=blue           % color of external links
% see:  http://www.tug.org/applications/hyperref/manual.html
}

%\usepackage{amssymb}
%\usepackage{amsfonts}
%\usepackage{booktabs}

\renewcommand{\thefootnote}{\fnsymbol{footnote}}
%\newcommand{\bv}{\begin{verbatim}}
%\newcommand{\ev}{\end{verbatim}}

% document starts here
\begin{document}
\parskip=\bigskipamount
\parindent=0.0in
\thispagestyle{empty}
{\large ECON-UB 233 \hfill Dave Backus @ NYU}

\bigskip\bigskip
\centerline{\Large \bf Lab Report \#0:  Matlab Practice}
\centerline{Revised: \today}

\bigskip
{\it This should be completed before the second class,
but you can easily do it before the term starts.
It will not be collected or graded, but you should do it anyway.
Seriously. You'll regret it if you don't.}

Matlab is a popular high-level computer language,
which means it's easier to use than C++ or Java
and (far) more powerful than Excel.
It's widely used in quantitative segments of the business world,
including consulting, banking, and asset management.

What follows will get you started.  It should take you no more than an hour.
%
\begin{questions}
\question {\it Accessing Matlab at NYU.\/}
You can access Matlab in three ways at NYU:
\begin{itemize}
\item Online with NYU id:  \url{vcl.nyu.edu}
\item Online with Stern id: \url{apps.stern.nyu.edu}
\item On your own computer:  student versions cost roughly \$100 for a one-year license
\end{itemize}
%I haven't used either of the online versions,
%but have been told they are sensitive to the browser you use. % and behave better with IE.

Choose one of the above.
Once you do, start it up and type
{\tt demo} on the command line to get a demonstration of Matlab's basic
features.

\question {\it Scalar operations.\/} We'll start by entering commands on the command line
in what Matlab calls the Command Window.
If you can't find it, look for ``{\tt >>}'' (the ``prompt'').
\begin{parts}
\part Type each at the prompt:
\begin{verbatim}
x=7
x = 7       
x = 7;
x
x/2
7^2
ans
\end{verbatim}
What do each of these do? What does the semi-colon do?  What if you skip it?
\part Now type these: {\tt pi}, {\tt exp(1)}, {\tt e}, {\tt x = exp(2)}, {\tt log(x)}.
What do they do?
\end{parts}

\question {\it Vector operations.\/}
Matlab treats everything as a vector or matrix.
What this means is that you can do a bunch of things at once rather than copying the same command
over and over again as you would in (say) Excel.  Here's an example:
\begin{parts}
\part Generate a grid with the command:  {\tt x = [-3:0.5:3]}.
What does {\tt x} look like?
(Type {\tt x} at the prmpt to find out.) 
\part Now compute $x^2$ for all values of $x$:
\begin{verbatim} 
xsquared1 = x.^2 
xsquared2 = x.*x 
\end{verbatim}
The dot or period here means do the operation element by element.
That way, we can compute the square for every element of $x$ with one line.
\part Plot the results this way:  {\tt plot(x,xsquared)}.
Type {\tt help plot} to learn how to control the type and color of the line.
Can you make the line red?  Dashed?
\part Using the same {\tt x},
plot the functions {\tt normpdf(x)} and {\tt normcdf(x)}.
(You can enter these expressions directly in the plot command.)
Use the help command to find out what these functions are ---
for example, {\tt help normpdf}.
What do you see?  How would you make the line smoother?
\end{parts}

\question {\it Vector operations (continued).\/}
We can also work with data vectors. 
Suppose, for example, we have five observations
of some variable {\tt x} defined with the command 
{\tt x = [1; 7; -2; 4; 8]}.  

Explain each of the following:
{\tt mean(x)}, 
{\tt sum(x)}, 
{\tt length(x)}, and 
{\tt sum(x)/length(x)} 
%
As we noted earlier, you can get more information about these commands by 
typing {\tt help command} at the Matlab prompt.  
You can also Google ``matlab command''.

\question {\it Symbolic math.\/}
There are lots of computer programs that do symbolic math:
solve equations, differentiate functions, and so on.
We'll use Matlab to derive the moments
of the normal distribution from its moment generating function.
You'll use similar commands in future assignments, so pay attention.

Type these commands in the command line:
%
\begin{verbatim}
syms mu sigma s
mgf = exp(mu*s + (sigma*s)^2/2)
mu1 = subs(diff(mgf,s,1),s,0)
mu2 = subs(diff(mgf,s,2),s,0)
mu4 = subs(diff(mgf,s,4),[mu sigma s],[0 1 0])
\end{verbatim}
%
What do they do?
What happens if you skip the first line?
What is {\tt mgf}?
What is {\tt diff(mgf,s,2)}?
How would you use {\tt help} to find out more about {\tt diff}?
What does {\tt subs} do?


\question {\it Data input and scripts.\/}
Our last task is to input data from a spreadsheet
using a ``script'':
that is, we put the commands in a file so we can run them over again without typing
everything in from scratch.
\begin{parts}
\part Enter the following data into a spreadsheet:
\begin{eqnarray*}
    \begin{array}{cc}
        x1 & x2  \\ 1 & 2 \\ 3 & 4 \\ 5 & 6
    \end{array}
\end{eqnarray*}
Save it as an Excel workbook with name (say) {\tt testdata.xls}.
Make sure you know what directory it's in.

Comments:  (i)~You can use any file name you like,
but the same name must be used in the script below.
(ii)~You need to do something to tell Matlab what directory the file is in.
How that works depends on what version you're using.
(iii)~{\bf Warning for Mac users:}
In Windows, all Excel formats are recognized,
but Matlab for Macs may not recognize xlsx files.

\part We're going to wite a Matlab script to read the contents of the spreadsheet into Matlab.
Inside Matlab, go to the upper left corner,
click on File, and choose New and Script.
This should open the Matlab editor.
Now enter these commands: \\
\begin{verbatim}
%  Practice script for data input from spreadsheet (this is a comment)
format compact
disp('Spreadsheet input')
data = xlsread('testdata.xls')
\end{verbatim}

\bigskip
Save these commands under some appropriate file name
in the same directory as the spreadsheet.
It will automatically be given the file extension {\tt m} (for Matlab).

Now run the program by clicking on the green triangle at the top of
the editor.
(An alternative is to type the file name, without the {\tt .m},
on the command line.)
Then go to the Command Window to see if it works.
If not, welcome to the world of programming, where most of your time is
spent fixing bugs.
Remind yourself that patience is a virtue.
And speak to others about your problem,
that often helps even if they don't have the answer.

\part Add to your program a line that plots x1 against x2
and marks each data point with a circle.
\end{parts}

\question {\it Functions.\/}
Matlab allows you to write functions and store them in separate files.
If you put a function {\tt helloworld} in the file {\tt helloworld.m},
then any time you type  {\tt helloworld} it will execute that function.
(Assuming Matlab finds it; for now, we'll hope for the best.)
That's what many of the commands we've used so far do:
they call the Matlab file of the same name.

Here's an example:
\begin{verbatim}
function helloworld(x)
% Comments
% If you type "help helloworld" you'll get these comments
% The program prints "Hello World!" and the value of the variable x
% Examples:
%   helloworld(7)
%   helloworld([1:7])
disp('Hello World!')        % displays text in quotes
x                           % print x
return                      % end program and return
\end{verbatim}
Enter these commands in the file {\tt helloworld.m} and save them.
Now type at the prompt:
\begin{verbatim}
help helloworld
helloworld(pi)
\end{verbatim}

\end{questions}

\vfill \centerline{\it \copyright \ \number\year \
NYU Stern School of Business}

\end{document}


EXTRA QUESTION

\question (optional)
It's not at all necessary,
but if you'd like to see Matlab put to work on some interesting math problems,
Cleve Moler (one of the founders of the Mathworks, the publisher of Matlab)
has a cool book that does just that:

\medskip
\centerline{\url{http://www.mathworks.com/moler/exm/chapters.html}}
%
I'd start with the chapters on iteration and matrices.

