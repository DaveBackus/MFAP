\documentclass[11pt]{exam}

\oddsidemargin=0.25truein \evensidemargin=0.25truein
\topmargin=-0.5truein \textwidth=6.0truein \textheight=8.75truein

%\RequirePackage{graphicx}
\usepackage{verbatim}
\usepackage{booktabs}
\usepackage{comment}
\usepackage{hyperref}
\urlstyle{rm}   % change fonts for url's (from Chad Jones)
\hypersetup{
    colorlinks=true,        % kills boxes
    allcolors=blue,
    pdfsubject={ECON-UB233, Macroeconomic foundations for asset pricing},
    pdfauthor={Dave Backus @ NYU},
    pdfstartview={FitH},
    pdfpagemode={UseNone},
%    pdfnewwindow=true,      % links in new window
%    linkcolor=blue,         % color of internal links
%    citecolor=blue,         % color of links to bibliography
%    filecolor=blue,         % color of file links
%    urlcolor=blue           % color of external links
% see:  http://www.tug.org/applications/hyperref/manual.html
}

\renewcommand{\thefootnote}{\fnsymbol{footnote}}
\newcommand{\var}{\mbox{\it Var\/}}

\printanswers

% document starts here
\begin{document}
\parskip=\bigskipamount
\parindent=0.0in
\thispagestyle{empty}
{\large ECON-UB 233 \hfill Dave Backus @ NYU}

\bigskip\bigskip
\centerline{\Large \bf Quiz \#2}
\centerline{Revised: \today}

\bigskip
{\it Please write your name below.
Then complete the exam in the space provided.
There are THREE questions.
You may refer to one page of notes:
standard paper, both sides, any content you wish.}

\bigskip
\begin{flushleft}
\rule{4in}{0.5pt} \\ (Name and signature)
\end{flushleft}

\begin{questions}
%-----------------------------------------------------------------------
\item covariances.
Financial economist Scott Richard once said:
``I learned two basic lessons about financial mathematics
that I've always found useful.
One is that risk premiums stem from the covariance, not the variance.
The other is that it's the risk-neutral probabilities that matter.''
\begin{parts}
\item Explain his first lesson in terms we have used in class.
\item Explain his second lesson. 
\item In what sense are the two lessons in contradiction to each other?  
\end{parts}

\question {\it State prices and related objects (30 points).\/}
Consider a world with two dates ($t=0$ and $t=1$)
and two states at date $t=1$ ($z=1$ and $z=2$).
The probabilities of the two states are $p(1) = 1/4$ and $p(2) = 3/4$.
State prices are $Q(1) = 1/3$ and $Q(2) = 2/3$.
%
\begin{parts}
\item What is the price of a one-period bond in this world?
(5~points)
\item What is the pricing kernel in each state?
(5~points)
\item If this were a representative agent economy,
which state would be the good one for the agent?
Why?
(5~points)
\item What are the risk-neutral probabilities?
(5~points)
\item What is the maximum Sharpe ratio these state prices can generate?
(10~points)
\end{parts}

\begin{solution}
\begin{parts}
\item A bond consists of one unit of each Arrow security.
Its price is $q^1 = Q(1) + Q(2) = 1$.
\item The pricing kernel $m(z)$ is $ m(z) = Q(z)/p(z)$,
which gives us $m(1) = 4/3$ and $m(2) = 8/9$.
\item In the good state, the pricin kernel is low: payoffs
have low value when consumption is high.
Here that's state 2.
\item The risk-neutral probabilities are
$ p^*(z) = Q(z)/q^1 $,
which gives us back the state prices in this case:
$p^*(1) = 1/3$ and $p^*(2) = 2/3 $.
\item The Hansen-Jagannathan bound gives us the maximum Sharpe ratio
as the ratio of the standard deviation of the pricing kernel to its mean,
the price of a (one-period) bond.
Here the bond price is one, so all we need is the standard deviation.
The usual calculations give us
\begin{eqnarray*}
    \mbox{Var}(m) &=& E(m^2) - E(m)^2
            \;\;=\;\; 4/9 + 16/27 - 1 \;\;=\;\; 0.03704.
\end{eqnarray*}
The standard deviation is the square root,  $0.1925$,
which is the largest Sharpe ratio possible in this world.

\end{parts}
\end{solution}

%\pagebreak \phantom{xx} \pagebreak
%-----------------------------------------------------------------------
\question {\it Entropy with gamma risks (30 points).\/}
Consider our usual representative agent economy with power utility
and log consumption growth $ \log g = x$.
The log pricing kernel is therefore $\log m(x) = \log \beta -\alpha x $.
Log consumption growth $x$ has a gamma distribution with
cgf $ k(s; x) = - \theta \log (1-s/\lambda) $
and
density function $p(x) = x^{\theta-1} e^{-\lambda x} [\lambda^\theta/\Gamma(\theta)] $.
The parameters $(\lambda,\theta)$ are positive.

You may recall from earlier work that $x$ has mean and variance
\begin{eqnarray*}
    \kappa_1 &=& \theta /\lambda \\
    \kappa_2 &=& \theta /\lambda^2
\end{eqnarray*}
and skewness and excess kurtosis
\begin{eqnarray*}
    \gamma_1 &=& 2/\theta^{1/2} \\
    \gamma_2 &=& 6/\theta .
\end{eqnarray*}
%
\begin{parts}
\item What is the cgf of the log pricing kernel?
(10~points)
\item What is the entropy of the pricing kernel?
(10~points)
\item How does entropy compare to a lognormal benchmark
with the same mean and variance?
(10~points)
\end{parts}

\begin{solution}
\begin{parts}
\item Since it's a linear transformation of $x$, we have
\begin{eqnarray*}
    k(s; \log m) &=& s \log \beta + k(-\alpha s; x)
        \;\;=\;\; s \log \beta - \theta \log (1 + \alpha s/\lambda) .
\end{eqnarray*}
[Recall: if $y = a + bx$, then
$E(e^{sy}) = e^{sa} E(e^{sbx}) $.]

\item The term $ \log E(m) $ is just the cgf of $\log m$
evaluated at $s = 1$, namely
\begin{eqnarray*}
    \log E(m) &=& \log E (e^{\log m})
            \;\;=\;\; \log \beta - \theta \log (1+\alpha/\lambda) .
\end{eqnarray*}
The mean of $\log m$ is $ \log \beta - \alpha \theta/\lambda $.
The difference is entropy:
\begin{eqnarray*}
    H(m) &=& \log E (m) - E (\log m)
            \;\;=\;\; - \theta \log (1+\alpha/\lambda) + \alpha \theta/\lambda.
\end{eqnarray*}

\item In the lognormal case, entropy is the variance of $\log m$ over two,
which here is $ \alpha^2 \theta /(2 \lambda^2)$.
The expressions aren't particularly friendly,
but if you do a Taylor series expansion of the previous
expression, you find
\begin{eqnarray*}
    H(m) &=& - \theta \left[ (\alpha/\lambda) - (\alpha/\lambda)^2/2
        + (\alpha/\lambda)^3/3! + \cdots \right]
        + \alpha \theta/\lambda \\
        &=& - \theta \left[ - (\alpha/\lambda)^2/2
        + (\alpha/\lambda)^3/3! + \cdots \right] .
\end{eqnarray*}
So the variance term corresponds to the lognormal case,
but we have other terms after that.
\end{parts}
% ?? Dmitry f(lambda) = log normal - gamma, positive...
\end{solution}

%\pagebreak \phantom{xx} \pagebreak
%-----------------------------------------------------------------------
\question {\it Option on mixture of exponentials (40 points).\/}
Suppose the risk-neutral distribution of the future value of the underlying
is a mixture of $x_1$ and $x_2$:
\begin{eqnarray*}
    s_{t+1} &=&
        \left\{
        \begin{array}{ll}
            x_1 & \mbox{with probability } 1-\omega \\
            x_2 & \mbox{with probability } \omega
        \end{array}
        \right.
\end{eqnarray*}
for some $\omega$ between zero and one.
Each $x_j$ is exponential with density
\begin{eqnarray*}
    p(x_j) &=& \lambda_j \exp(-\lambda_j x_j)
\end{eqnarray*}
for $x_j \geq 0$ and $\lambda_j > 0$.
Each $x_j$ has a mean of $1/\lambda_j$.

\begin{parts}
\item What is the no-arbitrage condition for this asset?
(10~points)
\item Consider a put option giving the owner the right to sell
the asset for price $k$ at $t+1$.
What cash flow is generated by this option?
(10~points)
\item Suppose $\omega = 0$.
What is the option's value?
(10~points)
\item What is the option's value for some arbitrary value of $\omega$?
(10~points)
\end{parts}

\begin{solution}
\begin{parts}
\item We have
\begin{eqnarray*}
    s_t &=& q^1_t E^* (s_{t+1}) \\
            &=& q^1_t [(1-\omega) E^*(x_1) + \omega E^*(x_2)]
            \;\;=\;\; q^1_t [(1-\omega)/\lambda_1 + \omega /\lambda_2] .
\end{eqnarray*}
\item The cash flow is (as usual) $ (k-s_{t+1})^+ $.
\item If $\omega=0$ we can skip the second term.
The put price is
\begin{eqnarray*}
    q^p_t &=& q^1_t E^* (k-s_{t+1})^+  \\
            &=& q^1_t \int_{0}^{k} (k-x) \lambda_1 e^{-\lambda_1 x } dx
            \;\;=\;\;  q^1_t \big[ k - ( 1-e^{-\lambda_1 k})/\lambda_1 \big] .
\end{eqnarray*}
\item Here we have, by the same logic,
\begin{eqnarray*}
    q^p_t &=& q^1_t \Big\{
        (1-\omega) \big[ k - ( 1-e^{-\lambda_1 k})/\lambda_1 \big]
        + \omega \big[ k - ( 1-e^{-\lambda_2 k})/\lambda_2 \big]
        \Big\} .
\end{eqnarray*}

\end{parts}
\end{solution}

%\pagebreak %\phantom{xx} \pagebreak
\end{questions}

\phantom{xx}
\vfill \centerline{\it \copyright \ \number\year \
NYU Stern School of Business}
\end{document}

Option on $s^2$.



