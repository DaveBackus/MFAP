\documentclass[11pt]{exam}

\oddsidemargin=0.25truein \evensidemargin=0.25truein
\topmargin=-0.5truein \textwidth=6.0truein \textheight=8.75truein

%\RequirePackage{graphicx}
\usepackage{comment}
\usepackage{hyperref}
\usepackage{booktabs}
\urlstyle{rm}   % change fonts for url's (from Chad Jones)
\hypersetup{
    colorlinks=true,        % kills boxes
    allcolors=blue,
    pdfsubject={ECON-UB233, Macroeconomic foundations for asset pricing},
    pdfauthor={Dave Backus @ NYU},
    pdfstartview={FitH},
    pdfpagemode={UseNone},
%    pdfnewwindow=true,      % links in new window
%    linkcolor=blue,         % color of internal links
%    citecolor=blue,         % color of links to bibliography
%    filecolor=blue,         % color of file links
%    urlcolor=blue           % color of external links
% see:  http://www.tug.org/applications/hyperref/manual.html
}

\renewcommand{\thefootnote}{\fnsymbol{footnote}}
\newcommand{\var}{\mbox{\it Var\/}}

\printanswers

% document starts here
\begin{document}
\parskip=\bigskipamount
\parindent=0.0in
\thispagestyle{empty}
{\large ECON-UB 233 \hfill Dave Backus @ NYU}

\bigskip\bigskip
\centerline{\Large \bf Quiz \#1}
\centerline{Revised: \today}

\bigskip
{\it Please write your name below,
then complete the exam in the space provided.
You may refer to one page of notes:
standard paper, both sides, any content you wish.
{\bf There are FOUR questions.}
}

\bigskip
\begin{flushleft}
\rule{4in}{0.5pt} \\ (Name and signature)
\end{flushleft}

% ----------------------------------------------------------------------
\begin{questions}
\question {\it Moments (30~points).\/}
Consider a random variable $x$ that takes on the values
\begin{eqnarray*}
    x &=& \left\{
          \begin{array}{ll}
          - \delta  & \mbox{with probability } 1-\omega \\
          +\delta    & \mbox{with probability } \omega
          \end{array}
          \right.
\end{eqnarray*}
with $\delta > 0$.
%
\begin{parts}
\item Do the probabilities constitute a legitimate probability distribution?
(5~points)
\item What is the moment generating function (mgf) for $x$?
(10~points)
\item Use the mgf to derive the mean and variance of $x$.
(15~points)
\end{parts}

\begin{solution}
\begin{parts}
\item The probabilities must be nonnegative and sum to one.
They sum to one by construction.
They're nonnegative if $ 0 \leq \omega \leq 1$.
\item The mgf is
\begin{eqnarray*}
    h(s) &=& E (e^{sx}) \;\;=\;\;
            (1-\omega) e^{-\delta s} + \omega e^{\delta s} .
\end{eqnarray*}
\item The first two derivatives of the mgf are
\begin{eqnarray*}
    h'(s) &=& - \delta (1-\omega) e^{-\delta s} + \delta \omega e^{\delta s} \\
    h''(s) &=& \delta^2 (1-\omega) e^{-\delta s} + \delta^2 \omega e^{\delta s} .
\end{eqnarray*}
The mean is $k'(0) = \delta(2\omega-1)$.
The variance is
%\begin{eqnarray*}
$  k''(0) - k'(0)^2 = 4 \delta^2 \omega(1-\omega)$.
%\end{eqnarray*}
\end{parts}
\end{solution}

%\pagebreak \phantom{xx} \pagebreak
% ----------------------------------------------------------------------
\question {\it Geometric risk (25 points).\/}
Consider a power utility agent facing ``geometric risk.''
Utility is $u(c) = c^{1-\alpha}/(1-\alpha) $ with risk aversion
parameter $\alpha > 0$.
Consumption is $ c = e^x$
for $ x = 0,1,2, \ldots$ with probabilities
$ p(x) = (1-\omega) \omega^x$ for some parameter $0< \omega < 1$.
We say that $x$ has a {\it geometric distribution\/}.
%
\begin{parts}
\item Show that $p(x)$ is a legitimate probability distribution.
(5~points)
\item What is the mgf of $x$?
(5~points)
\item Suppose $\omega e < 1$.  What is the mean of $c$?
(5~points)
\item What is expected utility?
(5~points)
\item What is the certainty equivalent?
(5~points)
\end{parts}

\begin{solution}
\begin{parts}
\item Probabilities must be nonnegative and sum to one.
Here they're positive, and the sum is
\begin{eqnarray*}
    \sum_x p(x) &=& \sum_{x=0}^\infty (1-\omega) \omega^x
        \;\;=\;\; (1-\omega) / (1-\omega) \;\;=\;\; 1.
\end{eqnarray*}

\item The mgf is
\begin{eqnarray*}
    h(s) &=& E (e^{sx}) \;\;=\;\;
            \sum_{x=0}^\infty (1-\omega) \omega^x e^{sx}
            \;\;=\;\; (1-\omega) / ( 1 - \omega e^s) .
\end{eqnarray*}

\item The mean of $c$ is
\begin{eqnarray*}
    E(c) &=& E (e^{x}) \;\;=\;\; h(1)
            \;\;=\;\; (1-\omega) / ( 1 - \omega e) .
\end{eqnarray*}

\item Expected utility is
\begin{eqnarray*}
    E(c^{1-\alpha})/(1-\alpha) &=&  h(1-\alpha)/(1-\alpha)
            \;\;=\;\; (1-\omega) / [( 1 - \omega e^{1-\alpha})(1-\alpha)] .
\end{eqnarray*}
\item The certainty equivalent $\mu$ is the solution to
\begin{eqnarray*}
    \mu^{1-\alpha}/(1-\alpha) &=&  (1-\omega) / [( 1 - \omega e^{1-\alpha})(1-\alpha)] ,
\end{eqnarray*}
namely
\begin{eqnarray*}
    \mu &=&  \left[
                \frac{ 1-\omega} { 1 - \omega e^{1-\alpha} }
             \right]^{1/(1-\alpha)} .
\end{eqnarray*}
\end{parts}
\end{solution}

%\pagebreak \phantom{xx} \pagebreak
%-----------------------------------------------------------------------
\question {\it General equilibrium (20 points).\/}
Consider a two-period economy with dates $t=0$ and $t=1$.
At $t=1$, a state $z$ occurs, an element of a finite set $\mathcal{Z}$.
There is one good in each date and state.
A single agent's preferences over these goods are described
by the utility function
\begin{eqnarray*}
    U &=& u(c_0) + \beta \sum_z p(z) u[c_1(z)] .
\end{eqnarray*}
%The agent has a positive endowment of the good in each date and state.
There is also a production opportunity:  if we use $x$ units of the date-0 good as an input,
we produce $f(x)$ units of the date-1 good in all states $z$.
%
\begin{parts}
\item What ingredients do you need to turn this into a complete model economy?
(10~points)
\item What is the Pareto problem associated with this economy?
(10~points)
\end{parts}

\begin{solution}
\begin{parts}
\item The standard list:
\begin{itemize}
\item List of commodities:  1 at $t=0$, as many as states at $t=1$.
\item List of agents:  one.
\item Preferences and endowments:  preferences above; need to specify endowments.
\item Technology:  $f(x)$.
\item Resource constraints:  If the endowments are $y$, we have
\begin{eqnarray*}
    c_0 + x &\leq& y_0 \\
    c_1(z)  &\leq& y_1(z) + f(x) \mbox{ in each state } z.
\end{eqnarray*}
\end{itemize}

\item We want to maximize the utility of the agent subject to the resource constraints.
You could simply write down the problem:  utility function and resource constraints.
Or you could report the Lagrangian,
{\small
\begin{eqnarray*}
    \mathcal{L} &=& u(c_0) + \beta \sum_z p(z) u[c_1(z)]
        + q_0 (y_0 - c_0 - x) + \sum_z q_1(z)
        [y_1(z) + f(x) - c_1(z) ] ,
\end{eqnarray*}
}
which contains the same information.

\end{parts}
\end{solution}

%\pagebreak \phantom{xx} \pagebreak
%-----------------------------------------------------------------------
\question {\it Short answers (25 points).\/}

\begin{parts}
\item How would you compute excess kurtosis in a sample of data?
(10~points)
\item Consider the asset prices and dividends
\begin{eqnarray*}
    \mbox{Asset 1:}&&  q^1 = 3/4, \;\; d^1(1) = 1, \;\; d^1(2) = 1 \\
    \mbox{Asset 2:}&&  q^2 = 1, \;\; d^2(1) = 1, \;\; d^2(2) = 2.
\end{eqnarray*}
What are the returns on the two assets?
What are the implied prices of Arrow securities?
(15~points)
\end{parts}

\begin{solution}
\begin{parts}
\item First, we need to compute the sample mean and variance:
\begin{eqnarray*}
    \bar{x} &=& T^{-1} \sum_t x_t \\
    s^2  &=& T^{-1} \sum_t (x_t -\bar{x})^2 .
\end{eqnarray*}
Excess kurtosis is then
\begin{eqnarray*}
    \gamma_2  &=& \frac{T^{-1} \sum_t (x_t -\bar{x})^4} {s^4} - 3.
\end{eqnarray*}

\item The returns on the first asset are
$ r^1(1) = r^1(2) = 4/3$.
The returns on the second are
$ r^2(1) = 1$ and $r^2(2) = 2$.
The state prices are $Q(1) = 1/2$ and $Q(2) = 1/4$.
\end{parts}
\end{solution}


%\pagebreak \phantom{xx} %\pagebreak
\end{questions}

%\phantom{xx}
\vfill \centerline{\it \copyright \ \number\year \
NYU Stern School of Business}
\end{document}



