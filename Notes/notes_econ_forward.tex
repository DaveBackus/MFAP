\documentclass[11pt]{article}

\input{../LaTeX/preamble.tex}

\begin{document}
\parskip=\bigskipamount
\parindent=0.0in
\thispagestyle{empty}
\input{../LaTeX/header.tex}

\bigskip\bigskip
\centerline{\Large \bf Forward-Looking Models}
\centerline{Revised: \today}

\begin{comment}
??

Think about how this is organized, also how it ties
in with recursive methods.  

Do recursive methods first?  

Lucas:
http://www.nber.org/chapters/c6264.pdf

Page 199:
I take the purpose of this session to be to elicit views on economic
policy from economists of different points of view. The title
of the session, “Macroeconomic Policy, 1974/75 : What Should Have
Been Done?” does not seem to me useful for this purpose.

Page 205:
our ability as economists to predict the responses of
agents rests, in situations where expectations about the future matter,
on our understanding of the stochastic environment agents believe themselves
to be operating in. In practice, this limits the class of policies the
consequences of which we can hope to assess in advance to policies
generated by fixed, well understood, relatively permanent rules (or functions
relating policy actions taken to the state of the economy).
\end{comment}

\bigskip
Lots of things in economics and finance are forward-looking:
decisions made now depend on things we expect to happen in the future.
%Education is a classic example:  you're investing in skills you expect to have
%value in the future.
We'll take what we might call a classical approach to this issue,
illustrating in one-dimensional linear models how expectations of
the future affect the present.
%The core example is hyperinflation, but the idea is a general one.
This builds on work done in the 1970s by lots of people,
Tom Sargent and Neil Wallace among them.
In one respect, it's a step backwards for us:  we'll focus on the conditional mean
and ignore risk.
No need to panic, we'll bring back risk shortly.

Our leading example here is inflation.
It's an introduction of sorts to bond yields.
Since bonds are claims to money in the future,
the inflation rate is a central component.


\section{Hyperinflation and the quantity theory}

{\it Hyperinflation\/} is the term we use for high rates of inflation:
over one hundred percent a year.
There have been many examples of inflation rates above a thousand percent.
See, for example, the
\href{http://en.wikipedia.org/wiki/Hyperinflation}{Wikipedia entry}.

So where do hyperinflations come from?
As economists, we're incredibly pleased with ourselves that we figured this one out.
Tom Sargent put it this way in an interview:
\begin{quote}
%I wrote a paper a long time ago about how to start and stop a hyperinflation
The way to start a hyperinflation is run sustained government deficits and then have the monetary authority print money to pay for it.  That always works.  How do you stop a hyperinflation?  You stop doing it.  This isn't high economic theory.
\end{quote}
Here's a
\href{http://youtu.be/bVIOClT4Rws}{link}
to the whole interview.
The comments about hyperinflation come at 23:40.

The ``print money'' step of this process is often interpreted with the
{\it quantity theory of money\/}:
\begin{eqnarray*}
    M V &=& P Y .
\end{eqnarray*}
Here $M$ is the quantity of money (currency, for example) in circulation,
$V$ is ``velocity'' (more shortly),
$P$ is the price level, and $Y$ is GDP.
This is a description of how money is used to make transactions.
The dollar value of transactions is $PY$, the product of the price and the quantity.
Velocity $V$ is a measure of how quickly a unit of  money is reused in another transaction.
In a hyperinflation, rapid growth in money is associated
with rapid growth in prices --- inflation, in other words.

This theory works reasonably well in the sense that we always see high money growth in periods
of high inflation.
We also see government deficits, as Sargent suggested.
In these respects the theory works pretty well.
But it misses on the timing at both the start and the end.
Often hyperinflations break out before money growth picks up,
and they end before money growth stops.
The question is why.


\section{Digression:  the law of iterated expectations}

Consider a stochastic process $x$ that evolves somehow over time.
What is the expectation of $x_{t+k}$ conditional on whatever we know at date $t$?
If $I_t$ is the information available at date $t$, we might write
\begin{eqnarray*}
    E_t (x_{t+k}) &=&  E \left( x_{t+k} | I_t \right) .
\end{eqnarray*}
The left side here is shorthand notation for the right side.
In a Markovian environment with state $z_t$, we can replace $I_t$ with $z_t$.

The {\it law of iterated expectations\/} says that we can compute this one period at
a time.
If $k=2$, we have
\begin{eqnarray*}
    E_t (x_{t+2}) &=&  E_t \big[ E_{t+1} (x_{t+2} ) \big].
\end{eqnarray*}
And so on for longer time horizons $k$.
[Write this out to make sure you're following.]

{\it Example.\/}
Let's see how this works for an AR(1):
\begin{eqnarray*}
    x_{t+1} &=& \varphi x_t + \sigma w_{t+1} ,
\end{eqnarray*}
where $\{ w_t \}$ is (as usual) a sequence of iid standard normal random variables.
Here
\begin{eqnarray*}
    E_{t+1} (x_{t+2}) &=&  \varphi x_{t+1} ,
\end{eqnarray*}
so
\begin{eqnarray*}
    E_{t} (x_{t+2}) &=&  E_t \big[ E_{t+1} (x_{t+2})\big]
        \;\;=\;\; E_t (\varphi x_{t+1})
        \;\;=\;\; \varphi^2 x_{t} .
\end{eqnarray*}
You get the idea.  We do this one period at a time, but end up with the same
answer we had before.



\section{Forward-looking models}

Now let's think about forward-looking models.
There are lots of situations in economics in which current decisions are based on
what people expect to happen in the future:
consumption (what is future income?), investment (what is future demand for my product?),
asset valuation (what are future cash flows?), and so on.

A canonical version of a {\it forward-looking difference equation\/} looks like this:
\begin{eqnarray}
    y_t &=& \lambda E_t (y_{t+1}) + x_t .
    \label{eq:canonical}
\end{eqnarray}
This idea here is that $y_t$ depends on the variable $x_t$,
and also on the expectation of itself at $t+1$.
That leads to the question: What drives $y_t$?
Is it $x_t$ or the expectation $E_t y_{t+1}$?
%If the latter, don't we get into an infinite loop?

{\it Example.\/}
Here's an example that shows up in finance classes: equity as a claim to future dividends.
Suppose the price of equity is
\begin{eqnarray*}
    q_t = \delta (d_t + E_t q_{t+1} ) .
\end{eqnarray*}
Here $\delta$ is the discount factor we apply to future cash flows,
something you might write as $1/(1+i)$ in a finance or accounting class.
So we see that today's stock price depends on dividends, but also
on what we expect the stock price to be next period.
So we wonder:  Is the price is high today because we expect it to be high tomorrow?
How do we get out of this circular reasoning?


Let's go back to our canonical example.
In the deterministic version, with no uncertainty,
equation (\ref{eq:canonical}) becomes
\begin{eqnarray*}
    y_t &=&  \lambda y_{t+1} + x_{t} .
\end{eqnarray*}
If we substitute, we get
\begin{eqnarray*}
    y_t &=&  x_t + \lambda x_{t+1} + \lambda^2 y_{t+2}  \\
        &=&  x_t + \lambda x_{t+1} + \cdots \lambda^k x_{t+k} + \lambda^{k+1} y_{t+k+1} .
\end{eqnarray*}
If $|\lambda | < 1$ and $y$ doesn't explode somehow (we're glossing over some technical details here),
then we have
\begin{eqnarray*}
    y_t &=&  \sum_{j=0}^\infty \lambda^j x_{t+j} .
\end{eqnarray*}
That is:  $y_t$ depends on the current and future values of $x_t$.
That's the sense in which this is a forward-looking model.

In the stochastic version, the same logic (repeated substitution)
plus the law of iterated expectations gives us
\begin{eqnarray}
    y_t &=&  \sum_{j=0}^\infty \lambda^j E_t (x_{t+j}) .
    \label{eq:canonical-solution}
\end{eqnarray}
Thus we have connected the price to expectations of future dividends,
discounted back to the present.
We might say that the variable $y$ responds to changes in expected fundamentals $x$.


{\it Example.\/}
With some structure on $x$, we can be more specific.
Suppose, for example, that $x$ is AR(1):
$ x_{t+1} = \varphi x_t + \sigma w_{t+1}$.
Then $ E_t (x_{t+j}) = \varphi^j x_t $ and
\begin{eqnarray*}
    y_t &=&  \sum_{j=0}^\infty \lambda^j \varphi^j x_t
            \;\;=\;\; x_t /(1- \lambda \varphi).
\end{eqnarray*}
This works as long as $ |\lambda \varphi | < 1$.

There's a more direct route that you might have run across in
a differential equations course:  the {\it method of undetermined coefficients\/}.
Continuing with the same example,
we guess a linear solution:  $ y_t = a x_t$ for some parameter $a$ to be determined.
Then (\ref{eq:canonical}) becomes
\begin{eqnarray*}
    a x_t &=& \lambda a (\varphi x_t) + x_t .
\end{eqnarray*}
Since this holds for all $x_t$, we must have
$ a = \lambda \varphi a + 1 $ or $ a = 1/(1-\lambda \varphi)$.
This works, but since it shortcuts the infinite sum it doesn't remind
us that we need $ |\lambda \varphi | < 1 $.


\begin{comment}
Adam asked how we compute expectations of future (say) dividends in forward-looking models.  In all of our examples, dividends follow some stationary process, which we then used to compute expectations.

Is that reasonable?  I'd say sometimes yes, sometimes no.  What do we do if it's not?  Hard to say.  Here's Robert Lucas on a similar issue, the impact of monetary policy:

Page 199:
I take the purpose of this session to be to elicit views on economic
policy from economists of different points of view. The title
of the session, “Macroeconomic Policy, 1974/75 : What Should Have
Been Done?” does not seem to me useful for this purpose.

What he means, I think, is that we need to know more about future policy to be able to assess the impact of current policy.  How much more?

Page 205:
[O]ur ability as economists to predict the responses of
agents rests, in situations where expectations about the future matter,
on our understanding of the stochastic environment agents believe themselves
to be operating in. In practice, this limits the class of policies the
consequences of which we can hope to assess in advance to policies
generated by fixed, well understood, relatively permanent rules (or functions
relating policy actions taken to the state of the economy).

That is:  without some "understanding of the stochastic environment" it's impossible to say what agents should expect, and therefore impossible to say what the impact will be.

Here's the whole thing:  http://www.nber.org/chapters/c6264.pdf
\end{comment}


\section{What can we expect?}

You'll have noticed that our solution (\ref{eq:canonical-solution})
to (\ref{eq:canonical}) requires expectations of the exogenous variable $x$.  
In all of our examples, $x$ follows some well-behaved stochastic process, which we then used to compute expectations.

Is that reasonable?  I'd say sometimes yes, sometimes no.  What do we do if it's not?  Hard to say.  
Here's \href{http://www.nber.org/chapters/c6264.pdf}{Robert Lucas} 
on a similar issue, the impact of monetary policy in the distant past:
\begin{quote}
I take the purpose of this session to be to elicit views on economic
policy from economists of different points of view. The title
of the session, ``Macroeconomic Policy, 1974/75 : What Should Have
Been Done?'' does not seem to me useful for this purpose. 
\end{quote} 
What he means, I think, is that the impact of current policy depends, in part, 
on what it leads people to expect about future policy.
It's simply not possible in these settings to speak separately
about current and future policy.  
Or, in our example, current and future dividends.  
He continues:
\begin{quote} 
[O]ur ability as economists to predict the responses of
agents rests, in situations where expectations about the future matter,
on our understanding of the stochastic environment agents believe themselves
to be operating in. In practice, this limits the class of policies the
consequences of which we can hope to assess in advance to policies
generated by fixed, well understood, relatively permanent rules (or functions
relating policy actions taken to the state of the economy).
\end{quote} 
%
That is:  without some ``understanding of the stochastic environment'' 
it's impossible to say what agents should expect.
In a sense, he's throwing up his hands at the idea that we can do more than this.  
Perhaps we can, but it's good to keep in mind that we're taking 
on something inherently challenging if we do.  


\section{Bubbles}

The 2013 Nobel Prize in economics went to three people known for their
contributions to financial economics:  Gene Fama, Lars Hanse, and Robert Shiller.
We ran across some of Hansen's work when we derived the Hansen-Jagannathan bound.
Fama and Shiller are known for having very different views of
We've noted that Shiller often sees bubbles in asset markets,
while Fama says he doesn't know what a bubble is.
Oddly enough, I'd say both make useful points.
The term bubble is often used to describe a market in which prices
seem to have lost connection to ``fundamentals.''
We might think that equity prices, for example, should be connected to dividends.
Shiller argues that the loose connection between the two suggests we need another model,
perhaps one in which investors are ``irrational,''
or even a ``bubble.''
%To be sure, a lot of what we observe in equity prices remains mysterious.
Fama argues instead that the term bubble is typically used to mean ``we don't know,''
which is hardly a theory of anything.

Whether any particular situation is a bubble is hard to say,
but we do have mathematical theories in which the behavior of asset prices is
disconnected from fundamentals.
The most popular version gives us solutions to (\ref{eq:canonical})
other than (\ref{eq:canonical-solution}).
The idea?
Solving a difference equation like (\ref{eq:canonical}) is analogous to solving a differential
equation, where we have a constant of integration to think about.

Here's how that works.
We add a term $c \lambda^{-t}$ to the solution, turning (\ref{eq:canonical-solution}) into
\begin{eqnarray*}
    y_t &=&  \sum_{j=0}^\infty \lambda^j E_t (x_{t+j}) + c \lambda^{-t}
\end{eqnarray*}
for some constant $c$.
Does this satisfy (\ref{eq:canonical})?
We have
\begin{eqnarray*}
    y_{t+1} &=& \sum_{j=0}^\infty \lambda^{j} E_{t+1} (x_{t+j+1}) + c \lambda^{-(t+1)} \\
    \lambda E_t(y_{t+1})
        &=& \sum_{j=0}^\infty \lambda^{j+1} E_t (x_{t+j+1}) + c \lambda^{-t}
        \;\;=\;\; \sum_{j=1}^\infty \lambda^{j} E_t (x_{t+j}) + c \lambda^{-t} .
\end{eqnarray*}
Subtracting this from $y_t$ gives us zero, so our new solution satisfies
(\ref{eq:canonical}) for any constant $c$.

How does this new solution behave?  The term $c \lambda^{-t}$ tends to grow,
since $ |\lambda| < 1$.  The traditional approach is to set $c=0$, since
any other solution blows up.
Since we don't see asset prices growing forever, or turning negative,
that seems persuasive.

But here's a variant that avoids this problem.
Consider the term $c_t \lambda^{-t}$.
The same logic as before tells us that this satisfies (\ref{eq:canonical})
as long as
\begin{eqnarray*}
    E_t (c_{t+1} \lambda^{-t}) &=& c_{t} \lambda^{-t} .
\end{eqnarray*}
This is true if
\begin{eqnarray*}
    E_t (c_{t+1}) &=& c_t .
\end{eqnarray*}
This condition defines $c_t$ as a {\it martingale\/}:  a stochastic process
whose conditional mean is its current value.

Here's an example.  Suppose
\begin{eqnarray*}
    c_{t+1} &=&
            \left\{
            \begin{array}{ll}
            \lambda^{-1} c_t & \mbox{with probability } \lambda \\
            0       &  \mbox{with probability } 1-\lambda .
            \end{array}
            \right.
\end{eqnarray*}
It should be clear that this is a martingale.
This is like the previous example, in which prices rise at rate $1/\lambda$.
But eventually the bubble pops, as the term goes to zero.
[Graph this term over time.]


[This section is adapted from Tom Sargent's
\href{https://files.nyu.edu/ts43/public/teaching/NYU_course_all_2011.pdf}{course notes},
pages 27-29 and 55-58.]


\section{Hyperinflation revisited}

So what does this have to do with hyperinflation?
There's a more sophisticated version of the quantity theory in which
velocity varies with the expected inflation rate.
In logs, the quantity theory is
\begin{eqnarray*}
    m_t + v_t &=& p_t + y_t .
\end{eqnarray*}
Now add velocity:
\begin{eqnarray*}
    v_t &=& \alpha (E_t p_{t+1} - p_t )
\end{eqnarray*}
for some $\alpha > 0$.
Here we see that as expected inflation rises,
velocity rises, too.
The idea is that money loses value through inflation.
At faster rates of inflation people work harder to get rid of money,
which raises velocity.

Putting these two pieces together, we have
\begin{eqnarray*}
    m_t + \alpha (E_t p_{t+1} - p_t ) &=& p_t + y_t
\end{eqnarray*}
or
\begin{eqnarray*}
    p_t  &=& [\alpha/(1+\alpha)] E_t (p_{t+1})  + (1+\alpha)^{-1} (m_t- y_t).
\end{eqnarray*}
This has the same form as (\ref{eq:canonical}),
so we can apply the solution (\ref{eq:canonical-solution}).
As a result, the price level $p$ depends not only on the current money
supply, it also depends on expectations of the future money supply.
Anything that changes expectations of future $m$ will change current $p$.
This line of reasoning offers hope, and perhaps more,
of fixing up the timing problems noted earlier.



\section*{Bottom line}

Solutions to forward-looking models connect current decisions
to the expected future values of fundamentals.
Bubbles add an extra term to these solutions.


\section*{Practice problems}

\begin{enumerate}
%-----------------------------------------------------------------------
\item {\it Stock prices.\/}
Suppose stock prices are ex-dividend, so that
\begin{eqnarray*}
    q_t = \delta E_t (d_{t+1} + q_{t+1}) .
\end{eqnarray*}
How is the current stock price $q_t$ related to expected future
dividends?
%
\needspace{2\baselineskip}
Answer. The same logic we used earlier gives us
\begin{eqnarray*}
    q_t &=&  \sum_{j=1}^\infty \delta^j E_t (d_{t+j}) .
\end{eqnarray*}
That is:  we start the sum at $j=1$ rather than $j=0$.

%-----------------------------------------------------------------------
\item {\it ARMA(1,1) fundamentals.\/}
Suppose in our canonical model (\ref{eq:canonical}) that
$x_t$ is ARMA(1,1):
\begin{eqnarray*}
    x_{t+1} &=& \varphi x_t + w_{t+1} + \theta w_t ,
\end{eqnarray*}
with $| \varphi| < 1$ and $\{ w_t \}$ the usual sequence of independent
standard normals.
What is the solution?
That is:  how is the endogenous variable $y_t$ connected to $x_t$ and $w_t$?
%
\needspace{2\baselineskip}
Answer.
The simplest approach is the method of undetermined coefficients.
The state here is $z_t = (x_t, w_t)$.
We guess a solution of the form
$ y_t = a x_t + b w_t $.
Since $ E_t (x_{t+1}) = \varphi x_t + \theta w_t $,
we have
\begin{eqnarray*}
    E_t (y_{t+1}) &=& a E_t (x_{t+1})
            \;\;=\;\; a (\varphi x_t + \theta w_t ).
\end{eqnarray*}
Equation (\ref{eq:canonical}) then implies
\begin{eqnarray*}
        a x_t + b w_t &=& \lambda a (\varphi x_t + \theta w_t ) + x_t .
\end{eqnarray*}
Equating coefficients of $x_t$ and $w_t$, respectively,
gives us $a = 1/(1-\lambda\varphi)$ and
$b = \lambda a \theta = \lambda \theta /(1-\lambda\varphi)$.


%-----------------------------------------------------------------------
\item  {\it Inflation and the Taylor rule.\/}
We can get a sense of the impact of monetary policy on
inflation with the equations
\begin{eqnarray*}
    i_t &=& r + E_t (\pi_{t+1}) \\
    i_t &=& r + \tau \pi_t + s_t.
\end{eqnarray*}
The first equation is the Fisher equation, which says that the nominal interest rate $i_t$
equals the (constant) real interest rate $r$ plus expected inflation.
The second equation is Taylor's rule, which tells the central bank to set the
nominal interest rate equal to the real rate plus an adjustment for current inflation.
By assumption $\tau > 1$, so that an increase in inflation leads to a greater increase
in the nominal interest rate.
The ``shock'' $s_t$ to monetary policy is AR(1):
$s_{t+1} = \varphi s_t + \sigma w_{t+1}$ with $0 < \varphi < 1$
and independent standard normal innovations $w_t$.

The idea behind this policy rule is to respond aggressively to inflation;
the larger is $\tau$ the more aggressive the response.
%
\begin{enumerate}
\item Express these two equations in the form of a forward-looking difference equation
in the inflation rate $\pi_t$.

\item In what sense does future monetary policy affect the current inflation rate?

\item Solve the model.  How does inflation depend on the shock $s_t$?

\item In what sense might we say that larger values of $\tau$ are more successful
in eliminating inflation?
\end{enumerate}

\needspace{4\baselineskip}
Answer.
\begin{enumerate}
\item If we set the first equation
equal to the second, we get
\begin{eqnarray*}
    E_t (\pi_{t+1}) &=& \tau \pi_t + s_t .
\end{eqnarray*}
This isn't quite in the form of (\ref{eq:canonical}), but it's close.

\item Repeated substitution gives us
\begin{eqnarray*}
    \pi_{t} &=&  \sum_{j=0} \tau^{-j} E_t \big( \tau^{-1} s_{t+j} \big) ,
\end{eqnarray*}
so that future shocks to monetary policy affect the current inflation rate.


\item We guess the solution has the form:  $\pi_t = a s_t $ for some coefficient $a$.
The equation becomes
\begin{eqnarray*}
    a \varphi s_t  &=& \tau a s_t + s_t ,
\end{eqnarray*}
which implies $ a = -1/(\tau-\varphi) < 0$.

\item As we increase $\tau$, we reduce the impact $a$ of the shock on the inflation rate.
More precisely, the variance of the inflation rate is
\begin{eqnarray*}
    \mbox{Var}(\pi_t) &=& (\tau-\varphi)^{-2} \mbox{Var}(s_t)
                \;\;=\;\; \frac{1} {(\tau-\varphi)^{2} (1-\varphi^2)} ,
\end{eqnarray*}
which is decreasing in $\tau$.

\end{enumerate}


\begin{comment}
%-----------------------------------------------------------------------
\item  {\it Inflation and government deficits.\/}
Here's a variant of our forward-looking inflation model.
We have, as usual, the quantity theory plus a velocity equation:
\begin{eqnarray*}
    m_t + v_t &=& p_t + y_t \\
    v_t &=& \alpha \left( E_t p_{t+1} - p_t \right) .
\end{eqnarray*}
The endogenous variable here is $p_t$.
We set $y_t = 0$ (to keep things simple) and connect the money supply $m_t$
to the government deficit $d_t$:
\begin{eqnarray*}
        m_{t} &=& \delta d_t \\
        d_{t+1} &=& \varphi d_t + \sigma w_{t+1} .
\end{eqnarray*}
%
\begin{enumerate}
\item Express this model as a forward-looking difference equation in
which the price level $p_t$ is a function of its expected future value
and the deficit.
\item How is the price level connected to the current deficit?
\end{enumerate}
%
\needspace{2\baselineskip}
Answer.
\begin{enumerate}
\item Substitutions give us
\begin{eqnarray*}
    p_t &=& [\alpha/(1+\alpha)] E_t p_{t+1} + [\delta/(1+\alpha)] d_t \\
        &=& \lambda E_t p_{t+1} + \delta^\prime d_t ,
\end{eqnarray*}
where the second line is compact notation for the first.
\item This has the usual forward-looking solution in which $p_t$
is connected to expected future deficits.
We could grind through this, but it's easier to use the method of undetermined
coefficients.
If we guess the solution has the form $ p_t = a d_t$ for some coefficient $a$ to be determined,
then
\begin{eqnarray*}
    p_{t+1} &=& a d_{t+1} \;\;=\;\; a (\varphi d_t + \sigma w_{t+1}) \\
   E_t (  p_{t+1})  &=&  a \varphi d_t .
\end{eqnarray*}
If we substitute into the difference equation above and collect coefficients
of $d_t$, we get
\begin{eqnarray*}
    a &=& \frac{\delta'}{1-\lambda \varphi}  \;\;=\;\; \frac{\delta}{1+\alpha(1-\varphi)} .
\end{eqnarray*}
In words:  the impact of the current deficit on the price is larger if
(i)~money is more strongly tied to deficits (large $\delta$),
(ii)~deficits are more persistent (large $\varphi$),
and (iii)~velocity is less sensitive to expected inflation (small $\alpha$).
\end{enumerate}
\end{comment}


\end{enumerate}


\input{../LaTeX/footer.tex}

\end{document}


