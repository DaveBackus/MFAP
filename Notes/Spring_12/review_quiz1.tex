\documentclass[11pt]{article}

\oddsidemargin=0.25truein \evensidemargin=0.25truein
\topmargin=-0.5truein \textwidth=6.0truein \textheight=8.75truein

%\usepackage{graphicx}
\usepackage{verbatim}
%\usepackage{booktabs}
\usepackage{comment}
\usepackage[dvipdfm]{hyperref}
\urlstyle{rm}   % change fonts for url's (from Chad Jones)
\hypersetup{
    colorlinks=true,        % kills boxes
    allcolors=blue,
    pdfsubject={ECON-UB233, Macroeconomic foundations for asset pricing},
    pdfauthor={Dave Backus @ NYU},
    pdfstartview={FitH},
    pdfpagemode={UseNone},
%    pdfnewwindow=true,      % links in new window
%    linkcolor=blue,         % color of internal links
%    citecolor=blue,         % color of links to bibliography
%    filecolor=blue,         % color of file links
%    urlcolor=blue           % color of external links
% see:  http://www.tug.org/applications/hyperref/manual.html
}
% more from Chad Jones
%\usepackage{tweaklist}
%\renewcommand{\enumhook}{\setlength{\topsep}{0pt}%
%  \setlength{\itemsep}{0pt}}

\renewcommand{\thefootnote}{\fnsymbol{footnote}}

% document starts here
\begin{document}
\parskip=\bigskipamount
\parindent=0.0in
\thispagestyle{empty}
{\large ECON-UB 233 \hfill Dave Backus @ NYU}

\bigskip\bigskip
\centerline{\Large \bf Review for Quiz \#1}
\centerline{(Started: February 10, 2012; Revised: \today)}

\bigskip
I'll focus on the big picture to give you a sense of
what we've done and how it fits together.
The key sections are the first one, random variables,
and the last one, asset pricing.
The sections in between combine practice with random variables
with insight into various approaches to asset pricing.


\subsection*{Random variables}

Random variables.  They're the input to everything we do.
Formally, we start with a state $z$, associated probabilities $p(z)$,
and random variables $x(z)$.

Generating functions.
We used the moment generating function $h(s) = E (e^{sx})$
and cumulant generating function $k(s) = \log h(s)$ to
generate moments and cumulants, resp.
The latter are especially useful, since ``high-order'' cumulants
(cumulants $\kappa_j$ for $j \geq 3$)
are zero for the normal distribution.
Measures of skewness and excess kurtosis are therefore
indications that the distribution is something other than normal.

Common distributions:  Bernoulli, Poisson, normal, exponential.
The normal and Poisson are commonly used in option pricing.


\subsection*{Risk and risk aversion}

We'll use expected utility with a power function most of the time,
but the more general treatment sets up
the possibility of more general preferences.
We may come back to that at the end of the course
if we have time.
Also important is the role of high-order moments and cumulants,
which we'll see again when we look at the equity premium.


\subsection*{Consumption, saving, and portfolio choice}

Asset pricing starts with portfolio choice.
We worked our way up to a two-period example
with dates 0 and 1 and a number of different states $z$ at date 1.
Clean solutions are a rarity, but we saw that a theoretical agent
holds less of the risky asset when we increase her risk aversion.
We also saw the first version of the equation,
\begin{eqnarray}
    \sum_z p(z) \{ \beta u'[c_1(z)]/u'(c_0) \} r^j(z) &=&
        E(m r^j) \;\;=\;\; 1,
        \label{eq:Emr}
\end{eqnarray}
for all traded assets $j$.
Here it's a first-order condition for the agent:
given returns $r^j$, choose consumption and portfolios.
Later on it will reappear as an asset pricing relation:
given $m$, find the price and return of an asset.


\subsection*{Two-period economies}

General equilibrium models are a basic tool of economics.
The representative agent version is a good starting point
for macroeconomics and finance.
It's the predominant model in macro-finance,
even as people work on extensions with more complex preferences
or multiple agents.

{\it Arrow securities\/} are an important concept here:
claims to one unit of the good in a specific state $z$ at date 1.
We denote their prices, in units of the date-0 good, by
$q(z)$, which we call the {\it state prices\/}.
The first-order condition of a representative agent implies
\begin{eqnarray}
    q(z) &=& p(z) \beta u'[c_1(z)]/u'(c_0) ,
\end{eqnarray}
a short step from (\ref{eq:Emr}).
If we have as many assets as states, we can
use them to construct state prices.
The possibility of using the mysterious
state prices in practical situations is a
significant insight.


\subsection*{Asset pricing}

We change perspectives here.
In the portfolio choice problem,
asset prices and returns are given,
agents simply decide how much of each asset to buy.
In the general equilibrium problem,
both prices and quantities are endogenous:
they come out of the model.
Both paths lead to (\ref{eq:Emr}), but the interpretation is different.
Here we don't worry about where (\ref{eq:Emr}) comes from.
We simply say that if we know $m$, we can compute asset prices and returns.

%We have three versions, each of which gives us a different perspective.
The first result, which I regard as one of the great accomplishments of modern finance,
is that we can price assets with Arrow securities if the economy is ``arbitrage-free.''
Again, apologies for all the $q$'s.
If asset $j$ has dividends $d^j(z)$, its price satisfies
\begin{eqnarray}
    q^j &=& \sum_z q(z) d^j(z) .
    \label{eq:qj-state-price}
\end{eqnarray}
The theorem says we can always find positive state prices $q(z)$ that satisfy this
equation for every traded asset $j$.
We haven't said anything about what the state prices are
or where they come from,
only that there must be some.

We have two versions of this result that we'll use repeatedly.
The first version is based on a {\it pricing kernel\/} $m$,
defined implicitly by $ q(z) = p(z) m(z)$.
The pricing relation (\ref{eq:qj-state-price}) becomes
\begin{eqnarray}
    q^j &=& \sum_z p(z) m(z) d^j(z) \;\;=\;\; E (m d^j) .
    \label{eq:q=Emd}
\end{eqnarray}
This turns into (\ref{eq:Emr})
if we divide by $q^j$ and
note that the return is $r^j(z) = d^j(z)/q^j$.
We can breathe some life into $m$ by equating it to the
marginal rate of substitution of a representative agent,
as in (\ref{eq:Emr}).
The second version is based on
{\it risk-neutral probabilities\/} $p^*$,
defined implicitly by $ q(z) = p(z) m(z) = q^1 p^*(z)$.
Here
\begin{eqnarray*}
    q^1 &=& \sum_z p(z) m(z) \;\;=\;\; E(m)
\end{eqnarray*}
is the price of a one-period riskfree bond.
The pricing relation (\ref{eq:qj-state-price}) turns into
\begin{eqnarray*}
    q^j &=& q^1 \sum_z p^*(z) d^j(z)  \;\;=\;\; q^1 E^* (d^j) ,
\end{eqnarray*}
where $E^*$ means the expectation computed from the risk-neutral probabilities.


\subsection*{Risk and return}

We've barely touched on this, but the pricing kernel
version (\ref{eq:q=Emd}) contains an explanation for why some assets have
higher expected returns than others.
There are two levels to this explanation:
the covariance of the return with $m$ and the
connection between $m$ and consumption.

To keep things simple, let's scale the units of
every asset so that they have the same expected dividend $E(d^j)$.
Then any difference in expected return must come from the price:
$ E(r^j) = E(d^j)/q^j$.
The price of an asset follows from
\begin{eqnarray*}
    q^j &=& E (m d^j)
            \;\;=\;\; E(m) E(d^j) + \mbox{Cov}(m,d^j) .
\end{eqnarray*}
The first term is the same for all assets,
but the second need not be.
Evidently assets whose dividends have the greatest negative covariance
with $m$ have the lowest prices, hence the highest expected returns.
Put differently:  assets are penalized, in terms of price and return,
for having high payoffs in states when $m$ is low.

The second level tells us what states those are.
It stems from the connection
in representative agent models between
the pricing kernel and consumption:
\begin{eqnarray*}
    m(z) &=& \beta u'[c_1(z)]/u'(c_0) .
\end{eqnarray*}
Here we see that states with high $c_1(z)$ have low $m(z)$.
Why?  Because marginal utility is decreasing.
It says, in words, that payoffs when consumption is high
are less valuable than payoffs when consumption is low.
That's a good rationale for high average returns on assets like equity,
whose payoffs are highest in good times,
when consumption is high and the value of an additional
unit of consumption is low.


\vfill \centerline{\it \copyright \ \number\year \
NYU Stern School of Business}
\end{document}
