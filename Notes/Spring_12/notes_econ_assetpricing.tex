\documentclass[11pt]{article}

\oddsidemargin=0.25truein \evensidemargin=0.25truein
\topmargin=-0.5truein \textwidth=6.0truein \textheight=8.75truein

%\usepackage{graphicx}
\usepackage{verbatim}
%\usepackage{booktabs}
\usepackage{comment}
\usepackage[dvipdfm]{hyperref}
\urlstyle{rm}   % change fonts for url's (from Chad Jones)
\hypersetup{
    colorlinks=true,        % kills boxes
    allcolors=blue,
    pdfsubject={ECON-UB233, Macroeconomic foundations for asset pricing},
    pdfauthor={Dave Backus @ NYU},
    pdfstartview={FitH},
    pdfpagemode={UseNone},
%    pdfnewwindow=true,      % links in new window
%    linkcolor=blue,         % color of internal links
%    citecolor=blue,         % color of links to bibliography
%    filecolor=blue,         % color of file links
%    urlcolor=blue           % color of external links
% see:  http://www.tug.org/applications/hyperref/manual.html
}

\renewcommand{\thefootnote}{\fnsymbol{footnote}}

% document starts here
\begin{document}
\parskip=\bigskipamount
\parindent=0.0in
\thispagestyle{empty}
{\large ECON-UB 233 \hfill Dave Backus @ NYU}

\bigskip\bigskip
\centerline{\Large \bf Approaches to Two-Period Asset Pricing}
\centerline{(Started: July 19, 2011; Revised: \today)}

\bigskip
Darrell Duffie notes that the 1970s were a ``golden age''
for dynamic asset pricing theory,
but suggests that the period since has been ``a mopping-up operation''
(Duffie, {\it Dynamic Asset Pricing Theory\/}, Preface).
That takes the glamor out of the subject,
but it's true, the basic theory has been worked out.
The same is true of calculus, but that doesn't make it any less useful.
Our goal here is to summarize the concepts and results,
including the remarkable no-arbitrage result and
the mysterious ``risk-neutral'' probabilities.


\subsection*{Overview}

The setting is the two-period event tree:
there are two dates, 0 and 1,
and in period 1 a particular state $z$ occurs with probability $p(z)$.
A particular asset $j$ is a claim to a date-1 ``dividend'' $d^j(z$),
a function of the state  $z$ (a random variable, in other words).
If the date-0 price of this asset is $q^j$,
the gross return is $r^j(z) = d^j(z) / q^j$.

[Draw event tree.]

The basic idea behind modern asset pricing theory is to
derive prices of assets from state prices --- prices of Arrow securities.
Why we can do that is a subtle issue that we'll address later,
but for now note that if we know the dividends and state prices,
the asset's price is
\begin{eqnarray}
    q^j   &=& \sum_z q(z) d^j(z) .
    \label{eq:state-prices}
\end{eqnarray}
We'll refer to (\ref{eq:state-prices}) and its successors
as the {\it pricing relation\/}.
The pricing relation makes life really easy.
We break an asset down into its component parts,
price them one at a time, then add them together.
In this respect, the basic theory of asset pricing is very simple,
everything else is just decoration.

[Comment:  bad notation, too many $q$'s.]

We'll look at two other versions of the same equation ---
same theory, just different notation.
One is based on a {\it pricing kernel\/}
or {\it stochastic discount factor\/} $m$,
defined implicitly  by $ q(z) = p(z) m(z) $.
Equation (\ref{eq:state-prices}) becomes
\begin{eqnarray}
    q^j   &=& \sum_z p(z) m(z) d^j(z)  \;\;=\;\; E (m d^j) .
    \label{eq:pricing-kernel}
\end{eqnarray}
Dividing by $q^j$ gives us the familiar $E(mr^j) = 1$.
Here $m$ plays the same role the market plays in the CAPM:
the return on an asset depends on the the relation between the return and the pricing kernel.
More on that shortly.

Another is based on so-called {\it risk-neutral probabilities\/},
defined implicitly by $ p(z) m(z) = q^1 p^*(z) $,
where $q^1 = \sum_z p(z) m(z) = E(m)$ is the price of a one-period riskfree bond.
Since
\begin{eqnarray*}
    p^*(z) &=& p(z) m(z) /q^1 ,
\end{eqnarray*}
the $p^*$'s are positive if the $m$'s are, and they sum to one,
so they are legitimate probabilities.
The pricing relation (\ref{eq:pricing-kernel}) becomes
\begin{eqnarray}
    q^j   &=& q^1 \sum_z p^*(z) d^j(z)  \;\;=\;\; q^1 E^* (d^j) ,
    \label{eq:risk-neutral-probs}
\end{eqnarray}
where $E^*$ means the expectation based on the $p^*$'s.
In (\ref{eq:pricing-kernel}), the pricing kernel performed two roles:
discounting and risk adjustment.
Here the same roles are divided between $q^1$ (discounting)
and $p^*$ (risk adjustment).
If it were up to me, I'd call them {\it risk-adjusted probabilities\/},
but I haven't had much success selling that to others.


The idea in each of these cases is to go from
state prices (or pricing kernel or risk-neutral probabilities)
to prices of specific assets.
We say we ``price'' these assets.
The perspective is the reverse of the portfolio choice problem,
where the prices of assets are given and we choose consumption,
but the connection between the two is the same.


\subsection*{Approach 1:  State prices}

First we look at the remarkable no-arbitrage result
connecting arbitrage and state prices.
Versions of this were developed by
Ross and Harrison-Kreps in the late 1970s.

We'll start with a definition of arbitrage opportunities
and what it means for an economy to be free of them.
Let $q$ be a column vector of asset prices,
with each element corresponding to a different asset.
And let $D$ be a matrix whose $j$th column is the dividend
vector $d^j$ for the $j$ asset,
with each row corresponding to a specific state.
If $J$ is the number of assets and $Z$ the number of states,
then $D$ is $Z$ by $J$.
The asset structure of an economy is thus summarized by $D$.

A portfolio is a set of quantities $a$,
which element $a_j$ the number of shares of asset $j$.
An arbitrage is a portfolio that generates positive dividends ($D a > 0$)
at nonpositive cost ($q \cdot a \leq 0$).
The first condition means here that all the elements
of $Da$ are greater than or equal to zero and at least one is strictly
positive.
We say $(q,D)$ is ``arbitrage free'' if there are no portfolios
that allow arbitrage.

Now what about state prices?
We're looking for prices $q(z)$, with one element for each state,
such that the prices of assets equal the sums of the values
of their state-specific dividends.
In matrix terms:
\begin{eqnarray}
    q &=&  D^\top q(z),
    \label{eq:state-prices-matrix}
\end{eqnarray}
which is equation (\ref{eq:state-prices}) rewritten in matrix form.

Here's the result:

{\it Theorem.\/}  There exist positive state prices consistent with (\ref{eq:state-prices})
if and only if the economy is arbitrage free.

What's remarkable about this is that it's so general.
Absence of arbitrage is ruled out in general equilibrium models,
because it's inconsistent with equilibrium.
But here we get state prices with a lot less structure than that.

The theorem tells us that we can always find state prices that value assets
correctly, but doesn't tell us much more.
And note that it goes both ways:
if we value assets with some set of state prices,
the result is free of arbitrage opportunities.
We won't prove this, but some examples should give you the idea.

{\it Examples.\/}
Which of these price-dividend combinations are arbitrage free?
What are the state prices?
\begin{eqnarray*}
  (a) \;\;  q &=& \left[
            \begin{array}{c}
             1 \\ 2
            \end{array}
          \right], \;\;\;
        D \;\;=\;\;
            \left[
            \begin{array}{cc}
             1 & 0 \\ 0 & 1
            \end{array}
          \right]    \hspace*{0.5in}  \\
  (b) \;\;  q &=& \left[
            \begin{array}{c}
             2 \\ 2
            \end{array}
          \right], \;\;\;
        D \;\;=\;\;
            \left[
            \begin{array}{cc}
             1 & 1 \\ 1 & 2
            \end{array}
          \right]      \\
  (c) \;\;    q &=& \left[
            \begin{array}{c}
             2 \\ 3
            \end{array}
          \right], \;\;\;
        D \;\;=\;\;
            \left[
            \begin{array}{cc}
             1 & 1 \\ 1 & 2
            \end{array}
          \right]      \\
  (d) \;\;    q &=& \left[
            \begin{array}{c}
             3 \\ 6
            \end{array}
          \right], \;\;\;
        D \;\;=\;\;
            \left[
            \begin{array}{cc}
             1 & 1 \\ 1 & 2 \\ 1 & 3
            \end{array}
          \right]    .
\end{eqnarray*}
You should stare at this and think about what you see.
In (a) we have Arrow securities,
so their prices are the state prices.
In (b) you might notice that the second asset dominates the first:
same price, but the second has greater dividends in state 2
(and the same in state 1).
This can't work.  If you shorted asset 1 and used the proceeds to invest
in asset 2, you get something for nothing.
More  formally, let $a^\top = (-1,1)$ and see what you get.
What about the state prices?
Since $D$ is square, we can solve (\ref{eq:state-prices-matrix})
for the state prices.
We get $q(z)^\top = (1, 0)$, so they're not strictly positive.
In (c) we have the same dividends, but a higher price of asset 2,
which kills off the arbitrage opportunity.
In (d) we have fewer assets than states, but the same logic applies.
You won't be able to find an arbitrage.

[Write out the equations.]


\subsection*{Complete and incomplete markets}

Talk about relation between asset prices and state prices...  



\subsection*{Approach 2:  Pricing kernels and representative agents}

We do two things here:  change notation from state prices to the pricing kernel,
and describe how the mrs of a representative agent macro model gives you a
pricing kernel that's connected to something we know about.
There's more structure than in the previous section,
which gives us sharper insights into how risk is priced.

When we define the pricing kernel $m(z)$ from state prices by $q(z) = p(z) m(z)$,
we introduce probabilities into the pricing equation.
That's helpful, because we often want to know about things like the mean
return, which depends (obviously) on the probabilities of various outcomes.
Consider the pricing of the one-period riskfree asset.
As we've seen before, the price is $q^1 = E(m)$ and the return is $r^1 = 1/E(m)$.

Now consider the pricing of an arbitrary asset $j$.
If $m$ is constant, the price of the asset depends only on its expected dividend:
\begin{eqnarray*}
    q^j &=& E (m d^j) \;\;=\;\; q^1 E(d^j) .
\end{eqnarray*}
[Here we're using equation (\ref{eq:pricing-kernel}).]
The return is therefore
$r^j = d^j/q^j$ and the expected return $E(r^j) = E(d^j)/[q^1 E(d^j)] = r^1$.
Thus there's no risk premium, the return is the same as the one-period
riskfree rate.
In general, the connection depends on the relation between
the two random variables, $m$ and $d$:
whether $E(md)$ is greater or less than $E(m) E(d)$.
If it's greater, the price is higher and the expected return lower
than the riskfree rate.
[Can you show this?]

The representative agent framework puts some economic content into
$m$:  it's the marginal rate of substitution of the representative agent.
Given preferences, we could presumably infer its properties
from those of aggregate consumption.
It also gives us useful qualitative information.
Since marginal utility is decreasing, $m$ is lower in
good states --- states when consumption is high ---
than in bad states.
An asset, like an equity index, that that pays off most in good time will
therefore have a lower price [$E(md) < E(m) E(d)$]
and a higher mean return.

% ?? derive covariance?

There's a neat example that shows how a representative agent's portfolio
choice generates state prices automatically,
even when the agent trades a limited set of assets.
(I got it from Sargent, not sure where he got it.)
In this respect it has the same ``extension'' flavor that the no-arbitrage
theorem has, in which the prices of a small number of assets
is extended to a potentially larger number of states.

The portfolio choice problem here consists of the utility function
\begin{eqnarray*}
    u(c_0) + \beta \sum_z p(z) u[c_1(z)]
\end{eqnarray*}
and the budget constraints
\begin{eqnarray*}
    c_0 + \sum_j a^j q^j &=& y_0 \\
    c_1(z)  &=& y_1(z) + \sum_j a^j d^j(z), \;\; \mbox{ for all } z.
\end{eqnarray*}
We solve it with Lagrangean methods using $q_0$ and $q_1(z)$
as multipliers.
The Lagrangean is
\begin{eqnarray*}
    \mathcal{L} &=& u(c_0) + \beta \sum_z p(z) u[c_1(z)] \\
            && + q_0 [ y_0 - c_0 - \sum_j a^j q^j ]
               + \; \sum_z q_1(z) [  y_1(z) + \sum_j a^j d^j(z) -  c_1(z) ] .
\end{eqnarray*}
The first-order conditions are
\begin{eqnarray*}
    c_0: &&     0 \;\;=\;\; u'(c_0) - q_0   \\
    c_1(z): &&  0 \;\;=\;\; \beta p(z) u'[c_1(z)] - q_1(z)  \\
    a^j:  &&    0 \;\;=\;\; - q_0 q^j + \sum_z q_1(z) d^j(z) .
\end{eqnarray*}
The first two give us the connection between state prices
and the mrs cum pricing kernel:
\begin{eqnarray*}
    q(z) &=& q_1(z) / q_0
        \;\;=\;\; p(z) \frac{\beta u'[c_1(z)]}{u'(c_0)}
        \;\;=\;\; p(z) m(z) .
\end{eqnarray*}
That is:  the problem identifies the state prices from the Lagrange multipliers.
The third equation gives us the pricing relation:
\begin{eqnarray*}
    q^j &=& \sum_z [q_1(z)/q_0] d^j(z)
        \;\;=\;\; \sum_z q(z) d^j(z) .
\end{eqnarray*}
Thus we have demonstrated that these are indeed state prices that satisfy the theorem.
We've also shown that we can recover the state prices from the mrs
of the agent.

[Examples:  in principle we should put all of our favorite
distributions to work:  Bernoulli, Poisson, normal.
We'll spread them out across sections.]


{\it Example.\/}
In practice, we typically specify a distribution for consumption
and use it to derive prices of assets.
It helps to be specific.
Suppose the state $z \sim \mathcal{N}(\kappa_1,\kappa_2)$.
(I know, you're getting tired of this one, but first things first.)
Let $y_1(z)/y_0 = e^z$, so that $z = \log (y_1/y_0) = \log y_1 - \log y_0$.
Assume also that we have power utility:  $u'(c) = c^{-\alpha}$.
Now some questions, with answers:
\begin{itemize}
\item What is the pricing kernel?  Its distribution?
The pricing kernel is
\begin{eqnarray*}
    \log m(z) &=& \log \beta - \alpha (\log y_1(z) - \log y_0)
            \;\;\sim \;\; \mathcal{N}(\log \beta - \alpha \kappa_1, \alpha^2 \kappa_2) .
\end{eqnarray*}
\item What are the price and return of a one-period (riskfree) bond?
They solve
\begin{eqnarray*}
    q^1 &=& E(m) \;\;=\;\; \beta e^{ - \alpha \kappa_1 + \alpha^2 \kappa_2/2 } \\
    r^1 &=& 1/q^1 \;\;=\;\; \beta^{-1} e^{\alpha \kappa_1 - \alpha^2 \kappa_2/2 } .
\end{eqnarray*}
\item What is the price of an asset (``equity'') with dividend
$d^e(z) = y_1(z)^\lambda$ for some
arbitrary value of $\lambda$.
This is a useful trick, since it allows us to vary the sensitivity of the dividend
to the endowment without killing off our convenient loglinear structure.
The price of this ``equity'' is
\begin{eqnarray*}
    q^e &=& E(md) \;\;=\;\; \beta e^{ (\lambda- \alpha) \kappa_1
    + (\lambda-\alpha)^2 \kappa_2/2 } .
\end{eqnarray*}
Note that when $\lambda = 0$ we get the one-period bond's dividend and price.
\item What is the return on equity?  The mean return?
The return is $r^e(z) = d(z)/q^e$, which satisfies
\begin{eqnarray*}
    \log r^e(z) &=& \lambda \log y_1(z) - \log q^e
            \;\;\sim \;\; \mathcal{N}
            (-\log \beta + \alpha \kappa_1 - (\lambda-\alpha)^2 \kappa_2/2, \lambda^2 \kappa_2) .
\end{eqnarray*}
(This takes patience, just stick with it.  Or use Matlab for the substitutions.)
The mean return is therefore
\begin{eqnarray*}
    E (r^e) &=& \beta^{-1} e^{\alpha \kappa_1 + [\lambda^2 - (\lambda-\alpha)^2] \kappa_2/2 } .
\end{eqnarray*}

\item Is there a risk premium on equity?
The standard definition of a risk premium is a difference in expected return:  $E(r^e-r^1)$.
The $\beta$ and $\kappa_1$ terms are the same, so we have (thankfully)
no risk premium when $\kappa_2 = 0$.
If we compare the $\kappa_2$ terms, we see that the risk premium is positive if
\begin{eqnarray*}
    0 &<& [\lambda^2 - (\lambda-\alpha)^2] + \alpha^2 \;\;=\;\; 2 \alpha \lambda .
\end{eqnarray*}
So the risk premium is positive and larger if risk aversion is larger (larger $\alpha$)
or the dividend is more sensitive to the endowment (larger $\lambda$).


\end{itemize}



\subsection*{Approach 3:  Risk-neutral probabilities}

This is just a change in notation, but it's a common one in finance.
As we noted, the risk-neutral probabilities are defined implicitly by
\begin{eqnarray*}
    q^1 p^*(z) &=& p(z) m(z) .
\end{eqnarray*}
They're probabilities in the sense that they're positive and sum to one,
but they incorporate risk pricing via $m$.
If $m$ is constant, then $p^*(z) = p(z)$
and all assets have the same expected return:  there's no adjustment for risk.

With risk-neutral probabilities,
the pricing relation becomes
\begin{eqnarray*}
    q^j &=& q^1 E^* (d^j).
\end{eqnarray*}
Dividing by $q^j$ and substituting $r^1 = 1/q^1$, we have
\begin{eqnarray*}
    r^1 &=& E^* (r^j).
\end{eqnarray*}
That is, under the risk-neutral probabilities all assets have the same expected return.
How does this work for assets with higher expected returns and positive risk premiums?
Evidently $p^*$ puts more weight on bad outcomes and less weight on good outcomes than $p$.


{\it Example.\/}
Work through the example of the previous section.
We have $ z = \log y_1 - \log y_0 \sim \mathcal{N}(\kappa_1, \kappa_2) $.
The pdf is therefore
\begin{eqnarray*}
    p(z) &=& (2 \pi \kappa_2)^{-1/2} \exp[ - (z-\kappa_1)^2/(2\kappa_2)]
\end{eqnarray*}
and the pricing kernel is $m(z) = \beta \exp(-\alpha z)$.
The one-period riskfree bond price is therefore
$ q^1 = E(m) = \beta \exp(-\alpha \kappa_1 + \alpha^2 \kappa_2/2)$,
as we saw earlier.
That gives us the risk-neutral pdf
\begin{eqnarray*}
    p^*(z)  &=& p(z) m (z) /q^1 \\
            &=& (2 \pi \kappa_2)^{-1/2} \exp[ - (z-\kappa_1)^2/(2\kappa_2)]
                        \beta \exp(-\alpha z)/q^1 \\
            &=& (2 \pi \kappa_2)^{-1/2} \exp[ - (z-\kappa_1+\alpha \kappa_2)^2/(2\kappa_2)] .
\end{eqnarray*}
That is, $z$'s true distribution is $\mathcal{N}(\kappa_1,\kappa_2)$,
but its risk-neutral distribution is $\mathcal{N}(\kappa_1-\alpha \kappa_2,\kappa_2)$:
we shift the distribution to the left (more pessimistic) to account
for risk.  How much depends on risk ($\kappa_2$) and risk aversion ($\alpha$).


\subsection*{More fun with cumulants}

What we just did was complicated, but generating functions give us
a compact representation of the relation between true and risk-neutral
probabilities for power utility.
(The reason for power utility is that it's power form matches up cleanly
with moment generating functions.)

Suppose, as in our examples,
that the state is $ z = \log y_1 - \log y_0$.
The cumulant generating function  of $z$ is
\begin{eqnarray*}
    k(s) &=& \log E (e^{sz}) .
\end{eqnarray*}
The log of the one-period bond price is
\begin{eqnarray*}
    \log q^1  &=& \log E (\beta e^{-\alpha z})
            \;\;=\;\; \log \beta - k(-\alpha).
\end{eqnarray*}
The cgf of the risk-neutral distribution is
\begin{eqnarray*}
    k^*(s) &=& \log [E (m e^{sz})/q^1]
        \;\;=\;\;   \log E \left( \beta e^{-\alpha z}e^{sz} \right) - \log q^1
        \;\;=\;\;  k(s-\alpha) - k(-\alpha)  .
\end{eqnarray*}
This is a thing of beauty, a wonderfully compact
summary of how true and risk-neutral distributions are connected.
It's also useful, as we now show.

{\it Example.\/}
We'll redo the calculation of the previous section:
the risk-neutral distribution of $z = \log y_1 - \log y_0$
when $z \sim \mathcal{N}(\kappa_1,\kappa_2)$.
We know that $k(s) = s \kappa_1 + s^2 \kappa_2/2$.
Therefore
\begin{eqnarray*}
    k^*(s) &=& [(s-\alpha) \kappa_1 + (s-\alpha)^2 \kappa_2/2]
            - [-\alpha \kappa_1 + \alpha^2 \kappa_2/2]  \\
           &=& s(\kappa_1 - \alpha \kappa_2) + s^2 \kappa_2/2 ,
\end{eqnarray*}
the same answer we had before.
Even better, you could have Matlab do the substitution.



\subsection*{Bottom line}

The no-arbitrage theorem tells us that we can always value assets with state prices.
State prices, pricing kernels, and risk-neutral probabilities all represent
the same idea.



\vfill \centerline{\it \copyright \ \number\year \
NYU Stern School of Business}
\end{document}
