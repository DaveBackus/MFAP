\documentclass[11pt]{article}

\oddsidemargin=0.25truein \evensidemargin=0.25truein
\topmargin=-0.5truein \textwidth=6.0truein \textheight=8.75truein

%\usepackage{graphicx}
\usepackage{verbatim}
\usepackage{booktabs}
\usepackage{comment}
\usepackage[dvipdfm]{hyperref}
\urlstyle{rm}   % change fonts for url's (from Chad Jones)
\hypersetup{
    colorlinks=true,        % kills boxes
    allcolors=blue,
    pdfsubject={ECON-UB233, Macroeconomic foundations for asset pricing},
    pdfauthor={Dave Backus @ NYU},
    pdfstartview={FitH},
    pdfpagemode={UseNone},
%    pdfnewwindow=true,      % links in new window
%    linkcolor=blue,         % color of internal links
%    citecolor=blue,         % color of links to bibliography
%    filecolor=blue,         % color of file links
%    urlcolor=blue           % color of external links
% see:  http://www.tug.org/applications/hyperref/manual.html
}

\renewcommand{\thefootnote}{\fnsymbol{footnote}}

% document starts here
\begin{document}
\parskip=\bigskipamount
\parindent=0.0in
\thispagestyle{empty}
{\large ECON-UB 233 \hfill Dave Backus @ NYU}

\bigskip\bigskip
\centerline{\Large \bf Introduction to Bond Pricing}
\centerline{(Started: April 1, 2012; Revised: \today)}

\bigskip
We define bond prices and related objects
and describe the connection between bond prices and the
pricing kernel in arbitrage-free economies.
The idea, first, is to show the properties of the pricing kernel
determine those of bond prices and, second,
to use observed properties of bond prices
(mean, standard deviation, autocorrelation)
to infer parameter values for some simple pricing kernels.
The pricing kernel is, so to speak, reverse engineered
from bond prices rather than built up from (say)
the marginal rate of substitution of a representative agent.
We leave for later the question of how to reconcile
the two approaches.


\subsection*{Bond prices, yields, and forward rates}

Bond prices:
$q^n_t$ is the price at date $t$ of one unit
at date $t+n$.
One unit of what?
For now, dollars, but we can reconsider units later.
Some people refer to these prices as {\it discount factors\/}
or prices of {\it zeros\/}, meaning zero-coupon bonds.

What about coupons?  Same thing, but apply appropriate price
to each payment.

{\it Yields\/} are a convention for reporting prices.
The continuously-compounded yield is connected to the price by
\begin{eqnarray}
    q^n_t &=&  \exp(- n y^n_t)
        \;\;\Leftrightarrow\;\;
        y^n_t \;\;=\;\; - n^{-1} \log q^n_t .
        \label{eq:yields}
\end{eqnarray}
It's the average rate of discount over the interval $[t, t+n]$.
Sometimes it's more useful to apply a different
rate to each period:
\begin{eqnarray}
    q^n_t &=&  \exp[- (f^0_t + f^1_t + \cdots + f^{n-1}_t)]
        \;\;\Leftrightarrow\;\;
        f^n_t \;\;=\;\;  \log q^n_t - \log q^{n+1}_t .
        \label{eq:forwards}
\end{eqnarray}
We refer to $f^j_t$ as the $j$-period ahead {\it forward rate\/}.
Together we have
\begin{eqnarray*}
    y^n_t &=&  n^{-1} \left( f^0_t + f^1_t + \cdots + f^{n-1}_t \right).
\end{eqnarray*}
In words:  yields are averages of forward rates.

Sometimes for convenience we'll refer to the {\it short rate\/}
$f^0_t = y^1_t$.
This is continuously compounded, not the same as the return on
a one-period bond (we'll get to returns shortly).

[Draw time line, show where yields and forwards go...]

[Draw $-\log q^n$ v. $n$, show yields and forwards...]

Evidently we can report bond prices $q^n_t$ (one number for each maturity $n$),
yields $y^n_t$, or forward rates $f^n_t$.
Once we know one, we can compute the others.
I think forwards are mathematically simpler,
but yields are more common.


\subsection*{Pricing kernels and bond prices}

The simplicity and elegance of bond pricing comes
from the lack of an uncertain dividend.
Since the cash flows are known ---
that's why we call the subject ``fixed income'' ---
prices depend only on the pricing kernel, as we'll see.
To be clear:  the pricing kernel alone determines
bond prices.  Period.


We price assets as we did before:  using the no-arbitrage theorem.
In a dynamic setting, this is
\begin{eqnarray}
    E_t \left( m_{t+1} r_{t+1} \right) &=& 1
    \label{eq:E(mr)=1}
\end{eqnarray}
for the (gross) return $r$ on any traded security.
The expectation $E_t$ means conditional on the state at date $t$.
That is:  there exists a positive pricing kernel $m>0$
that prices assets at every date and state.

[Note units:  $m$ has inverse units to $r$.
If $r_{t+1}$ is dollars tomorrow for dollars today,
then $m_{t+1}$ has units of 1/dollars tomorrow
for 1/dollars today.
That way the units cancel and we're left with the unit-free ``1.''
Why bring this up?  It'll come up later.]

What are the {\it bond returns\/} we plug into this pricing relation?
The relevant returns are sometimes called {\it holding period returns\/}
for a holding period of one.
If we buy an $n+1$-period bond at $t$ and hold it for one period,
we're left with an $n$-period bond at $t+1$.
Our (gross) return is
\begin{eqnarray}
        r^{n+1}_{t+1} &=&  q^n_{t+1}/q^{n+1}_t .
        \label{eq:bond-returns}
\end{eqnarray}
Same object for bonds as the returns we've computed on
other assets.


Now put this to work.  If we substitute the return (\ref{eq:bond-returns})
into the pricing relation (\ref{eq:E(mr)=1}) we have
\begin{eqnarray}
    q^{n+1}_t &=& E_t \left( m_{t+1} q^n_{t+1} \right) ,
    \label{eq:bond-recursion}
\end{eqnarray}
which we compute starting with $q^0_t = 1$ (one now costs one).
The first one is what we've been doing all along:
$ q^1_t = E_t (m_{t+1}) $.
Long-maturity bonds follows from the same logic repeated:
given the price of an $n$-period bond,
find the price of an $n+1$-period bond from (\ref{eq:bond-recursion}).

Optional digression.
We've done this recursively, but you can imagine doing it ``all at once.''
The price of a two-period bond, for example, can be written
\begin{eqnarray*}
    q^{2}_t &=& E_t \left( m_{t+1} q^1_{t+1} \right) \\
            &=& E_t \left( m_{t+1} E_{t+1} [m_{t+2}] \right) \\
            &=& E_t \left( m_{t+1} m_{t+2} \right) .
\end{eqnarray*}
The last line follows from the law of iterated expectations.
Repeating the same logic, we find
\begin{eqnarray*}
    q^{n}_t  &=& E_t \left( m_{t+1} m_{t+2} \cdots m_{t+n} \right) .
\end{eqnarray*}
If this seems overly mysterious, just leave it alone,
we won't be using it.
It shows, though, how dynamics in $m$ might affect bond prices.



\subsection*{The iid pricing kernel}

Before we get started, what would happen if $\{ m_t \}$ were iid?
Let's see how that works.
The one-period bond price is
\begin{eqnarray*}
    q^{1}_t &=& E_t \left( m_{t+1} \right) .
\end{eqnarray*}
Since $m$ is iid, this is just a number $q^1$,
the same in all date-$t$ states.
[Does this need explaining?]

% ?? this is easier with the product of m's version


Now let's do the same thing again.
Since $q^1 $ is a number, we have
\begin{eqnarray*}
    q^{2}_t &=& E_t \left( m_{t+1} q^1 \right)
            \;\;=\;\;  q^1 E_t \left( m_{t+1}  \right)
            \;\;=\;\;  (q^1)^2 .
\end{eqnarray*}
Using the same logic, we get
\begin{eqnarray*}
    q^{n} &=& (q^1)^n .
\end{eqnarray*}
If we apply their definitions,
we see that bond yields and forward rates are constant over time.
They're also the same at all maturities and equal to each other.
Bottom line:  there's no interest rate risk if $m$ is iid.

The conclusion, evidently, is that we need dynamics in $m$
to generate reasonable bond prices.
But what kind of dynamics?
We'll start simple and add complexity when called for.


\subsection*{The moving average pricing kernel}

If we need dynamics, one idea that may come to mind is to use
our linear dynamic models, specifically the infinite moving average.
Whether it comes to mind or not, that's what we're going to do.

Let us say, then, that the pricing kernel has the form
\begin{eqnarray}
    \log m_t &=& \delta + \sum_{j=0}^\infty a_j w_{t-j} ,
    \label{eq:logm-ma}
\end{eqnarray}
where $\{ w_t \} \sim \mbox{NID}(0,1)$ and
the $a_j$'s are square-summable moving average coefficients.
[What does this mean?]
If $a_j = 0$ for $j \geq 1$ the pricing kernel is iid,
otherwise its dynamics
are governed by the moving average coefficients.
The $\log$ is essential and possible because the theorem
says $m$ is strictly positive.

Let's look at the one-period ``short'' rate and see how it works.
If the pricing kernel is (\ref{eq:logm-ma}),
then next period's pricing kernel is
\begin{eqnarray*}
    \log m_{t+1} &=& \delta + \sum_{j=0}^\infty a_{1+j} w_{t-j} + a_0 w_{t+1} .
\end{eqnarray*}
The first two terms are known at $t$, the last one is not.
Other than the conditioning, we're back in our familiar lognormal world.
The log pricing kernel is normal with conditional mean and variance
\begin{eqnarray*}
    E_t ( \log m_{t+1}) &=& \delta + \sum_{j=0}^\infty a_{1+j} w_{t-j} \\
    \mbox{Var}_t ( \log m_{t+1}) &=& a_0^2 .
\end{eqnarray*}
The usual ``mean plus variance over two'' gives us
\begin{eqnarray*}
    \log q^1_t &=& \log E_t \left(m_{t+1}\right)
            \;\;=\;\; \delta + \sum_{j=0}^\infty a_{1+j} w_{t-j} + a_0^2/2 ,
\end{eqnarray*}
which has the same infinite moving average form as the pricing kernel.
Note, too, the moving average coefficients:
we simply moved them over one place.


Can we express long-term bond prices in similar fashion?
The answer is yes,
but it's best to start out with the solution:

{\it Proposition\/}.  If the pricing kernel is (\ref{eq:logm-ma}),
forward rates are
\begin{eqnarray}
    - f^n_t &=& \delta + A_n^2/2 + \sum_{j=0}^\infty a_{n+1+j} w_{t-j} ,
\end{eqnarray}
where $A_n = a_0 + a_1 + \cdots + a_n = \sum_{j=0}^n a_j $.
Bond prices and yields follow from their definitions.

Now how did we get there?
Suppose bond prices have the form
\begin{eqnarray*}
    \log q^n_t &=& \delta^n + \sum_{j=0}^\infty b^n_{j} w_{t-j}
\end{eqnarray*}
with parameters $\delta^n$ and $\{b^n_j \}$ to be determined.
Since $q^0_t = 1$, we have $\delta^0 =b^0_j = 0$.
Given parameters for some maturity $n$,
we use the pricing relation (\ref{eq:bond-recursion})
to find the parameters of maturity $n+1$.


Here's a typical step.
The pricing relation uses the conditional distribution of
\begin{eqnarray*}
    \log m_{t+1} + \log q^n_{t+1} &=&  (\delta + \delta^n)
                + \sum_{j=0}^\infty (a_{1+j} + b^n_{j+1}) w_{t-j}
                + (a_0 + b^n_0) w_{t+1} .
\end{eqnarray*}
As before the first two terms on the right are known at date $t$
and the last one is not.
The conditional distribution is therefore normal with mean and variance
\begin{eqnarray*}
  E_t (  \log m_{t+1} + \log q^n_{t+1})  &=&  (\delta + \delta^n)
                + \sum_{j=0}^\infty (a_{1+j} + b^n_{j+1}) w_{t-j} \\
  \mbox{Var}_t (  \log m_{t+1} + \log q^n_{t+1})
                &=&  (a_0 + b^n_0)^2  .
\end{eqnarray*}
The usual ``mean plus variance over two gives us
\begin{eqnarray*}
  \log q^{n+1}_t  &=&  (\delta + \delta^n)
                + \sum_{j=0}^\infty (a_{1+j} + b^n_{j+1}) w_{t-j} +
             (a_0 + b^n_0)^2 / 2 \\
                &=& \delta^{n+1} + \sum_{j=0}^\infty b^{n+1}_{j} w_{t-j} .
\end{eqnarray*}
Lining up terms gives us
\begin{eqnarray*}
    \delta^{n+1} &=& \delta + \delta^n + (a_0 + b_0^n)^2/2\\
    b_j^{n+1} &=& b_{j+1}^{n} + a_{j+1} ,
\end{eqnarray*}
two recursions that we need to beat into more useful shape.

The trick is to be patient and work this out step by step.
Let's take the second one.
For $n = 0$,  we have $b^0_j = 0$ all $j$.
For $n=1$, we have
\begin{eqnarray*}
    b^1_j &=& b^0_{j+1} + a_{1+j}
            \;\;=\;\; 0 + a_{1+j}.
\end{eqnarray*}
That's what we saw earlier:  we simply shift the $a_j$'s over one.
For $n=2$, we have
\begin{eqnarray*}
    b^2_j &=& b^1_{j+1} + a_{1+j}
            \;\;=\;\; a_{2+j} + a_{1+j} ,
\end{eqnarray*}
and for $n=3$ we have
\begin{eqnarray*}
    b^3_j &=& b^2_{j+1} + a_{1+j}
            \;\;=\;\; a_{3+j} + a_{2+j} + a_{1+j} .
\end{eqnarray*}
With a little work, we see that
\begin{eqnarray*}
    b^n_j &=& A_{n+j} - A_j  ,
\end{eqnarray*}
and we're done with the $b$'s.

Now the $\delta$'s.
We start again with $n=0$, in which case $\delta^0 = 0$.
We'll need $b^n_0 = A_{n} - A_0 $,
which implies $a_0 + b^n_0 = A_n$.
We find, in order,
\begin{eqnarray*}
    \delta^1 &=& \delta + \delta^0 + a_0^2/2
            \;\;=\;\; \delta + a_0^2/2 \\
    \delta^2 &=& 2 \delta + a_0^2/2 + A_1^2/2 \\
    \delta^3 &=& 3 \delta + a_0^2/2 + A_1^2/2 + A_2^2/2\\
    \delta^n &=& n \delta + A_0^2/2 + A_1^2/2 + \cdots + A_{n-1}^2/2 .
\end{eqnarray*}
that does it for bond prices.

The solution is simpler if we express it in terms of forward rates,
as we did in the proposition.
We can always reconstitute bond prices from (\ref{eq:forwards}).
From the definition,
\begin{eqnarray*}
    - f^n_t &=& \log q^{n+1}_t - \log q^n_t \\
            &=& (\delta^{n+1} - \delta^n) + \sum_{j=0}^\infty
                    (b^{n+1}_j - b^n_j) w_{t-j} \\
            &=&  \delta + A_n^2/2 + \sum_{j=0}^\infty
                    a_{n+1+j} w_{t-j}  ,
\end{eqnarray*}
as stated in the proposition.


Properties.
Here's a list:
\begin{itemize}
\item The iid case revisited.
In the iid case, $a_j = 0$ for $j\geq 0$ and $A_n = A_0$ for all $n$.
As a result, $f^n$ is constant and equal for all $n$.
Ditto yields (averages of forwards).
Otherwise, the $a$'s affect both
the average shape of the forward rate curve
and the dynamics of interest rates.
\item Mean forward rates.
Since the $w$'s have zero mean, mean forward rates are
\begin{eqnarray*}
  -  E( f^n_t)
            &=&  \delta + A_n^2/2 .
\end{eqnarray*}
The only term here that varies with $n$ is the last one,
so the shape of the mean forward rate curve depends on the $a$'s through
the partial sums $A_n$.
The mean forward rate spread is
\begin{eqnarray}
    E( f^n - f^0 )
            &=&  (A_0^2 - A_n^2)/2 .
            \label{eq:vasicek-mean-forwards}
\end{eqnarray}
You might think about how we might generate an upward-sloping mean forward
rate curve.  What does that imply for the $a$'s?

\item Interest rate dynamics.
The dynamics of forward rates also depend on the $a$'s,
but through the coefficients of $w_{t-j}$.
Note that as we increase maturity $n$, we shift the coefficients (increase the subscript).

\item We see, therefore, that the dynamics of the pricing kernel,
reflected by its moving average coefficients,
determines both the shape of the average yield curve
and the dynamics of forward rates.
This interaction between cross-section and time-series properties
is inherent to bond pricing.

\item What happens as we increase maturity $n$?
If the $a$'s go to zero, as they must, then the forward rate becomes constant:
its dependence on the $w$'s goes away.
This is a fairly general result:  that forward rates converge
to a constant at long enough maturities.
\end{itemize}



\subsection*{The Vasicek model}

In the previous section, we found bond prices and forward rates
from the pricing kernel.
Here we do the reverse:  How can we use what we know about forward rates
to estimate the parameters of the pricing kernel?

The idea is to combine time series and cross section information
about interest rates to estimate the parameters of the pricing kernel.
We can use information about the dynamics of interest rates
to infer some of the $a_j$'s:  note, for example, that the short
rate includes $a_1, a_2, \ldots $.
It also allows us to identify the $w_t$'s.
But what about $a_0$?
This we need the cross-section information for.
The mean forward rate spread is given by
(\ref{eq:vasicek-mean-forwards}),
so the partial sums play a central role.


It's helpful to simplify the model first.
Suppose $\log m_t$ is ARMA(1,1).
Then it has the structure:  $a_0$, $a_1$,
and $a_{j+1} = \varphi a_j = \varphi^j a_1$ for $j\geq 1$.
The three parameters $(a_0,a_1,\varphi)$ are enough to nail it down.
The autoregressive parameter $\varphi$ is less than one in absolute value,
otherwise the process isn't square summable.

With this structure, the one-period bond price is an AR(1):
\begin{eqnarray*}
    \log q^1_t &=& \delta + a_0^2/2 + \sum_{j=0}^\infty a_{1+j} w_{t-j}  \\
             &=& \delta + a_0^2/2 + \sum_{j=0}^\infty \varphi^j a_{1} w_{t-j}  \\
             &=& (1-\varphi) (\delta + a_0^2/2) + \varphi \log q^1_{t-1} + a_1 w_t .
\end{eqnarray*}
The short rate $f^0_t = - \log q^1_t$ is the same with the opposite sign:
\begin{eqnarray*}
     f^1_t  &=& - (1-\varphi) (\delta + a_0^2/2) + \varphi f^0_{t-1} - a_1 w_t .
\end{eqnarray*}
It has autocorrelation $\varphi$ and variance $a_1^2/(1-\varphi^2)$.
The mean we're less concerned with, but it's $ - (\delta + a_0^2/2)$.


Now let's estimate parameters from properties of forward rates.
Some ballpark numbers are (we'll discuss this more in class)
%
\begin{center}
\begin{tabular}{lrrr}
\toprule
Forward rate $f^n_t$    &  Mean  &  Std Dev  &  Autocorr \\
\midrule
$f^0_t$                 &  6.683 & 2.703     &  0.959   \\
$f^{12}_t$              &  7.921 & 2.495     &  0.969   \\
$f^{120}_t$             &  8.858 & 1.946     &  0.980   \\
\bottomrule
\end{tabular}
\end{center}
%
The numbers are  computed from monthly data but reported as annual
percentages, something that will show up in our calculations.

Now let's put these numbers to work.
From the autocorrelation, we have $\varphi = 0.959$.
From the standard deviation, we have
\begin{eqnarray*}
    (2.703/1200)^2 &=&  a_1^2/(1-\varphi^2)
            \;\;\Rightarrow\;\; a_1 \;\;=\;\; - 0.645 \times 10^{-3} .
\end{eqnarray*}
With this sign convention, an increase in $w_t$ corresponds to an
increase in the short rate.
The small magnitude stems largely from short rates being small in magnitude.
What about $a_0$?  This comes from the mean forward rate.
If we choose it to match the mean forward premium at maturity $n=120$,
we need:
\begin{eqnarray*}
    (A_0^2 - A_{120}^2)/2 &=& (8.858 - 6.683)/1200 .
\end{eqnarray*}
If we choose $a_0 = 0.124$ we're pretty close.

[See Matlab program.]

What do we have?
(i)~The moving average coefficients start with an initial big one,
then much smaller ones that decay slowly to zero.
(ii)~They flip sign, too:  if you want the mean forward rate curve to slope up,
then $a_0$ and $a_j$ for $j\geq 1$ have to have opposite signs.


\begin{comment}
\subsection*{The expectations hypothesis}

Martingales...


%\subsection*{The Cox-Ingersoll-Ross model}



%\subsection*{Exponential-affine models}
\end{comment}


\vfill \centerline{\it \copyright \ \number\year \
NYU Stern School of Business}



\end{document}
