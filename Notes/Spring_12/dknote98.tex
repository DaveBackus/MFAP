\documentstyle[11pt]{article}

\oddsidemargin=0.25truein
\evensidemargin=0.25truein
\topmargin=0.0truein
\textwidth=6.0truein
\textheight=8.00truein

\newcommand{\var}{\mbox{\em Var\/}}
\newcommand{\cov}{\mbox{\em Cov\/}}
\newcommand{\corr}{\mbox{\em Corr\/}}
\newcommand{\nid}{\mbox{NID}}
\newcommand{\ds}{\displaystyle}

% next three lines for biblio setup
\newcommand{\paperstart}[0]{\par\penalty 1 \hangindent=32pt \hangafter=1}
\newcommand{\paperend}[0]{\penalty -1 \par\vskip 10pt plus 10pt minus 10pt}
%\newcommand{\paperend}[0]{\penalty -1 \par\vskip 15pt plus 25pt minus 10pt}
\newcommand{\paper}[1]{\paperstart #1 \paperend}

% document starts here
\begin{document}
\parskip=\bigskipamount
\parindent=0.0in
\thispagestyle{empty}
\begin{flushright} Wu \& Backus \end{flushright}

\bigskip
\centerline{\Large \bf Notes on Duffie and Kan}
\centerline{(June 17, 1998; revised September 13, 1998)}

\bigskip
The idea is to work through Duffie and Kan's representation
theorem for affine models in our discrete-time notation and setting.
This is an attempt at understanding, not necessarily
an accurate translation.

\section*{Affine Models}

Quick summary of affine models.
All arbitrage-free models obey the pricing relation,
\begin{equation}
        b^{n+1}_t \;=\; E_t \left( m_{t+1} b^n_{t+1} \right)
        \label{eq:foc}
\end{equation}
for some choice of $m$.
An affine model is one for which the solution is log-linear in
a vector $z$ of state variables:
\begin{equation}
        - \log b^{n}_t \;=\; A_n + B_n z_t .
        \label{eq:loglin}
\end{equation}
Our approach bases such models on a law of motion for the state vector,
\begin{equation}
        z_{t+1} \;=\; M(z_t) + V(z_t) \varepsilon_{t+1} ,
        \label{eq:z}
\end{equation}
and a pricing kernel,
\begin{equation}
        - \log m_{t+1} \;=\; N(z_t) + \lambda^\top W(z_t) \varepsilon_{t+1} ,
        \label{eq:m}
\end{equation}
where $\{ \varepsilon_t \} \sim \mbox{NID}(0,I)$.
We say (\ref{eq:z})
is affine if $M$ and $VV^\top$
[resp $N$ and $WW^\top$]
are linear in $z$.
Similarly, (\ref{eq:m}) is affine if
$N$ and $WW^\top$ are linear in $z$.


\section*{Representation Theorem}

{\bf Proposition. }
Consider an arbitrage-free model of the term structure.
Then:
\begin{enumerate}
\item[(a)] If (\ref{eq:z}) and (\ref{eq:m}) are affine with $W(z) = V(z)$,
        then bond prices satisfy (\ref{eq:loglin}).
\item[(b)] If bond prices are affine [ie, they satisfy (\ref{eq:loglin})]
        and the state equation is affine [$M$ and $VV^\top$ are linear in $z$],
        then the pricing kernel satisfies (\ref{eq:m}) with
        $N$, $WW^\top$, and $WV^\top$ linear in $z$.
        (We need invertibility, too.)
%\item[(c)]
\end{enumerate}

{\bf Proof. }
Start with applying the pricing relation.
We get
\begin{equation}
        A_n + B_n^\top z \;=\;
                A_n + B_n^\top M(z) + N(z)
                + \frac{1}{2}
                  \left[ B_n^\top V(z) + \lambda^\top W(z) \right]
                  \left[ B_n^\top V(z) + \lambda^\top W(z) \right]^\top .
        \label{eq:foc-aff}
\end{equation}
Part (a). Clearly this works if $ V(z) = W(z) $.

Part (b).
Take the univariate case.
Equation (\ref{eq:foc-aff}) becomes
\[
        A_n + B_n z \;=\;
                A_n + B_n M(z) + N(z)
                + \frac{1}{2} \left(
                   [B_n V(z)]^2 + [\lambda W(z)]^2
                  2[B_n V(z) \lambda W(z)]   \right) .
\]
Hence,
\[
              N(z)  + [\lambda W(z)]^2/2 +  [B_n \lambda V(z) W(z)]
\]
must be linear in $z$.
The invertibility condition means $B_n = 0 $ for some $n$.
You can get this to work three different ways:
(i)~Vasicek makes $V$ and $W$ independent of $z$.
(ii)~CIR make $V(z)$ proportional to $W(z)$ based on square root terms so that
$W(z)^2$ and $V(z)W(z)$ are linear in $z$.
(iii)~Duffee/Dai suggest making $V(z)$ independent of $z$
and $W(z)$ linear,
with $N(z) + [\lambda W(z)]^2/2 = 0 $ (fortuitous cancellation).
I think you can show these are the only possibilities.


\section*{Variants}


Jumps...


\section*{Appendix: Duffie and Kan}

Quick summary of their environment, using their notation and
equation numbers.
\begin{itemize}
\item Notation:  prices of zeros are $f$, short rate is $R$.
\item Behavior of state variables:
$$
        dX_t  \;=\; \mu(X_t) dt + \sigma(X_t) dW_t
                                        \eqno{(2.3)}
$$
\item Prices of zeros are expected discounted value using the risk neutral measure:
$$
        f(X_t,T-t) \;=\; E \left[ \exp
                \left( - \int_{t}^T R(X_s) ds \right) | X_t \right]
                                        \eqno{(2.4)}
$$
[Comment:  like our foc, but not clear what price of risk.]
A model that satisfies (2.3,2.4) is said to be {\it compatible\/}.

\item We say $(f,\mu,\sigma)$ is {\it exponential affine\/} if
$$
        \log f(x,\tau) \;=\; A(\tau) + B(\tau) x
                                        \eqno{(3.1)}
$$



\end{itemize}


{\bf Proposition. }
Consider a compatible term structure model.
\begin{enumerate}
\item [(a)] If $\mu$, $\sigma \sigma^\top$, and $R$ are affine,
the $f$ is exponential affine.
\item [(b)] If $f$ is exponential affine, then $R$ is affine.
If there e
\end{enumerate}


\section*{References}

\paper{Duffie and Kan, 1996,
        ``A yield-factor model of interest rates,''
        {\it Mathematical Finance\/} 6, 379-406.}


\end{document}



