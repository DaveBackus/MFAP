\documentclass[11pt]{article}

\oddsidemargin=0.25truein \evensidemargin=0.25truein
\topmargin=-0.5truein \textwidth=6.0truein \textheight=8.75truein

%\usepackage{graphicx}
\usepackage{verbatim}
%\usepackage{booktabs}
\usepackage{comment}
\usepackage[dvipdfm]{hyperref}
\urlstyle{rm}   % change fonts for url's (from Chad Jones)
\hypersetup{
    colorlinks=true,        % kills boxes
    allcolors=blue,
    pdfsubject={ECON-UB233, Macroeconomic foundations for asset pricing},
    pdfauthor={Dave Backus @ NYU},
    pdfstartview={FitH},
    pdfpagemode={UseNone},
%    pdfnewwindow=true,      % links in new window
%    linkcolor=blue,         % color of internal links
%    citecolor=blue,         % color of links to bibliography
%    filecolor=blue,         % color of file links
%    urlcolor=blue           % color of external links
% see:  http://www.tug.org/applications/hyperref/manual.html
}

\renewcommand{\thefootnote}{\fnsymbol{footnote}}

% document starts here
\begin{document}
\parskip=\bigskipamount
\parindent=0.0in
\thispagestyle{empty}
{\large ECON-UB 233 \hfill Dave Backus @ NYU}

\bigskip\bigskip
\centerline{\Large \bf Two-Period Macroeconomic Models}
\centerline{(Started: July 19, 2011; Revised: \today)}

\bigskip
We examine, in the same two-period setting we used earlier,
the consumption,
saving, and portfolio decisions of households
and how they interact with the
production and investment decisions of firms.
We simplify each of these things as much as we can
 to see how this might work.
%Later on, we can ask ourselves which of these simplifications
%was essential, and which were not.

\subsection*{Keep it simple}

Einstein is reported to have said:
``Make things as simple as possible, but not simpler.''
It's not clear exactly what this means, or whether Einstein said it,
but I'd like to think he'd agree we should keep things simple.
Economics, in any case, is a blend of realism and simplicity,
with most of the weight on simplicity.
Some people find the apparent lack of realism in economic models puzzling,
but it's often necessary to gain sharp insights.
The idea is to focus on the things that matter,
and simplify or ignore the ones that don't.
The artist Georgia O'Keeffe said something similar:
``Nothing is less real than realism. Details are confusing. It is only by
selection, by elimination, by emphasis that we get to the real meaning of
things.''

Here's another version of the same sentiment,
an exchange between Robert Mundell and banker Alex McLeod.
The paper in question helped Mundell get the Nobel prize,
but McLeod thought it was unrealistic.
As a banker, he knew how things worked in the real world.
%(We have common friends, and I've been assured he's a reasonable guy.)
I've taken some liberty with the quotations, but the exchange went something like this:

McLeod:
\begin{quote}
Mundell's article [includes] a number of incongruous assumptions.
One is that complications associated with speculation in the forward market do not exist.
It can only bring discredit on the economics profession to leave unchallenged
his attempt to draw from the model policy conclusions that are applicable in the real world.
\end{quote}
Mundell:
\begin{quote}
Theory is the poetry of science.
It is simplification, abstraction, the exaggeration of truth.
Through simplification, theory creates a caricature of reality.
The caricature itself is not the real world --- it mocks it.
Yet mind true things by their mockeries!

Dr McLeod does not like my caricature; he calls my assumptions unrealistic.
I certainly hope he is right.  I hope my assumptions are unrealistic.
%If they were not, I could not have made a contribution to theory.
I left out a million variables that made my caricature of reality unrealistic.
At the same time, it enabled me to find fruitful empirical generalizations.
\end{quote}
%
[From:  {\it Canadian Journal of Economics and Political Science\/}, vol 30, 1964, pp 413-414, 419-421.]

The theory of Mundell's paper is a long way from anything
we'll do in this class,
but you have to like a guy who has the nerve to say
``theory is the poetry of science.''
The substantive point is that our goal isn't realism, it's insight.
The art of economics lies in making unrealistic simplifying assumptions that
deliver clear results that we believe generalize to more complex environments.
The art lies in making that distinction ---
in understanding which assumptions are crucial and
which simply make the algebra easier.


\subsection*{General equilibrium models}

In the 1950s, economists formalized some ideas that date back to Adam Smith and before.
Lots of people were involved, but the central players
were Kenneth Arrow, Gerhard Debreu, and Lionel McKenzie.
The first idea is to consider equilibrium (think: supply equals demand)
for all markets at once ---
what economists refer to as general equilibrium.
That's missing in the usual supply and demand diagram,
one of the reasons we don't stop there.
The second is to address the welfare properties of general equilibrium.
Is the equilibrium allocation of resources a good one in some sense?
Does Smith's invisible hand work?
We'll do a quick summary of the principle results.

{\it Ingredients.\/}  Here are the ingredients of a general equilibrium model ---
what we might term the physical environment, which includes
the objects involved but says nothing about how they interact.
The list would include:
%
\begin{itemize}
\item List of commodities.   What's produced?  What's consumed?
\item List of agents.  Who are the people in the model?  How many are there?
\item Preferences and endowments.  Each agent has a utility function
that describes her preferences over commodities.
Also a vector of endowments:  quantities of the various commodities that she starts with.
\item Technologies.  Production functions that transform combinations some commodities (inputs)
into other commodities (outputs).
\item Resource constraints.  Uses (consumption) of each commodity is limited by
its sources (endowment net of other uses plus production).
\end{itemize}
Most theoretical economies consist of these ingredients in one form or other.
If there are no technologies, we refer to it as an {\it exchange economy\/};
since production is ruled out, all agents can do is exchange their endowments.


{\it Equilibrium.\/}
A {\it competitive equilibrium\/} for an environment like this consists of
%
\begin{itemize}
\item Agent maximization.
Agents choose consumption quantities that maximize utility given
prices and budget constraints.
\item Firm maximization.
Firms maximize profits given prices and technologies.
\item Markets clear.
Sources (``supply'') equals demand (``uses'') for every commodity.
\end{itemize}
That sounds simple, but it was a significant achievement to formalize it
and prove (under suitable conditions on preferences and technologies)
that an equilibrium exists.
Note that it's a {\it competitive\/} equilibrium:
agents and firms are price-takers, which my
industrial organization colleagues find amusing.
(Equilibrium when agents or firms can influence prices
is a much more demanding problem,
not one we can leave to macroeconomists, I'm told.)

{\it Welfare.\/}  Now the invisible hand:
Is there a sense in which a competitive equilibrium delivers a
good allocation of resources?
It depends what you mean by ``good.''
Suppose we have two agents and consider two allocations,
one in which agent 1 gets a lot and one in which agent 2 gets a lot,
with the other getting what's left in each case.
Which one do we think is better?
We generally resist making comparisons between allocations like
this, because it forces us to decide which agent we like best.
Like parents, we care about all our agents equally.
We settle for a weaker definition of good:
we say an allocation is {\it Pareto optimal\/} (ie, good)
if we can't make one agent better off without making at least one
other agent worse.
This leaves some allocations noncomparable,
but seems like a useful condition to apply to an economy.
If we can make someone better off without hurting anyone,
surely the allocation isn't all that good.

With that background, we have two classical results:
%
\begin{itemize}
\item First welfare theorem.  A competitive equilibrium produces
a Pareto optimal allocation of resources.
\item Second welfare theorem.
A Pareto optimal allocation of resources can be reproduced
as a competitive equilibrium for some allocation
of endowments.
\end{itemize}
The first is the modern version of Adam Smith's invisible hand.
The second is (among other things) a useful shortcut for computing
a competitive equilibrium:  we compute a Pareto optimum instead.

All of this might seem an excess of formalism.
On the contrary, I find it extremely helpful to keep track
of the ingredients and how they fit together.


\subsection*{Finding an equilibrium as a Pareto problem}


We'll generally attack equilibrium indirectly by finding a Pareto optimum.
By the second welfare theorem, this corresponds to a competitive equilibrium for some
allocation of endowments.
In much of macroeconomics, finding the right allocation of endowments is easy
because we only have one agent, who therefore owns everything.

Let's see how this works using an example, a simple one.
The ingredients are:
%
\begin{itemize}
\item List of commodities.   Two:  apples and bananas.
\item List of agents.  One.
\item Preferences and endowments.  The utility function is
\begin{eqnarray*}
    U(c_a,c_b) &=& \beta \log c_a + (1-\beta) \log c_b ,
\end{eqnarray*}
where $(c_a,c_b)$ is consumption.
The endowment is $(y_a,y_b)$.
\item Technologies.  None, this is an exchange economy.
\item Resource constraints.  They are
\begin{eqnarray*}
    c_a &\leq& y_a \\
    c_b &\leq& y_b .
\end{eqnarray*}
\end{itemize}
That's the economy:  one agent, two goods, no production.
You can't get much simpler than that,
but it serves to illustrate our approach.

The Pareto problem for this economy consists of maximizing utility subject
to the resource constraints:
\begin{eqnarray*}
    \max_{c_a,c_b} &&  \beta \log c_a + (1-\beta) \log c_b \\
    \mbox{s.t.}   &&  c_a \;\;\leq\;\; y_a \\
                  &&  c_b \;\;\leq\;\; y_b .
\end{eqnarray*}
The solution is obvious (consume the endowment),
but we'll solve it by Lagrangean methods using
multipliers $q_a$ and $q_b$ on the two constraints.
(This notation suggests we know something about the answer,
which we do.)
The Lagrangean is
\begin{eqnarray*}
    \mathcal{L} &=&  \beta \log c_a + (1-\beta) \log c_b
                + q_a (y_a - c_a) + q_b (y_b - c_b) .
\end{eqnarray*}
The first-order conditions are
\begin{eqnarray*}
    \beta/c_a - q_a &=& 0 \\
   (1-\beta)/c_b - q_b &=& 0 .
\end{eqnarray*}
The constraints tell us to consume the endowments,
but the nice thing about this approach is that
we get the prices, too:
$q_a = \beta/y_a$ and $q_b = (1-\beta)/y_b $.
We could recast the problem as a competitive equilibrium
with these prices --- but we won't bother.

[Draw picture:  indifference curves, endowment point,
slope of tangent line.]

We can easily modify this economy to incorporate production.
Suppose we have no endowment of bananas ($y_b=0$),
but can transform apples into bananas with the
production function $f(k) = k^\theta$.
If we use $k$ to denote use of apples to produce bananas,
the Pareto problem becomes
\begin{eqnarray*}
    \max_{c_a,k,c_b} &&  \beta \log c_a + (1-\beta) \log c_b \\
    \mbox{s.t.}   &&  c_a + k \;\;\leq\;\; y_a \\
                  &&  c_b \;\;\leq\;\; k^\theta ,
\end{eqnarray*}
with parameters $ 0 < \beta,\theta < 1$.
The Lagrangean is
\begin{eqnarray*}
    \mathcal{L} &=&  \beta \log c_a + (1-\beta) \log c_b
                + q_a (y_a - c_a - k) + q_b (k^\theta - c_b ) .
\end{eqnarray*}
The first-order conditions are
\begin{eqnarray*}
    \beta/c_a - q_a &=& 0 \\
    - q_a + q_b \theta k^{\theta-1} &=& 0 \\
   (1-\beta)/c_b - q_b &=& 0 .
\end{eqnarray*}
They imply
\begin{eqnarray*}
    (y_a - k) \beta /(1-\beta) &=& k / \theta ,
\end{eqnarray*}
which is easily solved for $k$.
Consumption quantities $(c_a,c_b)$ then follow from the
constraints.

[Update picture, add production possibilities frontier.]



\subsection*{Two-period deterministic economies}

In macroeconomics, a useful starting point is an
economy with a single agent and some form of dynamics.
We keep the dynamics very simple here with two periods,
dates 0 and 1.
Then we develop a series of examples of increasing complexity
to illustrate how the approach works and what it implies for asset pricing.

If we follow the formalism described above, we have the following ingredients:
%
\begin{itemize}
\item List of commodities.  Two:  the consumption good at dates 0 and 1.
\item List of agents.  One: a ``representative'' agent.
\item Preferences and endowments.  Preferences governed by the utility function
\begin{eqnarray*}
    U(c_0,c_1) &=& u(c_0) + \beta u(c_1) .
\end{eqnarray*}
Endowments are $(y_0,y_1)$.
\item Technologies.  The production function $f$ transforms input of the date-0 good
(label this $k$)
into output of the date-1 good.
\item Resource constraints.  For the two commodities,
we have
\begin{eqnarray*}
    c_0 + k &\leq& y_0 \\
    c_1 &\leq&  y_1 + f(k) .
\end{eqnarray*}
\end{itemize}
The idea is that investment $k$ produces higher future output and consumption.
%If we keep track of consumption quantities,
%that gives us $c_0 = y_0 - k$
%and $ c_1 = f(k) + y_1$.

%[?? direct method first?]

We could compute an equilibrium directly,
but it's easier to compute a Pareto optimum and
use the second welfare theorem to tell us it's also
a competitive equilibrium.
With only one agent, a Pareto optimum is the solution to:
\begin{eqnarray*}
    \max_{c_0,k,c_1} && u(c_0) + \beta u(c_1) \\
    \mbox{s.t.}   &&  c_0 + k \;\;\leq\;\; y_0 \\
                  &&  c_1 \;\;\leq\;\;  y_1 + f(k) .
\end{eqnarray*}
We could simply substitute the constraints into the objective
function and maximize, but using Lagrange multipliers gives us
the prices, too.
The Lagrangean is
\begin{eqnarray*}
    \mathcal{L} &=&  u(c_0) + \beta u(c_1) + q_0 (y_0 - c_0 - k)
                    + q_1 ( y_1 + f(k) - c_1)  .
\end{eqnarray*}
The first-order conditions are
\begin{eqnarray*}
    c_0: &&  u'(c_0) - q_0 \;\;=\;\; 0 \\
    k: &&  - q_0 + q_1 f'(k)  \;\;=\;\; 0 \\
    c_1: &&  \beta u'(c_1) - q_1 \;\;=\;\; 0 .
\end{eqnarray*}
They imply that the marginal rate of substitution (mrs)
and marginal rate of transformation (mrt)
equal the price ratio:
\begin{eqnarray*}
    \frac{\beta u'(c_1)}{u'(c_0)} &=& \frac{q_1}{q_0} \;\;=\;\; \frac{1}{f'(k)} .
\end{eqnarray*}
If we define the (gross) interest rate $r$ by $q_1/q_0 = 1/r$,
then we also have $ [\beta u'(c_1)/u'(c_0)] r = 1$ and $f'(k) = r$.
This differs from the consumption-saving problem in having curvature
in $f$, so that $f'(k) = r$ isn't constant.

[Draw indifference curves and ppf again.]

 Where does the interest rate $r$ come from?
 It depends on preferences ($u$ and $\beta$)
and technology ($f$).
A high interest rate is associated with productive capital
and high date-1 consumption.


\subsection*{Two-period stochastic economies:  exchange version}

[Draw an event tree]

We deal with uncertainty as we did with consumption and portfolio choice.
At date 1, any of a number of states $z$ can occur.
If there are $Z$ states (a finite number), the ingredients of
an exchange economy with a single representative agent are
\begin{itemize}
\item List of commodities.  $Z+1$:  the consumption good at date 0
and in each state $z$ at date 1.
\item List of agents.  One: a ``representative'' agent.
\item Preferences and endowments.  Preferences are governed by the utility function
\begin{eqnarray*}
    U(c_0,c_1) &=& u(c_0) + \beta \sum_z p(z) u[c_1(z)] .
\end{eqnarray*}
Endowments are $\{y_0,y_1(z)\}$.
\item Technologies.  None.
\item Resource constraints.  For the two commodities,
we have
\begin{eqnarray*}
    c_0 &\leq& y_0 \\
    c_1(z) &\leq& y_1(z), \mbox{ for each } z .
\end{eqnarray*}
\end{itemize}
This has what we call {\it complete markets\/}:  we have one market for every state.

The Pareto problem has a similar form:  maximize utility subject to
resource constraints.
The main difference from the previous section is that we have lots of resource
constraints, one for each state.
The problem is
\begin{eqnarray*}
    \max_{c_0,c_1(z)} && u(c_0) + \beta \sum_z p(z) u[c_1(z)] \\
    \mbox{s.t.}   &&  c_0  \;\;\leq\;\; y_0 \\
                  &&  c_1(z) \;\;\leq\;\; y_1(z) .
\end{eqnarray*}
The Lagrangean is
\begin{eqnarray*}
    \mathcal{L} &=&  u(c_0) + \beta \sum_z p(z) u[c_1(z)] + q_0 (y_0 - c_0)
                    + \sum_z q_1(z)  [ y_1(z) - c_1(z)]  .
\end{eqnarray*}
The first-order conditions are
\begin{eqnarray*}
    c_0: &&  u'(c_0) - q_0 \;\;=\;\; 0 \\
    c_1(z): &&  \beta p(z) u'[c_1(z)] - q_1(z) \;\;=\;\; 0 .
\end{eqnarray*}
The solution, of course, is to consume the endowment in all states,
but this gives us the prices too.

The foc's lead to the usual $E(mr) = 1$.
First note that they imply the mrs equals the price ratio:
\begin{eqnarray}
    \frac{\beta p(z) u'[c_1(z)]}{u'(c_0)} &=&
    \frac{\beta p(z) u'[y_1(z)]}{u'(y_0)}
    \;\;=\;\; \frac{q_1(z)}{q_0}
    \label{eq:state-price}
\end{eqnarray}
The ratio $q(z) = q_1(z)/q_0$ is the price at date 0 of one unit in state $z$ at date 1:
the {\it state price\/}, we might call it.
If we think of this as the price of an Arrow security ---
a security that pays off in one state only, like a digital option ---
then its return is the inverse:  $r(z) = 1/q(z) = q_0/q_1(z)$.
If we substitute, the equation becomes
\begin{eqnarray}
    \beta p(z) u'[c_1(z)]/u'(c_0) r(z) &=& p(z) m(z) r(z) \;\;=\;\;  1 ,
    \label{eq:mr}
\end{eqnarray}
where $m(z) = \beta u'[c_1(z)]/u'(c_0)$ is the marginal rate of substitution.
Note that we have lots of mrs's, one for each state $z$.
Summing across states gives us the familiar
\begin{eqnarray*}
    \sum_z \beta p(z) \{ u'[c_1(z)]/u'(c_0)\} r(z) &=&
    E \left[ \beta  [u'(c_1)/u'(c_0)] r \right]
    \;\;=\;\; E(mr)
    \;\;=\;\; 1 .
\end{eqnarray*}
This last equation shows up in lots of models,
but the previous one shows up only when we have complete markets.

How do prices and returns differ across states?
Consider the role of probability.
The state price formula (\ref{eq:state-price})
and return formula (\ref{eq:mr})
tell us that states with higher probability
have higher prices and lower returns.
Probabilities enter both proportionately.
More interesting to us is the role of the date-1 endowment.
The same equations tell us that with higher output have lower prices
and higher returns.
Why?  Because $u'[c_1(z)] = u'[y_1(z)]$ is a decreasing function:
if we increase $y_1(z)$, then we decrease marginal utility $u'[y_1(z)]$.
In words: a payoff of one is more valuable in states where output is scarce
than in states when output is plentiful.
Securities that pay off in good states (high endowment) are therefore
cheaper than those that pay off in bad states (low endowment).
We'll see later that this is the source of risk premiums.
Securities that pay off mostly in good states have lower prices
and higher returns, on average, than those that pay off in bad states.

[Draw indifference curves between states and endowment point.]

\subsection*{Asset pricing in the exchange economy}

We can be more specific about asset prices and returns
if we put more structure on the economy.
We apply two kinds of structure:  we give the agent power utility
and make the date-1 endowment lognormal (that is, the log of the
endowment is normal).
This is our first look at a collection of equations we'll
see over and over again.

Before turning to asset pricing, it's helpful to remind ourselves
of some properties of normal and lognormal random variables.
Let us say that $x$ is normal.
In short-hand notation:  $ x \sim \mathcal{N}(\kappa_1,\kappa_2)$.
Result 1:  Linear functions of $x$ are also normal.
Specifically:  $ y = a + b x \sim \mathcal{N}(a + b \kappa_1, b^2 \kappa_2)$.
(If you have any question about this, write down the mgf of $y$.)
Alternatively, let $\log y = x$ be normal.
Then we say $y$ is lognormal.
Result 2:  The mean of $y$ is $E(y) = E(e^x) = e^{\kappa_1 + \kappa_2/2}$.
(This also follows from the mgf of $x$:  evaluate it as $s=1$.)

Here's how asset pricing works.
An asset consists, in general,
of a state-dependent dividend or payoff $d(z)$.
If its price is $q$, its return is $ r(z) = d(z)/q$.
We typically value assets in this order:
specify the dividend, compute its price, then calculate its return.
Given a marginal rate of substitution $m(z)$ and probabilities $p(z)$,
pricing follows from
\begin{eqnarray*}
    1 &=& E (mr) \;\;=\;\; \sum_z p(z) m(z) [d(z)/q] \;\;=\;\;  E [m (d/q)]] .
\end{eqnarray*}
Since $q$ doesn't depend on $z$, we can take it outside the expectation.
If we multiply both sides by $q$, we have
\begin{eqnarray*}
    q &=& E (md)  .
\end{eqnarray*}
Hence the term ``asset pricing.''


Let's do a concrete example and see how this works.
We let $u(c) = c^{1-\alpha}/(1-\alpha)$ (power utility)
and
$\log y_1(z) \sim \mathcal{N}(\kappa_1,\kappa_2)$
(lognormal endowment).
For convenience, we also let $y_0 = 1$, so that it drops out of
calculations below.
With power utility, the mrs is
\begin{eqnarray*}
        m(z) \;\;=\;\; {\beta u'[y_1(z)]}/{u'(y_0)} &=&  \beta [y_1(z)/y_0]^{-\alpha} ,
\end{eqnarray*}
a function of the growth rate of the endowment.
Since $y_0 = 1$, $\log m(z)$ is
\begin{eqnarray*}
    \log m(z) &=& \log \beta - \alpha \log y_1(z)
        \;\;\sim\;\; \mathcal{N}(\log \beta - \alpha \kappa_1, \alpha^2 \kappa_2).
\end{eqnarray*}
The second step follows from the properties of lognormals we outlined above.

Now consider two specific assets:  a ``bond'' that pays off one in every state
and a share of ``equity'' that pays of the endowment in each state.
If the bond has price $q^1$ then its return is $r^1 = 1/q^1$.
The price satisfies
\begin{eqnarray*}
     q^1 &=& E(m \cdot 1) \;\;=\;\; \exp(\log \beta - \alpha \kappa_1 + \alpha^2 \kappa_2/2 )
            \;\;=\;\; \beta \exp(- \alpha \kappa_1 + \alpha^2 \kappa_2/2 ),
\end{eqnarray*}
which gives us the return
\begin{eqnarray*}
     r^1 &=& 1/q^1
            \;\;=\;\; \beta^{-1} \exp( \alpha \kappa_1 - \alpha^2 \kappa_2/2 ),
\end{eqnarray*}
Higher mean growth $\kappa_1$ is associated with a lower price
and therefore a higher interest rate.
The greater is risk aversion the greater is this effect.
The same is true in the deterministic case,
which corresponds to $\kappa_2 = 0$.
An increase in the variance $\kappa_2$ raises the price and lowers the interest rate.

Now consider equity, a claim to the dividend $d(z) = y_1(z)$.
Its return is $r^e(z) = d(z)/q^e = y_1(z)/q_1$.
Its price is therefore $q^e = E( m y_1)$.
Collecting terms, we have
\begin{eqnarray*}
    \log m(z) + \log d(z) &=& \log \beta  + (1-\alpha) \log y_1(z)
    \;\;\sim\;\; \mathcal{N}(\log \beta + (1-\alpha) \kappa_1, (1-\alpha)^2 \kappa_2).
\end{eqnarray*}
The price is therefore
\begin{eqnarray*}
    q^e &=& \exp[\log \beta + (1- \alpha) \kappa_1 + (1-\alpha)^2 \kappa_2/2 ]
        \;\;=\;\; \beta \exp[(1- \alpha) \kappa_1 + (1-\alpha)^2 \kappa_2/2 ] .
\end{eqnarray*}
Its return is
\begin{eqnarray*}
    r^e(z) \;\;=\;\; y_1(z)/q^e &=&
        y_1(z) \beta^{-1} \exp[-(1- \alpha) \kappa_1 - (1-\alpha)^2 \kappa_2/2 ] .
\end{eqnarray*}

The expected return is (using our lognormal result again)
\begin{eqnarray*}
    E(r^e)  &=&
    \exp(\kappa_1 + \kappa_2/2)  \beta^{-1}
            \exp[-(1- \alpha) \kappa_1 - (1-\alpha)^2 \kappa_2/2 ]  \\
    &=& \beta^{-1} \exp\{\alpha \kappa_1 + [1 - (1-\alpha)^2] \kappa_2/2 \} .
\end{eqnarray*}
(Hang in there, we're making progress.)

Now the question:  Does equity have a higher or lower expected return than the bond?
We refer to the difference as a risk premium.
We suggested that an asset that pays off mostly in good states,
as equity does, would demand a higher return as a result.
We would say, than, that equity has a positive risk premium.
Let's see if that's true.
First, let's check the certainty case to make sure we're on the right track.
If $\kappa_2 = 0$, we have
\begin{eqnarray*}
    r^1 \;\;=\;\; E(r^e)  &=&    \beta^{-1} \exp( \alpha \kappa_1 ) .
\end{eqnarray*}
Ditto if $\alpha = 0$.
In general we have $E(r^e) > r^1$ if
\begin{eqnarray*}
    \exp\{ [1 - (1-\alpha)^2] \kappa_2/2 \}
            &>& \exp(-\alpha^2 \kappa_2/2) .
\end{eqnarray*}
A little algebra should convince you that's the case.
(Or some numerical examples.)

In loglinear models like this,
including some of the most common ones in finance,
it's simpler to compute the expected difference in returns in logs;
namely, $ E \log r^e - \log r^1 $.
Here we get
\begin{eqnarray*}
    E \log r^e - \log r^1 &=& (2 \alpha - 1) \kappa_2 /2 ,
\end{eqnarray*}
which is positive if $\alpha > 1/2$.
[Check this.]



\subsection*{Two-period stochastic economies 2:  production}

We can do the same thing with production, which here means
we allow investment at date 0 to increase output and consumption
at date 1.
The standard version
is the same as above, except:
\begin{itemize}
\item Technologies.  An input of $k$ units of the date-0 good
produces $ z f(k)$ units of the date-1 good.
The output of this investment is, therefore, uncertain.
\item Resource constraints.  They become
\begin{eqnarray*}
    c_0 &\leq& y_0 - k \\
    c_1(z) &\leq& y_1(z) + z f(k), \mbox{ for each } z .
\end{eqnarray*}
\end{itemize}
The idea here is that investment is risky:  the marginal product
of an additional unit of capital is $z f'(k)$,
which depends on the date-1 state $z$.


With these changes, the Pareto problem is
\begin{eqnarray*}
    \max_{c_0,k, c_1(z)} && u(c_0) + \beta \sum_z p(z) u[c_1(z)] \\
    \mbox{s.t.}   &&  c_0  + k \;\;\leq\;\; y_0 \\
                  &&  c_1(z) \;\;\leq\;\; y_1(z) + z f(k) .
\end{eqnarray*}
The Lagrangean is
\begin{eqnarray*}
    \mathcal{L} &=&  u(c_0) + \beta \sum_z p(z) u[c_1(z)] + q_0 (y_0 - c_0 - k)
                    + \sum_z q_1(z)  [ y_1(z) + z f(k) - c_1(z)]  .
\end{eqnarray*}
The first-order conditions are
\begin{eqnarray*}
    c_0: &&  u'(c_0) - q_0 \;\;=\;\; 0 \\
    c_1(z): &&  \beta p(z) u'[c_1(z)] - q_1(z) \;\;=\;\; 0 \\
     k:   && - q_0 + \sum_z q_1(z) z f'(k) \;\;=\;\; 0 .
\end{eqnarray*}
%This gives us, in every state $z$,
%\begin{eqnarray*}
%    \frac{\beta p(z) u'[c_1(z)]}{u'(c_0)} &=& \frac{q_1(z)}{q_0}
%    \;\;=\;\; \frac{1}{z f'(k)} . [??]
%\end{eqnarray*}
%As before, the mrs equals the mrt equals the relative price.
[??]
We can solve this by hand for a special case.
Let $u(c) = \log c$ and $f(k) = k^\theta$, with
$0 \leq \theta \leq 1$.
Equity is a claim to $d(z) = z f'(k) k$
[each unit of capital is paid its marginal product $z f'(k)$].
I'll leave the rest to you.
% ????


What's useful about this model is that we can say where
aggregate dividends come from:
they're payments to capital.
At the margin,
an increase in $k$ at date 0 generates $z f'(k)$ units of
output at date 1 in state $z$.
Since $z$ is uncertain, so is the dividend.


\subsection*{What are we missing?}

Give me your suggestions in class.


\subsection*{Bottom line}

General equilibrium models are useful devices for thinking
about where prices and quantities come from.
From a macro-finance perspective,
even simple models give us some insights.
Among them I'd include:
%
\begin{itemize}
\item Asset prices and returns reflect the marginal rates of substitution
of the people who buy them.
Our old favorite, $E(mr) = 1$, is a good example.
\item That gives us a connection between returns and dividends.
Assets that pay off mostly in good states have less value
and higher returns, on average, than an asset that pays off the same in all states.
\end{itemize}
%

\vfill \centerline{\it \copyright \ \number\year \
NYU Stern School of Business}
\end{document}
