\documentclass[11pt]{article}

\oddsidemargin=0.25truein \evensidemargin=0.25truein
\topmargin=-0.5truein \textwidth=6.0truein \textheight=8.75truein

%\usepackage{graphicx}
\usepackage{verbatim}
%\usepackage{booktabs}
\usepackage{comment}
\usepackage[dvipdfm]{hyperref}
\urlstyle{rm}   % change fonts for url's (from Chad Jones)
\hypersetup{
    colorlinks=true,        % kills boxes
    allcolors=blue,
    pdfsubject={ECON-UB233, Macroeconomic foundations for asset pricing},
    pdfauthor={Dave Backus @ NYU},
    pdfstartview={FitH},
    pdfpagemode={UseNone},
%    pdfnewwindow=true,      % links in new window
%    linkcolor=blue,         % color of internal links
%    citecolor=blue,         % color of links to bibliography
%    filecolor=blue,         % color of file links
%    urlcolor=blue           % color of external links
% see:  http://www.tug.org/applications/hyperref/manual.html
}

\renewcommand{\thefootnote}{\fnsymbol{footnote}}

% document starts here
\begin{document}
\parskip=\bigskipamount
\parindent=0.0in
\thispagestyle{empty}
{\large ECON-UB 233 \hfill Dave Backus @ NYU}

\bigskip\bigskip
\centerline{\Large \bf Option Pricing}
\centerline{(Started: November 21, 2011; Revised: \today)}

\bigskip
We have two goals here.
The first is to derive the Black-Scholes-Merton (``BSM'') formula from a model
in which the price of the ``underlying'' ---
the asset the option is based on ---
has a lognormal risk-neutral distribution.
This is expressed most simply in terms of a flat volatility smile.
The second is to show how other distributions can generate other
shapes for the smile, including shapes similar to those
we observe for real-world options.
The focus is on options on S\&P 500 index futures,
a popular and convenient contract.
The index has a macro flavor:  its ups and downs
are closely connected to ups and downs in the economy.


We'll continue to use the $t, t+1$ timing,
with today being $t$ and next period (next year) $t+1$.
In some cases we'll generalize this to $t,t+\tau$,
where $\tau$ is the length of the time interval.
The reason to do this is to match up with the observed maturities of options.


\subsection*{Option cash flows}

An asset is defined by its cash flow or dividend, as we've called it.
Here we buy an asset at $t$ and realize its (generally uncertain) cash flow at $t+\tau$.
An option is a particular kind of cash flow.
An option gives the owner right to buy or sell another asset at a preset price
within a given time period.
For example, movie producers might have a 5-year option to produce a
movie based on a book.
Or a real estate developer might have a 6-month option to purchase
a piece of property at a preset price.
After that time, the option expires and
the seller of the option can do what she wants with the asset.
In financial markets you see options on lots of things:  stocks, bonds,
foreign currencies, pork belly futures, and so on.
They're the classic ``financial derivative'' and show up all over the place.
Options are useful to buyers because they give them flexibility.
In return for this flexibility, the seller collects a fee --- the price
of the option.

A typical option is defined by these features:
\begin{itemize}
\item The underlying:  the asset on which the option is based.
In our case this is the S\&P 500 futures contract.
\item The strike price: the price at which you can buy or sell
the underlying.
\item The term or maturity: the period of time
over which the option can be exercised.
%We'll use the Greek letter $\tau$.
\item Call or put:  whether it's an option to buy (a call) or sell (a put).
\item Type of option:  ``European,'' ``American,'' or something else (``Asian''?  ``Greek''?).
Most traded options can be exercised any time up to maturity (American),
but it's mathematically simpler to work with options that can be
exercised only at maturity (European).
That's what we'll do.
\end{itemize}
That's the basics.

With a small investment in notation,
we can express an option's cash flows mathematically.
Let us say we have a $\tau$-period option.
If the price of some underlying asset at time $t$ is $s_t$
then a $\tau$-period call option at strike price $k$
generates a cash flow at time $t+\tau$ of
$ ( s_{t+\tau} - k )^+ $,
where $x^+ \equiv \max(x,0) $.
Why?  Because you would only exercise the option if the price is above $k$.
Otherwise you let it expire.
Similarly, a $\tau$-period put option at strike price $k$
generates the cash flow $ (k - s_{t+\tau} )^+ $.

If you graph the cash flow against the future price, you'll see
it's convex for both puts and calls.
That means, via Jensen's inequality, that the value increases with the amount
of risk in the underlying.
More simply:  put and call options give you the upside
without the downside.
The more risk there is, the more valuable they are.


\subsection*{Put-call parity}


There's a useful connection between European put and call prices
that tells us, in effect, that if you know the price of one you can
easily compute the price of the other.
With a call option, you buy the asset if its price is above $k$,
and with a put you sell it if the price is below $k$.
So buying a call and selling a put at the same strike leads you to own the stock in all cases.
The price is $k$, paid at $t+\tau$, so we discount it using
the price $q^\tau_t$ (the price at $t$ of one dollar paid at $t+\tau$).
That gives us two ways to buy the stock at date $t$, with options or directly,
and they should have the same price:
\begin{eqnarray*}
    \underbrace{q^c_t}_{\mbox{buy call}} -
    \underbrace{q^p_t}_{\mbox{sell put}} +
    \underbrace{q^\tau_t k}_{\mbox{present value of strike}}
    &=&
    \underbrace{s_t}_{\mbox{buy stock}} .
    \label{eq:put-call-parity}
\end{eqnarray*}
This is known as {\it put-call parity\/}.
A fine point:  this is for an asset that doesn't pay a dividend
between $t$ and $t+\tau$.
If it does, we need to work that cash flow into the equation.


\subsection*{The normal distribution function}


Recall that if $x$ is normal with mean zero and variance one (``standard normal''),
its probability density function (pdf) is
\begin{eqnarray*}
    p(x) &=& (2 \pi)^{-1/2} \exp(-x^2/2)
\end{eqnarray*}
for any real number $x$.
It's positive and integrates to one, as usual.
It's also symmetric:  $p(x) = p(-x)$.

The BSM formula involves the cumulative distribution function (cdf) $N$,
defined by
\begin{eqnarray*}
    \mbox{Prob} (x \leq x^*)  &=& \int_{-\infty}^{x^*} p(x) dx
        \;\;=\;\; N(x^*).
\end{eqnarray*}
The last is common notation (or sometimes people use $\Phi$ instead of $N$).
There's no simple antiderivative of $p$, but it comes up enough that
we give it a letter.
It's also a common function in lots of software.
The function $N$ corresponds to {\tt normcdf} in Matlab
and {\tt NORMSDIST} in spreadsheet programs.

We can do the same with other normal random variables.
If $ y = \mu + \sigma x$,
then it's normal with mean $\mu$ and variance $\sigma^2$.
We can compute cumulative probabilities for $y$ using
\begin{eqnarray*}
    \mbox{Prob} (x \leq x^*)  &=& \mbox{Prob} [x \leq (y^*-\mu)/\sigma ]
        \;\;=\;\; N\left[ (y^*-\mu)/\sigma \right].
\end{eqnarray*}
Matlab does this calculation for you with the command {\tt normcdf(ystar,mu,sigma)}.

One final property:  since $p$ is symmetric, $N(-x) = 1 - N(x)$.
Why?
Note that
\begin{eqnarray*}
\int_{-\infty}^{x^*} p(x) dx + \int_{x^*}^{\infty} p(x) dx &=& 1 .
\end{eqnarray*}
This implies
\begin{eqnarray*}
     1 - N(x^*) &=& \int_{x^*}^{\infty} p(x) dx \;\;=\;\; \int_{-\infty}^{-x^*} p(-x) dx
            \;\;=\;\; N(-x^*)
\end{eqnarray*}
by symmetry.
It's easier to show in a picture than this suggests:
graph the pdf and its tail integrals.


\subsection*{The BSM formula and the volatility smile}


We refer to this as a {\it formula\/} because that's what it is:
a function that relates option prices to the strike price
and a few other things.  We'll derive it from a model shortly,
but for now it's just a formula.

The BSM formula for the price $q^c_t$ of a call option is
\begin{eqnarray}
        q^c_t &=& s_t N(d) - q^\tau_t b N (d - \tau^{1/2}\sigma ),
        \label{eq:bsm-call}
\end{eqnarray}
where
\begin{eqnarray*}
          d &=& \frac{\log(s_t/q^\tau_t k) + \tau \sigma^2/2}{\tau^{1/2}\sigma} .
\end{eqnarray*}
With put-call parity, we can show that put prices are
\begin{eqnarray}
        q^p_t &=& q^\tau_t b N(-d + \tau^{1/2} \sigma) - s_t N(-d) .
        \label{eq:bsm-put}
\end{eqnarray}
[You might try this.]
This uses the symmetry property of $N$.

Let's review the ingredients:
\begin{itemize}
\item $s_t$ is the current price of the underlying.
\item $\tau$ is the maturity of the option.
\item $q_t^\tau$ is the price at $t$ of ``one'' at date $t+\tau$  (the bond price, in other words).
This is commonly written in terms of the implied (continuously compounded) interest rate $r$:
$q_t^\tau = \exp(-r \tau)$.
We won't do that, because $r$ means something else to us, but you might imagine
a variant like this:  $q_t^\tau = \exp(-y_t^\tau \tau)$
where $y_t^\tau$ is the continuously compounded yield on a bond with maturity $\tau$.
Not something we need, but it has one useful feature:
the suggestion that we need not have a constant interest rate.
\item $k$ is the strike price.
\item $\sigma$ is a parameter we refer to as ``volatility.''
\end{itemize}
All of these are observable except the last one,
which means we can derive it from the others.
The output of this calculation is commonly referred to as {\it implied volatility\/}
or simply volatility.
A plot of implied volatility against the strike price
is referred to as a volatility smile.
The term smile refers to the shape, but for equity index options
it's less a smile than a downward-sloping line, possible with some
convexity to it.


\subsection*{The S\& P 500 E-mini futures contract}

We're interested in options on a broad-based equity index
like the S\&P 500,
which we've seen is closed related to the state of the economy.
(I have in mind here the scatterplot of equity returns against
consumption growth.)
As it happens, it's common to use options on the futures contract.
There's no need to pay any attention to this,
but if you're interested, this is the one,
as described in Wikipedia (lightly edited):
%
\begin{quote}
The E-mini S\&P futures contract,
often called simply the ``E-mini,''
is a stock market index futures contract traded
on the Chicago Mercantile Exchange's Globex electronic trading platform.
The notional value of one contract is US\$50 times the value of the S\&P 500 stock index.
It was introduced by the CME on September 9th, 1997,
after the value of the existing S\&P contract became too large for many small traders.
The E-mini quickly became the most popular equity index futures contract in the world.
\end{quote}

Options on these contracts are among the most liquid we have.
Most work on equity index options therefore starts with
options on index futures.


\subsection*{Risk-neutral option pricing}

If you recall our work on asset pricing,
the fundamental result is what I'll call the no-arbitrage theorem.
With a finite set of states,
the theorem says if prices and dividends do not contain any arbitrage opportunities,
then there are positive state prices $p(z)$ that price all
assets:
\begin{eqnarray}
    q^j_t  &=& \sum_z q(z) d_{t+1}(z) .
    \label{eq:theorem}
\end{eqnarray}
The only question left is what the state prices are.

We also looked at two alternative ways to express this result:
\begin{eqnarray*}
    q(z) &=& \left\{
                \begin{array}{l}
                    p(z) m(z) \\
                    q^1_t p^*(z) .
                \end{array}
            \right.
\end{eqnarray*}
The first defines a pricing kernel $m(z) = q(z)/p(z)$,
where $p(z)$ is the probability of state $z$.
We've seen that risk premiums follow from the covariance of returns with $m$.
We'll use that representation when we get to bonds.
The second defines the risk-neutral probability
$p^*(z) = p(z)m(z)/q_t^1$.
The denominator $q_t^1 = \sum_z p(z) m(z) = E(m)$ guarantees
that the risk-neutral probabilities sum (or integrate) to one.

We could attack option prices either way.
We use risk-neutral probabilities because
then we need only one object ($p^*$)
rather than two ($p$ and $m$), but it should be clear that
they're equivalent.
So from now on, we'll think about $p^*$.
The pricing relation (\ref{eq:theorem}) becomes
\begin{eqnarray}
    q^j_t  &=&  q^1_t \sum_z p^*(z) d_{t+1}(z)
            \;\;=\;\; q^1_t  E^* d_{t+1} ,
              \label{eq:arb-condition}
\end{eqnarray}
where $E^*$ means the expectation using the risk-neutral distribution.


If we know the risk-neutral distribution, we're all set.
More commonly we guess a functional form and infer parameters from
observed prices.
When we do that, we need to make sure we satisfy (\ref{eq:arb-condition}).
I refer to the resulting restriction as the ``no-arbitrage condition,''
because existence of risk-neutral probabilities satisfying (\ref{eq:arb-condition})
follows from the no-arbitrage theorem.

This is clearer if we look at an example.
Suppose $x_{t+1} = \log d_{t+1} \sim \mathcal{N} (\kappa_1,\kappa_2)$.
Then (\ref{eq:arb-condition}) implies
\begin{eqnarray*}
    s_t &=& q^1_t e^{\kappa_1 + \kappa_2/2}.
\end{eqnarray*}
This says, essentially, that in risk-neutral terms
the underlying and the riskfree bond have the same expected return.
Typically we would estimate $\kappa_2$ somehow
then set $\kappa_1 = \log s_t -\log q_t^1 - \kappa_2/2$.
We'll see this again, so don't panic if it hasn't sunk in.
More generally, the no-arbitrage condition can be written
\begin{eqnarray*}
    s_t &=& q^1_t e^{k(1)},
\end{eqnarray*}
where $k(1)$ is the cumulant generating function evaluated at $s=1$
(and yes, $s$ means something different here).
Think about this if it's not clear to you.


For an option, the dividend or cash flow of the underlying is
the same asset at $t+1$:
$ d_{t+1} = q_{t+1}$.
We choose a risk-neutral distribution that satisfies the no-arbitrage condition.
Then we use it to value options.
Put and call options then follow from
\begin{eqnarray*}
        q^p_t &=&  q_t^1 E^* (k-s_{t+1})^+  \\
        q^c_t &=&  q_t^1 E^* (s_{t+1}-k)^+ .
\end{eqnarray*}
Further progress requires more information about the risk-neutral distribution.

At this point we have two choices.
One is to get lucky and come up with a relatively convenient formula.
The other is to apply some kind of numerical method.
In this context, a relatively simply one is to convert the
risk-neutral distribution into a discrete distribution defined
over some grid of points.
This is brute force, but it's reasonably general and easy to program.


\subsection*{The BSM model}


The BSM formula follows directly from using a lognormal risk-neutral distribution
for the underlying.  Everything else is just integration.
We'll revert to a maturity of 1, but come back to that in the following section.

We'll derive the put formula (\ref{eq:bsm-put})
using as an input a lognormal distribution
of the underlying:
$x_{t+1} = \log s_{t+1} \sim \mathcal{N} (\kappa_1,\kappa_2)$.
The no-arbitrage condition is therefore
\begin{eqnarray}
    s_t &=& q_t^1 \exp(\kappa_1 + \kappa_2/2 ) .
    \label{eq:bsm-no-arb}
\end{eqnarray}
We'll need this later.

The price of a put with strike $k$ is
\begin{eqnarray*}
    q^p_t &=&  q^1_t E^* (k - s_{t+1})^+  \\
            &=& q^1_t \int_{-\infty}^{\log k} (2 \pi \kappa_2)^{-1/2}
                    \exp[ - (x_{t+1} - \kappa_1)^2/2\kappa_2]
                    \left( k - e^{x_{t+1}} \right) d x_{t+1} \\
            &=& q^1_t k \int_{-\infty}^{\log k} (2 \pi \kappa_2)^{-1/2}
                    \exp[ - (x_{t+1} - \kappa_1)^2/2\kappa_2] d x_{t+1}
                    \;\;\; \mbox{(term 1)}\\
            &&    - \; q^1_t \int_{-\infty}^{\log k} (2 \pi \kappa_2)^{-1/2}
                    e^{x_{t+1}} \exp[ - (x_{t+1} - \kappa_1)^2/2\kappa_2] d x_{t+1} .
                     \;\;\; (\mbox{term 2})
\end{eqnarray*}
The rest is high school calculus:  tedious, but nothing particularly deep.
Term 1 is just the normal cdf evaluated at $x_{t+1} = \log s_{t+1} = \log k$:
\begin{eqnarray*}
    \mbox{term 1} &=&  q^1_t k N [ (\log k - \kappa_1)/\kappa_2^{1/2}] .
\end{eqnarray*}
This is just the discounted value of the strike price times
the risk-neutral probability the option is exercised.

Term 2 requires us to combine to two exponential terms, something
we've done a couple times before, starting with the normal
moment generating function (see notes on random variables).
The key step is showing
\begin{eqnarray*}
    e^{x_{t+1}} \exp[ - (x_{t+1} - \kappa_1)^2/2\kappa_2] &=&
        e^{\kappa_1 + \kappa_2/2} \exp\{ - [x_{t+1} - (\kappa_1+\kappa_2)]^2/2\kappa_2 \} .
\end{eqnarray*}
You can verify this by expanding the exponents on both sides.
That gives us
\begin{eqnarray*}
    \mbox{term 2} &=& q^1_t e^{\kappa_1 + \kappa_2/2}
            N \{ [\log k - (\kappa_1 + \kappa_2)]/\kappa_2^{1/2} \} .
\end{eqnarray*}
The put price is therefore
\begin{eqnarray}
    q^p_t &=& q^1_t k N (f) - q^1_t e^{\kappa_1 + \kappa_2/2}
            N ( f - \kappa_2^{1/2} ),
    \label{eq:bsm-put-kappas}
\end{eqnarray}
with
\begin{eqnarray*}
    f &=&  (\log k - \kappa_1)/\kappa_2^{1/2} .
\end{eqnarray*}
This is, in fact, a useful formula, and we'll come back to it later.
But it's not the BSM formula.

All that's left to derive the BSM formula (\ref{eq:bsm-put})
is to apply the no-arbitrage condition (\ref{eq:bsm-no-arb})
to (\ref{eq:bsm-put-kappas}).
That gives us
\begin{eqnarray}
    q^p_t &=& q^1_t k N (f) - s_t N ( f - \kappa_2^{1/2} ),
    \label{eq:bsm-put-kappas}
\end{eqnarray}
with
\begin{eqnarray*}
    f &=&  [\log (q^1_t k/s_t) + \kappa_2/2]/\kappa_2^{1/2} .
\end{eqnarray*}
If we substitute $\kappa_2 = \sigma^2$ and $\tau = 1$ (more on that next),
we get exactly the BSM put formula.

Bottom line:
If the risk-neutral distribution of the underlying
is lognormal,
then the volatility smile is flat.
Generally it's not,
so we need to consider other distributions.
That's the ultimate goal of this set of notes,
but we'll work up to it gradually.


\subsection*{Summing and dividing random variables}

The goal here is to adapt the BSM and related formulas
to time intervals different from a year.
That has some practical importance, because
most traded options have maturities significantly less than a year.

We'll attack this indirectly, starting with sums
of independent random variables.
Consider the sum $y = x_1 + x_2$, with $x_1$ and $x_2$ independent.
Since they're independent,
the joint probability is the product of the individual probabilities:
\begin{eqnarray*}
    p(x_1,x_2) &=& p_1(x_1) p_2(x_2) .
\end{eqnarray*}
As a result, the moment generating function of the sum
is the product of the individual mgfs:
\begin{eqnarray*}
    h(s; y) &=& E \left( e^{s (x_1 + x_2)} \right)
            \;\;=\;\; E \left( e^{s x_1} e^{x_2} \right)
            \;\;=\;\; E \left( e^{s x_1} \right)
                E \left( e^{s x_2} \right)
            \;\;=\;\; h(s; x_1) h(s; x_2) .
\end{eqnarray*}
Since the cumulant generating function is the log of the cgf,
the cgf of the sum is the sum of the cgfs:
\begin{eqnarray*}
    k(s; y) &=& \log h(s; y)
            \;\;=\;\; \log h(s; x_1) + \log h(s; x_2)
            \;\;=\;\; k(s; x_1) + k(s; x_2) .
\end{eqnarray*}
That gives us an easy way to find the distribution
of the sum:  we simply add the cgfs.

We can extend the idea to sums of multiple random variables.
Suppose, to keep things simple, that $ y = x_1 + x_2 + \cdots + x_n$,
with the $x_j$'s iid or independent and identically distributed.
That is, they're not only independent, they have the same distribution.
Then
\begin{eqnarray*}
    k(s; y) &=& \sum_j k(s; x_j)
            \;\;=\;\; n k(s; x) ,
\end{eqnarray*}
where $k(s; x)$ is the cgf of a particular $x$ ---
it doesn't matter which one, they're all the same.

We can always do this, but it doesn't always have a nice form.
Two examples that do are the normal and Poisson.
If $x_j \sim \mathcal{N}(\kappa_1, \kappa_2)$, m
then $ k(s; x_j) = \kappa_1 s + \kappa_2 s^2/2 $.
The cgf of the sum of $n$ such random variables is
the same thing multiplied by $n$:
\begin{eqnarray*}
    k(s; y) &=& n ( \kappa_1 s + \kappa_2 s^2/2 )
            \;\;=\;\; (n \kappa_1) s + (n\kappa_2) s^2/2 .
\end{eqnarray*}
That is, it's still normal, but the mean $\kappa_1$
and variance $\kappa_2$ are multiplied by $n$.

Sums of Poisson random variables also have a clean structure.
Recall that $x$ is Poisson, it assumes the values $x= 0, 1, 2, \ldots $
with probabilities
\begin{eqnarray*}
    p(x) &=& e^{-\omega} \omega^x / x! ,
\end{eqnarray*}
for some parameter $\omega > 0$, referred to as the ``intensity.''
Its cgf is (this is from the notes on random variables)
\begin{eqnarray*}
    k(s; x) &=& \omega \left( e^{s} -1 \right) .
\end{eqnarray*}
Now consider the sum $y$ of $n$ such random variables.
Its cgf is
\begin{eqnarray*}
    k(s; y) &=& n \omega \left( e^{s} -1 \right) .
\end{eqnarray*}
That is, we just multiply the intensity parameter $\omega$ by $n$.

The obvious application for us is time.
If have a random variable $y$ that applies to (say) a month,
and let the distribution be the same at all dates,
then we can construct the random variable for a year by adding
the appropriate number of months.
If we're lucky, we can also go the other way.
If we have the distribution for a year, we can divide it nicely
into months like this:
\begin{eqnarray*}
    k(s; \tau) &=&  \tau k(s; y) ,
\end{eqnarray*}
where $\tau = 1/12$ is the fraction of a year in one month.
[Awkward notation?]
It's not hard to see how this would work with normal
and Poisson random variables.

In the normal case, we see that the variable is scaled by $\tau$.
That's the one missing piece in our derivation.
We can now replace $\kappa_2$ with $\tau \sigma^2$ in our formulas,
and get the BSM formula for any maturity.

This doesn't work with all random variables.
For example, the Bernoulli has cgf
\begin{eqnarray*}
    k(s; y) &=& \log \left( 1-\omega + \omega e^s \right) .
\end{eqnarray*}
It's not divisible.


\subsection*{The Merton model}

If we want to account for observed option prices,
we evidently need a risk-neutral distribution that's not normal.
There's no shortage of examples.
We'll focus on one that's widely used in finance:
a combination of a normal random variable and a Poisson mixture of normals.
Robert Merton has the classic paper on the subject.

Let us say that
\begin{eqnarray*}
    \log s_{t+1} &=& x_1 + x_2 ,
\end{eqnarray*}
where
\begin{eqnarray*}
    x_1 &\sim& \mathcal{N}(\mu, \sigma^2)
\end{eqnarray*}
is the ``normal component.''
The so-called ``jump component'' $x_2$ is a Poisson mixture of normals.
If $j$ is a Poisson random variable, then conditional on $j$
\begin{eqnarray*}
    x_2 | j &\sim& \mathcal{N}(j\theta, j\delta^2) .
\end{eqnarray*}
The idea behind this is that during the relevant time interval,
we may get 0, 1, 2, or more jumps, each of which is normal
with mean $\theta$ and variance $\delta^2$.
If there are $j$ jumps, then all together we have
a normal random variable with $j\theta$ and variance $j\delta^2$.
When we compute the pdf for $x_2$,
we take each of these normal pdfs and compute the sum weighted
by the Poisson probabilities $e^{-\omega} \omega^j / j!$.


We won't burden ourselves with the algebra,
but we should mention two features of this setup.
The first is the no-arbitrage condition.
In this setting, it's the sum of the two components,
which have the form
\begin{eqnarray*}
    \log s_t - \log q^1_t &=& (\mu + \sigma^2/2 )
            + \omega \left( e^{\theta+\delta^2/2} - 1 \right) .
\end{eqnarray*}
You can see the normal and Poisson inputs here.
Given values of the other parameters, our operating procedure is
to choose $\mu$ to satisfy this equation.
The second feature is the time interval.
If we change from a time interval of one to $\tau$, we simply multiply
$(\mu, \sigma^2, \omega)$ times $\tau$.
That's the beauty of the Poisson, it adapts readily to different time intervals.
We'll take both of these on faith, but you have enough tools
to derive them yourself if you're interested.

If we add the two components together, we have a distribution
of $\log s_{t+1}$ is normal for every $j$:
\begin{eqnarray*}
    \log s_{t+1} | j &\sim& \mathcal{N}(\mu + j \theta, \sigma^2 + j \delta^2).
\end{eqnarray*}
Therefore it can be computed using the BSM-like formula (\ref{eq:bsm-put-kappas}).
The put price is the weighted average of these, with weights
on each $j$ of $e^{-\omega} \omega^j / j!$.
You can find a more explicit version in my paper on disasters.

This model has enough flexibility to reproduce the shapes
we see of volatility smiles in a range of markets.
[Talk about role of $\delta$ (generates curvature)
and $\omega$ (generates slope).]


\subsection*{Bottom line}

The BSM formula is the result of a model in which the risk-neutral distribution
 of the underlying is lognormal.
 It's most useful as a way to compute implied volatilities.
 The resulting volatility smiles show clearly that the distribution
 isn't lognormal.
That leaves us looking for alternatives,  including the Merton model.

\vfill \centerline{\it \copyright \ \number\year \
NYU Stern School of Business}
\end{document}
