\documentclass[11pt]{article}

% spacing on page
\oddsidemargin=0.25truein \evensidemargin=0.25truein
\topmargin=-0.5truein \textwidth=6.0truein \textheight=8.75truein

\usepackage{comment}
\usepackage{graphicx}
\usepackage{amssymb}
\usepackage{amsmath}

\usepackage[margin=0pt, labelsep=period, font=large, labelfont=bf]{caption}
%\usepackage{float}

\usepackage{hyperref}
\urlstyle{rm}   % change fonts for url's (from Chad Jones)
\hypersetup{
    colorlinks=true,        % kills boxes
%    allcolors=blue,
    pdfsubject={ECON-UB233, Macroeconomic foundations for asset pricing},
    pdfauthor={Dave Backus @ NYU},
    pdfstartview={FitH},
    pdfpagemode={UseNone},
%    pdfnewwindow=true,      % links in new window
%    linkcolor=blue,         % color of internal links
%    citecolor=blue,         % color of links to bibliography
%    filecolor=blue,         % color of file links
%    urlcolor=blue           % color of external links
% see:  http://www.tug.org/applications/hyperref/manual.html
}

% for listing code in tt font
\usepackage{verbatim}

% for table spacing
\usepackage{booktabs}

% section headers and spacing
\usepackage[small, compact]{titlesec}

% list spacing
\usepackage{enumitem}
\setitemize{leftmargin=*, topsep=0pt}
\setenumerate{leftmargin=*, topsep=0pt}

% attach files to the pdf
\usepackage{attachfile}
    \attachfilesetup{color=0.75 0 0.75}

\usepackage{needspace}
% \needspace{4\baselineskip} makes sure we have four lines available before a pagebreak

%\renewcommand{\thefootnote}{\fnsymbol{footnote}}
%\newcommand{\var}{\mbox{\it Var\/}}

\newcommand{\cbar}{\bar{c}}
\newcommand{\rp}{\mbox{\em rp\/}}
\newcommand{\tp}{\mbox{\em tp\/}}
\newcommand{\bp}{\mbox{\em bp\/}}


\begin{document}
\parskip=\bigskipamount
\parindent=0.0in
\thispagestyle{empty}
{\large ECON-UB 233 \hfill Dave Backus @ NYU}% 

\bigskip\bigskip
\centerline{\Large \bf Consumption, Saving, \& Portfolio Choice}
\centerline{Revised: \today}

\medskip
Another building block for thinking about business cycles
and asset returns is what we'll call the consumer's problem:
how an individual might decide between current consumption
and saving for future consumption,
and how that saving is allocated across assets.
We'll see how this works in a two-period setting,
which captures the essential ideas in a relatively simple setting.
The solutions, however, are not simple.
This is disappointing, but it's not insurmountable:
we can compute solutions numerically (see the problems) or use Merton's convenient
formula as an approximation.

The more important issue is that we get our first glimpse of the central equation in this course,
\begin{eqnarray}
    E (mr) &=& 1 .
    \label{eq:E(mr)=1}
\end{eqnarray}
Variants of this equation account for
an embarrassingly large part of macroeconomics and finance,
including virtually all of modern asset pricing.
Don't bother to memorize it,
you'll soon find that it's impossible to forget.


\section{Kuhn-Tucker conditions for constrained optimization}

The KT conditions are a solution method for a common class of optimization problems.
In the standard treatment, optimization means minimization.
Economists prefer to maximize, but the ideas are interchangeable:
maximizing $f(x)$ is the same as minimizing $-f(x)$.

The canonical constrained optimization problem in economics
is to choose a variable $x$ to maximize a function
$f(x)$ subject to a constraint $g(x) \geq 0$.
Formally, we might write it as
\begin{eqnarray*}
    \max_x && f(x) \\
    \mbox{subject to} && g(x) \geq 0 .
\end{eqnarray*}
We find the solution using the so-called Lagrangean function
\begin{eqnarray*}
    \mathcal{L}(x,\lambda) &=& f(x) + \lambda g(x) ,
\end{eqnarray*}
where $\lambda$ is called the {\it Lagrange multiplier\/}.
The first-order (necessary) conditions for a maximum are
\begin{eqnarray*}
   \partial \mathcal{L}/\partial x \;\;=\;\; f'(x) + \lambda g'(x) &=&  0 \\
   \lambda g(x) &=& 0 ,
%    g(x) &=& 0 \mbox{ if } \lambda > 0 .
\end{eqnarray*}
with  $\lambda \geq 0$ and $g(x) \geq 0$.
These are first-order conditions,
but with the right structure --- which includes any problem in this course ---
they are also sufficient.
Typically it's sufficient for $f$ and $g$ to be concave.

The Lagrange multiplier is the value, in units of $f$,
of loosening the constraint $g$ a little.
Suppose we're at an optimum $x^*$.
Then the value of our function is $f(x^*) = f(x^*) + \lambda g(x^*)$.
That works because of the second condition:  $\lambda g(x^*) = 0 $.
Now suppose we loosen the constraint a little to $g(x^*) + \varepsilon$,
where $\varepsilon$ is a small positive number.
Then the value of the function rises to $ f(x^*) + \lambda \varepsilon$.
Economists refer to it as a ``shadow price,''
the price of the constraint.
In some economic applications the analogy goes further:
the multipliers are the prices of the goods to which the constraint applies.
We'll see that in the next chapter.
% ?? loose!!

What we've done here is add an extra variable $\lambda$
and an extra equation to the usual unconstrained maximization problem.
The second line says that if the multiplier is positive,
the constraint binds (holds with equality):  $g(x) = 0$.
But if $g(x) > 0$, the multiplier must be zero:  $\lambda = 0$.
{\it If we're sure the constraint binds, we can replace the second equation
with $ \partial \mathcal{L}/\partial \lambda = g(x) = 0$.
That will be true for every example we study in this class.\/}

The same idea extends to problems where $x$ and $g$
are vectors.
If $x$ is a vector, we differentiate $\mathcal{L}$ with
respect to each of them.
And if $g$ is a vector, we make $\lambda$ a vector of the same
dimension.
You'll see how this works when we put it to work.


{\it Example.\/}
I stole this one from Tom Sargent, not sure where he got it, but it's clever.
Let $f(x) =- (x+1)^2/2$  and $g(x) = x \geq 0$.
The Lagrangean is
\begin{eqnarray*}
    \mathcal{L} &=& - (x+1)^2/2 + \lambda x .
\end{eqnarray*}
The first-order conditions are
\begin{eqnarray*}
    -(x+1) + \lambda &=& 0 \\
    \lambda x &=& 0.
\end{eqnarray*}
The first equation implies $\lambda = x + 1$.
If we substitute this into the second, we get $x(x+1) = 0$,
which is quadratic and therefore has two solutions.
One is $x=-1$, which violates the constraint.
The answer is evidently the other solution, $x=0$, which implies $\lambda = 1$.


{\it Example.\/}
Consider an agent with income $y$ who is deciding
how many apples $a$ and bananas $b$ to consume.
Her utility function is
\begin{eqnarray*}
    U(a,b) &=& \beta \log a + (1-\beta) \log b .
\end{eqnarray*}
If prices of apples and bananas are $q_a$ and $q_b$,
respectively, her budget constraint is
\begin{eqnarray*}
    q_a a + q_b b  &\leq& y  .
\end{eqnarray*}
The inequality is there to give this the same form as the
generic problem, but you can guess that it binds:
there's no reason here not to spend all your money.

We find the solution using the same methods.
The Lagrangean is
\begin{eqnarray*}
    \mathcal{L} &=& \beta \log a + (1-\beta) \log b
            + \lambda \left[ y - (q_a a + q_b b) \right] .
\end{eqnarray*}
The first-order conditions are
\begin{eqnarray*}
        \beta / a &=& \lambda q_a \\
        (1-\beta) / b &=& \lambda q_b \\
   \lambda \left[ y - (q_a a + q_b b) \right] &=& 0 .
\end{eqnarray*}
The solution includes $\lambda = 1/y $,
$ a = \beta y/q_a $, and $b = (1-\beta) y /q_b $.


\section{Consumption and saving}

We approach consumption and saving the same way we approached
consumption of apples and bananas.
The goods here are consumption now ($c_0$)
and consumption later ($c_1$).
The utility function has a special form we use for dynamic
problems,
\begin{eqnarray}
    U(c_0,c_1) &=& u(c_0) + \beta u(c_1) ,
    \label{eq:utility-2period}
\end{eqnarray}
where $\beta$ is the {\it discount factor\/} (typically a number between zero and one)
and $u$ is some convenient function.
We'll say that $u$ is increasing ($u' > 0$) and concave ($u'' < 0$).
Power utility is a common choice:
$u(c) = c^{1-\alpha}/(1-\alpha)$ with $\alpha > 0$.
Log utility $u(c) = \log c$ corresponds to $\alpha = 1$.
[This is a tricky one, ask me if you're interested.]

We're going to think about the problem of choosing
$c_0$ and $c_1$ all at once, at date 0.
At date 1, we simply execute our plan:
we consume current income and the proceeds of any saving we do at date 0.
More on that shortly.
If incomes in the two periods are $(y_0,y_1)$,
the natural budget constraint is
\begin{eqnarray*}
   q_0 c_0 + q_1 c_1 &\leq& q_0 y_0 + q_1 y_1  .
%    \label{eq:budget-2period}
\end{eqnarray*}
We don't need both prices, all we need
is the relative price $Q = q_1/q_0$.
Then we can rewrite the constraint as
\begin{eqnarray}
    c_0 + Q c_1 &\leq& y_0 + Q y_1  \;\;=\;\; Y .
    \label{eq:budget-2period}
\end{eqnarray}
Here $Q$ is the date-0 price of one unit of date-1 consumption in ``$c_0$ units.''
It may look more familiar if we write $Q = 1/(1+r)$, so that
$Q$ discounts date-1 quantities at the appropriate rate.
We'll write $Q = 1/r$ instead, simply because we're lazy:
it's easier to write $r$ than $1+r$, especially if you have
to do it many times.
We refer to $r$ defined this way as the {\it gross\/} interest rate.


Now what, you might ask, does this have to do with saving?
The choice of first-period consumption also dictates a choice for saving.
Since $Q = 1/r$, we can rewrite the budget constraint as
\begin{eqnarray*}
    c_0 + (1/r) c_1 &\leq& y_0 + (1/r) y_1 .
\end{eqnarray*}
If we define saving as $s = y_0 - c_0$,
this becomes
\begin{eqnarray*}
    c_1  &\leq& r s +  y_1 .
\end{eqnarray*}
That is:  tomorrow's consumption is tomorrow's income
plus today's saving with interest.
We'll ignore saving from now on, but it's there behind the scenes.

We now have the pieces of the consumption-saving problem:
the utility function (\ref{eq:utility-2period})
and the budget constraint (\ref{eq:budget-2period}).
We attack it with the usual Lagrangian methods
using $\lambda$ as the Lagrange multiplier on the constraint.
[What is the Lagrangian function?]
The first-order conditions are
\begin{eqnarray*}
        u'(c_0)  &=& \lambda  \\
       \beta u'(c_1)  &=& \lambda Q
\end{eqnarray*}
plus the budget constraint, which we assume binds (it does).

One implication of these conditions is a relation
between consumption and the interest rate.
If we take the ratio of the two first-order conditions,
we have
\begin{eqnarray*}
       \frac {\beta u'(c_1)}{u'(c_0)}  &=& Q \;\;=\;\; 1/r.
\end{eqnarray*}
This says that the marginal rate of substitution
between current and future consumption equals
the relative price.
If we denote the marginal rate of substitution or mrs by $m$, we have
\begin{eqnarray}
       \frac{\beta u'(c_1)}{u'(c_0)} \ r  &=& m r \;\;=\;\;  1 .
       \label{eq:euler-deterministic}
\end{eqnarray}
This is a variant of (\ref{eq:E(mr)=1});
since there's no risk, we don't need the expectation $E$.

[Draw budget line and indifference curves, show tangency.
Budget line is $c_0 + c_1/r = Y$, so slope in $(c_0,c_1)$-space
is $-r = - 1/Q$.]

So what do we have here?  We have a connection between
preferences, consumption, and the interest rate.
If we discount the future more ($\beta$ is smaller),
then for a given interest rate we'll choose
more consumption now and less later.
This is implied because $u'$ is decreasing (recall $u''<0$).
Makes sense:  if we put less weight on the future,
we allocate less of our income to future consumption.
The interest rate works the same way.
If we raise $r$, which makes future consumption cheaper,
we'll do more of it.
With power utility, we can be even more specific:
$u'(c) = c^{-\alpha}$ and
\begin{eqnarray*}
       \beta (c_1/c_0)^{-\alpha}  r  &=& 1 .
\end{eqnarray*}
Here a high interest rate is associated with
high consumption growth,
hence high future consumption and low current consumption.

This connection between consumption and the interest rate
will hold in all of our models.
Here the interest rate is given and we choose consumption.
In some settings, we reverse the causality
and say that the high interest rate is a consequence
of high future consumption growth.
More generally, they're both outcomes of market decisions.
However we interpret it, the positive association between the interest rate
and consumption growth is something we see in the data:
interest rates are high, for the most part,
when the economy (including consumption) is growing quickly
and low when it's not.


{\it Example (log utility).\/}
Let's return to our problem and solve it explicitly
for the case of log utility.
We solve the first-order conditions for consumption:
\begin{eqnarray*}
        c_0  &=& 1/\lambda  \\
        c_1  &=& \beta/(\lambda Q) .
\end{eqnarray*}
If we substitute into the budget constraint,
we have
\begin{eqnarray}
    c_0 + Q c_1 &=& \lambda^{-1}
            \left(1 + \beta \right) \;\;=\;\; Y.
\end{eqnarray}
That gives us (see the first-order conditions)
\begin{eqnarray*}
        c_0  &=& [1/(1+\beta)] Y \\
        c_1  &=& [\beta/(1+\beta)] Y / Q  .
\end{eqnarray*}
This will look familiar from the second example in the previous section.

%You might try the more general power utility case yourself.


\section{Consumption and portfolio choice 1:  Arrow securities}

We can do the same thing in risky settings.
We'll continue our two-period environment,
but let the state $z$ in the second period be random.
We treat consumption in each of these states
as separate decisions, which we choose in the usual way
at date 0.

[Draw event tree.]

Our agent is now choosing current consumption $c_0$
and future consumption $c_1(z)$ for each state $z$.
Utility is
\begin{eqnarray}
    u(c_0) + \beta \sum_z p(z) u[c_1(z)]
    &=&     u(c_0) + \beta E u(c_1) ,
    \label{eq:utility-2period-stochastic}
\end{eqnarray}
where $p(z)$ is the probability that state $z$ occurs
(remember, rational expectations).
The budget constraint becomes
\begin{eqnarray}
    c_0 + \sum_z Q(z) c_1(z) &\leq& y_0 + \sum_z Q(z) y_1(z) \;\;=\;\; Y ,
    \label{eq:budget-2period-stochastic}
\end{eqnarray}
where $Q(z)$ is today's price of one unit of date-1 consumption in state $z$.


We refer to $Q(z)$ as the {\it state price\/}:
for $Q(z)$ units of date-0 consumption today, we get one
unit of date-1 consumption if state $z$ occurs --- and nothing otherwise.
Another way to think about $Q(z)$ is that it's the price of a risky asset
that we call an {\it Arrow security\/}.
It's like a digital option:  fixed payoff in some situations, nothing in others.
If you're familiar with linear algebra,
you might guess that {Arrow securities}
form a convenient basis for the state space.
In words, this means that it's easy to see how you could
use this collection of assets to construct any consumption outcome you want.
One of the central results of modern asset pricing is that
we can generally do this with assets, too:  decompose them
into collections of Arrow securities and value them with state prices.
More on that in a couple weeks.

{\it Example (``complete markets'').\/}
Suppose we have two states and two assets.
Asset 1 is a riskfree bond, paying one in each state.
Asset 2 we'll call equity.  It pays one in state 1, two in state 2.
We can think of each of them as collections of the two Arrow securities.
The bond consists of one unit of each Arrow security.
Why?  Because Arrow security $z$ pays one in state $z$, nothing in other states.
So if we have one unit of each Arrow security,
we have the same payoff in all states as the bond.
Equity pays one in state 1, two in state 2, and three in state 3,
so we can replicate its payoffs with one unit of the first Arrow security,
two units of the second, and three of the third.

This replication allows us to compute the prices of the assets from the state prices.
If $q^1$ is the price of the bond and $q^e$ is the price of equity, we have
\begin{eqnarray*}
    q^1 &=& Q(1) + Q(2) \\
    q^e &=& Q(1) + 2 Q(2) .
\end{eqnarray*}
In this case, because the numbers of states and assets are the same,
we can also do the reverse:  if we know the prices of the assets,
we can find the prices of the Arrow securities.


{\it Example  (``incomplete markets'').\/}
Here's another example, which makes a subtle point we'll come
back to later on.
Suppose we have three states and two assets.
Asset 1 is the riskfree bond and pays one in each state.
Asset 2 is equity and pays one in state 1, two in state 2, and one in state 3.
Replication here gives us
\begin{eqnarray*}
    q^1 &=& Q(1) + Q(2) + Q(3) \\
    q^e &=& Q(1) + 2 Q(2) + 1 Q(3) .
\end{eqnarray*}
Is there a solution?
Well, we have more unknowns than equations, so there
should be lots of solutions.
Suppose, for example, $q^1 = 1 $ and $q^e = 4/3$.
Then $ Q(1) = Q(2) = Q(3) = 1/3$ is a solution.
But so is $Q(2) = 1/3$ and any combination of $Q(1)$ and $Q(3)$ that sums to 1/3.
Some of these solutions give us negative state prices,
which we'll have to think about, but not all of them.

Back to our problem:  choose $\{c_0, c_1(z)\}$ to maximize
utility (\ref{eq:utility-2period-stochastic}) subject to the budget constraint
(\ref{eq:budget-2period-stochastic}).
If we use the Lagrange multiplier $\lambda$ on the budget constraint,
the first-order conditions are
\begin{eqnarray*}
    u'(c_0)  &=& \lambda  \\
       \beta p(z) u'[c_1(z)]  &=& \lambda Q(z),
\end{eqnarray*}
with a version of the second equation for every value of $z$.
The ratio of the second equation to the first is
\begin{eqnarray}
    \beta p(z) u'[c_1(z)]/u'(c_0) &=& Q(z) ,
    \label{eq:con-stateprices}
\end{eqnarray}
a convenient expression relating consumption to state prices.
This is the usual condition equating the marginal rate of substitution
to the relative price, but here we do it in every state.

\begin{comment}
The definition $ r(z) = 1/Q(z)$ gives us
\begin{eqnarray*}
       \{ \beta p(z) u'[c_1(z)] / u'(c_0) \} r(z)  &=& 1 ,
\end{eqnarray*}
a variant of our favorite equation.
It's not all that useful here,
because $r(z)$ for these assets is zero in all states but one,
but if we sum over $z$ we get
\begin{eqnarray*}
     \sum_z  \frac { \beta p(z) u'[c_1(z)] }{ u'(c_0)}\ r(z)  &=&
     E \left( \frac {\beta u'(c_1)}{u'(c_0)} \ r \right)
     \;\;=\;\; 1 ,
     \label{eq:euler-stochastic}
\end{eqnarray*}
a stochastic version of (\ref{eq:euler-deterministic}).
It's a somewhat strange object in this context,
since $r(z)$ is zero in all states by one,
but it illustrates the underlying unity of what we're doing.
\end{comment}


\section{Consumption and portfolio choice 2:  arbitrary assets}

Here's a version with more explicit portfolio choice
and an arbitrary set of assets which in principle could
look more like the assets we see in modern economies.
We start with a description of assets and returns.

{\it Assets and returns.\/}
We need to be clear about what we mean by an asset.
An asset here is a claim to a random ``dividend,''
with dividend defined broadly to include anything the asset is a claim to.
Let us say, to be concrete, that one unit of asset $j$ is a claim to $d^j(z)$ units
of the consumption good at date 1 in state $z$.
Different dividends constitute different assets.
To complete the notation,
let us say that asset $j$ sells at date 0
for $q^j$ units of the date-0 consumption good.
So we pay $q^j$ at date 0 for $d^j(z)$ at date 1 in state $z$.
Uncertainty over the payoff is built into the dependence
of $d^j$ on the state $z$.

[Draw event tree. Show when prices are quoted, and when dividends are paid.]

The (gross) {\it return\/} on asset $j$ between dates 0 and 1
is the ratio of the dividend to the price: $r^j(z) = d^j(z)/q^j$.
Since $d^j(z)$ and $r^j(z)$ are functions of the state $z$,
they're random variables as we defined them earlier.
We reserve $j=1$ for the (one period) riskfree asset,
which has constant dividend $d^1(z) = 1$ and price $q^1$.

We generally define the {\it excess return\/} on asset $j$ as the difference
between its return and the return on the riskfree bond:
$ x^j(z) = r^j(z) - r^1 $.
The mean excess return, $ E[x^j(z)]$ is typically referred
to as the {\it risk premium\/}.


In most applications, asset prices are arbitrary but returns are not.
What on earth could that mean?
It means that the units are arbitrary.
The price of equity depends on how we define a share.
Suppose the price for some definition of a share is $q^j$.
Then the price of two shares is $2q^j$.
These two shares pay a dividend of $2 d^j(z)$.
The return is therefore $ 2d^j(z)/(2^jq) = d^j(z)/q^j$,
which is the same as the return on one share.
That will come as no surprise, but it's helpful to remember
that the units --- the definition of a share --- don't matter.
In the next section, we'll use this to define
units so that the price is one.
In this case the dividend is the return:  $r^j(z) = d^j(z)/q^j = d^j(z)$.

{\it Portfolio choice.\/}
Now that we know what assets are, we can describe portfolio choice.
Here the problem consists of the same preferences,
equation (\ref{eq:utility-2period-stochastic}),
and a suitably modified budget constraint.
At date 0, the agent's income goes either to consumption
or asset purchases:
\begin{eqnarray}
    c_0 + s \;\;=\;\; c_0 + \sum_j a^j q^j &=& y_0 ,
    \label{eq:budget-2period-c0}
\end{eqnarray}
where $a^j$ ($a$ for asset) is the number of units
purchased of asset $j$ and $q^j$ is the price of each unit.
At date 1, the agent consumes her income
plus the payoffs from her asset positions:
\begin{eqnarray}
    c_1(z)  &=& y_1(z) + \sum_j a^j d^j(z) .
    \label{eq:budget-2period-c1}
\end{eqnarray}
Once we choose asset positions $a^j$,
this determines consumption $c_1(z)$.


We solve the problem by our usual methods.
We maximize (\ref{eq:utility-2period-stochastic})
subject to the constraints (\ref{eq:budget-2period-c0}) and (\ref{eq:budget-2period-c1}).
We apply the Lagrange multipliers $\lambda_0$ and $\lambda_1(z)$ (one for each $z$) to the constraints,
giving us the Lagrangian
\begin{eqnarray*}
    \mathcal{L} &=& u(c_0) + \beta \sum_z p(z) u[c_1(z)] \\
        &&   + \; \lambda_0 \Big[ y_0 - c_0 - \sum_j a^j q^j \Big]
        + \sum_z \lambda_1(z) \Big[ y_1(z) + \sum_j a^j d^j(z) - c_1(z) \Big] .
\end{eqnarray*}
The first-order conditions include
\begin{eqnarray*}
  c_0: &        0 \;\;=& u'(c_0) - \lambda_0 \\
  c_1(z): &     0 \;\;=& \beta p(z) u'[c_1(z)] - \lambda_1(z) \\
  a^j: &        0 \;\;=& - q^j \lambda_0 + \sum_z \lambda_1(z) d^j(z) .
\end{eqnarray*}
There are lots of the second equation, one for each state $z$.
The ratio of the second to the first gives us, for each $z$,
\begin{eqnarray*}
    \beta p(z) u'[c_1(z)] / u'(c_0) &=& \lambda(z)/\lambda_0 .
\end{eqnarray*}
If we relabel by $Q(z) = \lambda(z)/\lambda_0$, we have the same relation
we had earlier connecting  the marginal rate of substitution to state prices,
equation (\ref{eq:con-stateprices}).
Evidently there are state prices hiding behind the scenes here,
even though we made no mention of Arrow securities when we set up the problem.

We can also express asset prices in terms of state prices.
The third equation can be written
\begin{eqnarray*}
    q^j &=& \sum_z Q(z) d^j(z) ,
\end{eqnarray*}
which is another equation we'll run across again.
It decomposes the price of the asset into pieces associated with each state.
We did this by example in the previous section.

Finally, we can write the solution in the form of
our asset pricing relation (\ref{eq:E(mr)=1}).
If we substitute
$ Q(z) = \beta p(z) u'[c_1(z)] / u'(c_0) = p(z) m(z)$
and $r^j(z) = d^j(z)/q^j $,
we have
\begin{eqnarray*}
    1 &=& \sum_z Q(z) r^j(z) \;\;=\;\; \sum_z p(z) m(z) r^j(z) \;\;=\;\; E (mr^j) ,
\end{eqnarray*}
the equation we promised at the start.
Here the asset price and return are given,
and the consumer's choice leads to a connection with consumption.
Later on we'll turn that around:
take consumption decisions as given and explore
the implications for asset prices.

Remark:  I got this version
of the problem from Tom Sargent.
You could attack the portfolio decision directly,
substituting the $c$'s into the utility function
and maximizing with respect to the $a^j$'s.
The advantage is this approach is that the state prices
pop out from the multipliers.


\section{Merton's portfolio formula}

Portfolio problems don't typically lend themselves
to simple solutions.
You can solve the first-order conditions numerically,
but that leaves us without much in the way of sharp insights.
Here's an example, though, that works.

We'll change the notation and consider the portfolio decision over two assets,
one risky (``equity'') and one not (``riskfree bond'').
Our agent has income $y_0$ at date 0, consuming $c_0$ and investing the rest,
an no income at date 1 other than the proceeds of from investments.
If she invests a fraction $a$ in the risky asset, date-1 consumption is
\begin{eqnarray*}
    c_1(z) &=& (y_0 - c_0) [(1-a) r^1 + a r^e(z)],
\end{eqnarray*}
where $r^1$ is the (gross) return on the bond and $r^e$ is the return on equity.
You might verify for yourself that this is a two-asset version
of the problem we looked at in the previous section,
expressed in somewhat different notation.

The agent's decision problem is therefore to choose $(c_0,a)$ to maximize
\begin{eqnarray*}
    u(c_0) + \beta \sum_z p(z) u \left\{ (y_0 - c_0) [(1-a) r^1 + a r^e(z)] \right\} .
\end{eqnarray*}
The first-order conditions are
\begin{eqnarray*}
    c_0 :&  0 &=\;\; u'(c_0) - \beta \sum_z p(z) u'[c_1(z)] [(1-a) r^1 + a r^e(z)] \\
    a:&     0 &=\;\; \beta \sum_z p(z) u'[c_1(z)] (y_0-c_0) [ r^e(z)-r^1] .
\end{eqnarray*}
Once you substitute for $c_1(z)$, you have two equations in the two unknowns,
$c_0$ and $a$.

This simplifies if we have power utility:
$u(c) = c^{1-\alpha}/(1-\alpha)$ with $\alpha > 0$.
Then $u'(c) = c^{-\alpha} $ and the second equation becomes
\begin{eqnarray*}
    0 &=& \beta \sum_z p(z) \left\{ (y_0-c_0)[(1-a) r^1 + a r^e(z)]\right\}^{-\alpha} (y_0-c_0) [ r^e(z)-r^1] \\
    0 &=& \sum_z p(z) [(1-a) r^1 + a r^e(z)]^{-\alpha} [ r^e(z)-r^1] .
\end{eqnarray*}
Note the value of power utility:  the portfolio weight $a$ doesn't depend
on the scale of consumption and can be solved separately.
You can imagine calculating the rhs in Matlab
and varying $a$ until it equals zero.

Now the insight.  You might guess that how much equity you hold
depends on its risk and return (the distribution of its return)
and the agent's preference toward risk (risk aversion, $\alpha$).
Merton's portfolio formula makes this precise:
\begin{eqnarray*}
    a &=& \frac{1}{\alpha} \frac{E (r^e - r^1)}{\mbox{Var}(r^e)} .
\end{eqnarray*}
This combines several ingredients:
the expected excess return on the risky asset $E (r^e - r^1)$,
its risk $\mbox{Var}(r^e)$,
and the agent's risk aversion $\alpha$.
%And it requires a lognormal distribution of $r^e$.
The result is fragile, in the sense that many modest changes to the problem
change the answer as well,
but it's a useful starting point for an unusually tricky problem.

The result is harder to prove than say, so we won't do it.
Maybe another time.
If you come up with Merton's formula as some kind of approximation
of the first-order conditions, let me know.

% ??
[Albert Zevelev suggests I look into Stein's lemma, but I haven't
had a chance yet.]
% see p 150 or so of
% https://www.dropbox.com/s/j1jq7yie4oo0qff/CLOSED_FORM.pdf?dl=0

\begin{comment}
Merton derived it in continuous time using a set of tools
we will diligently avoid.
Here we might start with the lognormal assumption.
Let us say that $x = \log r^e - \log r^1 \sim \mathcal{N}(\kappa_1,\kappa_2)$.
Then $r^e = r^1 e^x$ and
$r^e-r^1 = r^1 (e^x-1)$.
\end{comment}


\section*{Bottom line}

We derived the fundamental equation of the course for the first time:
\begin{eqnarray*}
    E \left( \frac{\beta u'(c_1)}{u'(c_0)}\; r^j \right) &=&
        E \left( m r^j \right) \;\;=\;\;  1
\end{eqnarray*}
for (gross) returns $r^j$ on all assets $j$.
Here it's the first-order condition for a consumer's portfolio choice
problem.
Later on it will show up as an asset pricing relation.
We also defined Arrow securities and state prices,
two theoretical concepts that reappear even in practical applications.

\section*{More}

Wikipedia has a nice
\href{http://en.wikipedia.org/wiki/Merton's_portfolio_problem}{entry}
on Merton's portfolio problem.


\section*{Practice problems}

\begin{enumerate}
\item {\it Lagrangian methods.\/}
Consider the constrained optimization problem:  choose $\{x,y\}$
to maximize
\begin{eqnarray*}
    f(x,y) &=& x + \log y
\end{eqnarray*}
subject to $ x + y \leq 7$.
%
\begin{enumerate}
\item What is the Lagrangian function associated with this problem?
\item What are the first-order conditions?
\item What values of $x$ and $y$ solve the problem?
\end{enumerate}
%
\needspace{2\baselineskip}
Answer.
\begin{enumerate}
\item The Lagrangian is
\begin{eqnarray*}
    \mathcal{L} &=& x + \log y  + \lambda (7 - x - y ).
\end{eqnarray*}
\item The first-order conditions are
\begin{eqnarray*}
    1 &=& \lambda \\
    1/y &=& \lambda .
\end{eqnarray*}
\item The solution is $\lambda = 1 = y$ and $x = 6$.
\end{enumerate}

\item {\it Consumption, saving, and the interest rate.\/}
Consider the two-period consumer problem:  choose consumptions $(c_0,c_1)$
to maximize utility
\begin{eqnarray*}
    U(c_0,c_1) &=& c_0^{1-\alpha}/(1-\alpha) + \beta c_1^{1-\alpha}/(1-\alpha)
\end{eqnarray*}
subject to the budget constraint $ c_0 + Q c_1 \leq y_0 $
(that is, there's no date-1 income).  %(\ref{eq:budget-2period}).
%
\begin{enumerate}
\item What is the Lagrangian function associated with this problem?
What are the first-order conditions?
\item Express consumptions as functions of income $y_0$ and the
price $Q$.
\item How does date-0 consumption $c_0$ depend on $Q$?
\item How does saving depend on $Q$?  On the interest rate?
\end{enumerate}
%
\needspace{2\baselineskip}
Answer.
\begin{enumerate}
\item The Lagrangian is
\begin{eqnarray*}
    \mathcal{L} &=& c_0^{1-\alpha}/(1-\alpha) + \beta c_1^{1-\alpha}/(1-\alpha)
            + \lambda (y_0 - c_0 - Q c_1 ).
\end{eqnarray*}
The first-order conditions are
\begin{eqnarray*}
        0 &=& c_0^{-\alpha} - \lambda \\
        0 &=& \beta c_1^{-\alpha} - \lambda Q .
\end{eqnarray*}
\item We have the two first-order conditions plus the budget constraint
to solve for the three unknowns $(c_0,c_1,\lambda)$.
We solve the foc's for the $c$'s, then substitute back into the budget constraint:
\begin{eqnarray*}
    \lambda^{-1/\alpha} &=& y_0/ \big( 1 + Q^{1-1/\alpha} \beta^{1/\alpha} \big) .
\end{eqnarray*}
Now we plug this into the foc's to get
\begin{eqnarray*}
    c_0 &=& \lambda^{-1/\alpha}  \;\;=\;\;
            y_0/ \big( 1 + Q^{1-1/\alpha} \beta^{1/\alpha} \big) \\
    c_1 &=& (\lambda Q/\beta)^{-1/\alpha}  \;\;=\;\;
            y_0 Q^{-1/\alpha} \beta^{1/\alpha} / \big( 1 + Q^{1-1/\alpha} \beta^{1/\alpha} \big).
\end{eqnarray*}
\item The impact of an increase in $Q$ on $c_0$ depends on $\alpha$.
If $\alpha<1$, $c_0$ rises, otherwise it falls.

\item Since saving is $ s = y_0-c_0$,
the impact of an increase in the interest rate is the reverse of the impact on $c_0$.
We change signs again if we use the interest rate, which is connected to $Q$ by $Q=1/r$.

\end{enumerate}

\item {\it Arrow securities 1.\/}
Consider an economy with two assets and two states.
Asset 1 (a bond) pays 1.0 in each state and sells for $q^1 = 0.9$.
Asset 2 (equity) pays 1.0 in state 1, 1.5 in state 2, and sells for 1.2.
\begin{enumerate}
\item What is the return on the bond?
\item What is the return on equity?
\item Describe how each asset can be decomposed into a collection of Arrow securities.
\item What are the prices of the Arrow securities (aka the state prices)?
\end{enumerate}
%
\needspace{2\baselineskip}
Answer.
\begin{enumerate}
\item Returns are ratios of payoffs $d(z)$ to price $q$.
For the bond, the payoff is the same in both states, so we have
$ r^1 = d/q = 1/0.9 $.
\item For equity, the return depends on the state:
\begin{eqnarray*}
    r^e(z) &=& \left\{
                \begin{array}{ll}
                1/1.2 & \mbox{if } z=1 \\  1.5/1.2 & \mbox{if } z=2 .
                \end{array}
                \right.
\end{eqnarray*}
\item The descriptions tell us how the assets are connected to Arrow securities.
The bond consists of one unit of each Arrow security.
Equity consists of one unit of the first Arrow security and 1.5 units of the second.
\item Prices of assets and Arrow securities are connected by
\begin{eqnarray*}
    q^1 &=& 1.0 Q(1) + 1.0 Q(2) \\
    q^e &=& 1.0 Q(1) + 1.5 Q(2) .
\end{eqnarray*}
The prices of Arrow securities are
$Q(1) = 0.3$ and $Q(2) = 0.6$.
\end{enumerate}

\item {\it Arrow securities 2.\/}
Consider an economy with two assets and three states.
Asset 1 (a bond) pays one in each state and sells for $q^1 = 13/12$.
Asset 2 (equity) pays one in state 1, two in state 2, and three in state 3,
and sells for 27/12.

Can you find the prices of the Arrow securities?
What would you suggest?
%
\needspace{2\baselineskip}
Answer.
As before, the descriptions tell us how the assets are connected to Arrow securities.
They give us
\begin{eqnarray*}
    q^1 &=& 1.0 Q(1) + 1.0 Q(2) + 1.0 Q(3) \\
    q^e &=& 1.0 Q(1) + 2.0 Q(2) + 3.0 Q(3) .
\end{eqnarray*}
% Q = 1/4, 1/3, 1/2 works
We have three unknowns but only two equations, so we have some ambiguity over
the state prices.
There are several things we could do about this.
One is get more assets.  We could consider equity options, for example.
Another is not to worry about it:  any solution to the two equations prices
the assets correctly.
Yet another is to choose the ``smallest'' vector $Q$ that works.

\item {\it Portfolio choice.\/}
Portfolio choice problems are notoriously unfriendly,
but we can get a sense of their properties with an example.
The agent's consumption-saving-portfolio-choice problem is
\begin{eqnarray*}
   \max_{c_0, a} &&  u(c_0) + \beta \sum_z p(z) u[c_1(z)] \\
   \mbox{subject to} &&  c_1(z)\;\;=\;\; (y_0-c_0)[(1-a) r^1 + a r^e(z)] .
\end{eqnarray*}
The idea is that we have two assets,
a riskfree asset with gross return $r^1$ (``$1$'' for one-period bond)
and a risky asset with gross return $r^e$  (``$e$'' for equity).
We invest a fraction $a$ of our saving $s = y_0 - c_0$
in equity and the complementary
fraction $1-a$ in the riskfree bond.
If $a>1$, the agent has a levered position.
Note that this version of the problem is slightly different from the one we used in class.

We'll deal with a special case:
two states, $z=1$ and $z=2$;
equally likely, $p(1) = p(2) = 1/2$;
and power utility, $u(c) = c^{1-\alpha}/(1-\alpha)$.
Parameter values include
$\beta = 1/1.1$,
$r^1 = 1.1$,
$r^e(1) = 1.0$, $r^e(2) = 1.4$,
$ y_0 = 1$.

{\it Note:  Utility isn't defined
if $c_1(z)$ is negative.
That restricts your portfolio choice.\/}

\begin{enumerate}
\item What are the mean and variance of the return on equity?
\item What are the implied prices of Arrow securities?
%Hint:  Recall that Arrow securities pay off in one state only.
%Our assets are combinations of Arrow securities,
%so it's a question of unbundling them.
\item What are the first-order conditions for $c_0$ and $a$?
Show that with power utility,
the latter can be determined without knowing $c_0$.
\item Solve these conditions numerically for $\alpha = 5$.
What values do you get for $a$, $c_0$, $c_1(1)$, and $c_1(2)$?
%Comment:  I did this
%by varying $a$ until its first-order condition was satisfied.
%You could also compute the first-order condition for a grid of values for $a$,
%and choose the one that works best.
\item How does your answer change if you use $\alpha = 2$?
Does the difference make sense to you?
\item How do your answers compare to Merton's portfolio formula?
\end{enumerate}
%
\needspace{2\baselineskip}
Answer.
\begin{enumerate}
\item The mean is 1.2 (the average of 1.0 and 1.4)
and the variance is $0.2^2 = 0.04$.

\item
Let $Q(z)$ be the price of the Arrow security that pays one
in state $z$.
A one-period bond pays one in each state, so its price is the sum:
\begin{eqnarray*}
    q^1 \; (=1/r^1)  &=&  Q(1) + Q(2)  .
\end{eqnarray*}
Equity is a little more complicated, because we have to agree on
units.  Let's choose units to that the price of a share is one
unit of date-0 consumption.
Then it pays off 1.0 units of date-1 consumption in state $z=1$
and 1.4 units in state $z=2$:
\begin{eqnarray*}
    q^e \; (=1)  &=&  1.0 Q(1) + 1.4 Q(2)  .
\end{eqnarray*}
Now it's a matter of algebra to find
$Q(1) = 0.6818$ and $Q(2) = 0.2273$.


\item The first-order conditions are
\begin{eqnarray*}
    c_0:  &&  u'(c_0) \;\;=\;\; \beta \sum_z p(z)
%            u' \left\{  (y_0-c_0) [ (1-a)r^1 + a r^e(z)] \right\}
            u'[c_1(z)] [ (1-a)r^1 + a r^e(z)] \\
    a:    && 0 \;\;=\;\; \beta \sum_z p(z) u'[c_1(z)] [r^e(z) - r^1 ]  .
\end{eqnarray*}
Written out more completely, the latter becomes
\begin{eqnarray*}
    0 &=& \beta \sum_z p(z) (y_0-c_0)^{1-\alpha} [ (1-a)r^1 + a r^e(z)]^{-\alpha} [r^e(z) - r^1 ] \\
    &=&  \sum_z p(z) [ (1-a)r^1 + a r^e(z)]^{-\alpha} [r^e(z) - r^1 ] ,
\end{eqnarray*}
which depends on $a$ but not $c_0$.
This ``separation'' result is another reflection of the scaling property of
power utility:  the portfolio choice doesn't depend on the scale of saving, namely
$ s = y_0 - c_0 $.

\item
We'll talk about numerical methods for solving equations shortly (``root-finding''),
but the simplest way is to compute the right side of the previous equation
for a grid of values for $a$.
We take the value that produces the first-order condition closest to zero.
A grid of 0.01 gives us $a=1.70$ when $\alpha=2$.

Now that we know $a$, we can find $c_0$ from the first first-order condition:
\begin{eqnarray*}
    c_0^{-\alpha}  &=&  \beta (y_0-c_0)^{-\alpha}
            \sum_z p(z) [ (1-a)r^1 + a r^e(z)]^{1-\alpha}  .
\end{eqnarray*}
We compute the sum on the rhs first --- denote it by $S$ (for sum).
Then we solve for $c_0$:
\begin{eqnarray*}
    (y_0-c_0)/c_0 &=& (\beta S)^{-1/\alpha}
        \;\;\Rightarrow\;\;
        c_0  \;\;=\;\; y_0 /[ 1 + (\beta S)^{-1/\alpha}] .
\end{eqnarray*}
Second-period consumption follows from the budget constraint:
$c_1(z)=  (y_0-c_0)[(1-a) r^1 + a r^e(z)] $.
The numbers are listed below:
\begin{center}
\begin{tabular}{lccccc}
\toprule
                &  $a$  &  $S$    & $c_0$   &  $c_1(1)$ &  $c_1(2)$  \\
\midrule
$\alpha = 2$    & 1.702 & 0.8482  & 0.4676  & 0.4951    & 0.8576 \\
$\alpha = 5$    & 0.637 & 0.6135  & 0.4708  & 0.5484    & 0.6832 \\
\bottomrule
\end{tabular}
\end{center}

\item When $\alpha=5$, we get $a=0.64$ by the same method.
The punchline:  when we increase risk aversion, we hold less of the risky asset
and have less risky second-period consumption as a result.

\item Merton's formula gives us $a= 1.25$ with $\alpha = 2$
and $a = 0.5 $ with $\alpha = 5$.
There are ways to make the approximation better, but we'll leave that for someone else.
\end{enumerate}
%
\needspace{2\baselineskip}
Here's the Matlab code:
\begin{verbatim}
% inputs
beta = 1/1.1;
r1 = 1.1;
re_1 = 1.0;
re_2 = 1.4;
y0 = 1;

% Arrow securities
disp('Prices of Arrow securities')
Q = inv([1 1; 1 1.4])*[1/r1; 1]

% portfolio weight a: find value that sets foc closest to zero
agrid = [0.1:0.001:2];

disp('Portfolio shares for alpha equal to 2 and 5')
alpha = 2;
foc2 = 0.5*((1-agrid)*r1+agrid*re_1).^(-alpha).*(re_1-r1) + ...
        0.5*((1-agrid)*r1+agrid*re_2).^(-alpha).*(re_2-r1);
[focmin,i] = min(abs(foc2));
a2 = agrid(i)

alpha = 5;
foc5 = 0.5*((1-agrid)*r1+agrid*re_1).^(-alpha).*(re_1-r1) + ...
        0.5*((1-agrid)*r1+agrid*re_2).^(-alpha).*(re_2-r1);
axis = 0*agrid;
[focmin,i] = min(abs(foc5));
a5 = agrid(i)

plot(agrid,foc2,'b')
hold on
plot(agrid,foc5,'m')
plot(agrid,axis,'k')
ylabel('First-order condition')
xlabel('Portfolio weight a')
title('Blue is \alpha=2, Magenta is \alpha=5')

disp(' ')
disp('Consumption quantities')
alpha = 5; a = a5;          % change as needed
S = 0.5*((1-a)*r1+a*re_1)^(1-alpha) + 0.5*((1-a)*r1+a*re_2)^(1-alpha)
SBA = (S*beta)^(-1/alpha)

c0 = y0/(1+SBA)
saving = y0 - c0

c1_1 = saving*((1-a)*r1+a*re_1)
c1_2 = saving*((1-a)*r1+a*re_2)

% Merton's formula
% See:  http://en.wikipedia.org/wiki/Merton's_portfolio_problem
disp('Compare answers to Merton formula')
E_rx = 0.5*(re_1+re_2) - r1
var_re = 0.2^2

amerton2 = E_rx/(2*var_re)
amerton5 = E_rx/(5*var_re)
\end{verbatim}

\end{enumerate}


\section*{Matlab code}

To access related Matlab code, download this document as a pdf, open it in Adobe Reader or the equivalent, and click on the pushpin:
\attachfile{../Code/Matlab/notes_cons.m}


{\vfill
{\bigskip \centerline{\it \copyright \ \number\year \
David Backus $|$ NYU Stern School of Business}%
}}


\end{document}

