\documentclass[11pt]{article}

\oddsidemargin=0.25truein \evensidemargin=0.25truein
\topmargin=-0.5truein \textwidth=6.0truein \textheight=8.75truein

%\usepackage{graphicx}
\usepackage{verbatim}
%\usepackage{booktabs}
\usepackage{comment}
\usepackage[small, compact]{titlesec}

\usepackage{hyperref}
\urlstyle{rm}   % change fonts for url's (from Chad Jones)
\hypersetup{
    colorlinks=true,        % kills boxes
    allcolors=blue,
    pdfsubject={ECON-UB233, Macroeconomic foundations for asset pricing},
    pdfauthor={Dave Backus @ NYU},
    pdfstartview={FitH},
    pdfpagemode={UseNone},
%    pdfnewwindow=true,      % links in new window
%    linkcolor=blue,         % color of internal links
%    citecolor=blue,         % color of links to bibliography
%    filecolor=blue,         % color of file links
%    urlcolor=blue           % color of external links
% see:  http://www.tug.org/applications/hyperref/manual.html
}

\renewcommand{\thefootnote}{\fnsymbol{footnote}}

% list spacing
\usepackage{enumitem}
\setitemize{leftmargin=*, topsep=0pt}
\setenumerate{leftmargin=*, topsep=0pt}

\usepackage{needspace}
% example:  \needspace{4\baselineskip} makes sure we have four lines available before pagebreak

% document starts here
\begin{document}
\parskip=\bigskipamount
\parindent=0.0in
\thispagestyle{empty}
{\large ECON-UB 233 \hfill Dave Backus @ NYU}

\bigskip\bigskip
\centerline{\Large \bf Math Tools:  Stochastic Processes}
\centerline{Revised: \today}


\begin{comment}
{\bf
*** Meyn and Tweedie re stability \\
*** Recursive approach \\
*** Limiting behavior of mean and (esp) variance \\
*** Law of iterated expectations
}
\end{comment}

*** think of this as diff rvs **** \\
stationary = time-homogeneous

\bigskip
Hansen and Sargent:  an economic model is a stochastic process...

We look at some simple examples of {\it stochastic processes\/},
which combine randomness with dynamics.
Stochastic processes are a fundamental tool of modern macroeconomics and finance,
and many other fields as well.

** Sargent:  An econmic model is a stoch process ***

%There's a particularly clear connection to bond pricing,
%where the prices of bonds of long maturities reflect the
%evolution of events between now and maturity.
%That evolution is --- wait for it --- a stochastic process.
%We introduce this into asset pricing in the usual way:
%through the pricing kernel.
%The dynamics of the pricing kernel, reflected by its stochastic process,
%tells us all we need to characterize bonds of all maturities.

We start where we left off,  with an event tree,
which we extend to more than two periods.
This isn't difficult, but we need some new notation to track how
the uncertainty unfolds with time.
We then proceed to impose more structure on the process, to simplify it.
Typically we'd like uncertainty to have some regularity to it,
so that behavior at one date is similar to the behavior at another.
Without something like this, it's hard to know how we'd compare models with data.
%One kind of structure is limited memory:
%the probabilities over future events depend only
%on the current situation, somehow defined.
%Another is that the dependence of
%the distribution of next period's events on the current situation
%has the same form at all dates.
%And finally, we generally deal with models in which the distribution
%over events in the distant future ``settle down'' in some well-defined way.
This is easier to do than say, so let's charge ahead.


\section{Event trees}

??? Change $p$ to $Prob$ ...

We can expand the concept of an event tree to more than the two
dates we've used so far.
An example is Figure 1, but we can imagine extending it further.

An event tree starts with a single event or state that we'll label $z_0$:
the state $z$ that occurred at date $t=0$.
At date $t=1$ another state $z_1$ occurs.
In the figure, there are two possible states, $z_1 = 1$ and $z_1 = 2$,
each represented by a branch stemming from the initial state $z_0$.
The probability of each branch, conditional on starting at $z_0$,
is $p(z_1| z_0)$. These probabilities are all greater than or equal to zero,
and the sum across states $z_1$ is one.

We can extend the idea to as many periods as we like,
with the node at any date $t$ giving rise to further branches at $t+1$.
It's a little cumbersome, but nonetheless helpful,
to have notation for the various nodes we generate this way.
We refer to the {\it history\/} of states through date $t$
as the sequence $(z_0, z_1, \ldots, z_t)$, which we denote by $z^t$.
Each such history defines a specific node in the tree.
For example, the node labeled (A) in the figure comes from the history
$z^1 = (z_0, z_1 = 1)$.
The node in the upper right labeled (C)
comes from the history $z^2 = (z_0, z_1 = 1, z_2 = 1)$.
We're defining nodes by listing every step we followed to get there,
which is sufficient but seems like overkill.

We construct probabilities of nodes by stringing together probabilities
of the branches that give rise to them.
Suppose for example, that the probability of
taking branch $z_1=1$ from $z_0$ is $p(z_1=1| z_0)$,
and the probability of continuing with $z_2 = 1$ is $p[z_2=1 | z^1 = (z_0, z_1=1) ]$,
then the probability at $z_0$ of getting to the upper right node is the product:
\begin{eqnarray*}
    p[z_2=1 | z^1 = (z_0, z_1=1) ] &=& p(z_1=1| z_0) p[z_2=1 | z^1 = (z_0, z_1=1) ] .
\end{eqnarray*}
In words:  the probability of getting from $z_0$ to (C)
is the probability of getting from $z_0$ to (A)
times the probability of getting from (A) to (C).

All of this works reasonably well, but it's way too cumbersome.
And without some regularity to how uncertainty works at different dates,
it's hard to see how we'd be able to connect properties of the tree
to analogous properties of data.
What kind of tree, for example, corresponds to the behavior of the short-term
interest rate?  Or the growth rate of consumption?


\section{Well-behaved stochastic processes}


We will use stochastic processes with a lot more structure
on them than this.
Why?  Lots of reasons, but mainly because they're simpler,
a give us a clear link between models and evidence.

One kind of structure is what is called the {\it Markov property\/}:
the probabilities of possible outcomes next period
depend on this period's state, but not the complete history.
Formally, we would write
$p(z_{t+1} | z^t ) = p(z_{t+1} | z_t ) $.
In a sense, we're limiting the memory to one period,
but there are standard tricks to extend it without going back to the complete history.
If we want a two-period memory, for example, we might define the state by
$y_t = (z_t, z_{t-1})$ and apply the definition to
$y_t$ rather than $z_t$.
This works for any finite history, although it undercuts some
of the simplification we were looking for.

?? change to $Prob$ and $p_n$...

Another kind of structure is to use the same conditional
probabilities $p(z_{t+1} | z_t)$ at all dates.
We would refer to such a process as {\it stationary\/}
(it stays the same) or
{\it time-homogeneous\/} (the same at all times).
In this case the state $z_t$ tells us everything we need to know
about the situation at date $t$.
The date, for example, is irrelevant.
We have the same process in 2020, for example, as we had in 2010.
By that we mean that $p(z_{t+1}|z_t)$ has the same form,
although the current state $z_t$ may be different.
The point is that any difference between the two dates is captured
by the difference in their states --- between $z_{2010}$ and $z_{2020}$.

The Markov property and stationarity
are features of the conditional distribution $p(z_{t+1} | z_t) $.
But what about conditional distributions over multiple periods?
By stringing together one-period conditional distributions
we can construct multi-period conditional distributions
$p(z_{t+k} | z_t) $ for any $k \geq 1$.
We'll save the details until we have some specific candidates to work
with, but we can probably imagine doing something of this sort.

What happens as we extend $k$ to longer time horizons?
The structure we've imposed so far allows a lot of different kinds
of behavior, but in all the examples we'll use,
the conditional distribution will ``settle down.''
By that we mean that the distribution $ p(z_{t+k} = z | z_t)$
converges as we increase $k$ to a unique distribution $p(z)$
that doesn't depend on the current state $z_t$.
If it does, we say the process is {\it stable\/}.
There are two conditions here:  that it converges, and that it converges
to the same thing for every $z_t$.
Terminology varies, but we will refer to the resulting $p(z)$ as the
{\it equilibrium distribution\/}.

Despite the brevity of this section, it has a lot of content.
We'll see how it all works when we apply it.


\section{Digression:  matrix multiplication (skip)}

Here's a quick summary of some linear algebra we'll find helpful.
A matrix is a table of numbers.
The matrix $A$ for example has (say) $m$ rows and $n$ columns,
so we would say it is ``$m$ by $n$.''
The element in the $i$th row and $j$ column is usually written $a_{ij}$.
A (column) vector is a matrix with one column ($n=1$).

If the dimensions line up, we can multiply
one matrix times another.
Suppose we multiply $A$ times $B$ and call the result $C$ (a new matrix).
The rule for doing this is
\begin{eqnarray*}
    c_{ij} &=& \sum_k a_{ik} b_{kj} ,
\end{eqnarray*}
where $c_{ij}$ is the $ij$th element of $C$.
We apply the rule for all possible $i$ and $j$.
This works only if the second dimension of $A$ equals the first dimension of $B$:
if, for example, $A$ is $m$ by $n$ and $B$ is $n$ by $p$.
Then the subscript $k$ in the sum runs from 1 to $n$.

Matlab is set up to work directly with vectors and matrices.
When we define a matrix in Matlab, we separate rows with semi-colons.
The command {\tt A = [1 2 5; 6 4 2]}, for example,
defines the 2 (rows) by 3 (columns) matrix
\begin{eqnarray*}
    A &=& \left[
            \begin{array}{ccc}
            1 & 2 & 5 \\ 6 & 4 & 2
            \end{array}
            \right] .
\end{eqnarray*}
Similarly, {\tt B = [1 4; 3 2]} defines the matrix
\begin{eqnarray*}
    B &=& \left[
            \begin{array}{ccc}
            1 & 4 \\  3 & 2
            \end{array}
            \right] .
\end{eqnarray*}
You might verify by hand that
\begin{eqnarray*}
    BA &=& \left[
            \begin{array}{ccc}
            25 & 18 & 13 \\  15 & 14  & 19
            \end{array}
            \right] .
\end{eqnarray*}
You can do the same calculation in Matlab with the command
{\tt C = B*A}.

As a test of your knowledge, what is $A B$?
What happens if you type {\it D = A*B} in Matlab?


\section{Markov chains (skip)}

Markov chains are wonderful examples of Markov processes
for illustrating the concepts of stationarity and stability.
Both are relatively simple if you're comfortable with linear algebra
and matrix multiplication.


Here's the structure.
The state $z_t$ takes on a finite set of values, the same at all dates,
which we label with the integers $i$ and $j$.
The probability of state $j$ next period depends only on today's state $i$
(the Markov property) and is the the same at all dates (stationary):
\begin{eqnarray*}
    p_{ij} &=& \mbox{Prob} (z_{t+1} = j | z_t = i) .
\end{eqnarray*}
We can collect them in a matrix $P$, called the {\it transition matrix\/},
where $p_{ij}$ is the element in the $i$th row and $j$th column.
Thus  row $i$ gives us the probabilities of moving to any state $j$
next period from state $i$ now.

Here's an example.  Suppose
\begin{eqnarray*}
    P &=& \left[
            \begin{array}{cc}
            0.8 & 0.2 \\  0.3 & 0.7
            \end{array}
            \right] .
\end{eqnarray*}
Row $i$ is the probability distribution over states next period
if the current state is $i$.
If we're in state 1 now, there's an 80\% chance we'll stay there
and a 20\% chance we'll move to state 2.
These probabilities are nonnegative and sum to one.
Ditto for row 2.
Put another way:  any matrix $P$ is a legitimate transition matrix
if (i)~$p_{ij} \geq 0$ and (ii)~$\sum_j p_{ij} = 1$.
By choosing different numbers, we can generate a wide range of behavior.


The matrix $P$ describes probabilities of one-period changes in the state.
But what are the probabilities of moving from state 1 to state 2 in 2 periods?
In 10 periods?
As it happens, the probability of moving from state $i$ at date $t$ to state $j$
at date $t+k$ is the $ij$th element of $P^k$.
So we can compute the whole set of probabilities by computing powers of $P$.
In our example, we get
\begin{eqnarray*}
    P^2 &=& \left[
            \begin{array}{cc}
            0.70 & 0.30 \\  0.45 & 0.55
            \end{array}
            \right] , \;\;\;
    P^3 \;\;=\;\; \left[
            \begin{array}{cc}
            0.650 & 0.350 \\  0.525 & 0.475
            \end{array}
            \right] .
\end{eqnarray*}
Try it yourself in Matlab.


We're typically interested in examples in which the probabilities are stable.
If we increase the time horizon $k$, what happens to the probabilities?
With the right conditions,
$P^k$ converges to a matrix whose rows are all the same.
This says that the probability distribution across states doesn't depend
on the current state if we get far enough in the future.
Each row represents the {\it equilibrium distribution\/}.

Sometimes we converge to a unique equilibrium distribution,
and sometimes we don't.
Here are some examples to help us think about how this works.
(i)~In the example above, show that the equilibrium distribution is $ (0.6, 0.4)$.
[How?  Keep taking higher powers of $P$.
You should see that every row is $(0.6, 0.4)$.]
(ii)~For each of these choices of $P$, describe what happens as we compute
$P^k$ for larger values of $k$.
\begin{eqnarray*}
    P &=& \left[
            \begin{array}{cc}
            1 & 0 \\  0 & 1
            \end{array}
            \right], \;\;\;
    P \;\;=\;\; \left[
            \begin{array}{ccc}
            0.8 & 0.1 & 0.1 \\ 0 & 1 & 0 \\ 0 & 0 & 1
            \end{array}
            \right], \;\;\;
    P \;\;=\;\; \left[
            \begin{array}{cc}
            0 & 1 \\  1 & 0
            \end{array}
            \right] .
\end{eqnarray*}
In each case, verify that $P$ is a legitimate transition matrix,
then compute powers of $P$ and see how they behave.

We're interested, for the most part, in $P$'s that have
unique equilibrium distributions.
One way to guarantee that is the condition:  $p_{ij} > 0$.
It's sufficient, but stronger than we need.
The main point is that we need some structure to make it work.
We'll see the same thing in other contexts.


\section{Linear models 1:  autoregressions}
\label{sec:autoregressions}

Next up are linear stochastic processes.
Linear processes are incredibly useful because they illustrate the concepts
in relatively simple form and are capable of approximating
a lot of what we observe about the economy and asset returns.
It's possible nonlinearities are important in the real world,
but there's enough noise in most real-world situations
that it's typically hard to say whether that's true or not.
In any case, linear processes are the predominant tool
of modern macroeconomics and finance and a good starting point even
if we're ultimately interested in something else.

One of the simplest linear processes is a first-order autoregression or AR(1).
We'll use it over and over again.
A random variable $x_t$ evolves through time according to
according to
\begin{eqnarray}
    x_{t} &=& \varphi x_{t-1} + \sigma w_{t} .
    \label{eq:ar1}
\end{eqnarray}
Here $\{ w_t \}$ is a sequence of random variables,
all of them standard normal (normal with mean zero and variance one)
and mutually independent.
The $w$'s are commonly referred to as ``errors,'' ``disturbances,'' or ``innovations.''
Equation (\ref{eq:ar1}) is referred to as an
autoregression because $x_{t}$ is regressed on a lag of itself (``auto'' = ``same'')
and first-order because one lag of $x_{t}$ appears on the right.

Note that we've switched from $z_t$ to $x_t$ here.
In this case we use the letter $x$ to refer to both
the state and a specific random variable that depends on the state.
It's similar to our treatment of random variables earlier,
where the random variable is a function of the state, but in most cases
we could ignore the latter.
Here it's more complicated.
We'll see examples shortly in which
$x_t$ remains the random variable of interest but the state $z_t$ is a more
complicated object.

It's important to be clear about what we know and when we know it.
If we go back to the event tree, we usually assume that we know where
we are.
If we've experienced a history $z^t$, then we know that,
and therefore know our current location or node in the tree.
Here we make a similar assumption:  that we know all the
$x$'s and $w$'s up through date $t$, but do not know their future values.

Since each $w_t$ is independent of every other one,
their covariances are
\begin{eqnarray*}
    \mbox{Cov} (w_t, w_{t-k}) \;\;=\;\; E (w_t w_{t-k}) &=&
            \left\{
            \begin{array}{l}
            1  \mbox{ if } k=0  \\
            0  \mbox{ otherwise}.
            \end{array}
            \right.
\end{eqnarray*}
That is, the expectation of ``cross terms'' is zero.
These conditions define the $w_t$'s as having no dynamics:
what happens at one date is independent of all other dates.
Therefore any dynamics must come from the rest of the equation ---
in this case the $x_t$ terms.


Let's go back to the autoregression, equation (\ref{eq:ar1})
and ask ourselves:  (i)~Is it Markov?
(ii)~Is it stationary?
(iii)~Does it have a unique equilibrium distribution?
We'll consider each in turn.

On question (i): Yes, it's Markov.
If we take equation (\ref{eq:ar1}) updated one period,
we see that the conditional distribution of $x_{t+1}$ depends only in the state $x_t$.
What is the distribution?  Well, since
it comes from $w_{t+1}$, which is normal, it's also normal.
So once we know the mean and variance we're done.
The conditional mean is
\begin{eqnarray*}
    E ( x_{t+1} | x_t )
            &=& E (\varphi x_t + \sigma w_{t+1} | x_t )
                \;\;=\;\; \varphi x_t .
\end{eqnarray*}
Why?  Because we know $x_t$ (that's what we meant by conditional on $x_t$)
but we don't know $w_{t+1}$ (it hasn't happened yet).
So we take the mean of $w_{t+1}$, which is zero.
What about the variance of the conditional distribution --- what we might call the conditional variance?
The variance, as we know,
 is the expected squared difference between the random variable and its mean.
The conditional variance is therefore
\begin{eqnarray*}
    \mbox{Var}(x_{t+1} | x_t ) \;\;=\;\;
    E \left[(x_{t+1} - \varphi x_t)^2 | x_t \right]
            &=& E \left[(\sigma w_{t+1})^2 | x_t \right]
                \;\;=\;\; \sigma^2 .
\end{eqnarray*}
Thus we have:  $x_{t+1} | x_t \sim \mathcal{N} (\varphi x_t, \sigma^2)$.
This conditional distribution is the analog of a row
of the transition matrix in a Markov chain.

On question (ii):  Yes, it's stationary.
Nothing in (\ref{eq:ar1}) depends on the date, including the distribution
of the disturbance $w_t$.


Now about question (iii):  Is the process stable?
What is the equilibrium distribution?
Is it unique?
The answer is yes if $|\varphi| < 1$, but let's show that for ourselves.
One approach is to compute the distribution directly
from (\ref{eq:ar1}).
The logic here is to presume an equilibrium distribution exists and
compute its properties.
It's the same distribution for $x_t$ and $x_{t-1}$,
so it has the same mean at both dates:
\begin{eqnarray*}
    E( x ) &=& \varphi E(x) \;\;\;\;\Rightarrow\;\;\;\; E(x)
        \;\;=\;\; 0/(1-\varphi) \;\;=\;\; 0 .
\end{eqnarray*}
If we added an intercept to (\ref{eq:ar1}) we would get something else.
[Try it. The simplest approach is to define $y_t = x_t + \mu$
and see how $\mu$ changes our calculations.]
What about the variance?
Again, the distributions of $x_t$ and $x_{t+1}$ are the same,
so we have
\begin{eqnarray*}
    \mbox{Var}(x) &=& E (x_{t}^2)
            \;\;=\;\; E \left[ (\varphi x_{t-1} + \sigma w_{t})^2 \right]
            \;\;=\;\; \varphi^2 \mbox{Var}(x) + \sigma^2.
\end{eqnarray*}
That gives us $\mbox{Var}(x) = \sigma^2 /(1-\varphi^2)$.
This works if $\varphi^2 < 1$,
but if not the variance is negative, which
we know can't be.  So if $ \varphi^2<1$, the
equilibrium distribution is normal with mean zero
and variance $ \sigma^2/(1-\varphi^2)$.
If $\varphi^2 \geq 1$, we don't converge:  there is no equilibrium distribution.

% ?? show this from perspective of t, moving out one period at a time
The stability property is more obvious if we go through the effort
of deriving the equilibrium distribution as a limit, as we did with the Markov chain.
Suppose we think of $x_{t+k}$ from the perspective of date $t$.
If the current state is $x_{t}$, what is the distribution of $x_{t+k}$?
This is the conditional distribution of $x_{t+k}$:  conditional
on the current state $x_t$.
Repeated application of (\ref{eq:ar1}) gives us
\begin{eqnarray*}
    x_{t+k} &=& \varphi^k x_{t} +
            \sigma \left[ w_{t+k} + \varphi w_{t+k-1} +
                    \cdots + \varphi^{k-1} w_{t+1} \right].
%    \label{eq:ar1-tk}
\end{eqnarray*}
The conditional mean and variance are
\begin{eqnarray*}
    E (x_{t+k} | x_t ) &=& \varphi^k x_{t} \\
    \mbox{Var}(x_{t+k} | x_t )
            &=&  \sigma^2 \left[ 1 + \varphi^2 + \cdots + \varphi^{2(k-1)} \right]
                \;\;=\;\; \sigma ^2 (1-\varphi^{2k})/(1-\varphi^2) .
\end{eqnarray*}
The second one follows because $E(w_t^2) = 1$
and $E (w_t w_{t+j}) = 0$ for $ j\neq 0$.
Does this settle down as we increase $k$?
Yes, as long as $\varphi^2 < 1$.
The conditional mean converges to zero and
the conditional variance converges to $\sigma^2/(1-\varphi^2)$,
the same answers we had before.
%This approach has the advantage of describing how we approach the
%equilibrium distribution.

 
*** Law of iterated expecations... \\
\url{http://en.wikipedia.org/wiki/Law_of_total_expectation}

*** Var is var of cond mean plus forecast error \\
\url{http://en.wikipedia.org/wiki/Law_of_total_variance} 

*** Multistep forecasting...

We can do the same for high-order autoregressions:
versions of (\ref{eq:ar1}) with additional lags of $x_{t+1}$ on the right.
Examples include
\begin{eqnarray*}
    \mbox{AR(1)}:     &&  x_t \;\;=\;\; \varphi x_{t-1} + \sigma w_t  \\
    \mbox{AR($2$)}:   &&  x_t \;\;=\;\; \varphi_1 x_{t-1} + \varphi_2 x_{t-2}
                    +\sigma w_t  \\
    \mbox{AR($p$)}:   &&  x_t \;\;=\;\; \varphi_1 x_{t-1} + \varphi_2 x_{t-2}
                    + \cdots + \varphi_p x_{t-p} + \sigma w_t   .
\end{eqnarray*}
These models are stationary by construction.
They're Markov if we define the state at $t-1$ by
$ y_{t-1} = (x_{t-1}, \ldots, x_{t-p})$.
And they're stable under some conditions on the $\varphi$'s that we'll leave alone
for now.
We mention it briefly in Section \ref{sec:matrix}.


\section{Digression: components of variance}

The AR(1) is a good example for illustrating the behavior of the variance
over different time horizons.
Consider the behavior of $x_{t+k}$ from the perspective of date $t$.
We can write
\begin{eqnarray*}
    x_{t+k} &=& E (x_{t+k} | x_t ) + \big[ x_{t+k} - E (x_{t+k} | x_t )\big] ,
\end{eqnarray*}
the sum of two components.
Since we're lazy, our first step is to use more compact notation:
\begin{eqnarray*}
    x_{t+k} &=& E_t (x_{t+k}) + \big[ x_{t+k} - E_t (x_{t+k})\big] ,
\end{eqnarray*}
where $ E_t (x_{t+k}) = E (x_{t+k} | x_t )$ is shorthand for
``the mean of $x_{t+k}$ conditional on the state at date $t$.''
In both versions, the first component is the conditional mean,
which we might think of as a forecast.
The second component is the forecast error.
The components are independent,
because the first one depends on $w$'s that happened at $t$ or before
and the second one depends on $w'$ that happened after.
[Use the expressions in the previous section to remind yourself of this.]

The variance of $x_{t+k}$ can be broken into similar
components.
Since the two terms are independent, we have
\begin{eqnarray*}
    \mbox{Var}(x_{t+k}) &=& E \left\{ E_t (x_{t+k}) + \big[ x_{t+k} - E_t (x_{t+k})\big]\right\}^2 \\
        &=& E \big[ E_t (x_{t+k}) \big]^2 + E \big[ x_{t+k} - E_t (x_{t+k})\big]^2 .
\end{eqnarray*}
This seems a little mysterious, so let's substitute the expressions
for the AR(1).
That gives us
\begin{eqnarray*}
    \mbox{Var}(x_{t+k}) &=& E \big( \varphi^k x_t \big)^2
            + E \big[(\sigma ^2 (1-\varphi^{2k})/(1-\varphi^2)\big] \\
            &=& \varphi^{2k} \sigma ^2 /(1-\varphi^2)
            +  \sigma ^2 (1-\varphi^{2k})/(1-\varphi^2) .
\end{eqnarray*}
The first term is the part of the variance of $x_{t+k}$ that's predictable ---
it comes from the forecast after all.
The second is the part that's unpredictable.
You might verify that they sum to the variance.

Now think about how the components change as we increase the forecast horizon.
As we increase $k$, our ability to forecast declines, and the forecast
$  E (x_{t+k} | x_t ) = \varphi x_t $ approaches zero, a constant.
The variance of this term evidently also approaches zero.
The other term grows to make up the difference:
as we increase the time horizon $k$, the variance of the forecast error
approaches the equilibrium variance of $x$.
This takes a simple form in the AR(1) case, but the idea
is more general.


\section{Linear models 2:  moving averages}

Our next example is one we'll put to work repeatedly:
what we call a {\it moving average\/}.

The simplest example is  a first-order moving average
or MA(1):
\begin{eqnarray}
    x_{t}  &=& \sigma \left( \theta_0 w_{t} + \theta_1 w_{t-1} \right) .
    \label{eq:ma1}
\end{eqnarray}
Here, as in the previous section,
$w_t$ is a sequence of independent standard normal random variables.
The state $z_{t-1}$ at date $t-1$ is the lagged disturbance $w_{t-1}$.

Does equation (\ref{eq:ma1}) satisfy our conditions?
Well, it's stationary (it holds for all $t$)
and Markov (the state variable $w_t$ summarizes the situation at $t$).
This is essential, so let's go through it slowly.
How do we know what the state is?
The state is whatever we need to say what the distribution of $x_{t+1}$ is
as of date $t1$.
In this example, we need to know $w_{t}$,
but $w_{t+1}$ is simply part of the distribution of $x_{t+1}$.
So the state at date $t$ is $w_{t}$.


Is equation (\ref{eq:ma1}) stable?
Let's see.
The conditional mean and variance of $x_{t+1}$ are
\begin{eqnarray*}
    E_t (x_{t+1}) &=& \sigma \theta_1 w_t  \\
    \mbox{Var}_t(x_{t+1} )
            &=&  \sigma^2 \theta_0^2  .
\end{eqnarray*}
As above, the subscript $t$ on $E$ and $\mbox{Var}$
means ``conditional on the state $z$ at date $t$'':
$  E_t (x_{t+1} ) = E_t (x_{t+1} | z_{t} ) $ and
$  \mbox{Var}_t(x_{t+1}  ) =  \mbox{Var}(x_{t+1} | z_{t} ) $.

What about $x_{t+2}$?
From (\ref{eq:ma1}) we have
\begin{eqnarray*}
    x_{t+2}  &=& \sigma \left( \theta_0 w_{t+2} + \theta_1 w_{t+1} \right) .
\end{eqnarray*}
Viewed from the perspective of the state $w_t$ at $t$, none of this is known.
The conditional mean and variance are therefore
\begin{eqnarray*}
    E_t (x_{t+2} ) &=& 0   \\
    \mbox{Var}_t (x_{t+2} )
            &=&  \sigma^2 \left( \theta_0^2 + \theta_1^2 \right).
\end{eqnarray*}
Therefore, conditional on the state at $t$,
$x_{t+2}$ is normal with mean zero and variance $ \sigma^2 (\theta_0^2 + \theta_1^2)$.
What about $x_{t+3}$?
It should be clear that it has the same conditional distribution as $x_{t+2}$.
An MA(1) has a one-period memory, so once we get beyond one period
the conditional distribution is the same.
Evidently we've converged to the equilibrium distribution:
normal with mean zero and variance $ \sigma^2 (\theta_0^2 + \theta_1^2)$.

How about an MA(2)?
It has the form
\begin{eqnarray*}
    x_{t}  &=& \sigma \left( \theta_0 w_{t} + \theta_1 w_{t-1}
    + \theta_2 w_{t-2} \right) .
\end{eqnarray*}
It's Markov with state $z_{t} = (w_{t}, w_{t-1})$.
If we computed conditional distributions, we'd find that the effect
of the state disappears, in this case, after two periods.
We'll skip that for now, but consider a more general case shortly.

Moving averages in general have the form
\begin{eqnarray*}
    \mbox{MA(1)}:   &&  x_t \;\;=\;\;  \sigma \left( \theta_0 w_t + \theta_1 w_{t-1}\right) \\
    \mbox{MA($2$)}:   &&  x_t \;\;=\;\;  \sigma \left( \theta_0 w_t + \theta_1 w_{t-1}
                            + \theta_2 w_{t-2} \right) \\
    \mbox{MA($q$)}:   &&  x_t \;\;=\;\;  \sigma \left( \theta_0 w_t + \theta_1 w_{t-1}
                            + \theta_2 w_{t-2} + \cdots + \theta_q w_{t-q}\right) \\
    \mbox{MA($\infty$)}:   &&  x_t \;\;=\;\; \sigma \left( \theta_0 w_t + \theta_1 w_{t-1}
                            + \theta_2 w_{t-2} + \theta_3 w_{t-3} + \cdots \right).
\end{eqnarray*}
Since the infinite MA includes the others as special cases, let's consider
its properties.
Is it stationary?  Yes, the equation holds at all dates $t$.
Is it Markov?
Yes, if we allow ourselves an infinite dimensional state
$ z_{t} = (w_t, w_{t-1}, \ldots)$.
This goes against our goal of simplicity,
but we'll see that it's convenient for other reasons.

Is the infinite MA stable?
Let's see how the conditional distribution of $x_{t+k}$ changes as we increase $k$.
The random variable $x_{t+k}$ is, for any positive $k$,
\begin{eqnarray*}
    x_{t+k} &=& \sigma
        \left(\theta_0 w_{t+k} + \theta_1 w_{t+k-1} + \cdots + \theta_{k-1} w_{t+1}  \right)
            + \sigma \left( \theta_k w_{t} + \theta_{k+1} w_{t-1} + \cdots \right) .
\end{eqnarray*}
Conditional on $ z_{t} = (w_t, w_{t-1}, \ldots)$,
the second collection of $w_t$'s is known, but the first is not.
The conditional mean and variance are therefore
\begin{eqnarray*}
    E_t \left( x_{t+k} \right) &=&
        \sigma \left( \theta_k w_t + \theta_{k+1} w_{t-1} + \cdots \right) \\
    \mbox{Var}_t \left( x_{t+k} \right)
            &=& \sigma^2 \big( \theta_0^2 + \theta_1^2 + \cdots + \theta_{k-1}^2 \big).
\end{eqnarray*}
Do they converge as we increase $k$?
Evidently we need
\begin{eqnarray*}
    \lim_{k\rightarrow \infty} \theta_k &=& 0
\end{eqnarray*}
for the mean to converge.
[This isn't a tight argument, but it'll do for us.]
And we need
\begin{eqnarray*}
    \lim_{k\rightarrow \infty} (\theta_0^2 + \theta_1^2 + \cdots + \theta_{k-1}^2) &=&
            \sum_{k=0}^\infty \theta_k^2 \;\;<\;\; \infty
\end{eqnarray*}
for the variance to converge.
(The notation ``$< \infty$'' means here that the sum converges.)
The distribution of $x_{t+k}$ for $k$ sufficiently far in the future
is then normal with mean zero and variance $\sigma^2 \sum_{k=0}^\infty \theta_k^2 $.
The second condition is sometimes phrased as ``the moving average coefficients
are square summable.''
The second implies the first, so we'll stick with that.
Roughly speaking, we need the squared moving average coefficients to
go to zero at a fast enough rate.

One of the things that makes moving averages so useful is that even
autoregressions can be written this way.
Consider the AR(1), equation (\ref{eq:ar1}).
If we substitute backwards, we find
\begin{eqnarray*}
    x_{t} &=& \varphi x_{t-1} + \sigma w_t \\
            &=& \sigma w_t + \varphi \left( \varphi x_{t-2} + \sigma w_{t-1} \right) \\
            &=& \sigma \big( w_t + \varphi w_{t-1} + \varphi^2 w_{t-2} + \cdots \big) .
%            \label{eq:ar1asma}
\end{eqnarray*}
We refer to this as the ``moving average representation'' of the AR(1).
The AR(1) simply applies some structure to the moving average coefficients,
namely that they decline geometrically.

This representation is sometimes useful in finding the equilibrium distribution.
For example, the mean here is zero:  take expectations of both sides.
The variance is
\begin{eqnarray*}
    \mbox{Var} (x_{t})
            &=& \sigma^2 E  \big( w_t + \varphi w_{t-1} + \varphi^2 w_{t-2}
                + \cdots \big)^2 , %\;\;=\;\; \sigma^2/(1-\varphi^2) ,
\end{eqnarray*}
which converges, as we've seen, if $\varphi^2 < 1$.


\section{Linear models 3:  ARMA(1,1)'s}


We now have two linear models at our disposal:  autoregressions and moving averages.
Combining them gives us a wide range of behavior with a small number
of parameters.
One of my favorites is the ARMA(1,1):
\begin{eqnarray*}
    x_t &=& \varphi_1 x_{t-1} + \theta_0 w_t + \theta_1 w_{t-1} .
\end{eqnarray*}
This is Markov with date $t$ state $z_{t} = (x_t,w_t)$.

Its properties are evident from its moving average representation.
Repeated substitution gives us
\begin{eqnarray*}
    x_t &=&  \theta_0 w_t + (\varphi_1\theta_0+ \theta_1) w_{t-1}
                + (\varphi_1\theta_0+ \theta_1)\varphi_1 w_{t-2}
                +  (\varphi_1\theta_0+ \theta_1)\varphi_1^2 w_{t-3} + \cdots .
\end{eqnarray*}
That is:  the first two moving average coefficients are arbitrary,
then they decline at rate $\varphi_1$.
Variants of this model are the basis of the most popular models of bond pricing.
One useful feature is that its conditional mean is autoregressive:
\begin{eqnarray*}
  E_t ( x_{t+1} ) &=&  (\varphi_1\theta_0+ \theta_1) w_{t}
                + (\varphi_1\theta_0+ \theta_1)\varphi_1 w_{t-1}
                +  (\varphi_1\theta_0+ \theta_1)\varphi_1^2 w_{t-2} + \cdots .
\end{eqnarray*}
How do we know it's autoregressive?
Because the moving average coefficients decline at a constant rate.

Comment:  this is simpler if we set $\theta_0 = 1$.

\section{Matrix representations (optional)}
\label{sec:matrix}


We can extend all of this to vector processes and, moreover,
express univariate ARMA models in compact matrix form.
The standard form is an AR(1) in vector form:
\begin{eqnarray}
    x_{t} &=& A x_{t-1} + B w_{t} ,
    \label{eq:var}
\end{eqnarray}
where $x_t$ and $w_t$ are both vectors.
This has the same structure as equation (\ref{eq:ar1}) with the matrix $A$
playing the role of $\varphi$ and $B$ the role of $\sigma$.
It's a stationary Markov process in the state $x_t$.
It's stable if the eigenvalues of $A$ are all less than one in absolute value:
if $A^k$ approaches zero as we increase $k$.

Finite ARMA models ---
ARMA($p$,$q$) for finite $p$ and $q$ ---
are conveniently expressed in this form with the appropriate definition
of the state.
An ARMA(1,1), for example, can be expressed
\begin{eqnarray*}
    \left[
    \begin{array}{c}
    x_t \\ w_t
    \end{array}
    \right]
    &=&
    \left[
    \begin{array}{cc}
    \varphi_1 & \theta_1 \\ 0 & 0
    \end{array}
    \right]
    \left[
    \begin{array}{c}
    x_{t-1} \\ w_{t-1}
    \end{array}
    \right]
    +
    \left[
    \begin{array}{c}
    \theta_0 \\ 1
    \end{array}
    \right]
    [w_t ] .
\end{eqnarray*}
In practice, then,
if we can handle the vector process (\ref{eq:var})
we can handle finite ARMA models.

%We see, then, that the form (\ref{eq:var}) is flexible
%enough to cover a wide range of linear models.




\section{Autocorrelation functions}

We don't observe moving average coefficients,
but we do observe one consequence of them:
the autocovariance and autocorrelation functions.
% ?? intro


We can compute both from the moving average representation.
Suppose we have a time series with moving average representation
\begin{eqnarray*}
    x_t &=& \mu + \sum_{j=0}^\infty a_j w_{t-j} .
\end{eqnarray*}
(The reason for the change in notation should be apparent soon.)
The mean is evidently $\mu$.
What is the covariance between $x_t$ and $x_{t-k}$ for $k \geq 0$?
We've seen that we can easily compute sample analogs.
The covariance is
\begin{eqnarray*}
    \gamma_x(k)  &=& \mbox{Cov} (x_t,x_{t-k}) \;\;=\;\;
                 E \left[ (x_t - \mu) (x_{t-k} - \mu)\right] \phantom{\sum}\\
                &=&  E \ \Big( \sum_{j=0}^\infty a_j w_{t-j} \Big)
                  \Big( \sum_{j=0}^\infty a_j w_{t-k-j} \Big) \\
                &=&   \sum_{j=k}^\infty a_j a_{j-k}
                \;\;=\;\; \sum_{j=0}^\infty a_j a_{j+k} .
\end{eqnarray*}
We eliminated the cross terms because their expectation is zero
[$E (w_t w_{t-j}) = 0$ for $j\neq 0$].
The notation $\gamma_x(k)$ is standard.
The variance is the value for $k=0$,
\begin{eqnarray}
    \gamma_x(0)  &=& \mbox{Cov} (x_t,x_{t})
                \;\;=\;\;   \sum_{j=0}^\infty a_j^2  ,
                \label{eq:acf-from-ma}
\end{eqnarray}
which we've seen before.

We refer to $\gamma_x(k)$, plotted as a function of $k$,
as the {\it autocovariance function\/}.
The {\it autocorrelation function\/} or acf is the same thing scaled by the variance:
\begin{eqnarray*}
    \rho_x(k)  &=& \gamma_x(k)/\gamma_x(0) .
\end{eqnarray*}
The scaling insures that $|\rho(k)| \leq 1$.
By construction $\rho_x(0) = 1$.
Both are symmetric:  we compute them for $k\geq 0$,
but you get the same for positive and negative values of $k$.
(Try it, you'll see.)


{\it Example 1.\/}
An MA(1) has autocovariances
\begin{eqnarray*}
    \gamma_x(k) &=&
            \left\{
            \begin{array}{ll}
            \theta_0^2 + \theta_1^2 & k=0 \\
            \theta_0\theta_1  & k=1 \\
            0       &  k > 1 .
            \end{array}
            \right.
\end{eqnarray*}
The autocorrelations are
\begin{eqnarray*}
    \rho_x(k) &=& \gamma_x(k)/\gamma_x(0)
        \;\;=\;\;
            \left\{
            \begin{array}{ll}
            1  & k=0 \\
            \theta_0\theta_1/(\theta_0^2 + \theta_1^2)  & k=1 \\
            0       &  k > 1 .
            \end{array}
            \right.
\end{eqnarray*}
That's the defining property of an MA(1):  %its acf is zero for $k > 1$.
it has only a one-period memory.
You might verify for yourself that an MA($q$) has a $q$-period memory.

{\it Example 2.\/}
We've seen that an AR(1) has an infinite moving average representation
with MA coefficients $a_j = \varphi^j$.
Therefore the autocovariances (\ref{eq:acf-from-ma}) are
\begin{eqnarray*}
    \gamma_x(k) &=& \sum_{j=0}^\infty \varphi^j \varphi^{j+k}
            \;\;=\;\; \varphi^k/(1-\varphi^2) .
\end{eqnarray*}
The autocorrelations are therefore
\begin{eqnarray*}
    \rho_x(k) &=& \gamma_x(k) / \gamma_x(0)
            \;\;=\;\; \varphi^k .
\end{eqnarray*}
In words:  the acf declines at the geometric rate $\varphi$.
In contrast to an MA, it approaches zero gradually.


\section{Sample autocorrelations}

We can do the same thing with data.
You may recall that the sample mean is
\begin{eqnarray*}
    \bar{x} &=& T^{-1} \sum_{t=1}^T x_{t}
\end{eqnarray*}
and the sample variance is
\begin{eqnarray*}
    \gamma_x(0)  &=&   T^{-1} \sum_{t=1}^T (x_{t}-\bar{x})^2 .
\end{eqnarray*}
The notation follows that of the previous section.

We're typically interested in some aspects of the dynamics of $x$,
summarized in the sample autocovariance and autocorrelation functions.
The sample covariance is
\begin{eqnarray*}
    \gamma_x(k)  &=&  T^{-1} \sum_{t=k+1}^T (x_{t}-\bar{x})(x_{t-k}-\bar{x}) .
\end{eqnarray*}
Since we only have the observations $x_t$ for $t=1,...,T$,
we need to start the sum at $t=k+1$.
By longstanding convention, we nevertheless divide the sum by $T$ rather than $T-k$.
We could also consider negative values of $k$ and adjust the
range in the sum appropriately.
%We refer to $\gamma_x(k)$, a function of $k$, as the sample autocovariance function.

The shape of $\gamma_x(k)$ is useful in telling us about the dynamics of $x$,
but it's more common to scale it by $\gamma_x(0)$ and convert it to
a correlation.
The autocorrelation function $\rho_x(k)$ is defined by
\begin{eqnarray*}
    \rho_x(k)  &=&  \gamma_x(k)/\gamma_x(0).
\end{eqnarray*}
Obviously $\rho_x(0) = 1$:  $x_t$ is perfectly correlated with $x_t$.
But for other values of $k$ it can take a variety of forms.

When we compute autocorrelations with financial
data we see, for example, that autocorrelations for equity returns are very small:
returns are virtually uncorrelated over time.
Interest rates, however, are very persistent:
the autocorrelations decline slowly with $k$.


%\section{Cross-correlation functions}
%
%We can do the same thing for two random variables...


\section*{Bottom line}

There's a lot here, most of it essential to modeling
economies and asset returns over time.
We'll use stochastic processes with these three properties:
\begin{itemize}
\item Markov. The distribution over next period outcomes depends only
on this period's state.
\item Stationary. The conditional distribution is that same at all dates.
\item Stable. The distribution over future outcomes settles down as we increase
the time horizon.
\end{itemize}
We'll put all of them to work.


\section*{Practice problems}

\begin{enumerate}
\item {\it Probabilities in an event tree.\/}
Consider the event tree in Figure 1.  Suppose that at each node
the probability of taking the up branch is $\omega$ and the
probability of the down branch is $1-\omega$.
What are the probabilities of the four nodes at $t=2$?
%
\needspace{2\baselineskip}
Answer.
We simply multiply the probabilities
of the branches taken to get to each node.
Starting at the top and working our way down,
they are $ \omega^2 $, $ \omega (1-\omega)$, $ (1-\omega)\omega $,
and $(1-\omega)^2$.


\item {\it Two-state Markov chain.\/}
We can get a sense of how Markov chains work with a two-state example.
A two-state chain is characterized by a 2 by 2 transition matrix $P$.
Because the rows sum to one, $P$ has (essentially) two parameters.
A convenient parameterization is
\begin{eqnarray}
    P &=& (1-\varphi)
        \left[
        \begin{array}{cc}
        \omega & 1-\omega \\ \omega & 1-\omega
        \end{array}
        \right]
        + \varphi
        \left[
        \begin{array}{cc}
        1  & 0  \\  0  & 1
        \end{array}
        \right] ,
        \label{eq:p-simple}
\end{eqnarray}
where the two parameters are $\omega$ and $\varphi$.
%
\begin{enumerate}
\item Under what conditions on $(\omega, \varphi)$ is $P$ a legitimate
transition matrix?
\item What are the two-period transitions $P^2$?
You can either do this by hand or get Matlab to do it.  Either way,
the key is to arrange the terms into a form similar to (\ref{eq:p-simple}).
\item What about the $k$-period transitions?
\item What happens as we continue to increase $k$?
What is the equilibrium distribution?
\item (extra credit) What are the eigenvalues of $P$?
\end{enumerate}
%
\needspace{2\baselineskip}
Answer.
This question is useful if you're comfortable with linear algebra,
but skip it if you're not.
This is the simplest possible Markov chain,
but it illustrates a number of possibilities.
\begin{enumerate}
\item [(a)] The probabilities have to be between zero and one,
which gives us these inequalities:
\begin{eqnarray*}
    0 &\leq& (1-\varphi) \omega + \varphi \;\;\leq\;\; 1 \\
    0 &\leq& (1-\varphi) (1-\omega) \;\;\leq\;\; 1 \\
    0 &\leq& (1-\varphi) \omega \;\;\leq\;\; 1 \\
    0 &\leq& (1-\varphi) (1-\omega) + \varphi \;\;\leq\;\; 1 .
\end{eqnarray*}
That's sufficient for your answer.
If you'd like to go further, here's how it works.
The second and third inequalities imply (add them together)
$ - 1 \leq \varphi \leq 1 $.
The first and fourth imply $ 0 \leq \omega \leq 1$.
That's not quite sufficient though.
The second and third imply (directly, divide by $1-\varphi$)
\begin{eqnarray*}
        - \varphi /(1-\varphi) \;\;\leq\;\; \omega \;\;\leq\;\; 1/(1-\varphi),
\end{eqnarray*}
a joint restriction on $\omega$ and $\varphi$.
If $\varphi \geq 0$ this is irrelevant,
it's less restrictive than our earlier condition on $\omega$.
But if $\varphi < 0$, it limits the range of $\omega$.
For example, if $\varphi = -1/2$,
then $ 1/3 \leq \omega \leq 2/3$.

\item [(b,c)] The $k$-period transitions have the form
\begin{eqnarray*}
    P &=& (1-\varphi^k)
        \left[
        \begin{array}{cc}
        \omega & 1-\omega \\ \omega & 1-\omega
        \end{array}
        \right]
        + \varphi^k
        \left[
        \begin{array}{cc}
        1  & 0  \\  0  & 1
        \end{array}
        \right] .
\end{eqnarray*}

\item [(d)]
If $ | \varphi | < 1$,
then as we increase $k$, $\varphi^k \rightarrow 0$ and $P$ converges to the
first matrix.
The equilibrium distribution is evidently $(\omega, 1-\omega)$.

\item [(e)]
If you're comfortable with linear algebra, you might notice that
$P$ has eigenvalues of 1 and $\varphi$.
The first is a feature of all Markov chains:
since the rows sum to one, there's an eigenvalue of one.
The second tells us how fast we converge to the equilibrium
distribution.
\end{enumerate}

\item {\it Properties of an MA(2).\/}
An MA(2) can be written
\begin{eqnarray*}
    x_t &=& \delta + w_t + \theta_1 w_{t-1} + \theta_2 w_{t-2}
\end{eqnarray*}
with $\{ w_t \} \sim \mbox{NID}(0,1)$
(the $w$'s are independent normals with mean zero and variance one).
%
\begin{enumerate}
\item What is the equilibrium distribution of $x$?
\item What are the conditional means,
$E_t (x_{t+1})$, $E_t (x_{t+2})$, and  $E_t (x_{t+3})$?
\item What are the conditional variances,
$\mbox{Var}_t (x_{t+1})$,
$\mbox{Var}_t (x_{t+2})$,
and $\mbox{Var}_t (x_{t+3})$?
\item What is the autocovariance function,
\begin{eqnarray*}
    \gamma(k)  &=& \mbox{Cov}(x_t,x_{t-k}),
\end{eqnarray*}
for $k=0,1,2,3$?
\item What is the autocorrelation function?
Under what conditions are $\rho(1)$ and $\rho(2)$ positive?
\end{enumerate}
%
\needspace{2\baselineskip}
Answer.
\begin{enumerate}
\item Since $x$ is a linear combination of normals,
it's normal as well.
It's therefore sufficient to say what its mean and variance are.
Its mean is
\begin{eqnarray*}
   E( x_t) &=& E (\delta + w_t + \theta_1 w_{t-1} + \theta_2 w_{t-2})
        \;\;=\;\; \delta .
\end{eqnarray*}
Its variance is
\begin{eqnarray*}
   \mbox{Var} ( x_t) &=& E (x_t - \delta)^2
        \;\;=\;\;  1 + \theta_1^2 + \theta_2^2 .
\end{eqnarray*}
\item The conditional means are
\begin{eqnarray*}
   E_t ( x_{t+1}) &=& E_t (\delta + w_{t+1} + \theta_1 w_{t} + \theta_2 w_{t-1})
        \;\;=\;\; \delta + \theta_1 w_{t} + \theta_2 w_{t-1} \\
   E_t ( x_{t+2}) &=& E_t (\delta + w_{t+2} + \theta_1 w_{t+1} + \theta_2 w_{t})
        \;\;=\;\; \delta + \theta_2 w_{t} \\
   E_t ( x_{t+3}) &=& E_t (\delta + w_{t+3} + \theta_1 w_{t+2} + \theta_2 w_{t+1})
        \;\;=\;\; \delta .
\end{eqnarray*}
You can see that  as we increase the forecast horizon,
the conditional mean approaches the mean.

\item The conditional variances are
\begin{eqnarray*}
   \mbox{Var}_t ( x_{t+1}) &=& E_t [(w_{t+1})^2]
        \;\;=\;\; 1  \\
   \mbox{Var}_t ( x_{t+2}) &=& E_t [(w_{t+2} + \theta_1 w_{t+1})^2]
        \;\;=\;\; 1  + \theta_1^2 \\
   \mbox{Var}_t ( x_{t+3}) &=& E_t [(w_{t+3} + \theta_1 w_{t+2} + \theta_2 w_{t+1})^2]
        \;\;=\;\; 1  + \theta_1^2 + \theta_2^2 .
\end{eqnarray*}
You see here that as we increase the time horizon,
the conditional variance approaches the variance.

\item The autocovariance function is
\begin{eqnarray*}
    \mbox{Cov}(x_t,x_{t-k}) &=&
            \left\{
            \begin{array}{ll}
            1 + \theta_1^2 + \theta_2^2 & k=0 \\
            \theta_1 + \theta_1 \theta_2 & k=1 \\
            \theta_2                    & k=2 \\
            0                           & k \geq 3 .
            \end{array}
            \right.
\end{eqnarray*}
\item Autocorrelations are scaled autocovariances:
$\rho(k) = \gamma(k)/\gamma(0)$.
$\rho(2)$ is positive if $\theta_2$ is.
$\rho(1)$ is positive if $\theta_1 (1+\theta_2)$ is.
Both are therefore positive if $\theta_1$ and $\theta_2$ are
positive.
\end{enumerate}

\item {\it Combination models.\/}
Consider the linear time series model
\begin{eqnarray*}
    x_{t} &=& \varphi x_{t-1} +  w_t ,
\end{eqnarray*}
with $\{ w_t \}$ independent normal random variables with mean zero and variance one.
Now consider a second random variable $y_t$ built from $x_t$ and
the same disturbance $w_t$ by
\begin{eqnarray*}
    y_{t} &=& x_{t} + \theta w_t .
\end{eqnarray*}
The question is how this combination behaves.

\begin{enumerate}
\item Is there a state variable for which $x_t$ is Markov?
What is the distribution of $x_{t+1}$ conditional on the state at date $t$?
\item Express $x_t$ as a moving average.
What are its coefficients?
\item Is there a state variable for which $y_t$ is Markov?
What is the distribution of $y_{t+1}$ conditional on the state at date $t$?
\item Express $y_t$ as a moving average.
What are its coefficients?
\item Under what conditions is $y_t$ stable?
That is: under what conditions does the distribution of $y_{t+k}$,
conditional on the state at $t$, converge as $k$ gets large?
\item What is the equilibrium distribution of $y_t$?
\item What is the first autocorrelation of $y_t$?
\end{enumerate}
%
\needspace{2\baselineskip}
Answer.
\begin{enumerate}
\item It's Markov with state $x_t$.
The conditional distribution of $x_{t+1}$,
\begin{eqnarray*}
    x_{t+1} &=& \varphi x_{t} + w_{t+1},
\end{eqnarray*}
is normal with mean $\varphi x_t $ and variance one.
\item The moving average representation is
\begin{eqnarray*}
    x_{t} &=& w_t + \varphi w_{t-1} + \varphi^2 w_{t-2}  + \cdots .
\end{eqnarray*}
The coefficients are $(1,\varphi, \varphi^2, \ldots )$.
\item Two answers, both work:  the state can be $x_t$ or (more commonly)
the vector $(y_t,w_t)$.
The distribution of $y_{t+1}$:
\begin{eqnarray*}
    y_{t+1} &=& \varphi x_t + (1+\theta) w_{t+1}
                \;\;=\;\; \varphi (y_t - \theta w_t) + (1+\theta) w_{t+1}
\end{eqnarray*}
is (conditionally) normal with mean $\varphi x_t = \varphi (y_t - \theta w_t)$
and variance $(1+\theta)^2)$.
\item If we add $\theta w_t$ to the expression for $x_t$ above, we get
\begin{eqnarray*}
    y_{t} &=& (1+\theta) w_t + \varphi w_{t-1} + \varphi^2 w_{t-2}  + \cdots .
\end{eqnarray*}
\item It's stable if $|\varphi| < 1$:  we need the moving average coefficients
 to approach zero.
 \item Equilibrium distribution:  $x_t$ is normal with mean zero
 and variance
 \begin{eqnarray*}
    \mbox{Var}(x_t) &=& (1+\theta)^2 + \varphi^2 + \varphi^4 + \cdots
            \;\;=\;\; (1+\theta)^2 + \varphi^2/(1-\varphi^2) .
 \end{eqnarray*}
\end{enumerate}

\item {\it Vector autoregressions.\/}
We can write many linear models in the form
\begin{eqnarray}
    x_{t+1} &=& A x_t + B w_{t+1} .
    \label{eq:state-space}
\end{eqnarray}
Here $x$ is a vector, $w \sim \mathcal{N}(0,I)$
is also a vector (of possibly different dimension),
and $(A,B)$ are matrices.
%
\begin{enumerate}
\item Consider the ARMA(1,1):
\begin{eqnarray*}
    y_t &=& \varphi_1 y_{t-1} + \theta_0 w_t + \theta_1 w_{t-1} .
\end{eqnarray*}
Show that this can be expressed in the same form as (\ref{eq:state-space}).
\item Ditto for the ARMA(2,1):
\begin{eqnarray*}
    y_t &=& \varphi_1 y_{t-1} + \varphi_2 y_{t-2} + \theta_0 w_t + \theta_1 w_{t-1} .
\end{eqnarray*}
\item For the general model (\ref{eq:state-space}),
what is the distribution of $x_{t+2}$ given $x_t$?
\item Ditto for $x_{t+k}$.
Under what conditions does this converge as $k$ gets large?
\end{enumerate}
%
\needspace{2\baselineskip}
Answer.
This course is designed to use linear algebra sparingly.
Nevertheless, in this case you might want to know
that seemingly complex models can often
be written simply in matrix form.
In that case, we're dealing with essentially higher-dimensional
analogs of an AR(1), which makes programming and insight much easier.
If this line of thought doesn't work for you, just ignore it.
%
\begin{enumerate}
\item The ARMA(1,1) can be written
\begin{eqnarray*}
    \left[
    \begin{array}{c}
    y_{t+1} \\ w_{t+1}
    \end{array}
    \right]
    &=&
    \left[
    \begin{array}{cc}
    \varphi_1 & \theta_1 \\ 0 & 0
    \end{array}
    \right]
    \left[
    \begin{array}{c}
    y_{t} \\ w_{t}
    \end{array}
    \right]
    +
    \left[
    \begin{array}{c}
    \theta_0 \\ 1
    \end{array}
    \right]
    [w_{t+1} ] ,
\end{eqnarray*}
which you should recognize from the notes on stochastic processes.

\item The ARMA(2,1) becomes
\begin{eqnarray*}
    \left[
    \begin{array}{c}
    y_{t+1} \\ y_t \\ w_{t+1}
    \end{array}
    \right]
    &=&
    \left[
    \begin{array}{ccc}
    \varphi_1 & \varphi_2 & \theta_1 \\ 1 & 0 & 0 \\ 0 & 0 & 0
    \end{array}
    \right]
    \left[
    \begin{array}{c}
    y_{t} \\ y_{t-1} \\ w_{t}
    \end{array}
    \right]
    +
    \left[
    \begin{array}{c}
    \theta_0 \\ 0 \\ 1
    \end{array}
    \right]
    [w_{t+1} ] .
\end{eqnarray*}

\item Now we can put the AR(1) structure to work.  After substituting,
\begin{eqnarray*}
    x_{t+2} &=& A^2 x_t + B w_{t+2} + A B w_{t+1} .
\end{eqnarray*}
The first term ($A^2 x_t$) is the conditional mean.
The others generate variance.  It goes beyond this course,
but for a vector like $w_t$ the variance is a matrix,
with variances on the diagonal and covariances otherwise.
If that's confusing, just skip it.

If not, we write the variance matrix as $E (w_t w_t^\top)$,
where $^\top$ is a brute force way to write the transpose.
In our case, we assumed the variance matrix for $w_t$ is $I$,
with ones on the diagonal and zeros off.

What's the variance matrix for $x_{t+2}$?
It's defined by
\begin{eqnarray*}
        \mbox{Var}_t (x_{t+2}) &=&
        E_t \left[ (x_{t+2} - A x_t) (x_{t+2} - A x_t)^\top \right] \\
                &=&
                E_t \left[ (B w_{t+2} + A B w_{t+1}) (B w_{t+2} + A B w_{t+1})^\top \right] \\
        &=& B B^\top + A B B^\top A^\top .
\end{eqnarray*}

\item Same idea repeated:
\begin{eqnarray*}
        \mbox{Var}_t (x_{t+k}) &=&
         B B^\top + A B B^\top A^\top + A^2 B B^\top (A^2)^\top
                + \cdots + A^{k-1} B B^\top (A^{k-1})^\top .
\end{eqnarray*}
As $k$ gets large, we get more terms.
It converges, though, if the terms shrink fast enough,
which requires powers of $A$ to shrink.
If you're familiar with linear algebra,
you need the eigenvalues of $A$ to be less than one in
absolute value.

\end{enumerate}


\end{enumerate}


\bigskip
\vfill \centerline{\it \copyright \ \number\year \
NYU Stern School of Business}


\pagebreak
% *****************************************************************************
{\large\bf Figure 1}
\newline{\large\bf Representative Event Tree}

\bigskip%\bigskip
\unitlength=1mm
\begin{picture}(120,160)(-10,-80)
{\Large \thicklines
%
\put(0,0){\makebox(0,0)[l]{\framebox{$z_0^{\phantom{0}}$}}}
%
\put(50,45){\makebox(0,0){\framebox{$z^1 = (z_0,z_1=1)$}}}
\put(50,-45){\makebox(0,0){\framebox{$z^1 = (z_0,z_1=2)$}}}
%
\put(109,68){\makebox(0,0){\framebox{$z^2 = (z_0,1,1)$}}}
\put(109,22){\makebox(0,0){\framebox{$z^2 = (z_0,1,2)$}}}
\put(109,-22){\makebox(0,0){\framebox{$z^2 = (z_0,2,1)$}}}
\put(109,-68){\makebox(0,0){\framebox{$z^2 = (z_0,2,2)$}}}
%
% lines/arrows
\put(8,1){\vector(1,2){22}} \put(8,-1){\vector(1,-2){22}}
%
\put(70,46){\vector(1,1){22}} \put(70,44){\vector(1,-1){22}}
\put(70,-44){\vector(1,1){22}} \put(70,-46){\vector(1,-1){22}}
%
% labels
\put(12,30){\makebox(0,0){$z_1=1$}}
\put(12,-30){\makebox(0,0){$z_1=2$}}
%
\put(75,62){\makebox(0,0){$z_2=1$}}
\put(75,27){\makebox(0,0){$z_2=2$}}
\put(75,-27){\makebox(0,0){$z_2=1$}}
\put(75,-62){\makebox(0,0){$z_2=2$}}
%
\put(48,53){\makebox{(A)}}
\put(48,-37){\makebox{(B)}}
\put(106,56){\makebox{(C)}}
%
}
\end{picture}
\vspace*{-0.1in}

The figure illustrates how uncertainty unfolds over time. Time
moves from left to right, starting at date $t=0$. At each date
$t$, an event $z_t$ occurs. In this example, $z_t$ is drawn from
the set $\mathcal{Z} = \{ 1,2 \} $. Each node is associated with a
box and can be identified from the path of events that leads to
it, which we refer to as a history and denote by $z^t \equiv (z_0,
... ,z_t)$, starting with an arbitrary initial node $z_0$. Thus
the upper right node follows two up branches, $z_1 = 1$ and $z_2 =
1$, and is denoted $z^2 = (z_0,1,1)$. The set $\mathcal{Z}^2$ of
all possible 2-period histories is therefore $\{ (z_0,1,1),
(z_0,1,2),(z_0,2,1),(z_0,2,2) \} $, illustrated by the ``terminal
nodes'' on the right.
Not shown are conditional probabilities of particular branches,
from which we can construct probabilities for each node/history.


\end{document}
