\documentclass[11pt]{article}

\oddsidemargin=0.25truein \evensidemargin=0.25truein
\topmargin=-0.5truein \textwidth=6.0truein \textheight=8.75truein

%\RequirePackage{graphicx}
\RequirePackage[hypertex]{hyperref}
\RequirePackage{comment}
\usepackage{amssymb}
\usepackage{amsfonts}
\usepackage{booktabs}

\renewcommand{\thefootnote}{\fnsymbol{footnote}}
\newcommand{\var}{\mbox{\it Var\/}}

% document starts here
\begin{document}
\parskip=\bigskipamount
\parindent=0.0in
\thispagestyle{empty}
{\large ECON-UB 233 \hfill Dave Backus @ NYU} 

\bigskip\bigskip
\centerline{\Large \bf Notes ... %
\footnote{Working notes, no guarantee of accuracy or sense.}}
\centerline{(Started: March 9, 2011; Revised: \today)}

\bigskip
This pulls together a few things we've done over the years.
The question is the following:  given a pricing kernel,
how does the return that maximizes the mean log return
compare to the return that maximizes the Sharpe ratio?
The first one hits the entropy bound,
the second hits the Hansen-Jagannathan bound.
We follow Bryan's lead (his 2010 qualifier question)
and derive the returns that hit the bounds
from maximization problems.


\subsection*{Preliminaries}

Environment.  Two periods.
You might think of this as summarizing the relationship between
a node and its successors in a more general dynamic setting.
Finite state space with elements $s$.
Probabilities are $p(s)$ --- these would be conditional probabilities
in a dynamic setting.

Bounds.  As usual, there's a pricing kernel $m(s)>0$
that prices all returns $r(s)$:
\begin{eqnarray}
    1 &=& E ( m r )
            \;\;=\;\; \sum_s p(s) m(s) r(s) .
            \label{eq:foc}
\end{eqnarray}
The entropy bound is
\begin{eqnarray}
    E ( \log r - \log r^1 )
            &\leq& \log E m - E \log m ,
            \label{eq:bound-entropy}
\end{eqnarray}
where $r^1 = 1/E m $ is the return on a one-period bond.
The Hansen-Jagannathan bound applies to
excess returns $x$,  differences between any two returns
satisfying (\ref{eq:foc}).
The bound is
\begin{eqnarray}
    E x /\var (x)^{1/2}
            &\leq& \var (m)^{1/2} /E m .
            \label{eq:bound-hj}
\end{eqnarray}
The left side is the Sharpe ratio for the investment strategy with
excess return $x$.


\subsection*{Hitting the bounds}

Suppose you could choose any returns $\{ r(s) \}$
that satisfy (\ref{eq:foc}).
What choices hit the two bounds?

The entropy bound is easy.  We choose the elements $r(s)$ to maximize
$ E \log r$ subject to the pricing relation (\ref{eq:foc}).
The Lagrangian problem is
\begin{eqnarray*}
    \max_{\{ r(s) \}} \;\; \mathcal{L} &=&  E \log r + \lambda [1 - E(mr)]
            \;\;=\;\; \sum_s p(s) \log r(s)
        + \lambda \left(  1 - \sum_s p(s) m(s) r(s) \right) .
\end{eqnarray*}
The foc for $r(s)$ is
\begin{eqnarray*}
    p(s)/r(s) &=& \lambda p(s) m(s) .
\end{eqnarray*}
If we multiply by $r(s)$ and sum over $s$, we find $\lambda = 1$.
That gives us the return $r(s) = 1/m(s)$
or
\begin{eqnarray*}
     \log r(s) &=& - \log m(s) .
\end{eqnarray*}
Since $r^1 = E(m)$ is given, this maximizes the mean log excess return, too.

Now the HJ bound.
We maximize the (log) Sharpe ratio:
\begin{eqnarray*}
    \max_{\{ x(s) \}} \;\; \mathcal{L}
        &=& \log E(x) - (1/2) \log \var(x) + \lambda [0 - E(mx)] \\
        &=& \log \left[ \sum_s p(s) x(s) \right] - (1/2)
        \log \left[ \sum_s p(s) x(s)^2 -
                \left(\sum_s p(s) x(s)\right)^2 \right] \\
        && + \; \lambda \left[  0 - \sum_s p(s) m(s) x(s) \right] .
\end{eqnarray*}
% ugly ??
The foc for $x(s)$ is
\begin{eqnarray*}
    p(s)/E(x)- p(s) [x(s) - E(x)]/\var(x)  &=& \lambda p(s) m(s) .
\end{eqnarray*}
Sum over $s$ to get
\begin{eqnarray*}
    1/E(x)  &=& \lambda E(m) .
\end{eqnarray*}
The foc becomes
\begin{eqnarray}
    \frac{x(s) - E(x)}{\var(x)} &=& -
                \frac{m(s) - E(m)}{E(m)E(x)} .
                \label{eq:xs-hj-prelim}
\end{eqnarray}
That is, we have a perfect negative correlation between the
excess return $x$ and the pricing kernel $m$.

This isn't a solution yet, since the mean and variance of $x$
haven't been nailed down.
If we square equation (\ref{eq:xs-hj-prelim}) and take its expectation,
we get
\begin{eqnarray*}
    \frac{1}{\var(x)}  &=& \frac{\var(m)} {E(m)^2 E(x)^2}
    \;\;\;\; \Leftrightarrow \;\;\;\;
    E(x) E(m)  \;\;=\;\; \var(m)^{1/2}\var(x)^{1/2} .
\end{eqnarray*}
[Need to worry about sign of $E(x)$, but see below.]
This verifies that we've hit the bound (\ref{eq:bound-hj}).
It also allows us to rewrite (\ref{eq:xs-hj-prelim}) as
\begin{eqnarray}
    \frac{x(s) - E(x)}{\var(x)^{1/2}} &=& -
                \frac{m(s) - E(m)}{\var(m)^{1/2}} .
                \label{eq:xs-hj-final}
\end{eqnarray}
You can see from the bound that we've nailed down the Sharpe ratio,
but not the individual terms.
In fact, they're not determined.
The Sharpe ratio is invariant to leverage:
if $x$ is a solution, then so is $\lambda x$ for any $\lambda$.
If $E(x)$ is negative, we just choose $\lambda<0$.
So let's choose an arbitrary $E(x)$.
Then $\var(x)$ is determined by the bound and
(\ref{eq:xs-hj-final}) is a complete characterization of the solution.


\subsection*{Discussion}

1. Comparison.  The entropy bound gives us an exact loglinear relation
between the return and the pricing kernel.
The HJ bound gives us an exact linear relation.
They're clearly not the same if we have more than 2 states.

2. Lognormal case.
Suppose $\log m \sim \mathcal{N}(\kappa_1,\kappa_2)$.
Then the return that hits the entropy bound is
$\log r = - \log m \sim \mathcal{N}(-\kappa_1,\kappa_2)$.
%The return that hits the HJ bound is
[Should be able to show that the return that maxes the Sharpe
ratio is approximately the same for small time intervals:
think of $\kappa_1$ and $\kappa_2$ as being $o(dt)$.
But they're not exactly the same.]

3. Examples.  For a multiperiod investment, you want to
maximize the mean log return.
Then there are probably examples in the literature that generate higher
Sharpe ratios but do not contribute to mean log returns.
In this sense, the Sharpe ratio is misleading.
Does that sound right?  Do you know any good examples?  I'm guessing
something beyond the lognormal case would be helpful.

%There are some examples if you Google
%``Sharpe ratio and growth optimal portfolio''.



\end{document}
