\documentclass[11pt]{article}

% spacing on page
\oddsidemargin=0.25truein \evensidemargin=0.25truein
\topmargin=-0.5truein \textwidth=6.0truein \textheight=8.75truein

\usepackage{comment}
\usepackage{graphicx}
\usepackage{amssymb}
\usepackage{amsmath}

\usepackage[margin=0pt, labelsep=period, font=large, labelfont=bf]{caption}
%\usepackage{float}

\usepackage{hyperref}
\urlstyle{rm}   % change fonts for url's (from Chad Jones)
\hypersetup{
    colorlinks=true,        % kills boxes
%    allcolors=blue,
    pdfsubject={ECON-UB233, Macroeconomic foundations for asset pricing},
    pdfauthor={Dave Backus @ NYU},
    pdfstartview={FitH},
    pdfpagemode={UseNone},
%    pdfnewwindow=true,      % links in new window
%    linkcolor=blue,         % color of internal links
%    citecolor=blue,         % color of links to bibliography
%    filecolor=blue,         % color of file links
%    urlcolor=blue           % color of external links
% see:  http://www.tug.org/applications/hyperref/manual.html
}

% for listing code in tt font
\usepackage{verbatim}

% for table spacing
\usepackage{booktabs}

% section headers and spacing
\usepackage[small, compact]{titlesec}

% list spacing
\usepackage{enumitem}
\setitemize{leftmargin=*, topsep=0pt}
\setenumerate{leftmargin=*, topsep=0pt}

% attach files to the pdf
\usepackage{attachfile}
    \attachfilesetup{color=0.75 0 0.75}

\usepackage{needspace}
% \needspace{4\baselineskip} makes sure we have four lines available before a pagebreak

%\renewcommand{\thefootnote}{\fnsymbol{footnote}}
%\newcommand{\var}{\mbox{\it Var\/}}

\newcommand{\cbar}{\bar{c}}
\newcommand{\rp}{\mbox{\em rp\/}}
\newcommand{\tp}{\mbox{\em tp\/}}
\newcommand{\bp}{\mbox{\em bp\/}}


% document starts here
\begin{document}
\parskip=\bigskipamount
\parindent=0.0in
\thispagestyle{empty}
{\large ECON-UB 233 \hfill Dave Backus @ NYU}% 

\bigskip\bigskip
\centerline{\Large \bf Review for Quiz \#3}
\centerline{Revised: \today}

\bigskip
I'll focus as usual on the big picture.
For each topic/result/ concept, I recommend you construct a numerical example or two to
remind yourself how it works.

The topic is dynamics.  We explore various dynamic models in Markov environments.
The Markov structure means that randomness is connected to random variation in 
some state variable.  
Everything else is simply a function of the state.  
The question is what the function is and what properties it implies for the variable of interest. 
The Markov structure leads naturally to recursive methods, in which solutions
are characterized by recursions.  



\section{Stochastic processes}

The term for the combination of dynamics and randomness is {\it stochastic process\/}.
It's common to characterize a stochastic process with its one-period conditional distribution:
the distribution over outcomes next period given 
some description of the current situation, which we refer to as the {\it state\/}.
Turning this around:  the state is whatever definition of the current situation you need
to describe the distribution over next period outcomes. 

The simplest example is a tree.
A state in this case is a node in the tree.
The outcomes next period are the branches coming out of the node.
The conditional distribution consists of the probabilities of these branches.
We construct probabilities of longer paths by multiplying the probabilities of individual branches.

Trees are easy to describe but cumbersome to use;
the description of a state --- the node in a tree --- is the complete
history of outcomes to that point.
Generally we impose more structure than this.
We proposed stochastic processes with three properties:
%
\begin{itemize}
\item {\it Markov.\/} There's a (hopefully simple) description of the state
that tells us all we need to know about the current situation.
%More formally:  the state variable is sufficient for characterizing the distribution
%now over outcomes next period.
\item {\it Stationary.\/}
The distribution over next period's outcomes conditional on this period's state
has the same structure at all dates.
\item {\it Stable.\/}
The conditional distribution over states in the distant future settles down
to a unique equilibrium distribution.
\end{itemize}

There are lots of examples of such processes, but we'll use
examples that are linear:  AR, MA, and ARMA models.
An example is an ARMA(1,1):
\begin{eqnarray*}
    x_{t+1} &=& \varphi x_{t} + \sigma ( w_{t+1} + \theta w_t ) ,
\end{eqnarray*}
where $ w_t$ is a sequence of independent standard normal random variables.
This is Markov with state $z_t = (x_t,w_t)$.
Conditional on $z_t$, $x_{t+1}$ is normal with mean $\varphi x_t + \sigma\theta w_t$
and variance $\sigma^2$.
If $| \varphi | < 1$,
the distribution of $x_{t+k}$ conditional on $z_t$ settles down:
it's normal with mean zero and variance $ \sigma^2 + (\varphi+\theta)^2/(1-\varphi^2) $.

We can describe the properties of a stationary, stable, Markov process
with its autocovariance and autocorrelation functions:
covariances $\gamma(k)$ and correlations $\rho(k)$, respectively,
of a variable $x_t$ and its own past $x_{t-k} $.
We say $x$ is persistent if the autocorrelations $\rho(k)$ are ``large.''
Interest rates are a good example:
the first autocorrelation $\rho(1)$ is well above 0.9 for monthly data.



\section{Forward-looking models}

There's a sense in which time runs backwards in economics:
current decisions depend on what we expect to happen in the future.
It doesn't really run backwards, because what we expect depends on the past and present,
but {\it expectations\/} of future outcomes show up in lots of places:
in valuing assets (what do we expect dividends to be?),
in deciding how much to consume (what will our future income be?),
and so on.

The simplest example of this is the linear model
\begin{eqnarray*}
    y_t &=& \lambda E_t (y_{t+1}) + x_t .
\end{eqnarray*}
Here $y_t$ is the variable of interest (the stock price, for example)
and $x_t$ is the ``fundamental'' the drives it (the dividend).
We generally solve this forward in time and apply the law of iterated
expectations to get
\begin{eqnarray*}
    y_t &=& \sum_{j=0}^\infty \lambda^j E_t (x_{t+j}) .
\end{eqnarray*}
This works as long as $|\lambda | < 1$ and $x$ is stable.
Some models of bubbles add an extra term to the solution:
$ c_t \lambda^{-t}$ where $ c_t$ is a martingale.


\section{Recursive methods}

Our approach to forward-looking models builds in the idea that 
the present depends on what we expect to happen in the future, 
but we ignored risk.  
We focused on the mean --- the expectation --- and ignored 
the risk adjustments that we have emphasized so far.  

Recursive methods generalize the forward-looking thinking we used earlier
and allow us to incorporate risk.  
For bonds, this takes the form 
\begin{eqnarray*}
    q^{n+1}(z_t) &=&  E \left[ m(z_t,z_{t+1}) q^n(z_{t+1}) | z_t \right] .
\end{eqnarray*}
The $n$-period bond price $q^n$ at date $t$ depends on the state $z_t$.  
typically we find these functions one at a time, starting with $q^0(z_t) = 1$. 
Other applications of recursion include equity and American options.  


\section{Bond pricing}

We started with definitions:  given the prices of zero-coupon bonds ---
the prices at date $t$ of a claim to one at date $t+n$ --- 
we define continuously-compounded yields and forward rates.
Prices, yields, and forward rates are different ways of presenting
the same information:  given one, we can compute the other two.

The next steps are to (i) compute bond prices as a function of the current state
and (ii)~choose parameter values so that the model reproduces
the salient features of forward rates.  
(We use forward rates, but yields and prices would do, too;
they contain the same information.) 

We value assets as we always do:  the no-arbitrage theorem
tells us there exists a positive pricing kernel that values all cash flows.
With bonds, the cash flows are constant, the pricing kernel
is all we need.
We do this recursively:
\begin{eqnarray*}
    q^{n+1}_t &=& E_t \big( m_{t+1} q^n_{t+1} \big)
\end{eqnarray*}
starting with $q^0_t = 1$ (the price of one today is one).
%In words:  an $(n+1)$-period bond is a claim to an $n$-period bond
%in one period.

We focused our attention on loglinear models,
which sometimes go by the label ``exponential-affine.''
We can illustrate the idea with the Vasicek model,
which we'll express here by
\begin{eqnarray*}
    \log m_{t+1} &=& - \lambda^2/2 - z_t + \lambda w_{t+1} \\
         z_{t+1} &=& (1-\varphi) \delta + \varphi z_t + \sigma w_{t+1} .
\end{eqnarray*}
[You might ask yourself what the dynamics of $\log m$ look like once
we've substituted for $x$.]
With this structure, bond prices are loglinear in the state variable $x_t$.
We can express them by
\begin{eqnarray*}
    \log q^n_t &=& A_n + B_n x_t
\end{eqnarray*}
for some constants $(A_n,B_n)$.
[You might derive recursions that connect
$(A_{n+1},B_{n+1})$ to $(A_n,B_n)$.]


Given such a model, we choose parameter values to approximate the
salient features of interest rates.
We chose four such features and linked them to the four parameters
of the model:
the mean, variance, and autocorrelation of the short rate;
and the mean slope of the forward rate curve.
The Vasicek model is the simplest such example.

There are a couple conceptual issues here worth noting:
%
\begin{itemize}

\item Expectations of the future.
This model is forward-looking, but there's a change in perspective from our earlier work,
where the conditional mean was all that mattered.
There we had terms like $y_t = E_t (x_{t+1}) $.
%(not literally, we're making a conceptual point).
Here we have, in the case of a one-period bond,
\begin{eqnarray*}
    \log q^1_t &=& \log E_t \big( e^{\log m_{t+1}} \big) .
\end{eqnarray*}
If we reverse the order of the $\log$ and $\exp$, we have 
\begin{eqnarray*}
    \log q^1_t &=& E_t \log \big( e^{\log m_{t+1}} \big) 
            \;\;=\;\; E_t \big( {\log m_{t+1}} \big) 
\end{eqnarray*}
and only the mean matters.  
But if we stick with the original, 
we get the cumulant generating function of $\log m_{t+1}$
evaluated at $s=1$.
That introduces risk into the calculation, 
specifically all of the cumulants of $\log m_{t+1}$.  

\item Markov structure.
As long as the model has a stationary Markov structure,
bond prices of all maturities will be functions of the state.
Here the function is loglinear, which is really convenient.
\end{itemize}


\section{Predictable returns} 

We concluded by describing, in brief, some of the evidence on 
what has come to be called return predictability and 
interpreting some of it to the bond pricing models we worked with earlier.  


{\vfill
{\bigskip \centerline{\it \copyright \ \number\year \
David Backus $|$ NYU Stern School of Business}%
}}


\end{document}
