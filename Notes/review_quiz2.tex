\documentclass[11pt]{article}

\oddsidemargin=0.25truein \evensidemargin=0.25truein
\topmargin=-0.5truein \textwidth=6.0truein \textheight=8.75truein

%\usepackage{graphicx}
\usepackage{verbatim}
%\usepackage{booktabs}
\usepackage{comment}
\usepackage{hyperref}
\urlstyle{rm}   % change fonts for url's (from Chad Jones)
\hypersetup{
    colorlinks=true,        % kills boxes
    allcolors=blue,
    pdfsubject={ECON-UB233, Macroeconomic foundations for asset pricing},
    pdfauthor={Dave Backus @ NYU},
    pdfstartview={FitH},
    pdfpagemode={UseNone},
%    pdfnewwindow=true,      % links in new window
%    linkcolor=blue,         % color of internal links
%    citecolor=blue,         % color of links to bibliography
%    filecolor=blue,         % color of file links
%    urlcolor=blue           % color of external links
% see:  http://www.tug.org/applications/hyperref/manual.html
}
% more from Chad Jones
%\usepackage{tweaklist}
%\renewcommand{\enumhook}{\setlength{\topsep}{0pt}%
%  \setlength{\itemsep}{0pt}}

\usepackage[small, compact]{titlesec}
\renewcommand{\thefootnote}{\fnsymbol{footnote}}

% document starts here
\begin{document}
\parskip=\bigskipamount
\parindent=0.0in
\thispagestyle{empty}
{\large ECON-UB 233 \hfill Dave Backus @ NYU}

\bigskip\bigskip
\centerline{\Large \bf Review for Quiz \#2}
\centerline{Revised: \today}

\bigskip
I'll focus again on the big picture to give you a sense of
what we've done and how it fits together.
For each topic/result/concept, I recommend you construct a numerical example or two to
remind yourself how it works.


\section{Asset pricing:  summary}

This is fundamental, worth repeating.

The {\it no-arbitrage theorem\/} tells us that we can price
assets using state prices:
break an asset's payoff into state-contingent pieces,
multiply each one by its state price, and add them up.
That turns states and state prices from hopelessly abstract objects
into objects with practical value.

The no-abritrage theorem says:
Consider a collection of assets $j$ with prices $q^j$ that pay dividends $d^j(z)$
in state $z$.
If there are no arbitrage opportunities in this collection of assets,
then there are positive state prices $Q(z)$ that price the assets:
\begin{eqnarray}
    q^j &=& \sum_z Q(z) d^j(z)
    \label{eq:qj-state-price}
\end{eqnarray}
for all assets $j$.

There are two versions of this wonderful result that are more commonly used.
The first is based on a {\it pricing kernel\/} $m$,
defined implicitly by $ Q(z) = p(z) m(z)$.
(Solve for $m$ if you like.)
The pricing relation (\ref{eq:qj-state-price}) becomes
\begin{eqnarray}
    q^j &=& \sum_z p(z) m(z) d^j(z) \;\;=\;\; E (m d^j) .
    \label{eq:q=Emd}
\end{eqnarray}
From this, we can show that risk premiums stem from covariances of returns with $m$.
[Can you show this?]
There's also has a nice link to representative agent models,
where $m$ is the agent's marginal rate of substitution.

The second version is based on
{\it risk-neutral probabilities\/} $p^*$,
defined implicitly by $ Q(z) = p(z) m(z) = q^1 p^*(z)$.
%(Solve for $p^*$ if you like.)
Here
\begin{eqnarray*}
    q^1 &=& \sum_z p(z) m(z) \;\;=\;\; E(m)
\end{eqnarray*}
is the price of a one-period riskfree bond.
The pricing relation (\ref{eq:qj-state-price}) now turns into
\begin{eqnarray}
    q^j &=& q^1 \sum_z p^*(z) d^j(z)  \;\;=\;\; q^1 E^* (d^j) ,
    \label{eq:q=Estar-d}
\end{eqnarray}
where $E^*$ means the expectation computed from the risk-neutral probabilities.
Once we know $p^*$, it's a little simpler because we only need the
expectation of $d^j$, not the expectation of the product $m d^j$.
The reason, of course, is that the product is built into $p^*$.
This version is useful in pricing derivatives,
because the risk-neutral distribution is the same for all of them.


\section{Properties of pricing kernels}

Here are three applications of the pricing kernel.
The first is to representative agent models,
which provide a source of insight into which assets have
the highest expected returns.
The other two relate risk premiums to the dispersion of the pricing kernel.

{\it Application 1:  representative agent.\/}
Here the pricing kernel $m$ is the marginal rate of substitution
of a representative agent.
If $g(z)$ is the growth rate of consumption and the agent has power utility,
the pricing kernel is
$ m(z) = \beta g(z)^{-\alpha}$.

The classic example here is the equity premium.
Define equity as a claim to a dividend tied to the same growth rate,
such as $ d(z) = g(z)^\lambda$.
If we choose a distribution for $g$ and ``reasonable'' values for  parameters,
the question is whether a model of this sort can generate an equity premium similar
to what we observe.
The answer is, basically, no.
We saw that for returns measured in levels, returns in logs,
a two-state distribution, a lognormal distribution, and some others.
Unless we have a very large value of $\alpha$, we don't have a chance.
One mechanism that made modest progress was to have strong negative skewness
in (log) consumption growth, which tends to increase risk premiums.


{\it Application 2: HJ bound.\/}
The question is where we went wrong.
The Hansen-Jagannathan bound suggests that a power utility pricing kernel
doesn't have much of a chance to start with.
We start with the {\it Sharpe ratio\/}, the ratio of the mean of an excess return
to its standard deviation.
If we look at returns on lots of assets, the idea is to look for the asset
with the largest Sharpe ratio.
Hansen and Jagannathan show that the largest Sharpe ratio places a lower bound on
the ratio of the standard deviation of the pricing kernel to its mean:
\begin{eqnarray*}
  \frac{| E(x)|}{ \mbox{\it Std\/}(x)} &\leq& \frac{ \mbox{\it Std\/}(m)}{E(m)} .
\end{eqnarray*}
Roughly speaking, large Sharpe ratios imply large standard deviations of the pricing kernel.

One way to think about the representative agent model's failure with the equity premium
is that it doesn't deliver enough variation in the pricing kernel.


{\it Application 3: entropy bound.\/}
Similar idea, somewhat different implementation.
We define the entropy of the pricing kernel by
\begin{eqnarray*}
    H(m) &=& \log E(m) - E \log m .
\end{eqnarray*}
This is a measure of dispersion or variation:
it's nonnegative, and strictly positive unless $m$ is constant.
The bound here tells us that
\begin{eqnarray*}
    E  \log r^j - \log r^1  &\leq&  H(m) .
\end{eqnarray*}
The difficulty here with the representative agent model is
that it doesn't generate enough entropy in the pricing kernel.

One of the nice things about entropy as a measure is that it incorporates
things like skewness and excess kurtosis naturally.
We can think of the first term in entropy as the cgf of $\log m$
evaluated at $s=1$:
\begin{eqnarray*}
   \log E(m) &=& \log E \left( e^{\log m}\right)
        \;\;=\;\; k(1; \log m)
        \;\;=\;\; \kappa_1 + \kappa_2/ 2 + \kappa_3/3! + \kappa_4/4! + \cdots .
\end{eqnarray*}
When we subtract the mean, $E \log m = \kappa_1$, we get entropy:
\begin{eqnarray*}
   H(m)  &=& \log E(m) - E \log m
        \;\;=\;\; \kappa_2/ 2 + \kappa_3/3! + \kappa_4/4! + \cdots .
\end{eqnarray*}
In the lognormal case, only $\kappa_2$ is nonzero.
Otherwise, the other cumulants play a role.


\section{Option pricing}

We have several components that combine to give us a complete picture.
The notation below isn't self-contained, you may need to go back to the notes.


{\it Options.\/}
Options are the right to buy or sell an asset
(the ``underlying'')
at a fixed price $k$ (the ``strike'')
at (or by) some future date.
We'll label the dates $t$ (``now'') and $t+1$ (or $t+\tau$) (``later'').
The question is what that right is worth now.
We're going to value it using risk-neutral probabilities.

{\it Put-call parity.\/}
For European options, there's a connection between prices of put options
(the right to sell) and call options (the right to buy):
\begin{eqnarray*}
    \underbrace{q^c_t}_{\mbox{buy call}} -
    \underbrace{q^p_t}_{\mbox{sell put}} +
    \underbrace{q^\tau_t k}_{\mbox{present value of strike}}
    &=&
    \underbrace{s_t}_{\mbox{buy stock}} .
    \label{eq:put-call-parity}
\end{eqnarray*}
This holds for every strike $k$.
It works pretty well in practice, as we saw in Lab Report \#6.

{\it BSM formula and implied volatility.\/}
The standard textbook formula for a call option is
\begin{eqnarray*}
        q^c_t &=& s_t N(d) - q^\tau_t b N (d - \tau^{1/2}\sigma ) \\
%        \label{eq:bsm-call}
          d &=& \frac{\log(s_t/q^\tau_t k) + \tau \sigma^2/2}{\tau^{1/2}\sigma} .
\end{eqnarray*}
Everything is observable here except $\sigma$, which
we can back out from the price.
That is: given $\sigma$, we use the formula to compute the price.
But if we know the price, we can reverse the process and compute what is
commonly referred to as
{\it implied volatility\/}:  the value of $\sigma$ for which the formula
delivers the observed price.

One of the reliable facts about option prices is that volatility varies with the strike.
The shape of the line in a  graph of volatility against the strike price
is referred to as the {\it volatility smile\/}.
We'll think of it as a convenient way to summarize option prices.

{\it Risk-neutral pricing.\/}
We've seen that any cash flow can be valued using (\ref{eq:q=Estar-d}).
Options are no different.
The question is generally what the risk-neutral distribution is.
We need a risk-neutral distribution that prices both the underlying
and options on the underlying.

Let's be specific.
The risk-neutral pricing relation (\ref{eq:q=Estar-d})
applied to the underlying gives us
\begin{eqnarray*}
    s_t &=& q^1_t  E^* (s_{t+1}) .
\end{eqnarray*}
Given a risk-neutral distribution, this gives us $s_t$.
Usually we observe $s_t$, and this gives rise to a restriction
on the risk-neutral distribution that we call the
{\it no-arbitrage condition\/}.

The classic example is the lognormal case.
Suppose $x_{t+1} = \log s_{t+1} \sim \mathcal{N}(\kappa_1, \kappa_2)$
(remember:  this is the risk-neutral distribution).
Then the pricing relation implies
\begin{eqnarray*}
    s_t &=& q^1_t E^* (e^{x_{t+1}}) \;\;=\;\; q^1_t  e^{\kappa_1 + \kappa_2/2} .
\end{eqnarray*}
Usually we choose $\kappa_1$ to satisfy the equation given information
about the other components.

*****
{\it Option prices as integrals.\/}
Show how this works, in general.  ???

{\it BSM from lognormal risk-neutral underlying.\/}
With these ingredients, we can derive the BSM formula using
a lognormal risk-neutral distribution of the underlying.
A put price is at strike $k$ is
\begin{eqnarray*}
    q^p_t &=& q^1_t  E^* (k-e^{x_{t+1}})^+
          \;\;=\;\;  q^1_t  \int_{-\infty}^{\log k}
          (k-e^{x_{t+1}}) (2 \pi \kappa_2)^{-1/2}
          \exp[ - (x_{t+1} - \kappa_1)^2/2 \kappa_2] d x_{t+1} .
\end{eqnarray*}
Integrating gives us
\begin{eqnarray*}
    q^p_t &=& q^1_t k N(d) - q^1_t e^{\kappa_1 + \kappa_2/2} N(d-\kappa_2^{1/2}) \\
    d &=& (\log k - \kappa_1)/\kappa_2^{1/2} .
\end{eqnarray*}
We get the BSM formula for $\tau=1$ by setting $\kappa_2 = \sigma^2$
and using the no-arbitrage condition.


{\it Beyond BSM.\/}
It's not hard to get option formulas that differ from BSM:
we simply start with a risk-neutral distribution of the underlying
that's not lognormal.
There are lots of examples.
We used ``normal mixtures,"
which have some of the analytical convenience of normality
but more flexibility.


\vfill \centerline{\it \copyright \ \number\year \
NYU Stern School of Business}
\end{document}
