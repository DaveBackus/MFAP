\documentclass[11pt]{article}

% spacing on page
\oddsidemargin=0.25truein \evensidemargin=0.25truein
\topmargin=-0.5truein \textwidth=6.0truein \textheight=8.75truein

\usepackage{comment}
\usepackage{graphicx}
\usepackage{amssymb}
\usepackage{amsmath}

\usepackage[margin=0pt, labelsep=period, font=large, labelfont=bf]{caption}
%\usepackage{float}

\usepackage{hyperref}
\urlstyle{rm}   % change fonts for url's (from Chad Jones)
\hypersetup{
    colorlinks=true,        % kills boxes
%    allcolors=blue,
    pdfsubject={ECON-UB233, Macroeconomic foundations for asset pricing},
    pdfauthor={Dave Backus @ NYU},
    pdfstartview={FitH},
    pdfpagemode={UseNone},
%    pdfnewwindow=true,      % links in new window
%    linkcolor=blue,         % color of internal links
%    citecolor=blue,         % color of links to bibliography
%    filecolor=blue,         % color of file links
%    urlcolor=blue           % color of external links
% see:  http://www.tug.org/applications/hyperref/manual.html
}

% for listing code in tt font
\usepackage{verbatim}

% for table spacing
\usepackage{booktabs}

% section headers and spacing
\usepackage[small, compact]{titlesec}

% list spacing
\usepackage{enumitem}
\setitemize{leftmargin=*, topsep=0pt}
\setenumerate{leftmargin=*, topsep=0pt}

% attach files to the pdf
\usepackage{attachfile}
    \attachfilesetup{color=0.75 0 0.75}

\usepackage{needspace}
% \needspace{4\baselineskip} makes sure we have four lines available before a pagebreak

%\renewcommand{\thefootnote}{\fnsymbol{footnote}}
%\newcommand{\var}{\mbox{\it Var\/}}

\newcommand{\cbar}{\bar{c}}
\newcommand{\rp}{\mbox{\em rp\/}}
\newcommand{\tp}{\mbox{\em tp\/}}
\newcommand{\bp}{\mbox{\em bp\/}}


\begin{document}
\parskip=\bigskipamount
\parindent=0.0in
\thispagestyle{empty}
{\large ECON-UB 233 \hfill Dave Backus @ NYU}% 

\bigskip\bigskip
\centerline{\Large \bf Two-Period Macroeconomic Models}
\centerline{Revised: \today}

\bigskip
Our next step is to look at equilibrium prices and quantities
in the same two-period setting we used earlier.
The idea is to combine the consumption,
saving, and portfolio decisions of agents we studied earlier
with the production and investment decisions of firms
and find the prices and quantities that clear markets.
We simplify each of these things as much as we can
in the interest of clarity.
The result is a relatively simple environment in which
we can talk, for the first time, about the effects
of risk on asset prices and returns.



\section{Keep it simple}

Einstein is reported to have said:
``Make things as simple as possible, but not simpler.''
It's not clear exactly what that means, or even whether Einstein said it,
but I'd like to think he'd agree we should keep things simple.
Economics, in any case, is a blend of realism and simplicity,
with most of the weight on simplicity.
Some people find the apparent lack of realism of economic models puzzling,
but it's often necessary to gain sharp insights.
The idea is to focus on the things that matter,
and simplify or ignore the things that don't.
The artist Georgia O'Keeffe said something similar:
``Nothing is less real than realism. Details are confusing. It is only by
selection, by elimination, by emphasis that we get to the real meaning of
things.''

Here's another version of the same sentiment,
an exchange between economist Robert Mundell and banker Alex McLeod.
The paper in question helped Mundell get the Nobel prize,
but McLeod thought it was unrealistic.
As a banker, he knew how things worked in the real world.
%(We have common friends, and I've been assured he's a reasonable guy.)
I've taken some liberty with the quotations, but the exchange went something like this:

McLeod:
\begin{quote}
Mundell's article [includes] a number of incongruous assumptions.
One is that complications associated with speculation in the forward market do not exist.
It can only bring discredit on the economics profession to leave unchallenged
his attempt to draw from the model policy conclusions that are applicable in the real world.
\end{quote}
Mundell:
\begin{quote}
Theory is the poetry of science.
It is simplification, abstraction, the exaggeration of truth.
Through simplification, theory creates a caricature of reality.
The caricature itself is not the real world --- it mocks it.
Yet mind true things by their mockeries!

Dr McLeod does not like my caricature; he calls my assumptions unrealistic.
I certainly hope he is right.  I hope my assumptions are unrealistic.
%If they were not, I could not have made a contribution to theory.
I left out a million variables that made my caricature of reality unrealistic.
At the same time, it enabled me to find fruitful empirical generalizations.
\end{quote}
%
[From:  {\it Canadian Journal of Economics and Political Science\/}, vol 30, 1964, pp 413-414, 419-421.]

The theory of Mundell's paper is a long way from anything
we'll do in this class,
but you have to love a guy who has the nerve to say
``theory is the poetry of science.''
The substantive point is that our goal isn't realism, it's insight and understanding.
The art of economics lies in making unrealistic simplifying assumptions that
deliver clear results that we believe generalize to more complex environments.
The art lies in making that distinction ---
in deciding which assumptions are crucial and
which simply make the analysis clearer.


\section{General equilibrium models}

In the 1950s, economists formalized ideas that date back to Adam Smith and before.
Lots of people were involved, but the central players
were Kenneth Arrow, Gerard Debreu, and Lionel McKenzie.
The first idea is to consider equilibrium (think: supply equals demand)
for all markets at once ---
what economists refer to as general equilibrium.
That's missing in the usual supply and demand diagram,
one of the reasons we don't stop there.
The second is to address the welfare properties of general equilibrium.
Is the equilibrium allocation of resources a good one in some sense?
Does Smith's invisible hand work?
We'll do a quick summary of the principle results.

{\it Ingredients.\/}  Here are the ingredients of a general equilibrium model ---
what we might term the physical environment, which includes
the objects involved but says nothing about how they interact.
The list would include:
%
\begin{itemize}
\item List of commodities.   What's produced?  What's consumed?
\item List of agents.  Who are the people in the model?  How many are there?
\item Preferences and endowments.  Each agent has a utility function
that describes her preferences over commodities.
Also a vector of endowments:  quantities of the various commodities that she starts with.
\item Technologies.  Production functions that transform combinations of some commodities (inputs)
into other commodities (outputs).
\item Resource constraints.  Uses (consumption) of each commodity are limited by
its sources (endowment net of other uses plus production).
\end{itemize}
Most theoretical economies consist of these ingredients in one form or other.
If there are no technologies, we refer to it as an {\it exchange economy\/};
since production is ruled out, all agents can do is exchange their endowments.

\needspace{2\baselineskip}
{\it Equilibrium.\/}
A {\it competitive equilibrium\/} for an environment like this consists of
%
\begin{itemize}
\item Agents  maximize.
Agents choose consumption quantities that maximize utility given
prices and budget constraints.
\item Firms maximize.
Firms choose inputs and outputs to maximize profits given prices and technologies.
\item Markets clear.
Sources (``supply'') equal uses (``demand'') for every commodity.
\end{itemize}
That sounds simple, but it was a significant achievement to formalize it
and prove (under suitable conditions on preferences and technologies)
that an equilibrium exists.
Note that it's a {\it competitive\/} equilibrium:
agents and firms take prices as given,
which my industrial organization colleagues find amusing.
(Equilibrium when agents or firms can influence prices
is a much more demanding problem,
not one we can leave to macroeconomists, I'm told.)

{\it Welfare.\/}  Now the invisible hand:
Is there a sense in which a competitive equilibrium delivers a
good allocation of resources?
It depends what you mean by ``good.''
Suppose we have two agents and consider two allocations,
one in which agent 1 gets a lot and one in which agent 2 gets a lot,
with the other getting what's left in each case.
Which one do we think is better?
We generally resist making comparisons between allocations like
this, because it forces us to decide which agent we like best.
Like parents, we care about all our agents equally.
We settle for a weaker definition of good:
we say an allocation is {\it Pareto optimal\/} (ie, good)
if we can't make one agent better off without making at least one
other agent worse.
This leaves some allocations noncomparable,
but seems like a useful condition to apply to an economy.
If we can make someone better off without hurting anyone,
surely the allocation isn't all that good.

With that background, we have two classical results:
%
\begin{itemize}
\item First welfare theorem.  A competitive equilibrium produces
a Pareto optimal allocation of resources.
\item Second welfare theorem.
A Pareto optimal allocation of resources can be reproduced
as a competitive equilibrium for some allocation
of endowments.
\end{itemize}
The first is the modern version of Adam Smith's invisible hand.
The second is (among other things) a useful shortcut for computing
a competitive equilibrium:  we compute a Pareto optimum instead.

All of this might seem like excess formalism.
On the contrary, I find it extremely helpful to keep track
of the ingredients and how they fit together.


\section{Finding an equilibrium as a Pareto problem}


We'll generally attack equilibrium indirectly by finding a Pareto optimum.
By the second welfare theorem, this corresponds to a competitive equilibrium for some
allocation of endowments.
In much of macroeconomics, finding the right allocation of endowments is easy
because we only have one agent, who therefore owns everything.

{\it Example (exchange economy).\/}
Let's see how this works in an example, a really simple one.
The ingredients are:
%
\begin{itemize}
\item List of commodities.   Two:  apples and bananas.
\item List of agents.  One.
\item Preferences and endowments.  The utility function is
\begin{eqnarray*}
    U(c_a,c_b) &=& \beta \log c_a + (1-\beta) \log c_b ,
\end{eqnarray*}
where $(c_a,c_b)$ is consumption.
The endowment is $(y_a,y_b)$.
\item Technologies.  None, this is an exchange economy.
\item Resource constraints.  They are
\begin{eqnarray*}
    c_a &\leq& y_a \\
    c_b &\leq& y_b .
\end{eqnarray*}
\end{itemize}
That's the environment:  one agent, two goods, no production.
You can't get much simpler than that,
but it serves to illustrate our approach.

The Pareto problem for this economy consists of maximizing utility subject
to the resource constraints:
\begin{eqnarray*}
    \max_{c_a,c_b} &&  \beta \log c_a + (1-\beta) \log c_b \\
    \mbox{s.t.}   &&  c_a \;\;\leq\;\; y_a \\
                  &&  c_b \;\;\leq\;\; y_b .
\end{eqnarray*}
The solution is obvious (consume the endowment),
but we'll solve it by Lagrangean methods using
multipliers $q_a$ and $q_b$ on the two constraints.
(This notation suggests we know something about the answer,
which we do:  the multipliers $q_a$ and $q_b$ turn out to be the market prices
in the competitive equilibrium associated with this Pareto optimum.)
The Lagrangean is
\begin{eqnarray*}
    \mathcal{L} &=&  \beta \log c_a + (1-\beta) \log c_b
                + q_a (y_a - c_a) + q_b (y_b - c_b) .
\end{eqnarray*}
The first-order conditions are
\begin{eqnarray*}
    \beta/c_a - q_a &=& 0 \\
   (1-\beta)/c_b - q_b &=& 0 .
\end{eqnarray*}
The constraints tell us to consume the endowments,
which nails down the Lagrange multipliers:
$ q_a = \beta /y_a$ and $q_b = (1-\beta)/ y_b$.


The nice thing about this approach is that
we get the prices from the Lagrange multipliers.
It's not hard to show that these are, in fact,
the competitive equilibrium prices.
The first condition for a competitive equilibrium is that agents maximize
utility given their budget constraints.
Here the budget constraint is
\begin{eqnarray*}
    q_a c_a + q_b c_b &\leq& q_a y_a + q_b y_b \;\;=\;\; Y.
\end{eqnarray*}
(The letter $Y$ is here out of laziness:  it's easier than writing
the stuff it's equal to.)
We saw earlier that this leads to the demand functions
$ c_a = \beta Y/q_a$ and $c_b = (1-\beta) Y/q_b$.
Does demand equal supply when we substitute the prices we derived
from the Pareto problem?
With those prices,
$ Y = q_a y_a + q_b y_b = \beta + (1-\beta) = 1$.
Demand for apples is therefore
\begin{eqnarray*}
    c_a &=& \beta Y / q_a \;\;=\;\; \beta / (\beta/y_a) \;\;=\;\; y_ a,
\end{eqnarray*}
so demand equals supply for apples.
The same is true for bananas, too.

The bottom line:  When we solve a Pareto problem,
we not only get the equilibrium quantities,
we get the equilibrium prices, too.


[Draw picture:  indifference curves, endowment point,
slope of tangent line.]

{\it Example (production economy).\/}
We can easily modify this economy to incorporate production.
Suppose we have no endowment of bananas ($y_b=0$),
but can transform apples into bananas with the
production function $f(k) = k^\theta$.
If we use $k$ to denote use of apples to produce bananas,
the Pareto problem becomes
\begin{eqnarray*}
    \max_{c_a,k,c_b} &&  \beta \log c_a + (1-\beta) \log c_b \\
    \mbox{s.t.}   &&  c_a + k \;\;\leq\;\; y_a \\
                  &&  c_b \;\;\leq\;\; k^\theta ,
\end{eqnarray*}
with parameters $ 0 < \beta,\theta < 1$.
The Lagrangean is
\begin{eqnarray*}
    \mathcal{L} &=&  \beta \log c_a + (1-\beta) \log c_b
                + q_a (y_a - c_a - k) + q_b (k^\theta - c_b ) .
\end{eqnarray*}
The first-order conditions are
\begin{eqnarray*}
    \beta/c_a - q_a &=& 0 \\
    - q_a + q_b \theta k^{\theta-1} &=& 0 \\
   (1-\beta)/c_b - q_b &=& 0 .
\end{eqnarray*}
They imply
\begin{eqnarray*}
    (y_a - k) \beta /(1-\beta) &=& k / \theta ,
\end{eqnarray*}
which we can solve for $k$.
Consumption quantities $(c_a,c_b)$ then follow from the
constraints.

[Update picture, add production possibilities frontier.]



\section{Two-period deterministic economies}

In macroeconomics, a useful starting point is an
economy with a single agent and some form of dynamics.
We keep the dynamics simple here with two periods,
dates 0 and 1.
Then we develop a series of examples of increasing complexity
to illustrate how the approach works and what it implies for asset pricing.
The insight is to treat consumption now and consumption next period
as different commodities, just as we treated apples and bananas
as different commodities above.

If we follow the formalism described above, we have the following ingredients:
%
\begin{itemize}
\item List of commodities.  Two:  the consumption good at dates 0 and 1.
\item List of agents.  One: a ``representative'' agent.
\item Preferences and endowments.  The utility function is
\begin{eqnarray*}
    U(c_0,c_1) &=& u(c_0) + \beta u(c_1) .
\end{eqnarray*}
Endowments are $(y_0,y_1)$.
\item Technologies.  The production function $f$ transforms input of the date-0 good
(label this $k$)
into output of the date-1 good.
\item Resource constraints.  For the two commodities,
we have
\begin{eqnarray*}
    c_0 + k &\leq& y_0 \\
    c_1 &\leq&  y_1 + f(k) .
\end{eqnarray*}
The idea is that date-0 investment $k$ produces date-1 output $f(k)$.
\end{itemize}
That's the physical environment.
%If we keep track of consumption quantities,
%that gives us $c_0 = y_0 - k$
%and $ c_1 = f(k) + y_1$.

%[?? direct method first?]

We could compute a competitive equilibrium directly,
but it's easier to compute a Pareto optimum and
use the second welfare theorem to tell us it's also
an equilibrium.
With only one agent, a Pareto optimum is the solution to:
\begin{eqnarray*}
    \max_{c_0,k,c_1} && u(c_0) + \beta u(c_1) \\
    \mbox{s.t.}   &&  c_0 + k \;\;\leq\;\; y_0 \\
                  &&  c_1 \;\;\leq\;\;  y_1 + f(k) .
\end{eqnarray*}
We could simply substitute the constraints into the objective
function and maximize, but using Lagrange multipliers gives us
the prices, too.
The Lagrangean is
\begin{eqnarray*}
    \mathcal{L} &=&  u(c_0) + \beta u(c_1) + q_0 (y_0 - c_0 - k)
                    + q_1 ( y_1 + f(k) - c_1)  .
\end{eqnarray*}
The first-order conditions are
\begin{eqnarray*}
    c_0: &&  u'(c_0) - q_0 \;\;=\;\; 0 \\
    k: &&  - q_0 + q_1 f'(k)  \;\;=\;\; 0 \\
    c_1: &&  \beta u'(c_1) - q_1 \;\;=\;\; 0 .
\end{eqnarray*}
They imply that the marginal rate of substitution (mrs)
and marginal rate of transformation (mrt)
equal the price ratio:
\begin{eqnarray*}
    \frac{\beta u'(c_1)}{u'(c_0)} &=& \frac{q_1}{q_0} \;\;=\;\; \frac{1}{f'(k)} .
\end{eqnarray*}
If we define $Q = q_1/q_0$ and the (gross) interest rate $r$ by $Q = 1/r$,
then we also have $ [\beta u'(c_1)/u'(c_0)] r = 1$ and $f'(k) = r$.
This differs from the consumption-saving problem in (possibly) having curvature
in $f$, so that $f'(k) = r$ isn't constant.

[Draw indifference curves and ppf again.]

Where does the interest rate $r$ come from?
It depends on preferences ($u$ and $\beta$) and technology ($f$).
A high interest rate is associated with productive capital (high $f'$)
and high date-1 consumption [hence low $u'(c_1)$].


\section{Two-period stochastic economies:  exchange version}

Now let's add uncertainty to the model, so that we can give a general
equilibrium interpretation to asset prices.
We deal with uncertainty as we did with consumption and portfolio choice.
We make decisions at date 0.
At date 1, any of a number of states $z$ can occur.
[Draw our usual event tree.]
If there are $Z$ states (a finite number), the ingredients of
an exchange economy with a single representative agent are
\begin{itemize}
\item List of commodities.  $Z+1$:  the consumption good at date 0
and in each state $z$ at date 1.
\item List of agents.  One: a ``representative'' agent.
\item Preferences and endowments.  Preferences are governed by the utility function
\begin{eqnarray*}
    U(c_0,c_1) &=& u(c_0) + \beta \sum_z p(z) u[c_1(z)] .
\end{eqnarray*}
Endowments are $\{y_0,y_1(z)\}$.
\item Technologies.  None.
\item Resource constraints.  For the two commodities,
we have
\begin{eqnarray*}
    c_0 &\leq& y_0 \\
    c_1(z) &\leq& y_1(z), \mbox{ for each } z .
\end{eqnarray*}
\end{itemize}
This has what we call {\it complete markets\/}:  we have one market for every state $z$.

The Pareto problem has a similar form:  maximize utility subject to
resource constraints.
The main difference from the previous section is that we have lots of resource
constraints, one for each state.
The problem is
\begin{eqnarray*}
    \max_{c_0,c_1(z)} && u(c_0) + \beta \sum_z p(z) u[c_1(z)] \\
    \mbox{s.t.}   &&  c_0  \;\;\leq\;\; y_0 \\
                  &&  c_1(z) \;\;\leq\;\; y_1(z) .
\end{eqnarray*}
The Lagrangean is
\begin{eqnarray*}
    \mathcal{L} &=&  u(c_0) + \beta \sum_z p(z) u[c_1(z)] + q_0 (y_0 - c_0)
                    + \sum_z q_1(z)  [ y_1(z) - c_1(z)]  .
\end{eqnarray*}
As before, we represent the Lagrange multipliers by $q$'s:
$q_0$ for the date-0 constraint and $q_1(z)$ for the date-1 constraint in state $z$.
The first-order conditions are
\begin{eqnarray*}
    c_0: &&  u'(c_0) - q_0 \;\;=\;\; 0 \\
    c_1(z): &&  \beta p(z) u'[c_1(z)] - q_1(z) \;\;=\;\; 0 .
\end{eqnarray*}
The solution, of course, is to consume the endowment in all states,
but this gives us the prices too.

[Draw indifference curves between states and endowment point.]

If we take the ratio of the second equation to the first,
we have
\begin{eqnarray}
    \frac{\beta p(z) u'[c_1(z)]}{u'(c_0)} &=&
    \frac{\beta p(z) u'[y_1(z)]}{u'(y_0)}
    \;\;=\;\; \frac{q_1(z)}{q_0} \;\;=\;\; Q(z) .
    \label{eq:state-price}
\end{eqnarray}
The first equality substitutes the equilibrium consumption quantities:
the endowments.
The subsequent equalities connect them to prices.
The ratio $Q(z) = q_1(z)/q_0$ is the price at date 0 of one unit in state $z$ at date 1:
what we have called the {\it state price\/}, the price of an Arrow security.
Note that the state price is connected to the agent's marginal rate of substitution,
just as it was in the portfolio choice problem.
More formally, we can write
$ Q(z) = \beta p(z) u'[c_1(z)]/u'(c_0) = p(z) m(z) $,
with $m$ defined by $m(z) = \beta u'[c_1(z)]/u'(c_0)$.
We will typically refer to $m(z)$ as the mrs in state $z$,
rather than $p(z) m(z)$, but that's a story for another time.

How do prices differ across states?
Consider the role of probability.
The state price formula (\ref{eq:state-price})
tells us that states with higher probability have higher prices,
other things equal.
More interesting is the role of the date-1 endowment.
The same equations tell us that states with higher output have lower prices
and higher returns.
Why?  Because $u'[c_1(z)] = u'[y_1(z)]$ is a decreasing function.
If we increase $y_1(z)$, then we decrease marginal utility $u'[y_1(z)]$,
the marginal rate of substitution $m(z) = \beta u'[y_1(z)]/u'(y_0)$,
and the state price $Q(z) = p(z) m(z)$.
In words: a payoff of one unit is more valuable when output is scarce
than when output is plentiful.
Securities that pay off in good states (high endowment) are therefore
cheaper than those that pay off in bad states (low endowment).
We'll see later that this provides a basis for positive risk premiums on such assets.
Securities that pay off mostly in good states have lower prices
and higher average returns than those that pay off mostly in bad states.

% ?? tighten up?


\section{Asset pricing in the exchange economy: an example}

We can be more specific about asset prices and returns
if we put more structure on the economy.
We apply two kinds of structure:  we give the agent power utility
and make the date-1 endowment lognormal (that is, the log of the
endowment is normal).
This is our first look at a collection of equations we'll
see over and over again.

Here's how asset pricing works in settings like this.
An asset consists, in general,
of a state-dependent dividend or payoff $d(z)$.
If its price is $q$, its return is $ r(z) = d(z)/q$.
We typically value assets in this order:
specify the dividend, compute its price, then calculate its return.
Given a marginal rate of substitution $m(z)$ and probabilities $p(z)$,
pricing follows from
\begin{eqnarray*}
    1 &=& E (mr) \;\;=\;\; \sum_z p(z) m(z) [d(z)/q] \;\;=\;\;  E [m (d/q)]] .
\end{eqnarray*}
Since $q$ doesn't depend on $z$, we can take it outside the expectation.
If we multiply both sides by $q$, we have
\begin{eqnarray*}
    q &=& E (md)  .
\end{eqnarray*}
Hence the term asset ``pricing.''

Since we'll be using normal and lognormal random variables,
it's helpful to remind ourselves of some of their properties.
Let us say that $x = \log y$ is normal;
in short-hand notation, $ x \sim \mathcal{N}(\kappa_1,\kappa_2)$.
Then we say $y = \exp(x) $ is lognormal (its log is normal).
Result 1:  The mgf of $x$ is
$h(s) = E(e^{sx}) = e^{\kappa_1 s +\kappa_2 s^2/2}$.
The cgf, of course, is the log of this.
Result 2:  If $x = \log y$, the mean of
$y = e^x$ is $E(y) = E(e^x) = e^{\kappa_1 + \kappa_2/2}$.
This also follows from the mgf of $x$:  evaluate $h$ at $s=1$.
We say $ y$ is lognormal (its log is normal).
Result 3:  Linear functions of $x$ are also normal.
Specifically:  $ y = a + b x \sim \mathcal{N}(a + b \kappa_1, b^2 \kappa_2)$.
(If you have any question about this, write down the definition of the mgf of $y$.)
%Alternatively, let $\log y = x$ be normal.
%Then we say $y$ is lognormal.


Now the example.
Let $u(c) = c^{1-\alpha}/(1-\alpha)$ with $\alpha>0$ (power utility)
and
$\log y_1(z) \sim \mathcal{N}(\kappa_1,\kappa_2)$
(lognormal endowment).
For convenience, we set $y_0 = 1$ so that it drops out of
calculations.
With power utility, the mrs is
\begin{eqnarray*}
        m(z) \;\;=\;\; {\beta u'[y_1(z)]}/{u'(y_0)} &=&  \beta [y_1(z)/y_0]^{-\alpha} ,
\end{eqnarray*}
a function of the growth rate of the endowment.
Since $y_0 = 1$, $\log m(z)$ is
\begin{eqnarray*}
    \log m(z) &=& \log \beta - \alpha \log y_1(z)
        \;\;\sim\;\; \mathcal{N}(\log \beta - \alpha \kappa_1, \alpha^2 \kappa_2).
\end{eqnarray*}
The second step follows from Result 3 for lognormals.

Now consider two specific assets:  a ``bond'' that pays one in every state
and a share of ``equity'' that pays a dividend equal to the endowment in each state.
If the bond has price $q^1$ then its return is $r^1 = 1/q^1$.
The price satisfies
\begin{eqnarray*}
     q^1 &=& E(m \cdot 1) \;\;=\;\; \exp(\log \beta - \alpha \kappa_1 + \alpha^2 \kappa_2/2 )
            \;\;=\;\; \beta \exp(- \alpha \kappa_1 + \alpha^2 \kappa_2/2 ),
\end{eqnarray*}
which gives us the return
\begin{eqnarray*}
     r^1 &=& 1/q^1
            \;\;=\;\; \beta^{-1} \exp( \alpha \kappa_1 - \alpha^2 \kappa_2/2 ) .
\end{eqnarray*}
Higher mean growth $\kappa_1$ is associated with a lower price
and therefore a higher interest rate.
The greater is risk aversion the greater is this effect.
The same is true in the deterministic case,
which corresponds to $\kappa_2 = 0$.
An increase in the variance $\kappa_2$ raises the price and lowers the interest rate.

Now consider equity, a claim to the dividend $d^e(z) = y_1(z)$.
Its return is $r^e(z) = d^e(z)/q^e = y_1(z)/q^e$.
Its price is therefore $q^e = E( m d^e) = E( m y_1)$.
Taking logs and collecting terms, we have
\begin{eqnarray*}
    \log m(z) + \log d^e(z) &=& \log \beta  + (1-\alpha) \log y_1(z)
    \;\;\sim\;\; \mathcal{N}(\log \beta + (1-\alpha) \kappa_1, (1-\alpha)^2 \kappa_2).
\end{eqnarray*}
Result 1 for lognormals gives us the price:
\begin{eqnarray*}
    q^e &=& E \left( m d^e\right) \\
        &=& \exp[\log \beta + (1- \alpha) \kappa_1 + (1-\alpha)^2 \kappa_2/2 ] \\
        &=& \beta \exp[(1- \alpha) \kappa_1 + (1-\alpha)^2 \kappa_2/2 ] .
\end{eqnarray*}
The return is
\begin{eqnarray*}
    r^e(z) \;\;=\;\; y_1(z)/q^e &=&
        y_1(z) \beta^{-1} \exp[-(1- \alpha) \kappa_1 - (1-\alpha)^2 \kappa_2/2 ] .
\end{eqnarray*}

The expected return is (using Result 1 again)
\begin{eqnarray*}
    E(r^e)  &=&
    \exp(\kappa_1 + \kappa_2/2)  \beta^{-1}
            \exp[-(1- \alpha) \kappa_1 - (1-\alpha)^2 \kappa_2/2 ]  \\
    &=& \beta^{-1} \exp\{\alpha \kappa_1 + [1 - (1-\alpha)^2] \kappa_2/2 \} .
\end{eqnarray*}
(Hang in there, we're making progress.)

Now the question:  Does equity have a higher or lower expected return than the bond?
We refer to the difference as a risk premium.
We suggested that an asset that pays off mostly in good states,
as equity does, would demand a higher return as a result.
We would say, then, that equity has a positive risk premium.
Let's see if that's true here.
In general we have $E(r^e) > r^1$ if
\begin{eqnarray*}
    \exp\{ [1 - (1-\alpha)^2] \kappa_2/2 \}
            &>& \exp(-\alpha^2 \kappa_2/2) \\
     (-\alpha^2 + 2 \alpha) \kappa_2/2 &>& -\alpha^2 \kappa_2/2  .
\end{eqnarray*}
So it's true since $\alpha$ and $\kappa_2$ are positive.
When either is zero, the expected difference in returns
is also zero.


In loglinear models like this,
including some of the most common ones in finance,
it's simpler to compute the expected difference in returns in logs;
namely, $ E \log r^e - \log r^1 $.
Here we get
\begin{eqnarray*}
    E  \log r^e - \log r^1  &=& (2 \alpha - 1) \kappa_2 /2 ,
\end{eqnarray*}
which is positive if $\alpha > 1/2$.


\section{Two-period stochastic economies 2:  production version}

We can do the same thing with production, which here means
we allow investment at date 0 to increase output and consumption
at date 1.
The standard version
is the same as above, except:
\begin{itemize}
\item Technologies.  An input of $k$ units of the date-0 good
produces $ a(z) f(k)$ units of the date-1 good.
The output of this investment is, therefore, uncertain.
\item Resource constraints.  They become
\begin{eqnarray*}
    c_0 &\leq& y_0 - k \\
    c_1(z) &\leq& y_1(z) + a(z) f(k), \mbox{ for each } z .
\end{eqnarray*}
\end{itemize}
The idea is that investment is risky:  the marginal product
of an additional unit of capital is $a(z) f'(k)$,
which depends on the date-1 state $z$.


With these changes, the Pareto problem is
\begin{eqnarray*}
    \max_{c_0,k, c_1(z)} && u(c_0) + \beta \sum_z p(z) u[c_1(z)] \\
    \mbox{s.t.}   &&  c_0  + k \;\;\leq\;\; y_0 \\
                  &&  c_1(z) \;\;\leq\;\; y_1(z) + a(z) f(k) .
\end{eqnarray*}
The Lagrangean is
\begin{eqnarray*}
    \mathcal{L} &=&  u(c_0) + \beta \sum_z p(z) u[c_1(z)] + q_0 (y_0 - c_0 - k)
                    + \sum_z q_1(z)  [ y_1(z) + a(z) f(k) - c_1(z)]  .
\end{eqnarray*}
(As before, we've used $q$'s for the Lagrange multipliers,
because we know that they'll end up being the same as the equilibrium prices.)
The first-order conditions are
\begin{eqnarray*}
    c_0: &&  u'(c_0) - q_0 \;\;=\;\; 0 \\
    c_1(z): &&  \beta p(z) u'[c_1(z)] - q_1(z) \;\;=\;\; 0 \\
     k:   && - q_0 + \sum_z q_1(z) a(z) f'(k) \;\;=\;\; 0 .
\end{eqnarray*}
That gives us, in every state $z$,
\begin{eqnarray*}
    \frac{\beta p(z) u'[c_1(z)]}{u'(c_0)} &=& \frac{q_1(z)}{q_0}
    \;\;=\;\; Q(z)
    \;\;=\;\; \frac{1}{a(z) f'(k)} .
\end{eqnarray*}
The first part is the same as the exchange economy:
the marginal rate of substitution $\beta u'[c_1(z)]/u'(c_0)$
equals the state price $Q(z)$.
The second part concerns production:
the marginal rate of transformation $ 1 / [a(z) f'(k)]$ also equals the state price.

{\it Example.\/}
We can solve this by hand for a special case.
Let $u(c) = \log c$ and $f(k) = k^\theta$, with
$0 \leq \theta \leq 1$,
and $a(z) \sim \mathcal{N}(\kappa_1,\kappa_2)$.
Equity is a claim to $d(z) = a(z) f'(k) k$:
Each unit of capital is paid its marginal product $a(z) f'(k)$.
I'll leave it to you to derive its expected excess return.


What's useful about this model is that we can say where
aggregate dividends come from:
they're payments to capital.
At the margin,
an increase in $k$ at date 0 generates $z f'(k)$ units of
output at date 1 in state $z$.
Since $z$ is uncertain, so is the dividend.


\section*{Bottom line}

General equilibrium models are useful devices for thinking
about where prices and quantities come from.
From a macro-finance perspective,
even simple models give us some insights.
Among them are:
%
\begin{itemize}
\item Asset prices and returns reflect the marginal rates of substitution
of the people who buy them.
Our old favorite, $E(mr) = 1$, is a good example, with $m$
the mrs of a representative agent.
\item %That gives us a connection between returns and dividends.
Assets that pay off mostly in good states have less value
and higher returns, on average,
than either a riskfree asset that has the same payoff in all states
or an asset that pays off mostly in bad states.
Why?  Because marginal utility is lower in good states than bad.
\end{itemize}

\section*{Practice problems}

\begin{enumerate}
%-----------------------------------------------------------------------
\item {\it A three-good economy.\/}
Consider a static economy with three goods,
labelled 1, 2, and 3.
A representative agent has utility over consumption
of the three goods of
\begin{eqnarray*}
    U(c_1,c_2, c_3) &=& \log c_1 + \beta \log c_2 + \gamma \log c_3 .
\end{eqnarray*}
The same agent's endowments are $(y_1, y_2, y_3)$.
\begin{enumerate}
\item What are the ``ingredients'' for this economy?
\item What is the Lagrangian problem associated with a Pareto optimum?
\item What are its first-order conditions?
\item What are the optimal values of consumption?
What are the prices that support this as a competitive equilibrium?
\end{enumerate}
%
\needspace{2\baselineskip}
Answer.
\begin{enumerate}
\item The ingredients are
\begin{itemize}
\item List of commodities:  1, 2, and 3.
\item List of agents:  one.
\item Preferences and endowments:  see above.
\item Technology:  none.
\item Resource constraints:  $c_i \leq y_i$ for each good $i$.
\end{itemize}
\item The Lagrangian is
\begin{eqnarray*}
    \mathcal{L} &=& \log c_1 + \beta \log c_2 + \gamma \log c_3
            + \sum_i q_i (y_i - c_i) .
\end{eqnarray*}
The $q_i$'s are Lagrange multipliers on the resource constraints.
\item The first-order conditions are
\begin{eqnarray*}
    0 &=& 1/c_1 - q_1  \\
    0 &=& \beta/c_2 - q_2 \\
    0 &=& \gamma/c_3 - q_3 .
\end{eqnarray*}
\item Obviously we consume the whole endowment:  $c_i = y_i$ for $i=1,2,3$.
Why?  Because more is better, and that's the most we can consume.
The first-order conditions then give us the multipliers:
$ q_1 = 1/y_1$, $q_2 = \beta/y_2$, and $q_3 = \gamma/y_3 $.
\end{enumerate}

%-----------------------------------------------------------------------
\item {\it A two-good economy with power utility.\/}
Consider a static economy with two goods,
labelled 1 and 2.
A representative agent has utility
\begin{eqnarray*}
    U(c_1,c_2) &=& c_1^{1-\alpha}/(1-\alpha) + \beta c_2^{1-\alpha}/(1-\alpha).
\end{eqnarray*}
The endowments are $(y_1, y_2)$.
\begin{enumerate}
\item What are the ``ingredients'' for this economy?
\item What is the Lagrangian problem associated with a Pareto optimum?
\item What are its first-order conditions?
\item What are the optimal values of consumption?
What are the prices that support this as a competitive equilibrium?
\end{enumerate}
%
\needspace{2\baselineskip}
Answer.
\begin{enumerate}
\item The ingredients are
\begin{itemize}
\item List of commodities:  1 and 2.
\item List of agents:  one.
\item Preferences and endowments:  see above.
\item Technology:  none.
\item Resource constraints:  $c_i \leq y_i$ for each good $i$.
\end{itemize}
\item The Lagrangian is
\begin{eqnarray*}
    \mathcal{L} &=& c_1^{1-\alpha}/(1-\alpha) + \beta c_2^{1-\alpha}/(1-\alpha)
            + \sum_i q_i (y_i - c_i) .
\end{eqnarray*}
\item The first-order conditions are
\begin{eqnarray*}
    0 &=& c_1^{-\alpha} - q_1  \\
    0 &=& \beta c_2^{-\alpha} - q_2 .
\end{eqnarray*}
\item We consume the whole endowment: $c_i = y_i$ for $i=1,2$.
The prices are $ q_1 = y_1^{-\alpha}$ and $q_2 = \beta y_2^{-\alpha}$.
\end{enumerate}


%-----------------------------------------------------------------------
\item {\it Consumption and leisure.\/}
Most macroeconomic models include a role for work.
Our agent (let's stick with one) has one unit of time.
She supplies $n$ units to firms, who use it to produce output.
That leaves her with $1-n$ units to consume a leisure,
which includes here everything other than paid work:
sleeping, cleaning, watching television, biking, and so on.
Let us say, to be concrete, that the agent has utility function
\begin{eqnarray*}
    U(c,1-n) &=& \log c + \theta \log (1-n) ,
\end{eqnarray*}
and that each unit of labor hired by a firm generates $a$ units of output.
The overall resource constraint is therefore
\begin{eqnarray*}
        c &\leq& a n .
\end{eqnarray*}
Your mission is to find a Pareto optimum for this economy.
\begin{enumerate}
\item What are the ``ingredients'' for this economy?
\item What is the Lagrangian problem associated with a Pareto optimum?
\item What are its first-order conditions?
\item What are the optimal values of $c$ and $n$?
\end{enumerate}
%
\needspace{2\baselineskip}
Answer.
\begin{enumerate}
\item The ingredients are
\begin{itemize}
\item List of commodities:  the consumption good, labor.
\item List of agents:  one.
\item Preferences:  see utility function above.
Endowment:  one unit of labor.
\item Technology:  $y = an$, where $y$ is output of the consumption good.
\item Resource constraints:  for leisure we have $1-n$ of leisure plus $n$ of work
is no greater than one; for consumption, we have the constraint listed above.
\end{itemize}
\item The Lagrangian is
\begin{eqnarray*}
    \mathcal{L} &=& \log c + \theta \log (1-n) + \lambda (an - c)
\end{eqnarray*}
\item The first-order conditions are
\begin{eqnarray*}
    0 &=& 1/c - \lambda \\
    0 &=& - \theta/(1-n) + \lambda a .
\end{eqnarray*}
\item The first-order conditions plus the constraint give us
$ n = 1/(1+\theta) $ and $ c = an = a/(1+\theta) $.  \\
It wasn't asked, but this environment gives us prices, too.
This is a little terse, but we read them from the first-order conditions:
the price of consumption is $\lambda$, the price of labor is $\lambda a $.
\end{enumerate}

%-----------------------------------------------------------------------
\item {\it Pareto problem with production.\/}
Consider a theoretical economy with these ingredients:
\begin{itemize}
\item One agent.
\item Two goods, apples and bananas.
\item Utility function: $ \log c_a + \log c_b $.
\item Endowment:  6 apples, 4 bananas.
\item Technology:  apples can be converted to bananas one for one.
\end{itemize}

\begin{enumerate}
\item What are the resource constraints for this economy? 
\item What is the Pareto problem?  
\item What is the solution to the Pareto problem?
What are the optimal quantities?  What are the implied prices?
\end{enumerate}

\needspace{2\baselineskip}
Answer. 
\begin{enumerate}
\item If we let $a$ be the number of apples we convert to bananas,
the resource constraints for apples and bananas are
\begin{eqnarray*}
    c_a + a &\leq& y_a \;\;=\;\; 6 \\
    c_b &\leq& y_b + a \;\;=\;\; 4 + a.
\end{eqnarray*}
We might also say $ a\geq 0$ if we're worried about the reverse technology:
converting bananas to apples.
We'll ignore that from here on.

\item The Pareto problem is:  maximize utility subject to the resource constraints.
The Lagrangian is
\begin{eqnarray*}
    \mathcal{L} &=& \log c_a + \log c_b + \lambda_a (y_a - c_a - a) + \lambda_b (y_b + a - c_b) .
\end{eqnarray*}
The first-order conditions (the derivatives of $\mathcal{L}$) include
\begin{eqnarray*}
    c_a:&& 1/c_a - \lambda_ a \;\;=\;\; 0 \\
    c_b:&& 1/c_b - \lambda_b  \;\;=\;\; 0 \\
    a:&&    - \lambda_a + \lambda_b \;\;=\;\; 0.
\end{eqnarray*}
That gives us $\lambda_a = \lambda_b$ and (therefore) $c_a = c_b$.
From the resource constraints we have $c_a = a_b = 5$ and $a = 1$.
The relative price of bananas comes from the Lagrange multipliers:
$ q_b / q_a = \lambda_b / \lambda_a = 1$.

\end{enumerate}


%-----------------------------------------------------------------------
\item {\it Asset pricing with production and dividends.\/}
Consider a two-period economy with a linear technology.
What is equilibrium consumption growth?
What are the state prices?
What is a claim to one unit of capital worth?

To address these questions,
we use a variant of our two-period economy,
with dates 0 and 1 and states $z$ at date 1 that occur with probability $p(z)$.
The representative agent has utility function
\begin{eqnarray*}
    u(c_0) + \beta \sum_z p(z) u[c_1(z)] ,
\end{eqnarray*}
with $u(c) = c^{1-\alpha}/(1-\alpha)$ for $\alpha > 0$
(power utility).
She is endowed with $y_0$ units of the date-0 good, nothing at date 1.
The technology is linear:  $k$ units of the date-0 good invested in capital
generate $z k$ units of the good in state $z$ at date 1.
The resource constraints are therefore
\begin{eqnarray*}
    c_0 + k &=& y_0  \\
    c_1(z) &=& z k ,
\end{eqnarray*}
with one of the latter for each state $z$.
Think of the productivity factor $z$ as $a(z)$ with $a(z) = z$.

\begin{enumerate}
\item What are the classical ``ingredients'' of this economy?
\item What is the associated Pareto problem?
What are its first-order conditions?
\item Suppose $z$ is lognormal:  that is,
$\log z \sim \mathcal{N}(\kappa_1,\kappa_2)$.
Use the properties of lognormal random variables to show
that $E(z^a) = \exp( a \kappa_1 + a^2 \kappa_2/2)$
for any real number $a$.
\item Use this result to find the optimal values of $c_0$ and $k$.
Given these values, what is saving?
\item What is $c_1(z)$?
% ?? next part a little tough
What are the state prices?
\item What is the value of one unit of capital,
that is, a claim to $z$ units of output in each state $z$?
%What are the effects of the parameters $(\alpha, \kappa_1,\kappa_2)$?
\end{enumerate}
%
\needspace{2\baselineskip}
Answer.
\begin{enumerate}
\item Commodities:  the good at date 0 plus the good in each state at date 1. \\
Agents:  one \\
Preferences, endowment, and technologies:  given above. \\
Resource constraints:  ditto.
\item Pareto problem based on the Lagrangean:
\begin{eqnarray*}
    \mathcal{L}  &=&  \log c_0  + \beta \sum_z p(z) \log c_1(z)
          + q_0 (y_0 - c_0 - k)
            + \sum_z q_1(z) [z k - c_1(z)] .
\end{eqnarray*}
We choose $c_0$, $k$, and $c_1(z)$ (one for each $z$) to maximize this.
The first-order conditions are
\begin{eqnarray*}
    c_0: &&  c_0^{-\alpha} \;\;=\;\; q_0 \\
    k:   &&  q_0 \;\;=\;\; \sum_z q_1(z) z \\
    c_1(z): &&  \beta p(z) c_1(z)^{-\alpha} \;\;=\;\; q_1(z)
\end{eqnarray*}

\item Define $x=\log z$.
Its mgf is $e^{sx} = E(e^{s\log z}) = E(z^s) = e^{s\kappa_1 + s^2 \kappa_2/2}$.
Setting $s=a$ gives us the answer.

\item This is moderately demanding, but here's how it works.
With apologies for mixing sums and integrals,
we use the first and third first-order conditions to substitute
for $q_0$ and $q_1(z)$ in the second:
\begin{eqnarray*}
        1 &=& \sum_z \frac{q_1(z)}{q_0} z
            \;\;=\;\; \sum_z \frac{\beta p(z) c_1(z)^{-\alpha}}{c_0^{-\alpha}} \; z
            \;\;=\;\; \sum_z \beta p(z) \frac{(zk)^{-\alpha}}{(y_0-k)^{-\alpha}} \; z \\
          &=& \beta (y_0-k)^\alpha k^{-\alpha} \sum_z  p(z)  z^{1-\alpha}
             \;\;=\;\;  \beta (y_0-k)^\alpha k^{-\alpha} E (z^{1-\alpha} ) .
\end{eqnarray*}
To simplify, denote $ E(z^{1-\alpha}) = Z$.
Then consumption and capital are
\begin{eqnarray*}
    k &=& \frac{(\beta Z)^{1/\alpha}}{1+ (\beta Z)^{1/\alpha}} \; y_0,
        \;\;\;  c_0 \;\;=\;\; \frac{1}{1+ (\beta Z)^{1/\alpha}} \; y_0
\end{eqnarray*}
Using the lognormal result, we have
\begin{eqnarray*}
    Z &=& E(z^{1-\alpha}) \;\;=\;\; e^{(1-\alpha)\kappa_1 + (1-\alpha)^2 \kappa_2/2}
\end{eqnarray*}

\item Evidently
\begin{eqnarray*}
    c_1(z) &=& z k \;\;=\;\; z y_0 \frac{(\beta Z)^{1/\alpha}}{1+ (\beta Z)^{1/\alpha}} .
\end{eqnarray*}
The big part at the end is a constant, so it's ugly but innocuous.
The state prices are (more tedious substitution)
\begin{eqnarray*}
    q(z) &=& p(z) \beta [c_1(z)/c_0]^{-\alpha}
            \;\;=\;\; p(z) z^{-\alpha}/Z .
\end{eqnarray*}
As usual, the state price is the product of
a pricing kernel [here $m(z) = z^{-\alpha}/Z$]
and a probability [the normal density for $\log z$,
which we haven't bothered to write out].

\item
This is claim to $z$ next period,
with value at date 0 of
\begin{eqnarray*}
    q^e &=& E (z^{1-\alpha})/Z \;\;=\;\; 1.
\end{eqnarray*}
Hmmmm...
Why does this make sense?
Because one unit of capital is one unit of the good
at date 0, whose price is one since we're valuing
assets in units of the date-0 good.
\end{enumerate}

\begin{comment}
\item {\it Pricing contingent claims.\/}
Consider a two-period exchange economy with power utility
and endowment growth
and $\log c_1 - \log c_0 = \log y_1 - \log y_0 = z \sim \mathcal{N}(\kappa_1,\kappa_2)$
and $y_0 = 1$.
We'll compute the prices of equity and two equity derivatives
using a representative agent's marginal rate of substitution
as a pricing kernel.

\begin{enumerate}
\item What is the marginal rate of substitution for this economy?
\item What is the price $q^1$ of a bond
that pays one in each state?
Express it as a function of $\alpha$, $\beta$, $\kappa_1$, and $\kappa_2$.
\item What is the price $q^e$ of ``equity,''
a claim to the date-1 endowment $y_1(z) = e^z$?
What is its return? Expected return?
\item Consider an ``upside'' derivative that pays
$y_1 $ if $y_1 \geq b$
and a ``downside'' derivative that pays
$y_1 $ if $y_1 \leq b$.
Show that the prices (call them $q^u$ and $q^d$)
sum to the price of equity.

\item What is the price of the downside derivative
for the lognormal case?
Hint:  this involves integrating the normal distribution
as we did when we computed the normal mgf.
The difference is that we only integrate over the region in
which the dividend is positive.
\end{enumerate}
%
\needspace{2\baselineskip}
Answer.
\begin{enumerate}
\item The mrs is
\begin{eqnarray*}
    m(z) &=& \beta [c_1(z)/c_0]^{-\alpha}
        \;\;=\;\; \beta e^{-\alpha z} ,
\end{eqnarray*}
so $\log m \sim \mathcal{N}(\log \beta - \alpha \kappa_1, \alpha^2 \kappa_2) $.
\item The lognormal result again:
\begin{eqnarray*}
    q^1 &=& E(m) \;\;=\;\; \beta e^{-\alpha \kappa_1 + \alpha^2 \kappa_2/2} .
\end{eqnarray*}
The return is $r^1 = 1/q^1 $.

\item With our distributional assumptions, we have
\begin{eqnarray*}
    q^e  &=& E(m y_1) \;\;=\;\; \beta E ( e^{-\alpha z} e^z )
        \;\;=\;\; \beta E( e^{(1-\alpha) z}) \\
             &=&
             \beta e^{(1-\alpha)\kappa_1 + (1-\alpha)^2 \kappa_2/2 }.
\end{eqnarray*}
This should look familiar.

\item We're splitting the dividends in two --- upside and downside ---
so a claim to one unit of each is equivalent to a claim to equity.

\item This is moderately demanding, too,
but it's exactly the kind of math we'll use later to price options.
The price of the downside derivative is
\begin{eqnarray*}
    q^d &=&  \int_{-\infty}^{z^*} m(z) y_1(z) p(z) dz
\end{eqnarray*}
with $z^* = \log b$ the upper bound on $z$ corresponding to
$y_1 = b$.
Now we're back to the same math we used to compute the mgf for a normal
random variable:
\begin{eqnarray*}
     m(z) y_1(z) p(z)  &=& \beta e^{-\alpha z} e^z (2 \pi \kappa_2)^{-1/2}
        e^{-(z-\kappa_1)^2/2\kappa_2} \\
            &=& \beta e^{ (1-\alpha) \kappa_1 + (1-\alpha)^2 \kappa_2/2}
                (2 \pi \kappa_2)^{-1/2}
        e^{-(z + (1-\alpha) \kappa_2-\kappa_1)^2/2\kappa_2}  \\
        &=& q^e  (2 \pi \kappa_2)^{-1/2}
        \exp[-(w-\kappa_1)^2/2\kappa_2] ,
\end{eqnarray*}
where $ w = z + (1-\alpha)\kappa_2$.
The integral is therefore
\begin{eqnarray*}
    (2 \pi \kappa_2)^{-1/2} \int_{-\infty}^{\log b + (1-\alpha) \kappa_2}
            \exp[-(w-\kappa_1)^2/2\kappa_2] dw .
\end{eqnarray*}
More on this another time.
\end{enumerate}
\end{comment}

\end{enumerate}

{\vfill
{\bigskip \centerline{\it \copyright \ \number\year \
David Backus $|$ NYU Stern School of Business}%
}}


\end{document}
