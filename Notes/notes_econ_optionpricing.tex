\documentclass[11pt]{article}

% spacing on page
\oddsidemargin=0.25truein \evensidemargin=0.25truein
\topmargin=-0.5truein \textwidth=6.0truein \textheight=8.75truein

\usepackage{comment}
\usepackage{graphicx}
\usepackage{amssymb}
\usepackage{amsmath}

\usepackage[margin=0pt, labelsep=period, font=large, labelfont=bf]{caption}
%\usepackage{float}

\usepackage{hyperref}
\urlstyle{rm}   % change fonts for url's (from Chad Jones)
\hypersetup{
    colorlinks=true,        % kills boxes
%    allcolors=blue,
    pdfsubject={ECON-UB233, Macroeconomic foundations for asset pricing},
    pdfauthor={Dave Backus @ NYU},
    pdfstartview={FitH},
    pdfpagemode={UseNone},
%    pdfnewwindow=true,      % links in new window
%    linkcolor=blue,         % color of internal links
%    citecolor=blue,         % color of links to bibliography
%    filecolor=blue,         % color of file links
%    urlcolor=blue           % color of external links
% see:  http://www.tug.org/applications/hyperref/manual.html
}

% for listing code in tt font
\usepackage{verbatim}

% for table spacing
\usepackage{booktabs}

% section headers and spacing
\usepackage[small, compact]{titlesec}

% list spacing
\usepackage{enumitem}
\setitemize{leftmargin=*, topsep=0pt}
\setenumerate{leftmargin=*, topsep=0pt}

% attach files to the pdf
\usepackage{attachfile}
    \attachfilesetup{color=0.75 0 0.75}

\usepackage{needspace}
% \needspace{4\baselineskip} makes sure we have four lines available before a pagebreak

%\renewcommand{\thefootnote}{\fnsymbol{footnote}}
%\newcommand{\var}{\mbox{\it Var\/}}

\newcommand{\cbar}{\bar{c}}
\newcommand{\rp}{\mbox{\em rp\/}}
\newcommand{\tp}{\mbox{\em tp\/}}
\newcommand{\bp}{\mbox{\em bp\/}}


\begin{document}
\parskip=\bigskipamount
\parindent=0.0in
\thispagestyle{empty}
{\large ECON-UB 233 \hfill Dave Backus @ NYU}% 

\bigskip\bigskip
\centerline{\Large \bf Option Pricing: Black-Scholes-Merton \& Beyond}
\centerline{Revised: \today}

\begin{comment}
* Figure out graphs v underlying future value and v strike k
\end{comment}

\bigskip
Options are one of the most common financial derivatives, and
a nice application of asset pricing fundamentals.
We derive option prices from risk-neutral probabilities
and show how the prices of options depend on our choice of probability distribution.
There are lots of moving parts, but that's the idea.


We have two specific goals.
The first is to derive the Black-Scholes-Merton (``BSM'') formula from a model
in which the price of the {\it underlying\/} ---
the asset the option is based on ---
has a lognormal risk-neutral distribution.
This is expressed most simply in terms of a flat {\it volatility smile\/},
something we'll define shortly.
The second is to show how other distributions can generate other
shapes for the smile, including shapes similar to those
we observe for real-world options.
The focus is on options on S\&P 500 index futures,
a popular and convenient contract.
%The index has a macro flavor:  its ups and downs
%are closely connected to the ups and downs in the economy.
The same methods can be --- and have been ---
applied to other option contracts,
although the parameter values will be different.

%{\it Notation.\/}
A couple warnings about notation:
\begin{itemize}
\item Timing.
We'll continue to use the $(t, t+1)$ timing,
with today being $t$ and next period (next year) $t+1$.
In some cases we'll generalize this to $(t,t+\tau)$,
where $\tau$ is the length of the time interval,
so that we can value options of different maturities.

\item Letters.
We're going to repurpose some letters, including $s$ and $k$,
to stick closer to standard usage with options.
Here $s$ and $k$ will have nothing to do with the cumulant generating function here,
except in a couple instances I was unable to avoid.
We use the letter $d$ in two ways,
as the dividend or cash flow and as a component of the BSM formula.
\end{itemize}

%\begin{comment}
\section{Review:  risk-neutral asset pricing}
\label{sec:review}

Recall that one version of the no-arbitrage theorem connects an asset's price $q_t$
to its cash flows $d_{t+1}$ by
\begin{eqnarray}
    q_t &=& q^1_t E^* \left( d_{t+1} \right) ,
    \label{eq:pricing-relation}
\end{eqnarray}
where $q^1_t$ is the price of a one-period riskfree bond
and $E^*$ means the expectation using the risk-neutral distribution
of $d_{t+1}$.

We have then the following steps to price an asset:
\begin{itemize}
\item Identify the cash flows $d_{t+1}$.
\item Specify the risk-neutral distribution.
\item Use equation (\ref{eq:pricing-relation}) to price the asset.
\end{itemize}
This is the basic idea for any asset,
whether an option or something else.
%\end{comment}


\section{Option cash flows}

An asset is defined by its cash flow or dividend, as we've called it.
Here we buy an asset at $t$ and realize its (generally uncertain) cash flow at $t+\tau$.

An option is a particular kind of cash flow.
An option gives its owner the right to buy or sell another asset at a preset price
within a given time period.
For example, movie producers might have a 5-year option to produce a
movie based on a book.
Or a real estate developer might have a 6-month option to purchase
a piece of property at a preset price.
After that time, the option expires and
the seller of the option can do what she wants with the asset.
In financial markets we see options on lots of things:  stocks, bonds,
foreign currencies, pork belly futures, and so on.
They're the classic ``financial derivative'' and show up all over the place.
Options are useful to buyers because they give them flexibility.
In return for this flexibility, the seller collects a fee --- the price
of the option.

A typical option is defined by these features:
\begin{itemize}
\item The underlying:  the asset on which the option is based.
In our case this is the S\&P 500 futures contract.
\item The strike price: the price at which you can buy or sell
the underlying.
\item The term or maturity: the period of time
over which the option can be exercised.
%We'll use the Greek letter $\tau$.
\item Call or put:  whether it's an option to buy (a call) or sell (a put).
\item Type of option:  ``European,'' ``American,'' or something else (``Asian''?  ``Greek''?).
Most traded options can be exercised any time up to maturity (American),
but it's mathematically simpler to work with options that can be
exercised only at maturity (European).
That's what we'll do.
\end{itemize}
That's the basics.
%We'll focus on European options, because they're the simplest:
%the only cash flows are on the expiration date.

With a small investment in notation,
we can express an option's cash flows mathematically.
Let us say we have a $\tau$-period option.
If the price of some underlying asset at time $t$ is $s_t$
then a $\tau$-period call option at strike price $k$
generates a cash flow at time $t+\tau$ of
$ d_{t + \tau} = ( s_{t+\tau} - k )^+ $,
where $x^+ \equiv \max(x,0) $.
Why?  Because you would only exercise the option if the price is above $k$.
Otherwise you let it expire.
Similarly, a $\tau$-period put option at strike price $k$
generates the cash flow $ (k - s_{t+\tau} )^+ $.

If you graph the cash flow against the future price, you'll see
that it's convex for both puts and calls.
As a result, the value increases with the amount
of risk in the underlying.
The more risk there is, the more valuable they are.
That's a loose statement, but one we could make tighter using  logic
from the chapter on risk.
More simply:  put and call options give you the upside
without the downside.


\section{Put-call parity}


There's a useful connection between European put and call prices
that tells us, in effect, that if you know the price of one you can
easily compute the price of the other.
With a call option, you buy the asset if its price is above $k$,
and with a put you sell it if the price is below $k$.
So buying a call and selling a put at the same strike leads you to own the stock in all cases.
The price is $k$, paid at $t+\tau$, so we discount it using
the price $q^\tau_t$ (the price at $t$ of one dollar paid at $t+\tau$).

This gives us two ways to buy the stock at date $t$,
directly and with options.
Absence of arbitrage tells us both methods should cost the same:
\begin{eqnarray}
    \underbrace{q^c_t}_{\mbox{buy call}} -
    \underbrace{q^p_t}_{\mbox{sell put}} +
    \underbrace{q^\tau_t k}_{\mbox{present value of strike}}
    &=&
    \underbrace{s_t}_{\mbox{buy stock}} .
    \label{eq:put-call-parity}
\end{eqnarray}
This is known as {\it put-call parity\/}.
A fine point:  this is for an asset that doesn't pay a dividend
between $t$ and $t+\tau$.
If it does, we need to work that cash flow into the equation.


\section{The normal distribution function}


Recall that if $x$ is normal with mean zero and variance one (``standard normal''),
its probability density function (pdf) is
\begin{eqnarray}
    p(x) &=& (2 \pi)^{-1/2} \exp(-x^2/2)
    \label{eq:pdf-stdnormal}
\end{eqnarray}
for any real number $x$.
It's positive and integrates to one.
It's also symmetric:  $p(x) = p(-x)$.

We'll see shortly that the BSM formula
involves the cumulative distribution function (cdf) $N$,
defined by
\begin{eqnarray*}
    \mbox{Prob} (x \leq x^*)  &=& \int_{-\infty}^{x^*} p(x) dx
        \;\;=\;\; N(x^*).
\end{eqnarray*}
%This is common notation (although some people $\Phi$ instead of $N$).
There's no simple antiderivative of $p$, but it comes up enough that
we give it the label $N$.
It's also a common function in software packages.
The function $N$ corresponds to {\tt normcdf} in Matlab
and {\tt NORMSDIST} in spreadsheet programs.

We can do the same with other normal random variables.
The random variable $ y = \mu + \sigma x$
is normal with mean $\mu$ and variance $\sigma^2$.
Since $ x = (y-\mu)/\sigma$ is standard normal,
we can compute cumulative probabilities for $y$ using
\begin{eqnarray*}
    \mbox{Prob} (x \leq x^*)  &=& \mbox{Prob} [x \leq (y^*-\mu)/\sigma ]
        \;\;=\;\; N\left[ (y^*-\mu)/\sigma \right].
\end{eqnarray*}
[We're using $\sigma > 0$ here, which we take for granted below.]
Matlab does this calculation for us with the command {\tt normcdf(ystar,mu,sigma)}.

One final property:  since $p$ is symmetric, $N(-x) = 1 - N(x)$.
Why?
Note that
\begin{eqnarray*}
\int_{-\infty}^{x^*} p(x) dx + \int_{x^*}^{\infty} p(x) dx &=& 1 .
\end{eqnarray*}
This implies
\begin{eqnarray*}
     1 - N(x^*) &=& \int_{x^*}^{\infty} p(x) dx \;\;=\;\; \int_{-\infty}^{-x^*} p(-x) dx
            \;\;=\;\; N(-x^*)
\end{eqnarray*}
by symmetry.
It's easy to show in a picture:
graph the pdf and its tail integrals.


\section{The BSM formula and the volatility smile}


We refer to this as a {\it formula\/} because that's what it is:
a function that relates option prices to the strike price
and a few other things.  We'll derive it from a model shortly,
but for now it's just a formula.
We report it for an arbitrary option maturity $\tau$,
but in most of what we do we'll set $\tau = 1$.
In the general case, $q^\tau_t$ is the price of a riskfree bond
of maturity $\tau$.

The BSM formula for the price $q^c_t$ of a call option is
\begin{eqnarray}
        q^c_t &=& s_t N(d) - q^\tau_t k N (d - \tau^{1/2}\sigma ),
        \label{eq:bsm-call}
\end{eqnarray}
where
\begin{eqnarray*}
          d &=& \frac{\log(s_t/q^\tau_t k) + \tau \sigma^2/2}{\tau^{1/2}\sigma} .
\end{eqnarray*}
With put-call parity, we can show that put prices are
\begin{eqnarray}
        q^p_t &=& q^\tau_t k N(-d + \tau^{1/2} \sigma) - s_t N(-d) .
        \label{eq:bsm-put}
\end{eqnarray}
[You might try this. It uses the symmetry property of $N$.]

Let's review the ingredients:
\begin{itemize}
\item $s_t$ is the current price of the underlying.
\item $\tau$ is the maturity of the option.
\item $q_t^\tau$ is the price at $t$ of ``one'' at date $t+\tau$  (the bond price, in other words).
This is commonly written in terms of the implied (continuously compounded) interest rate $r$:
$q_t^\tau = \exp(-r \tau)$.
We won't do that, because $r$ means something else to us, but you might imagine
a variant like $q_t^\tau = \exp(-y_t^\tau \tau)$
where $y_t^\tau$ is the continuously compounded yield on a bond with maturity $\tau$.
Not something we need, but it has one useful feature:
the suggestion that we need not have a constant interest rate.
\item $k$ is the strike price.
\item $\sigma$ is a parameter we refer to as ``volatility.''
\end{itemize}
All but the last of these ingredients are observed.
But if we know the others,
and observe the option price, we back out $\sigma$.
The output of this calculation is commonly referred to as {\it implied volatility\/}
or simply volatility.


A plot of implied volatility against the strike price
is commonly referred to as a {\it volatility smile\/}.
The term smile refers to the shape, but for equity index options
it's less a smile than a downward-sloping line, possible with some
convexity to it.
We'll treat the shape of the volatility smile as a feature of the data we'd like
to explain.

A common variant is to graph implied volatility against
some measure of ``moneyness.''
Options are said to be ``at the money'' if $ k = s_t$.
We say a put is ``in the money'' if $k > s_t $ and ``out of the money''
if $ k< s_t $.
And the reverse for calls.
So-called ``moneyness'' is often measured by $ k-s_t$ or $(k-s_t)/s_t$,
so that zero represents an at-the-money option.

\section{Derivatives of the BSM formula}

Most treatments go on to list derivatives of the BSM formula
with respect to its inputs.
You can look these up in
\href{http://en.wikipedia.org/wiki/Black-Scholes_model#The_Greeks}{Wikipedia}.
We'll use one:  the derivative with respect to volatility $\sigma$.
The so-called ``vega'' is
\begin{eqnarray*}
    \partial q^p_t /\partial \sigma &=&
            s_t N'(d) \tau^{1/2} ,
\end{eqnarray*}
where $N'$ is the standard normal pdf, equation (\ref{eq:pdf-stdnormal}).
We'll use this as an input to Newton's method when we compute
implied volatilities.


\section{The S\&P 500 E-mini futures contract}

We're interested in options on a broad-based equity index
like the S\&P 500,
which we've seen is closed related to the state of the economy.
(I have in mind here the scatterplot of equity returns against
consumption growth.)
As it happens, it's common to use options on the futures contract on the index,
rather than the index itself.
This is more information than we need,
but if you're interested, this is the contract,
as described in
\href{http://en.wikipedia.org/wiki/E-mini_S&P}{Wikipedia}
(lightly edited):
%
\begin{quote}
The E-mini S\&P futures contract,
often called simply the ``E-mini,''
is a stock market index futures contract traded
on the Chicago Mercantile Exchange's Globex electronic trading platform.
The notional value of one contract is US\$50 times the value of the S\&P 500 stock index.
It was introduced by the CME on September 9th, 1997,
after the value of the existing S\&P contract became too large for many small traders.
The E-mini quickly became the most popular equity index futures contract in the world.
\end{quote}

Options on these contracts are among the most liquid we have.
%Most work on equity index options therefore starts with
%options on index futures.



\section{Risk-neutral option pricing and the no-arbitrage condition}

\begin{comment}
If you recall our work on the theory of asset pricing,
the fundamental result is the no-arbitrage theorem.
With a finite set of states,
the theorem says that if prices and dividends do not contain any arbitrage opportunities,
then there are positive state prices $Q(z)$ that price all
assets:
\begin{eqnarray}
    q^j_t  &=& \sum_z Q(z) d^j_{t+1}(z) .
    \label{eq:theorem}
\end{eqnarray}
The only question left is what the state prices are.

We also looked at two alternative ways to express this result:
\begin{eqnarray*}
    Q(z) &=& \left\{
                \begin{array}{l}
                    p(z) m(z) \\
                    q^1_t p^*(z) .
                \end{array}
            \right.
\end{eqnarray*}
The first defines a pricing kernel $m(z) = Q(z)/p(z)$,
where $p(z)$ is the probability of state $z$.
We've seen that risk premiums follow from the covariance of returns with $m$.
We'll use that representation when we get to bonds.
The second defines the risk-neutral probability
$p^*(z) = p(z)m(z)/q_t^1 = Q(z)/q^1_t$.
The denominator $q_t^1 = \sum_z Q(z) = \sum_z p(z) m(z) = E(m)$ guarantees
that risk-neutral probabilities sum (or integrate) to one.

We could attack option prices either way.
We use risk-neutral probabilities because
that calls for only one object ($p^*$)
rather than two ($p$ and $m$), but it should be clear that
they're equivalent.
So from now on, we'll think about $p^*$.
The pricing relation (\ref{eq:theorem}) then becomes
equation (\ref{eq:pricing-relation}), which is where we started.
%\begin{eqnarray}
%    q^j_t  &=&  q^1_t \sum_z p^*(z) d^j_{t+1}(z)
%            \;\;=\;\; q^1_t  E^* d^j_{t+1} ,
%              \label{eq:arb-condition}
%\end{eqnarray}
%where $E^*$ means the expectation using the risk-neutral distribution.
%We can write this more transparently as
%\begin{eqnarray*}
%    r^1_t  &=&   E^* r^j_{t+1} ,
%%              \label{eq:arb-condition}
%\end{eqnarray*}
%where $r^1_t = 1/q^1_t $ and $r^j_{t+1} = d^j_{t+1} / q^j_t$.
%In words:  when we use risk-neutral probabilities,
%every asset has the same expected return.
\end{comment}

We reminded ourselves in Section \ref{sec:review} that
if we know the risk-neutral probabilities, then we can price options.
Commonly we guess a functional form for the probabilities
and infer parameters from observed prices.
When we do that, we need to make sure the probabilities satisfy
equation (\ref{eq:pricing-relation}) for the underlying.
We refer to this as the {\it no-arbitrage condition\/},
because (\ref{eq:pricing-relation}) reflects the no-arbitrage theorem.


The cash flow of the underlying is its price next period,
so the no-arbitrage condition becomes
\begin{eqnarray}
    s_t  &=&  q^1_t  E^* (s_{t+1}) .
    \label{eq:arb-condition}
\end{eqnarray}
[We could add a dividend, but this makes things more complicated
and we end up in a similar place anyway.]
Without this condition,
the chosen probabilities violate the no-arbitrage theorem for this asset.

This is clearer if we look at an example.
Suppose $x_{t+1} = \log s_{t+1} \sim \mathcal{N} (\kappa_1,\kappa_2)$.
%(Here I mean that the risk-neutral distribution of $\log d_{1+1}$ is
%normal with mean $\kappa_1$ and variance $\kappa_2$.)
Then (\ref{eq:arb-condition}) implies
\begin{eqnarray}
    s_t &=& q^1_t e^{\kappa_1 + \kappa_2/2}.
    \label{eq:bsm-no-arb}
\end{eqnarray}
%This says, essentially, that in risk-neutral terms
%the underlying and the riskfree bond have the same expected return.
Typically we would choose $\kappa_2$ to match option prices
(more on this shortly)
and set $\kappa_1 = \log s_t -\log q_t^1 - \kappa_2/2$.
We'll see this again, so don't panic if it hasn't sunk in yet.
More generally, the no-arbitrage condition can be written
\begin{eqnarray*}
    s_t &=& q^1_t h(1) , %\;\;=\;\; q^1_t e^{k(1)},
\end{eqnarray*}
where $h$ is the moment generating function of $\log s_{t+1}$.
% and $k(1)$ is the cumulant generating function, both evaluated at $s=1$.
%(And yes, we're using $s$ and $k$ as we did earlier,
%not as we have so far with options.)

Once we have the risk-neutral distribution,
we price options as we would any asset:
we characterize the cash flows and apply the pricing relation
(\ref{eq:pricing-relation}).
Prices of one-period put and call options are therefore
\begin{eqnarray*}
        q^p_t &=&  q_t^1 E^* (k-s_{t+1})^+  \\
        q^c_t &=&  q_t^1 E^* (s_{t+1}-k)^+ .
\end{eqnarray*}
The expectation $E^*$ is either a sum (if the distribution is discrete)
or (more commonly) an integral (if the distribution is continuous).
The problem now becomes a computational one.

There are two common shortcuts.
One is to approximate a continuous distribution with a discrete grid.
That's a brute-force approach, but works reasonably well.
The other is to use distributions that we can solve by hand ---
distributions that lead to convenient formulas for option prices.
Either would work.
We'll do mostly the second because we can handle the challenge,
and because it highlights the role of the distribution in
the BSM model and alternatives to it.


\section{The Black-Scholes-Merton model}


The BSM formula follows directly from using a lognormal risk-neutral distribution
for the underlying.  Everything else is just calculus.
We'll revert to a maturity of one, but consider alternatives in the next section.

We'll derive the put formula (\ref{eq:bsm-put})
using as an input a lognormal distribution
of the underlying:
$x_{t+1} = \log s_{t+1} \sim \mathcal{N} (\kappa_1,\kappa_2)$.
This is, to be clear, the risk-neutral distribution.
The no-arbitrage condition is therefore equation (\ref{eq:bsm-no-arb}).
%\begin{eqnarray}
%    s_t &=& q_t^1 \exp(\kappa_1 + \kappa_2/2 ) .
%    \label{eq:bsm-no-arb}
%\end{eqnarray}
We'll need it later.

The rest follows from evaluating moderately tedious integrals.
There's no magic here, just first-year calculus.
The price of a put option with strike $k$ is
\begin{eqnarray*}
    q^p_t &=&  q^1_t E^* (k - s_{t+1})^+  \\
            &=& q^1_t \int_{-\infty}^{\log k} (2 \pi \kappa_2)^{-1/2}
                    \exp[ - (x_{t+1} - \kappa_1)^2/2\kappa_2]
                    \left( k - e^{x_{t+1}} \right) d x_{t+1} \\
            &=& q^1_t k \int_{-\infty}^{\log k} (2 \pi \kappa_2)^{-1/2}
                    \exp[ - (x_{t+1} - \kappa_1)^2/2\kappa_2] d x_{t+1}
                    \;\;\; \mbox{(term 1)}\\
            &&    - \; q^1_t \int_{-\infty}^{\log k} (2 \pi \kappa_2)^{-1/2}
                    e^{x_{t+1}} \exp[ - (x_{t+1} - \kappa_1)^2/2\kappa_2] d x_{t+1} .
                     \;\;\; (\mbox{term 2})
\end{eqnarray*}
%The rest is high school calculus:  tedious, but nothing particularly deep.
Term 1 is just the normal cdf evaluated at $x_{t+1} = \log s_{t+1} = \log k$:
\begin{eqnarray*}
    \mbox{term 1} &=&  q^1_t k N [ (\log k - \kappa_1)/\kappa_2^{1/2}] .
\end{eqnarray*}
This is the discounted value of the strike price times
the risk-neutral probability the option is exercised.

Term 2 requires us to combine two exponential terms, something
we've done a couple times before, starting with the normal
moment generating function.
(You might want to go back and remind yourself how that worked.)
The key step is showing
\begin{eqnarray*}
    e^{x_{t+1}} \exp[ - (x_{t+1} - \kappa_1)^2/2\kappa_2] &=&
        e^{\kappa_1 + \kappa_2/2} \exp\{ - [x_{t+1} - (\kappa_1+\kappa_2)]^2/2\kappa_2 \} .
\end{eqnarray*}
You can verify this by expanding the exponents on both sides.
That gives us
\begin{eqnarray*}
    \mbox{term 2} &=& q^1_t e^{\kappa_1 + \kappa_2/2}
            N \{ [\log k - (\kappa_1 + \kappa_2)]/\kappa_2^{1/2} \} .
\end{eqnarray*}
The put price is therefore
\begin{eqnarray}
    q^p_t &=& q^1_t k N (f) - q^1_t e^{\kappa_1 + \kappa_2/2}
            N ( f - \kappa_2^{1/2} ),
    \label{eq:bsm-put-kappas}
\end{eqnarray}
with
\begin{eqnarray*}
    f &=&  (\log k - \kappa_1)/\kappa_2^{1/2} .
\end{eqnarray*}
This is a useful formula, and we'll come back to it later.
But it's not the BSM formula --- yet.

All that's left to derive the BSM formula (\ref{eq:bsm-put})
is to apply the no-arbitrage condition (\ref{eq:bsm-no-arb})
to (\ref{eq:bsm-put-kappas}).
That gives us
\begin{eqnarray}
    q^p_t &=& q^1_t k N (f) - s_t N ( f - \kappa_2^{1/2} ),
    \label{eq:bsm-put-kappas-bsm}
\end{eqnarray}
with
\begin{eqnarray*}
    f &=&  [\log (q^1_t k/s_t) + \kappa_2/2]/\kappa_2^{1/2} .
\end{eqnarray*}
If we substitute $\kappa_2 = \sigma^2$ and $\tau = 1$ (more on that next),
we get exactly the BSM put formula, equation (\ref{eq:bsm-put}).

From a practical perspective, we might ask how well the BSM model
approximates observed option prices.
The answer is that it misses the volatility smile.
In the model, volatility $\sigma$ is a parameter,
the standard deviation of $\log s_{t+1}$,
and  it's the same for all strike prices.
In the data, we typically find that the value of volatility that reproduces
option prices --- what we've called implied volatility ---
varies with the strike.
With equity index options, implied volatility declines with the strike.
This discrepancy between model and data lead us to search
for a better model, one that can reproduce the volatility smile.
That's the ultimate goal of this chapter,
but we'll work up to it gradually.


\section{Summing and dividing random variables}

The goal here is to adapt the BSM and related formulas
to time intervals different from a year.
That has some practical importance, because
most traded options have maturities significantly less than a year.

We'll attack this indirectly, starting with sums
of independent random variables.
As we've seen, the cumulant generating functions of
these sums are sums of cgfs of the components.
If $y = x_1 + x_2$, for example, then
\begin{eqnarray*}
    k(s; y) &=& \log h(s; y)
            \;\;=\;\; \log h(s; x_1) + \log h(s; x_2)
            \;\;=\;\; k(s; x_1) + k(s; x_2) .
\end{eqnarray*}
If $x_1$ and $x_2$ have the same distribution,
this becomes $ k(s; y) = 2 k(s; x)$.
Similarly the cgf for $ y = x_1 + x_2 + \cdots + x_n$,
with $n$ independent identically distributed components $x_j$, is
$  k(s; y) =  n k(s; x) $.

We can always do this, but it doesn't always have a nice form.
Two examples that do are the normal and Poisson.
If $x_j \sim \mathcal{N}(\kappa_1, \kappa_2)$,
then $ k(s; x_j) = \kappa_1 s + \kappa_2 s^2/2 $.
The cgf of the sum of $n$ such random variables is
the same thing multiplied by $n$:
\begin{eqnarray*}
    k(s; y) &=& n ( \kappa_1 s + \kappa_2 s^2/2 )
            \;\;=\;\; (n \kappa_1) s + (n\kappa_2) s^2/2 .
\end{eqnarray*}
That is, it's still normal, but the mean $\kappa_1$
and variance $\kappa_2$ are multiplied by $n$.
Poisson random variables also scale nicely.
Recall that if $x$ is Poisson, it assumes the values $x= 0, 1, 2, \ldots $
with probabilities
$  p(x) = e^{-\omega} \omega^x / x! $
for some parameter $\omega > 0$ (``intensity'').
Its cgf is (this is from the notes on random variables)
\begin{eqnarray*}
    k(s; x) &=& \omega \left( e^{s} -1 \right) .
\end{eqnarray*}
Now consider the sum $y$ of $n$ such random variables.
Its cgf is
\begin{eqnarray*}
    k(s; y) &=& n \omega \left( e^{s} -1 \right) .
\end{eqnarray*}
That is, we just multiply the intensity parameter $\omega$ by $n$.

The obvious application of sums for us is time.
If we have a random variable $y$ that applies to (say) a month,
and let the distribution be the same at all dates,
then we can think of  the random variable $y$ for a year
as the sum of 12 months $x_j$.
If we're lucky, we can also go the other way.
If we have the distribution for a year, we can divide it
into months like this:
\begin{eqnarray*}
    k(s; y) &=&  \tau k(s; x) ,
\end{eqnarray*}
where $y$ is the random variable for a month,
$x$ is the random variable for a year,
and $\tau = 1/12$ is the fraction of a year in one month.
Random variables for which this is possible are said to be {\it divisible\/}.

It's not hard to see how this would work with normal
and Poisson random variables.
In the normal case, the mean $\kappa_1$ and variance $\kappa_2$
are proportional to $\tau$.
That's the one missing piece in our derivation of the BSM formula.
We can now replace $\kappa_2$ with $\tau \sigma^2$
in equation (\ref{eq:bsm-put-kappas-bsm})
and get the BSM formula for any maturity.
Generally we scale the interest rate, too, replacing $ \log q^1_t$
with $\log q^\tau_t = \tau y^\tau_t $:  we discount shorter
intervals less than longer ones.

This doesn't work with all random variables.
For example, the Bernoulli has cgf
\begin{eqnarray*}
    k(s; y) &=& \log \left( 1-\omega + \omega e^s \right) .
\end{eqnarray*}
It's not divisible.
That's one of the reasons the Poisson distribution is preferred in applications
to finance:  it scales more nicely to different time intervals.


\section{The Merton model}
\label{sec:merton}

Let's return to our problem:  accounting for the
shape of the volatility smile in equity index options.
A lognormal risk-neutral distribution gives is a flat smile,
so that doesn't work ---
evidently we need something else.
But what?
There's no shortage of possibilities, but we'll focus on one that's widely used in finance:
a combination of a normal random variable and a Poisson mixture of normals.
Robert Merton has the classic paper on the subject,
so it's known as the Merton model.

Let us say that the log of the price of the underlying is the sum
of two independent components:
\begin{eqnarray*}
    \log s_{t+1} &=& x_1 + x_2 .
\end{eqnarray*}
The first component is normal:
\begin{eqnarray*}
    x_1 &\sim& \mathcal{N}(\mu, \sigma^2) .
\end{eqnarray*}
If we stop there, we're back to BSM.
The second component, sometimes called the ``jump component,'' is a Poisson mixture of normals.
If $j$ is a Poisson random variable, then conditional on $j$
\begin{eqnarray*}
    x_2 | j &\sim& \mathcal{N}(j\theta, j\delta^2) .
\end{eqnarray*}
The idea behind this is that during the relevant time interval,
we may get 0, 1, 2, or more jumps, each of which is normal
with mean $\theta$ and variance $\delta^2$.
If there are $j$ jumps, then $x_2$ is
a normal random variable with $j\theta$ and variance $j\delta^2$.
When we compute the pdf for $x_2$,
we take each of these normal pdfs and compute the sum weighted
by the Poisson probabilities $e^{-\omega} \omega^j / j!$.

The beauty of this approach is that we have, essentially, a weighted average
of BSM formulas.
The density function is a weighted average of normal pdf's, one for each $j$.
We know how to compute the put price for normals, namely equation (\ref{eq:bsm-put-kappas})
with the appropriate choices of $\kappa_1$ and $\kappa_2$.
Then the price is the weighted average of the put prices for each $j$.
This takes patience, but hang in there.
For each $j$, we might write (\ref{eq:bsm-put-kappas}) as
\begin{eqnarray*}
    q^p_t(j) &=& q^1_t k N (f_j) - q^1_t e^{\kappa_{1j} + \kappa_{2j}/2}
            N ( f_j - \kappa_{2j}^{1/2} ) \\
     &&  f_j \;\;=\;\; (\log k - \kappa_{1j})/\kappa_{2j}^{1/2} ,
\end{eqnarray*}
where $\kappa_{1j} = \mu + j \theta $ and $\kappa_{2j} = \sigma^2 + j \delta^2$.
The put price is then
\begin{eqnarray*}
    q^p_t &=& \sum_{j=0}^\infty p(j) q^p_t(j)
            \;\;=\;\; \sum_{j=0}^\infty e^{-\omega} (\omega^j / j!) q^p_t(j) .
\end{eqnarray*}
In practice, you usually need only a few terms in $j$ because $p(j)$
approaches zero rapidly.


There are two remaining issues.
The first one is the no-arbitrage condition.
In this setting, it takes the form
\begin{eqnarray*}
    \log s_t - \log q^1_t &=& \left(\mu + \sigma^2/2 \right)
            + \omega \left( e^{\theta+\delta^2/2} - 1 \right) .
\end{eqnarray*}
You can see the normal and Poisson inputs here.
Given values of the other parameters, our operating procedure is
to choose $\mu$ to satisfy this equation.
The second issue is the time interval.
If we change from a time interval of one to $\tau$, we simply multiply
$(\mu, \sigma^2, \omega)$ by $\tau$.
That's the beauty of the normal and Poisson distributions,
they adapt easily to different time intervals.

If we add the two components together, we have a distribution
of $\log s_{t+1}$ that is normal for every $j$:
\begin{eqnarray*}
    \log s_{t+1} | j &\sim& \mathcal{N}(\mu + j \theta, \sigma^2 + j \delta^2).
\end{eqnarray*}
Therefore it can be computed using the BSM-like formula (\ref{eq:bsm-put-kappas}).
The put price is the weighted average of these, with weights
on each $j$ of $p(j) = e^{-\omega} \omega^j / j!$.


This model has enough flexibility to reproduce many
of the volatility smiles we see in actual markets.
It's a common starting point for serious work on option pricing.
[Talk about role of $\delta$ (generates curvature)
and $\omega$ (generates slope).]
A lot of this can be illustrated more simply with a Bernoulli mixture (one jump),
but that model doesn't divide well into smaller time intervals.


\section*{Bottom line}

The BSM formula is the result of a model in which the risk-neutral distribution
of the underlying is lognormal.
It's most useful as a formula that we use to compute implied volatilities.
The resulting volatility smiles show clearly that the distribution
isn't lognormal.
That's a little weird, because we're using a formula that comes from a model
that doesn't work very well, but that's what we do.
That leaves us looking for alternative models, including the Merton model.

This is close to the state of the art in option pricing.
The one thing most applications add is stochastic volatility:
something like variation over time in $\sigma$ in the model of
Section \ref{sec:merton}.
That generates movements up and down of volatility smiles,
which we see all the time in these markets.
The VIX, for example, is (roughly) the value of at-the-money volatility.
If you're interested, let me know and I'll point you to the standard references.
If you're really really interested, you might start by taking
courses in stochastic processes and stochastic calculus.


\section*{Practice problems}

\begin{enumerate}
\item {\it Put and calls.\/}
Use put-call parity (\ref{eq:put-call-parity})
to derive the BSM price of a call option (\ref{eq:bsm-call})
from the price of a put (\ref{eq:bsm-put}).
%
\needspace{2\baselineskip}
Answer.  This one's all yours.

%-----------------------------------------------------------------------
\item {\it The BSM formula.\/}
Consider the BSM call option formula, equation (\ref{eq:bsm-call})
with $s_t = 100$, $q^1_t = 0.98$, and $\tau = 1$.
%
\begin{enumerate}
\item What is the price of a call option at strike $k = 95$ if $\sigma = 0.20$?
\item What is the price of a call option at strike $k=105$?
Why is it higher or lower than your answer to (a)?
\item What is the price of a call option at strike $k = 95$ if $\sigma = 0.25$?
Why is it higher or lower than your previous answer to (a)?
\item Suppose you observe a call price of 12.00.
Is implied volatility above or below 0.20?
Why?
What is implied volatility?
\end{enumerate}
%
\needspace{2\baselineskip}
Answer.
\begin{enumerate}
\item 11.625.
\item 6.713. The payoff is strictly less at all outcomes $s_{t+1}$.
\item 13.442.
It's higher because options have convex payoffs:
their value increases the more uncertainty there is.
\item We know the price is increasing in volatility.
So if the price is higher, the volatility must be higher.
If we play around with values, we find $\sigma = 0.2104$ does the trick.
\end{enumerate}

%-----------------------------------------------------------------------
\item {\it Digital options.\/}
A digital or binary option either pays some fixed amount or not.
Consider, for example, a digital option on the underlying $s_{t+1}$.
A digital call with strike $k$ pays 100 if $s_{t+1} > k$.
A digital put with strike $k$ pays 100 if $s_{t+1} \leq k$.
%
\begin{enumerate}
\item What is the analog of put-call parity for these options?

\item Suppose $\log s_{t+1}$ has a normal risk-neutral distribution
with mean $\kappa_1$ and variance $\kappa_2$.
What is the price of a put with strike $k$?

\item What is the no-arbitrage condition for this situation?
Use it to simplify your answer to (b).

\item {\it Extra credit.\/}
How does the price of a put respond to an increase in volatility
$ \sigma = \kappa_2^{1/2}$?
Why?
\end{enumerate}
%
\needspace{2\baselineskip}
Answer.
\begin{enumerate}
\item The idea behind put-call parity is that if we know the price of a put,
we can use it to find the price of a call.  And vice versa.
Here we see that a put gives us 100 at date $t+1$ if $s_{t+1} \leq k$,
and a call gives us 100 if $s_{t+1} > k$, so if we buy one of each we get
100 for sure.
that gives us the parity relation
\begin{eqnarray*}
    q^p_t + q^c_t &=& q^1_t \cdot 100 .
\end{eqnarray*}

\item The payoff from a put option is
\begin{eqnarray*}
    d_{t+1} &=&
        \left\{
        \begin{array}{rl}
            100    &  \mbox{if } s_{t+1} \leq k  \\
            0      &  \mbox{otherwise} .
        \end{array}
        \right.
\end{eqnarray*}
The price is therefore
\begin{eqnarray*}
    q^p_t &=& q^1_t E^* (d_{t+1})  \\
            &=& q^1_t \mbox{Prob}(s_{t+1}\leq k) 100 \\
            &=& q^1_t \mbox{Prob}(\log s_{t+1}\leq \log k) \cdot 100 \\
            &=& q^1_t N [ (\log k-\kappa_1)/\kappa_2^{1/2}] \cdot 100 .
\end{eqnarray*}
The last line uses the lognormal (risk-neutral) distribution, but
the others hold in general.

\item This is the same as the one we used for BSM.
The idea in general is that we must choose a risk-neutral distribution
that's consistent with the current price of the asset.
In this case we have
\begin{eqnarray*}
    s_t &=& q^1_t E^* s_{t+1} \;\;=\;\; q^1_t e^{\kappa_1 + \kappa_2/2} ,
\end{eqnarray*}
which should look familiar.
In practice, we would use this condition to eliminate $\kappa_1$,
as we did when we derived the BSM formula:
\begin{eqnarray*}
    (\log k - \kappa_1)/\kappa_2^{1/2} &=&
            [\log (q^1_t k/s_t) + \kappa_2/2]/\kappa_2^{1/2} .
\end{eqnarray*}
There's a similar expression in BSM.

\item This is terse, but the idea is that it depends on whether
the option is in the money or not.
Going beyond the formula, the payoff in this case isn't convex,
so the impact of volatility is different.
\end{enumerate}

%-----------------------------------------------------------------------
\item {\it Option on a uniform underlying.\/}
We'll explore the logic of option pricing in a setting in which the underlying
has a uniform distribution.
More formally, let $ x = s_{t+1} $ (not $x = \log s_{t+1}$!)
have a uniform distribution over the
interval $ [\mu-\theta, \mu + \theta]$;
that is, its probability density function is
%
\begin{eqnarray*}
    p(x) &=& \left\{
            \begin{array}{ll}
            (2\theta)^{-1} & \mbox{if } \mu-\theta \leq x \leq \mu + \theta \\
            0              & \mbox{otherwise}.
            \end{array}
            \right.
\end{eqnarray*}

\smallskip
\begin{enumerate}
\item What is the mean of $x$? The variance?
\item What is the no-arbitrage condition here?
\item What is the cash flow of a put option with strike price $k$?
What is its value?
\item How does the price of a put option change when we increase $\theta$?
Why?
\end{enumerate}

\needspace{2\baselineskip}
Answer.
\begin{enumerate}
\item You might guess the mean and variance.  If not, they follow from
\begin{eqnarray*}
    E(x) &=& (2\theta)^{-1} \int_{\mu-\theta}^{\mu+\theta} x \; dx
            \;\;=\;\; (2\theta)^{-1} [(\mu+\theta)^2 - (\mu-\theta)^2]/2 \;\;=\;\; \mu \\
    \mbox{Var}(x) &=& (2\theta)^{-1} \int_{\mu-\theta}^{\mu+\theta} (x-\mu)^2 dx
            \;\;=\;\; (2\theta)^{-1} [\theta^3 - (-\theta)^3]/3 \;\;=\;\; \theta^2/3 .
\end{eqnarray*}
So $\mu$ governs the mean and $\theta$ governs the variance.

\item The no-arb condition is
\begin{eqnarray*}
    s_t &=& q^1 E^*(s_{t+1}) \;\;=\;\; q^1 \mu .
\end{eqnarray*}

\item A put option is a claim to the cash flow $ d = (k - x)^+ $.
If $k$ is in the interval $ [\mu-\theta, \mu + \theta]$ we have
\begin{eqnarray*}
    q^p &=& q^1 (2\theta)^{-1} \int_{\mu-\theta}^k (k-x) dx \\
%        \;\;=\;\;
        &=&
            q^1 k \; [k-(\mu-\theta)]/(2\theta)
            - q^1 [k^2 - (\mu-\theta)^2]/(4\theta).
\end{eqnarray*}

\item The idea is that option prices increase with risk.
Here if we increase $\theta$, the parameter that controls the variance,
put prices seem to go up, but I haven't proven it.

\end{enumerate}


%-----------------------------------------------------------------------
\item {\it Option on an exponential underlying.\/}
Suppose the risk-neutral distribution of the future value of the underlying
$s_{t+1} = x$ has pdf
\begin{eqnarray*}
    p(x) &=& \lambda e^{-\lambda x}
\end{eqnarray*}
for $x \geq 0$ and $\lambda > 0$.
In  words: $x$ has an exponential distribution.
The mean of $x$ is $1/\lambda$.
%
\begin{enumerate}
\item What is the no-arbitrage condition for this asset?
\item Consider a put option giving the owner the right to sell
the asset for price $k$ at $t+1$.
What cash flow is generated by this option?
\item What is the price of the option?
\end{enumerate}
%
\needspace{2\baselineskip}
Answer.
\begin{enumerate}
\item We have
\begin{eqnarray*}
    s_t &=& q^1_t E^* (s_{t+1})
            \;\;=\;\; q^1_t E^*(x)
            \;\;=\;\; q^1_t (1/\lambda)  .
\end{eqnarray*}
\item The cash flow is (as usual) $ (k-s_{t+1})^+ $.
\item The put price is
\begin{eqnarray*}
    q^p_t &=& q^1_t E^* (k-s_{t+1})^+  \\
            &=& q^1_t \int_{0}^{k} (k-x) \lambda e^{-\lambda x } dx
            \;\;=\;\;  q^1_t \big[ k - ( 1-e^{-\lambda k})/\lambda \big] .
\end{eqnarray*}
\end{enumerate}

%-----------------------------------------------------------------------
\item {\it Option on a mixture of exponentials.\/}
Suppose the risk-neutral distribution of the future value of the underlying
is a Bernoulli mixture of $x_1$ and $x_2$:
\begin{eqnarray*}
    s_{t+1} &=&
        \left\{
        \begin{array}{ll}
            x_1 & \mbox{with probability } 1-\omega \\
            x_2 & \mbox{with probability } \omega
        \end{array}
        \right.
\end{eqnarray*}
for some $\omega$ between zero and one.
Each $x_j$ is exponential with density
\begin{eqnarray*}
    p(x_j) &=& \lambda_j e^{-\lambda_j x_j}
\end{eqnarray*}
for $x_j \geq 0$ and $\lambda_j > 0$.
Each $x_j$ has a mean of $1/\lambda_j$.
%
\begin{enumerate}
\item What is the no-arbitrage condition for this asset?
\item Consider a put option giving the owner the right to sell
the asset for price $k$ at $t+1$.
What cash flow is generated by this option?
\item What is the price of a put option?
\end{enumerate}
%
\needspace{2\baselineskip}
Answer.
\begin{enumerate}
\item We have
\begin{eqnarray*}
    s_t &=& q^1_t E^* (s_{t+1}) \\
            &=& q^1_t [(1-\omega) E^*(x_1) + \omega E^*(x_2)]
            \;\;=\;\; q^1_t [(1-\omega)/\lambda_1 + \omega /\lambda_2] .
\end{eqnarray*}
\item The cash flow is (as usual) $ (k-s_{t+1})^+ $.
\item The put price is
\begin{eqnarray*}
    q^p_t &=& q^1_t \Big\{
        (1-\omega) \big[ k - ( 1-e^{-\lambda_1 k})/\lambda_1 \big]
        + \omega \big[ k - ( 1-e^{-\lambda_2 k})/\lambda_2 \big]
        \Big\} .
\end{eqnarray*}
The logic follows the previous problem.
\end{enumerate}

%-----------------------------------------------------------------------
\needspace{3\baselineskip}
\item {\it Divisibility.\/} Explain intuitively why the Bernoulli distribution isn't divisible.

%\needspace{2\baselineskip}
Answer. A Bernoulli random variable takes on the values zero and one.
If we repeat it $n$ times, one way to think about it is that
we have $n$ chances of getting a one.
The probabilities of various outcomes are binomial.
If $n=1$, for example, we can get zero ones, or one one.
If $n=2$, we can get zero ones, one one, or two ones.
But what if $n=1/2$?
It's not possible to get half a one, it just doesn't work.
Wikipedia has a more formal treat of
\href{http://en.wikipedia.org/wiki/Indecomposable_distribution}{this example}
and of the issue
\href{http://en.wikipedia.org/wiki/Infinite_divisibility_(probability)}
{in general}.


%-----------------------------------------------------------------------
\item {\it Sums and mixtures (review).\/}
Let us say that the log-price of the underlying has two components:
\begin{eqnarray*}
    \log s_{t+1} &=& y_{t+1} \;\;=\;\; x_{1t+1} + x_{2t+1},
\end{eqnarray*}
with $( x_{1t+1}, x_{2t+1})$ independent.
The first component is normal:
$ x_{1t+1} \sim \mathcal{N}(\mu,\sigma^2)$.
The second component, the ``jump,'' is a mixture:
with probability $1-\omega$, $ x_{2t+1} = 0$,
and with probability $\omega$,
$ x_{2t+1} \sim \mathcal{N}(\theta,\delta^2)$.

With these inputs, the pdf for $y$ is a weighted average of normals:
\begin{eqnarray}
    p(y) &=& (1-\omega) \cdot ( 2 \pi \sigma^2)^{-1/2} \exp[ - (y-\mu)^2/2\sigma^2]
            \nonumber \\
    && + \; \omega \cdot [ 2 \pi (\sigma^2+\delta^2)]^{-1/2}
            \exp[ - [y-(\mu+\theta)]^2/2(\sigma^2 + \delta^2)] .
            \label{eq:bernoulli-mixture}
\end{eqnarray}
If $\omega = 0$, the second component drops out.
Otherwise, we have a weighted average of two normal densities.

\begin{enumerate}
\item Show that the cumulant generating function for $x_{1t+1}$ is
\begin{eqnarray*}
    k (s; x_1) &=&  \mu s + \sigma^2 s^2 / 2 .
\end{eqnarray*}

\item Show that the cumulant generating function for $x_{2t+1}$ is
\begin{eqnarray*}
    k (s; x_2) &=& \log \left[ (1-\omega) + \omega e^{ \theta s + \delta^2 s^2 / 2 }
            \right] .
\end{eqnarray*}

\item What is the cgf for $y_{t+1}$?
What are its mean, variance, skewness, and excess kurtosis?
What parameters determine the sign of skewness?
\end{enumerate}
%
\needspace{2\baselineskip}
Answer.
This is a reminder that we can use mixtures of normals
to generate nonnormal behavior.
\begin{enumerate}
\item  The usual normal cgf.
\item This is a Bernoulli mixture with one twist:
the ``$\omega$ branch'' has a normal mgf in it.
If $\delta = 0$, it's just like the Bernoulli we looked at in
Lab Report \#1.
Otherwise, we get some additional terms.

\item The cgf of the sum is the sum of the cgfs:
\begin{eqnarray*}
        k (s; y) &=& k (s; x_1) + k (s; x_2) \\
             &=& ( \mu s + \sigma^2 s^2 / 2 ) +
              \log \left[ (1-\omega) + \omega e^{ \theta s + \delta^2 s^2 / 2 }
            \right] .
\end{eqnarray*}
We find it cumulants from derivatives, which are conveniently
computed by Matlab.
The mean and variance are
\begin{eqnarray*}
    \kappa_1 &=& \mu + \omega \theta \\
    \kappa_2 &=& \sigma^2 + \omega (1-\omega) \theta^2 + \omega \delta^2 .
\end{eqnarray*}
Each has terms from both components.
The third and fourth cumulants come only from $x_2$, because the normal
component has zero cumulants beyond the first two.
The third and fourth cumulants are
\begin{eqnarray*}
    \kappa_3 &=& \omega (1-\omega) \theta [ 3 \delta^2 + (1-2\omega) \theta^2]  \\
    \kappa_4 &=& \omega (1-\omega)
        \{  \theta^4 [1 - 6 \omega(1-\omega)] + 3 \delta^4 + (1-2 \omega) 6 \delta^2 \theta^2
        \}
\end{eqnarray*}
This is a bit of a mess, but for $\omega$ small,
$\kappa_3$ depends on the sign of $\theta$
and $\kappa_4 >0 $,
so the mixture introduces skewness and excess kurtosis,
both of which are absent from normal random variables.
\end{enumerate}

%-----------------------------------------------------------------------
\item {\it Merton-like option pricing.\/}
With the same setup as the previous problem,
we can illustrate the value of mixtures in
generating nonnormal distributions and, as a result,
more flexible shapes of volatility smiles.

\begin{enumerate}
\item Risk-neutral asset pricing tells us, in general, that
\begin{eqnarray*}
    s_t &=& q^1_t E^* (s_{t+1} )
            \;\;=\;\; q^1_t E^* \left( e^{y_{t+1}}\right)
            \;\;=\;\; q^1_t  h(1; y)
            \;\;=\;\; q^1_t  e^{k(1; y)}  .
\end{eqnarray*}
We refer to this as the no-arbitrage condition.
What is the no-arbitrage condition for our example?

We'll use this condition to set $\mu$:
given values for everything else, we'll choose $\mu$ to satisfy this condition.

\item Recall that if the risk-neutral distribution is
 $\log s_{t+1} = y_{t+1} \sim \mathcal{N}(\kappa_1,\kappa_2)$,
 then the put price at strike $k$ is
\begin{eqnarray*}
    f(k; \kappa_1, \kappa_2) &=& q^1_t k N(d) - q^1_t e^{\kappa_1 + \kappa_2/2}
            N(d-\kappa_2^{1/2}) \\
            d&=& (\log k - \kappa_1)/\kappa_2^{1/2} .
\end{eqnarray*}
(Note:  modest change in notation.)
What is the call price?

\item
Use (b) to show that the put price in the mixture model is a weighted average:
\begin{eqnarray*}
    q^p_t &=& (1-\omega) \cdot f(k; \mu, \sigma^2) +
        \omega \cdot f(k; \mu+\theta, \sigma^2 + \delta^2) .
\end{eqnarray*}

\item Consider these inputs:
$\sigma = 0.04$, $\omega = 0.01$,
$\theta = -0.3$, $\delta = 0.15$,
$s_t = 100$, and $q^1_t = 1$.
What is $\mu$?
What are the prices of put options with strikes
$ k = (90, 94, 98, 102, 106, 110)$?
(Use a finer grid if you have this automated.)
What are the implied volatilities?

\item What happens to the volatility smile when you set
\begin{itemize}
\item $\theta = 0$?
\item $\theta = + 0.3$?
\item $\delta = 0.25$?
\end{itemize}
Make sure you adjust $\mu$ in each case.
\end{enumerate}
%
\needspace{2\baselineskip}
Answer.
\begin{enumerate}
\item The condition is
\begin{eqnarray*}
    s_t &=& q^1_t \exp( \mu + \sigma^2 / 2 )
            \left[ (1-\omega) + \omega e^{ \theta + \delta^2 / 2 } \right] .
\end{eqnarray*}
It's not pretty, but given values of the other inputs we can use it to set $\mu$.

\item This is a question of integrating over the distribution,
as we did when we derived the BSM formula.
We can then find the call price from put-call parity.

\item Since the pdf is a sum, we can integrate
each term separately (the integral of a sum is the sum of the integrals).
Each integral gives us a BSM-like formula, as noted,
but with different mean and variance.
The put price is the weighted average of the two formulas, as stated.

\item Now we can put all this to work.
With these inputs, the arbitrage condition gives us $ \mu = 4.6009$.
The put prices are

\begin{center}
\tabcolsep=0.15in
\begin{tabular}{rr}
\toprule
Strike    &  Put Price  \\
\midrule
  90   & 0.1606   \\
   94  &  0.2806  \\
   98  &  0.9285  \\
  102  &  2.8801  \\
  106  &  6.1523  \\
  110  & 10.0147 \\
\bottomrule
\end{tabular}
\end{center}

The implied volatilities give us a clearer picture;
run the Matlab code below and see.
We find that there's a distinct downward slope to the smile.
That disappears if we set $\theta = 0$, since $\theta$ is the source
of skewness in the model.
Instead we get a U-shape.
When we increase $\delta$, the U is more pronounced.
So roughly speaking, $\theta$ generates skewness and slope
in the smile,
and $\delta$ generates kurtosis and convexity in the smile.

Here's the code:
{\small
\begin{verbatim}
disp('Inputs')
tau = 1;
q1 = 1;
q_tau = q1;
s = 100.00
k = [85:2:115]';

sigma = 0.04;
omega = 0.01;
theta = -0.3;
delta = 0.15;

% apply arb condition
mu = log(s/q1) - sigma^2/2 - log((1-omega)+omega*exp(theta+delta^2/2))

% branch 1
d1 = (log(k)-mu)/sigma;
put1 = q1*k.*normcdf(d1) - q1*exp(mu+sigma^2/2)*normcdf(d1-sigma);

% branch 2
d2 = (log(k)-(mu+theta))/sqrt(sigma^2+delta^2);
put2 = q1*k.*normcdf(d2) - Q1*exp((mu+theta) ...
    + (sigma^2+delta^2)/2)*normcdf(d2-sqrt(sigma^2+delta^2));

puts = (1-omega)*put1 + omega*put2;
calls = puts + s - q1*k;

% BSM formula
% f = call price as function of sigma, two steps for clarity
% fp = the derivative (vega) for use in Newton routine
d = @(sigma,k) (log(s./(q_tau.*k))+tau*sigma.^2/2)./(sqrt(tau)*sigma);
f = @(sigma,k) s*normcdf(d(sigma,k)) - q_tau.*k.*normcdf(d(sigma,k) ...
        -sqrt(tau)*sigma) - calls;
fp = @(d) s*sqrt(tau)*exp(-d.^2/2)/sqrt(2*pi);

% convergence parameters
tol = 1.e-8;
maxit = 50;

% starting values
x_now = 0.12 + zeros(size(k));
f_now = f(x_now,k);

% compute implied vol
t0 = cputime;
for it = 1:maxit
    fp_now = fp(d(x_now,k));
    x_new = x_now - f_now./fp_now;
    f_new = f(x_new,k);
    diff_x = max(abs(x_new - x_now));
    diff_f = max(abs(f_new));

    if max(diff_x,diff_f) < tol, break, end

    x_before = x_now;
    x_now = x_new;
    f_before = f_now;
    f_now = f_new;
end

% plot smile
figure(1)
clf
plot(k, vol, 'b')
hold on
plot(k, vol, 'b+')
xlabel('Strike Price')
ylabel('Implied Volatility')
\end{verbatim}
} 
\end{enumerate}

\needspace{2\baselineskip}
\item {\it Today's option prices.\/}
Yahoo Finance has an
\href{http://finance.yahoo.com/options/}{Options Center},
where you can look up prices of a wide range of equity options.
Look up options on SPY, the ticker for
an S\&P 500 exchange traded fund.
Choose a contract, note the prices, and compute the volatility smile.
What does it look like?
%
\needspace{2\baselineskip}
Answer.  This one's all yours, too.

\end{enumerate}


{\vfill
{\bigskip \centerline{\it \copyright \ \number\year \
David Backus $|$ NYU Stern School of Business}%
}}


\end{document}
