\documentclass[11pt]{article}

% spacing on page
\oddsidemargin=0.25truein \evensidemargin=0.25truein
\topmargin=-0.5truein \textwidth=6.0truein \textheight=8.75truein

\usepackage{comment}
\usepackage{graphicx}
\usepackage{amssymb}
\usepackage{amsmath}

\usepackage[margin=0pt, labelsep=period, font=large, labelfont=bf]{caption}
%\usepackage{float}

\usepackage{hyperref}
\urlstyle{rm}   % change fonts for url's (from Chad Jones)
\hypersetup{
    colorlinks=true,        % kills boxes
%    allcolors=blue,
    pdfsubject={ECON-UB233, Macroeconomic foundations for asset pricing},
    pdfauthor={Dave Backus @ NYU},
    pdfstartview={FitH},
    pdfpagemode={UseNone},
%    pdfnewwindow=true,      % links in new window
%    linkcolor=blue,         % color of internal links
%    citecolor=blue,         % color of links to bibliography
%    filecolor=blue,         % color of file links
%    urlcolor=blue           % color of external links
% see:  http://www.tug.org/applications/hyperref/manual.html
}

% for listing code in tt font
\usepackage{verbatim}

% for table spacing
\usepackage{booktabs}

% section headers and spacing
\usepackage[small, compact]{titlesec}

% list spacing
\usepackage{enumitem}
\setitemize{leftmargin=*, topsep=0pt}
\setenumerate{leftmargin=*, topsep=0pt}

% attach files to the pdf
\usepackage{attachfile}
    \attachfilesetup{color=0.75 0 0.75}

\usepackage{needspace}
% \needspace{4\baselineskip} makes sure we have four lines available before a pagebreak

%\renewcommand{\thefootnote}{\fnsymbol{footnote}}
%\newcommand{\var}{\mbox{\it Var\/}}

\newcommand{\cbar}{\bar{c}}
\newcommand{\rp}{\mbox{\em rp\/}}
\newcommand{\tp}{\mbox{\em tp\/}}
\newcommand{\bp}{\mbox{\em bp\/}}


\begin{document}
\parskip=\bigskipamount
\parindent=0.0in
\thispagestyle{empty}
{\large ECON-UB 233 \hfill Dave Backus @ NYU}% 

\bigskip\bigskip
\centerline{\Large \bf Predicting Returns}
\centerline{Revised: \today}

\begin{comment}
Can we convert bond models to returns?

\end{comment}

\bigskip
We've seen that assets differ in their average excess returns,
evidently reflecting differences in risk premiums.
Long bonds have higher returns, on average, than short bonds,
and broad-based equity indexes have higher returns still.

Here we do two things related to what I'd call the predictability of
excess returns.  First, we describe the evidence in somewhat greater detail,
including evidence that expected excess returns vary through time and across assets.
Second, we return to ``exponential-affine models'' in which we can think
about similar properties of theoretical asset pricing models.
The two subjects are to some extent separate,
but models give us a useful perspective on the evidence.
This is an active topic of current research in which
the evidence is currently at a more advanced stage than the theory that might explain it.


\section{Predictable returns}

Portfolio choice is a complicated business:  you'd want to know,
in principle, the joint distribution of returns on all assets,
your preferences, and your circumstances (income, expenses, retirement plans, and so on).
It's not a simple problem, as we've seen, even in much simpler settings.
We can, however, make some progress on the properties of returns
on special collections of assets, which we'll refer to as portfolios.
Not your whole portfolio, but potentially components of it.

We've accumulated lots of evidence on the risk and average returns
on a large number of such portfolios, which we organize here by asset class.

{\it Equity.\/}
Equity portfolios have gotten a lot of attention.
Here's a short summary of the variables that are related to average excess returns:
%
\begin{itemize}
\item Size.  Small firms earn higher returns, on average, than large firms.
More concretely:  If we sort publicly traded stocks by market cap (the market value of their equity)
and put them in bins (small, medium, large), then
the average excess return is highest for small firms.
In this respect, size helps to predict returns.

\item Value and growth.
Another collection of portfolios is based on the ratio of book value to market value.
Book value is an accounting concept that we'll ignore, but the idea is that
firms with low ratios represent ``growth'' (that presumably is why they have high market value)
and those with high ratios represent ``value.''
Whatever the logic, the evidence is that excess returns are higher, on average,
for value firms.

\item Momentum.  Here we sort by past returns.
Stocks with high returns over the past year are referred to as ``winners,''
those with low returns are ``losers.''
The evidence shows that mean excess returns on winners are higher than those
on losers.
In this respect, returns have momentum:  high returns in the past suggest high returns
in the future.

\end{itemize}
%
Data for all of these portfolios --- and more ---
is available on Ken French's
\href{http://mba.tuck.dartmouth.edu/pages/faculty/ken.french/data_library.html}{website}.


{\it Currencies.\/}
Another well-established property concerns returns on currencies.
If we sort currencies by interest rate, then currencies with high interest rates
have higher excess returns in dollars than currencies with low interest rates.
An investment in high interest rate currencies financed by borrowing in low interest rate
currencies makes a positive return on average:  the so-called carry trade.

{\it Bonds.\/}
Excess returns on long bonds seem not to be constant.
In particular, excess returns are higher when the yield curve is steeper,
which is generally when short-term interest rates are low.
More on this shortly.

These facts are reasonably robust, but they're not bullet-proof.
Momentum, for example, went the other way during the crisis.
Ditto the carry trade.
Despite this, they're a good place to start in thinking
about the dynamics of asset returns.


\section{Predicting bond returns}

We're interested in bond returns, but focus on
closely related changes in forward rates.
Recall that bond prices are connected to forward rates by
\begin{eqnarray*}
    \log q^{n+1}_t &=& -(f^0_t + f^1_t + \cdots + f^{n}_t) .
\end{eqnarray*}
The log excess return on an $(n+1)$-period bond is therefore
\begin{eqnarray*}
    \log r^{n+1}_{t+1} - \log r^1_{t+1}
         &=& \log q^n_{t+1} - \log q^{n+1}_t + \log q^1_t \\
         &=& (f^0_{t+1} - f^1_t) + \cdots + (f^{n-1}_{t+1} - f^n_t)
         \;\;=\;\; \sum_{j=1}^n (f^{j-1}_{t+1} - f^j_t) .
\end{eqnarray*}
[This is tedious, but follows from the definitions of returns and forward rates.]
In short, if we account for changes in forward rates then we account for bond returns ---
and vice versa.

%Now how might we expect forward rates to change?
%Recall that the forward rate $f^n_t$ is the one-period interest rate
%arranged at date $t$ for the period $t+n$ to $t+n+1$.
%The forward rate that applies to the same period at $t+1$ is $f^{n-1}_{t+1}$.
%As we move forward in time, the maturity of the forward rate falls,
%but the rate applies to the same time period.


So how should we think about forward rates?
As always, it's helpful to have a benchmark.
Our benchmark is the {\it expectations hypothesis\/} (EH):
that changes in forward rates, suitably defined, are unpredictable.
More formally, the forward rates
$(f^n_t, f^{n-1}_{t+1}, ... , f^0_{t+n})$
form a martingale sequence:
\begin{eqnarray*}
    f^j_t &=& E_t \big( f^{j-1}_{t+1} \big) .
\end{eqnarray*}
Each of these rates applies to the same period,
the period between $t+n$ and $t+n+1$.
What's different is that they're quoted at different times.
The idea is that the rate quoted today is the best predictor of the rate
quoted next period.

Fifty years ago --- and even today in textbooks --- people thought the EH was a good idea.
Decades of evidence have convinced us otherwise, but it's still a useful benchmark.
%Now we know better, but it's still a good starting point for more sophisticated theories.
We can get a sense of its weaknesses by running the regression
\begin{eqnarray*}
    f^{n-1}_{t+1} &=& a_n + b_n f^n_t .
\end{eqnarray*}
If the EH is true, we should find that $b_n = 1$.
(We should also find $a_n=0$, but allowing it to be nonzero
takes care of any constant term premiums.
Can you see why?)
For technical reasons, we typically run the alternative
\begin{eqnarray}
    f^{n-1}_{t+1} - f^0_t &=& a_n + b_n (f^n_t - f^0_t) .
    \label{eq:forward-regressions}
\end{eqnarray}
If the EH is true, we should again find that $b_n = 1$.

The evidence suggests otherwise.
With monthly US data we find (approximately) $b_1 = 0.5$, $b_3 = 0.75$,
$b_{12} = 0.90$, and $b_{60} = 0.96$.
As we increase maturity, the EH looks better and better.
But for short maturities it's not very good.

So what's going on?
We'll look at short maturities, since that seems to be where the action is.
If we raise the forward rate $f^1_t$ (relative to $f^0_t$),
making the forward rate curve steeper,
we see that the future short rate $f^0_{t+1}$ goes up by 0.5,
not the 1.0 implied by the EH.
Thus a steeper yield curve implies higher returns on long bonds, because the
increase in the short rate --- and the ensuing capital loss on a two-period bond ---
is smaller than the EH predicts.
% ?? more



\section{Predictability in the Vasicek and CIR models}

If the evidence is summarized by (\ref{eq:forward-regressions}),
what do our models suggest?
We'll review the performance of the Vasicek and Cox-Ingersoll-Ross models
along this dimension.
In both models, bond prices are loglinear functions of the state:
$ \log q^n_t = A_n + B_n z_t$.
Excess returns, therefore, are
\begin{eqnarray*}
    \log r^{n+1}_{t+1} - \log r^{1}_{t+1} &=& \log q^n_{t+1} - \log q^{n+1}_t + \log q^1_t \\
            &=&  (A_n - A_{n+1}+A_1) + B_{n} z_{t+1} + (B_1 - B_{n+1} ) z_t   .
\end{eqnarray*}
The first term (the $A$'s) are constant;
any predictability must come from the others.

{\it Vasicek model.\/}
Consider the model
\begin{eqnarray*}
    \log m_{t+1} &=& - \lambda^2/2 - z_t + \lambda w_{t+1} \\
         z_{t+1} &=& (1-\varphi) \delta + \varphi z_t + \sigma w_{t+1} .
\end{eqnarray*}
The no-arbitrage theorem implies the recursions
\begin{eqnarray*}
    A_{n+1} &=& A_n + B_n (1-\varphi) \delta - \lambda^2/2 + (\lambda + B_n \sigma)^2/2 \\
    B_{n+1} &=& B_n \varphi - 1 .
\end{eqnarray*}
The second equation implies
$B_1 = -1$ and $B_2 = -(1+\varphi)$.
The first two forward rates are therefore
\begin{eqnarray*}
    f^0_t &=& \mbox{constants} + z_t \\
    f^1_t &=& \mbox{constants} + (B_1-B_2) z_t
            \;\;=\;\; \mbox{constants} + \varphi z_t .
\end{eqnarray*}
That gives us a rhs variable $f^1_t - f^0_t = (\varphi-1)z_t$.
The lhs is
\begin{eqnarray*}
    f^0_{t+1} - f^0_t &=& \mbox{constants} + z_{t+1} - z_t
                    \;\;=\;\; \mbox{constants} + (\varphi-1) z_t + \sigma w_{t+1}.
\end{eqnarray*}
Since $w_{t+1}$ is independent of the rhs (in fact of anything dated $t$ or before),
we're effectively regressing $(\varphi-1) z_t$
on itself.
The coefficient is therefore $b_1 = 1$, not the $b_1 \approx 0.5$ that we see in the data.

This model, in short, fails to account for the evidence against the expectations
hypothesis.
More work would show us that term premiums in this model are constant:
there's no variation in either risk or the price of risk,
so premiums are constant.
We need something else.

[Exercise.
What is the log excess return on an $n$-period bond in this model?
Show that it doesn't depend on $z_t$.]

{\it CIR model.\/}
This model has variation in risk at its core,
so we have a chance of doing better.
The model, you may recall,
is
\begin{eqnarray*}
    \log m_{t+1} &=& - (1+\lambda^2/2 ) z_t + \lambda z_t^{1/2} w_{t+1} \nonumber \\
         z_{t+1} &=& (1-\varphi) \delta + \varphi z_t + \sigma z_t^{1/2} w_{t+1} .
         \label{eq:square-root}
\end{eqnarray*}
It leads to the recursions
\begin{eqnarray*}
    A_{n+1} &=& A_n + B_n (1-\varphi) \delta \\
    B_{n+1} &=& \varphi B_n -(1+\lambda^2/2) + (\lambda + B_n \sigma)^2/2 .
\end{eqnarray*}
From this, we see that $B_1 = -1$ and $B_2 = -(1+\varphi) + \sigma(\lambda + \sigma/2)$.
That implies
\begin{eqnarray*}
    f^0_t &=& z_t \\
    f^1_t &=& \mbox{constants} + (B_1-B_2) z_t
            \;\;=\;\; \mbox{constants} + [\varphi-\sigma(\lambda+\sigma/2)] z_t ,
\end{eqnarray*}
which gives us the rhs variable
\begin{eqnarray*}
    f^1_t - f^0_t &=& \mbox{constants} + [\varphi-1-\sigma(\lambda+\sigma/2)] z_t .
\end{eqnarray*}
The lhs variable is
\begin{eqnarray*}
    f^0_{t+1} - f^0_t &=&  \mbox{constants} + (\varphi-1) z_t + \sigma z_t^{1/2} w_{t+1}.
\end{eqnarray*}
The regression coefficient is therefore
\begin{eqnarray*}
    b_1 &=& \frac{\varphi-1}{\varphi-1-\sigma(\lambda+\sigma/2)} .
\end{eqnarray*}
This can deliver lots of values of $b_1$, including our preferred value $b_1 = 1/2$.
But with the parameter values we chose earlier, we get $b_1 = 1.384$.
So we've deviated from the EH, but in the wrong direction.

The issue is $\lambda$.  We can choose $\lambda$ to get $b_1 = 1/2$:
\begin{eqnarray*}
    \lambda &=& (1-\varphi)/\sigma - \sigma/2 .
\end{eqnarray*}
But now the mean forward rate curve declines with maturity ---
it slopes the wrong way.

% ?? add stuff on risk and return
% also cylicality of risk


\section{Another exponential-affine model}

You might think otherwise, but the CIR model makes some progress in generating
predictable changes in forward rates.
It's just that the variation goes the wrong way.
What can we do?
If we look at the model, perhaps with an unreasonable amount
of insight,
we might notice that there are two things
that generate predictability in forward rate changes:
variation in risk (the term that includes $\sigma$)
and variation in the price of risk (the term that includes $\lambda$).
We had both in the CIR model.
The next model focuses on the latter,
our third and last example of an exponential-affine model.


The model is
\begin{eqnarray*}
    \log m_{t+1} &=& - (\lambda_0 + \lambda_1 z_t)^2/2 + \delta - z_t
            + (\lambda_0 + \lambda_1 x_t) w_{t+1} \\
            z_{t+1} &=& \varphi z_t + \sigma w_{t+1} .
\end{eqnarray*}
This structure is cleverly rigged to deliver loglinear bond prices.
If $\lambda_1 = 0$, we're back to Vasicek.
Otherwise the price of risk depends on the state $z_t$.

We solve it the usual way.
We guess  $ \log q^n_t = A_n + B_n z_t$ and substitute into the
pricing relation.
The conditional mean and variance are
\begin{eqnarray*}
   E_t \left( \log m_{t+1} + \log q^{n}_{t+1} \right) &=&
            - (\lambda_0 + \lambda_1 z_t)^2/2 + \delta + (\varphi B_n -1) z_t  \\
   \mbox{Var}_t \left( \log m_{t+1} + \log q^{n}_{t+1} \right) &=&
            \left[B_n \sigma + (\lambda_0 + \lambda_1 z_t)\right]^2  .
\end{eqnarray*}
The ``mean plus variance over two'' formula gives us, eventually,
the recursions
\begin{eqnarray*}
    A_{n+1} &=& A_n + \delta + B_n \sigma (B_n \sigma/2 + \lambda_0) \\
    B_{n+1} &=& \varphi B_n - 1 + B_n \sigma \lambda_1 .
\end{eqnarray*}
This implies
$B_1 = -1$ and $B_2 = -(1+\varphi) - \sigma \lambda_1 $.


After the usual manipulations,
we find the regression coefficient
\begin{eqnarray*}
    b_1 &=& \frac{\varphi-1}{\varphi-1+\sigma\lambda_1} .
\end{eqnarray*}
This model has one clear advantage over the previous ones:
we have two parameters governing the price of risk,
$\lambda_0$ and $\lambda_1$.
We can, in principle, choose $\lambda_1$ to reproduce
the regression coefficient and $\lambda_0$ to reproduce the
mean forward rate curve.

Here's how that might work:
%
\begin{itemize}
\item Set $\varphi = 0.959$, as before, to reproduce the
autocorrelation of the short rate.
\item Set $\delta = - 6.683/1200$ to reproduce the mean short rate.
\item Set $ \sigma^2 = (2.703/1200)^2 (1-\varphi^2)$.
\end{itemize}
All of this is similar to what we've done before.
Now add
\begin{itemize}
\item Set $\lambda_1=-63.5$ to reproduce the regression coefficient $b_1$.
If you plot the forward rate curve you'll see it has little effect on it.
\item Now set $\lambda_0$ to hit the long forward rate.  I get
$\lambda_0 = 0.234$.
You'll notice there's more curvature to the mean forward rate curve,
something we might comment on in class.
\end{itemize}


\section*{Bottom line}

We now know that risk premiums differ across asset classes
and time.
The evidence for this is clear, the reasons for it less so.
If you trust the evidence, you can start your own hedge fund.
Or you can write a research paper trying to explain it.
Either way, good luck!


\section*{Practice problems}


\begin{enumerate}
%-----------------------------------------------------------------------
\item  {\it Expected returns on bonds.\/}
Consider the bond pricing model
\begin{eqnarray*}
    \log m_{t+1} &=& - (\lambda_0 + \lambda_1 w_t)^2/2 + \delta - x_t
            + (\lambda_0 + \lambda_1 w_t) w_{t+1} \\
            x_{t} &=& \delta + \sigma (w_t + \theta w_{t-1}) ,
\end{eqnarray*}
where the $w_t$'s are independent standard normal random variables.
%
\begin{enumerate}
\item Suppose bond prices take the form
\begin{eqnarray*}
    \log q^n_t &=& A_n + B_n w_t + C_n w_{t-1} .
\end{eqnarray*}
How is $(A_{n+1},B_{n+1},C_{n+1})$ connected to $(A_{n},B_{n},C_{n})$?
What are the values of $(A_{n},B_{n},C_{n})$ for $n=0,1,2$?
\item Express the expected log excess return on a two-period bond as a function
of the coefficients $(A_{n},B_{n},C_{n})$ for $n=1,2$.
Use your solution for the coefficients to describe how expected log excess returns vary with the
interest rate innovation $w_t$.
\end{enumerate}
%
\needspace{2\baselineskip}
Answer.
\begin{enumerate}
\item We use the usual formula,
$ q^{n+1}_t = E_t (m_{t+1} q^n_{t+1})$.
Then with substitutions,
\begin{eqnarray*}
    \log (m_{t+1} q^n_{t+1}) &=& A_n - \delta + (C_n-\sigma) w_t - \sigma\theta w_{t-1}
            - (\lambda_0 + \lambda_1 w_t)^2/2 \\
            && + \; [B_n + (\lambda_0 + \lambda_1 w_t)] w_{t+1} .
\end{eqnarray*}
The first line on the rhs gives us the conditional mean, the second line gives us the variance.
Some intensive algebra gives us
\begin{eqnarray*}
    A_{n+1} &=& A_n - \delta + B_n^2/2 + B_n \lambda_0 \\
    B_{n+1} &=& C_n - \sigma + B_n \lambda_1 \\
    C_{n+1} &=& - \sigma \theta .
\end{eqnarray*}
That gives us
\begin{center}
\begin{tabular}{cccc}
$n$ & $A_n$ & $B_n$ & $C_n$ \\
\midrule
0 & 0 & 0 & 0 \\
1 & $-\delta$ & $-\sigma$ & $-\sigma \theta$ \\
2 & $- 2\delta -\lambda_0 \sigma + \sigma^2/2$ & $- \sigma (1+\theta+\lambda_1)$ & $-\sigma\theta$
\end{tabular}
\end{center}

\item The log excess return is
\begin{eqnarray*}
    \log r^2_{t+1} - \log r^1_{t+1} &=& \log q^1_{t+1} - \log q^2_t + \log q^1_t \\
            &=& (2 A_1 - A_2) + (C_1 + B_1 - B_2) w_t + (C_1 - C_2) w_{t-1} \\
            &&        + \; B_1 w_{t+1} .
\end{eqnarray*}
The expected excess return knocks out the last term.

If we substitute our solutions, we have
\begin{eqnarray*}
   E_t \left( \log r^2_{t+1} - \log r^1_{t+1}\right)
            &=& (\sigma \lambda_0 + \sigma^2/2)  + (\sigma\lambda_1) w_t  .
\end{eqnarray*}
So the key parameter is $\lambda_1$:  the sensitivity of the ``price of risk''
to $w_t$.
\end{enumerate}

%-----------------------------------------------------------------------
\item  {\it Stochastic volatility and equity pricing.\/}
The Cox-Ingersoll-Ross model of bond pricing consists of the equations
\begin{eqnarray*}
    \log m_{t+1} &=& -(1+\lambda^2/2) z_{t} + \lambda z_t^{1/2} w_{t+1} \\
        z_{t+1}  &=& (1-\varphi)\delta + \varphi z_t + \sigma z_t^{1/2} w_{t+1} ,
\end{eqnarray*}
with the usual independent standard normal disturbances $w_t$.
We add to it an equation governing the dividend $d_t$ paid by a one-period
equity-like claim:
\begin{eqnarray*}
    \log d_{t+1} &=& \alpha + \beta z_t + \gamma z_t^{1/2} w_{t+1} .
\end{eqnarray*}
Here $z_t$ plays the role of the state:  if we know $z_t$, we know the conditional
distributions of $ (m_{t+1}, z_{t+1}, d_{t+1})$.

\begin{enumerate}
\item  Conditional on $z_t$,
what are the mean and variance of $\log m_{t+1}$?
What is its distribution?
\item  What is the price $q^1_t$ of a one-period bond?
What is its return $ r^1_{t+1} = 1/q^1_t$?
\item  What is the price $q^e_t$ of equity, a claim to the dividend
$d_{t+1}$?
What is its return $r^e_{t+1}$?
\item  Conditional on $z_t$, what is the expected log excess return on equity,
$E_t (\log r^e_{t+1} - \log r^1_{t+1})$?
\end{enumerate}
%
\needspace{2\baselineskip}
Answer.
\begin{enumerate}
\item  Conditional on $z_t$, $\log m_{t+1}$ is normal with mean and variance
\begin{eqnarray*}
    E_t (\log m_{t+1}) &=& -(1+\lambda^2/2) z_{t} \\
    \mbox{Var}_t (\log m_{t+1}) &=& \lambda^2 z_t .
\end{eqnarray*}

\item  The price uses the ``mean plus variance over two'' formula:
\begin{eqnarray*}
    \log q^1_t  &=& - z_{t} .
\end{eqnarray*}
The log return is $\log r^1_{t+1} = z_t$.

\item  The price of equity is
\begin{eqnarray*}
    \log q^e_t &=& \log E_t (m_{t+1} d_{t+1} ) \\
            &=& \alpha + [\beta - (1+\lambda^2/2)] z_t
                + [(\gamma+\lambda)^2/2 ]  z_t \\
            &=& \alpha + (\beta - 1 + \gamma^2/2 + \gamma\lambda) z_t .
\end{eqnarray*}
The return is
\begin{eqnarray*}
    \log r^e_{t+1} &=& \log d_{t+1} - \log q^e_t
            \;\;=\;\; (1 - \gamma^2/2 - \gamma\lambda) z_t
                + \gamma z_t^{1/2} w_{t+1} .
\end{eqnarray*}
\item  Conditional on $z_t$, the expected excess return is
\begin{eqnarray*}
    E_t (\log r^e_{t+1} - \log r^1_{t+1} )
            &=& - (\gamma^2/2 + \gamma\lambda) z_t .
\end{eqnarray*}
\end{enumerate}

%-----------------------------------------------------------------------
\item {\it Exponential-affine equity valuation.\/}
We explore variation in expected excess returns
(``risk premiums'') in an environment like the one we used for bond pricing.
We say a return or expected return is predictable if it depends on
the state.
If we know the state, we can use it to ``predict'' the return.
In the asset management business, you might imagine strategies to exploit such
information.
For example, you might invest more in assets when their expected returns are high,
and less when they're low.

Here's the model.  The pricing kernel is
\begin{eqnarray*}
    \log m_{t+1} &=& - (\lambda_0 + \lambda_1 z_t)^2/2 - z_t
                + (\lambda_0 + \lambda_1 z_t) w_{t+1} \\
        z_{t+1} &=& (1-\varphi) \delta + \varphi z_t + \sigma w_{t+1} ,
\end{eqnarray*}
where $\{ w_t \}$ is (as usual) a sequence of independent standard normal random variables.
This model differs from Vasicek in having a price of risk $\lambda$ that
depends on the state $z_t$.
The dependence is carefully designed to deliver loglinear (``exponential-affine'')
bond prices.
%
\begin{enumerate}
\item  What is the (stationary) mean of $z_t$?  The variance?
\item  What is the price $q^1_t$ of a one-period bond?
What is its (gross) return $r^1_{t+1}$?
\item  What is the mean log return $E (\log r^1_{t+1})$?
What parameter(s) govern its value in the model?
\item  Now consider the price of ``equity,'' a claim to the dividend
\begin{eqnarray*}
    \log d_{t+1} &=&  \alpha + \beta w_{t+1} .
\end{eqnarray*}
What is its price $q^e_t$?
Its (gross) return $r^e_{t+1}$?
\item  What is the expected excess log return $ E_t (\log r^e_{t+1} - \log r^1_{t+1})$ in state $z_t$?
In what states is it high?  In what states low?
\item  Which equation is responsible for the predictability of the expected return?
\end{enumerate}
%
\needspace{2\baselineskip}
Answer.
The point is that we can generate predictable excess
returns for equity just as we did for bonds.
\begin{enumerate}
\item  $z_t$ is an AR(1).
The mean is $\delta$ and the variance is $\sigma^2/(1-\varphi^2)$.
\item   The model is set up to deliver
\begin{eqnarray*}
    \log q^1_t &=& \log E_t (m_{t+1}) \;\;=\;\; - z_t \\
    \log r^1_{t+1} &=& - \log q^1_t \;\;=\;\; z_t .
\end{eqnarray*}
\item  The mean log return is the mean of $z_t$, namely $\delta$.
\item  The price is
\begin{eqnarray*}
    \log q^e_t &=& \log E_t (m_{t+1} d_{t+1})
            \;\;=\;\; \alpha - z_t + \beta^2/2 + \beta ( \lambda_0+ \lambda_1 z_t) .
\end{eqnarray*}
The return is
\begin{eqnarray*}
    \log r^e_{t+1} &=& \log d_{t+1} - \log q^e_t
            \;\;=\;\;  z_t - \beta^2/2  -
                \beta (\lambda_0 + \lambda_1 z_t) + \beta w_{t+1}.
\end{eqnarray*}
\item  The expected excess return is
\begin{eqnarray*}
   E_t ( \log r^e_{t+1}- \log r^1_{t+1}) &=&
             - \beta^2/2  -  \beta (\lambda_0 + \lambda_1 z_t)
\end{eqnarray*}
The second term is predictable through $z_t$.
\end{enumerate}

\end{enumerate}

{\vfill
{\bigskip \centerline{\it \copyright \ \number\year \
David Backus $|$ NYU Stern School of Business}%
}}



\end{document}

%
\begin{table}[h]
%\begin{center}
\centering
\caption{Properties of forward rates}
\begin{tabular}{lrrr}
\toprule
Forward rate $f^n_t$    &  Mean  &  Std Dev  &  Autocorr \\
\midrule
$f^0_t$                 &  6.683 & 2.703     &  0.959   \\
$f^{12}_t$              &  7.921 & 2.495     &  0.969   \\
$f^{120}_t$             &  8.858 & 1.946     &  0.980   \\
\bottomrule
\end{tabular}
\label{tab:forward-moments}
\end{table}
%

