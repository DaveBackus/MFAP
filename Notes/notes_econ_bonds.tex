\documentclass[11pt]{article}

% spacing on page
\oddsidemargin=0.25truein \evensidemargin=0.25truein
\topmargin=-0.5truein \textwidth=6.0truein \textheight=8.75truein

\usepackage{comment}
\usepackage{graphicx}
\usepackage{amssymb}
\usepackage{amsmath}

\usepackage[margin=0pt, labelsep=period, font=large, labelfont=bf]{caption}
%\usepackage{float}

\usepackage{hyperref}
\urlstyle{rm}   % change fonts for url's (from Chad Jones)
\hypersetup{
    colorlinks=true,        % kills boxes
%    allcolors=blue,
    pdfsubject={ECON-UB233, Macroeconomic foundations for asset pricing},
    pdfauthor={Dave Backus @ NYU},
    pdfstartview={FitH},
    pdfpagemode={UseNone},
%    pdfnewwindow=true,      % links in new window
%    linkcolor=blue,         % color of internal links
%    citecolor=blue,         % color of links to bibliography
%    filecolor=blue,         % color of file links
%    urlcolor=blue           % color of external links
% see:  http://www.tug.org/applications/hyperref/manual.html
}

% for listing code in tt font
\usepackage{verbatim}

% for table spacing
\usepackage{booktabs}

% section headers and spacing
\usepackage[small, compact]{titlesec}

% list spacing
\usepackage{enumitem}
\setitemize{leftmargin=*, topsep=0pt}
\setenumerate{leftmargin=*, topsep=0pt}

% attach files to the pdf
\usepackage{attachfile}
    \attachfilesetup{color=0.75 0 0.75}

\usepackage{needspace}
% \needspace{4\baselineskip} makes sure we have four lines available before a pagebreak

%\renewcommand{\thefootnote}{\fnsymbol{footnote}}
%\newcommand{\var}{\mbox{\it Var\/}}

\newcommand{\cbar}{\bar{c}}
\newcommand{\rp}{\mbox{\em rp\/}}
\newcommand{\tp}{\mbox{\em tp\/}}
\newcommand{\bp}{\mbox{\em bp\/}}


\begin{document}
\parskip=\bigskipamount
\parindent=0.0in
\thispagestyle{empty}
{\large ECON-UB 233 \hfill Dave Backus @ NYU}% 

\bigskip\bigskip
\centerline{\Large \bf Models of Bond Prices}
\centerline{Revised: \today}

\begin{comment}
This is clunky...

Bodie Kane and Marcus as example?
\end{comment}

\bigskip
We define bond prices and related objects
and describe their connection with the pricing kernel in arbitrage-free economies.
Roughly speaking, the dynamics of the pricing kernel are reflected in the
slope and dynamics of the yield curve.

Formally, we apply the no-arbitrage theorem in a dynamic setting.
Consider a Markov environment with state variable $z_t$.
The no-arbitrage theorem might be stated:
there exists a positive pricing kernel $m$ that satisfies
\begin{eqnarray}
    E_t \big[ m(z_t, z_{t+1}) r(z_t, z_{t+1})^j \big] &=&
    E_t \big( m_{t+1} r_{t+1}^j \big) \;\;=\;\; 1
    \label{eq:E(mr)=1}
\end{eqnarray}
for gross returns $r^j$ on all assets $j$.
The middle expression is shorthand for the one on the left.
It's a little sloppy, but we'll use it out of laziness.

We show how this works for bonds of different maturities
and use observed properties interest rates
(means, standard deviations, autocorrelations)
to infer parameter values for some popular models.
Our examples are loglinear; they're what have come to be
called {\it exponential-affine\/} models.
This has the great advantage of tractability:
the solutions are generally clear enough that we can make some
sense of them.
%The convenience of linearity should be clear from our earlier work.
%The pricing kernel is, so to speak, reverse engineered
%from bond prices rather than built up from (say)
%the marginal rate of substitution of a representative agent.
%We leave for another time (never?) the question of how to reconcile
%the two approaches.


\section{Definitions:  bond prices, yields, and forward rates}


We'll work, by tradition, with zero-coupon bonds:
claims at date $t$ to ``one'' (one dollar, typically) at date $t+n$.
We denote the price of such a bond by $q^n_t$, a generalization
of the $q^1_t$ we've used since the start of the course.

{\it Yields and forward rates.\/}
We typically report prices as yields.
Like implied volatilities, they represent prices in a more user-friendly form.
But even a casual look at yields gives us a range of options.
Some bonds compute compound yields annually, some semi-annually,
and some monthly (mortgages, for example).
We'll use ``continuous'' compounding, but any such choice is a matter of convenience, not substance.

Consider the compounding of interest over one period.
If we compound $k$ times, the {\it yield\/} $y_t$ is the solution to
\begin{eqnarray*}
    q_t &=& \frac{1} {(1+ y_t/k)^k}
\end{eqnarray*}
for the given price $q_t$.
Evidently different choices of $k$ give us different solutions $y_t$.
They're all fine, they're just based on different compounding conventions.
Continuous compounding refers to the limit as $k$ approaches infinity.
Taking logs and substituting $ h = 1/k$ gives us
\begin{eqnarray*}
    \log q_t &=& - \frac {\log (1+ h y_t)} {h} .
\end{eqnarray*}
The limit as $h$ goes to zero (and $k$ goes to infinity) is (via l'Hopital's rule)
$ \log q_t = - y_t $, or $q_t = \exp(-y_t)$.
The convenience here is the loglinearity of the yield in terms of the bond price.

That leads us to continuously-compounded yields.
We define the continuously-compounded yield over $n$ periods by
\begin{eqnarray}
    q^n_t &=&  \exp(- n y^n_t)
        \;\;\Leftrightarrow\;\;
        y^n_t \;\;=\;\; - n^{-1} \log q^n_t .
        \label{eq:yields}
\end{eqnarray}
It's the average rate of discount over the interval $[t, t+n]$.


Sometimes it's more useful to apply a different
rate to each period:
\begin{eqnarray}
    q^n_t &=&  \exp[- (f^0_t + f^1_t + \cdots + f^{n-1}_t)]
        \;\;\Leftrightarrow\;\;
        f^n_t \;\;=\;\;  \log q^n_t - \log q^{n+1}_t .
        \label{eq:forwards}
\end{eqnarray}
We refer to $f^j_t$ as the $j$-period ahead {\it forward rate\/}.
Together we have
\begin{eqnarray*}
    y^n_t &=&  n^{-1} \left( f^0_t + f^1_t + \cdots + f^{n-1}_t \right).
\end{eqnarray*}
In words:  yields are averages of forward rates.
Yields and forwards are analogous to average and marginal cost in microeconomics.
We'll refer to $f^0_t = y^1_t$ as the {\it short rate\/}.

[Plot $-\log b^n_t$ versus maturity $n$.  Show yields and forwards.]

[Draw time line, show what periods yields and forwards apply to.]


{\it Digression on forward contracts.\/}
Another approach to forward rates is via forward contracts.
All of our assets to date have been purchases at some date $t$
of claims to cash flows at futures dates $t+n$.
A forward contract involves the agreement to purchase an asset in the future
at a price agreed on now.
Consider an agreement at date $t$  to buy a one-period bond at date $t+n$.
What should the price be?
The agreement involves the purchase at $t+n$ of a one-period bond at the forward price of (let us say) $g^n_t$.
The subscript $t$ here is the date of agreement, not the date of execution.
The one-period bond delivers one (dollar) at $t+n+1$.

We can replicate these cash flows with long and short positions in bonds.
If we purchase an $(n+1)$-period bond at $t$, it generates a cash flow of one at $t+n+1$,
but costs $q^{n+1}_t$ at $t$.
We can offset the initial cost if we sell the right amount of $n$-period bonds,
namely $ q^{n+1}_t/q^n_t$ of these bonds.  [Ask yourself why.]
That in turn calls for the payment of $ q^{n+1}_t/q^n_t$ (times one) at $t+n$.
But that's the same as the price of the forward contract.
Evidently the price of the forward contract is
\begin{eqnarray*}
    g^n_t &=&  q^{n+1}_t/q^n_t .
\end{eqnarray*}
The continuously-compounded interest rate on this contract --- the forward rate ---
is $ f^n_t = - \log g^n_t$.
Which is where we started.


Evidently we can report bond prices $q^n_t$,
yields $y^n_t$, or forward rates $f^n_t$.
In each case we have one number for each maturity.
Once we know of these sequences, we can compute the others.
Forwards are mathematically simpler, as we'll see, but yields are more common.

One last thing:  bond prices and returns have units.
Typically the units are dollars (or other currency):
bonds are claims to dollars, prices are measured in dollars,
and returns are dollars later per dollar now.
We could use other units, but those are the standard one.


\section{Pricing kernels and bond prices}

The simplicity and elegance of bond pricing comes
from the lack of an uncertain dividend.
Since the cash flows are known ---
that's why we refer to bonds and related securities as ``fixed income'' ---
prices depend only on the pricing kernel.
To make the point absolutely clear:  the pricing kernel alone determines
bond prices.  Period.


We value assets as we did before:  using the no-arbitrage theorem.
In a dynamic setting, this is equation (\ref{eq:E(mr)=1}).
The difference with our earlier work is that $m$ is now a stochastic process.

[Note units:  $m$ has inverse units to $r$.
If $r_{t+1}$ is dollars tomorrow for dollars today,
then $m_{t+1}$ has units of 1/dollars tomorrow
for 1/dollars today.
That way the units cancel and we're left with the unit-free ``1'' on the rhs.]

What are the {\it bond returns\/} we plug into this pricing relation?
The relevant returns are sometimes called {\it holding period returns\/}
for a holding period of one.
If we buy an $(n+1)$-period bond at $t$ and hold it for one period,
we're left with an $n$-period bond at $t+1$.
Our (gross) return is
\begin{eqnarray}
        r^{n+1}_{t+1} &=&  q^n_{t+1}/q^{n+1}_t .
        \label{eq:bond-returns}
\end{eqnarray}
Same object for bonds as the returns we've computed on
other assets.


Now put this to work.  If we substitute the return (\ref{eq:bond-returns})
into the pricing relation (\ref{eq:E(mr)=1}) we have
\begin{eqnarray}
    q^{n+1}_t &=& E_t \left( m_{t+1} q^n_{t+1} \right) .
    \label{eq:bond-recursion}
\end{eqnarray}
So if we know the price of an $n$-period bond,
we can compute the price of an $(n+1)$-period bond.
We do this recursively starting with $q^0_t = 1$ (one now costs one).
The first one is what we've been doing all along:
$ q^1_t = E_t (m_{t+1} q^0_{t+1}) = E_t (m_{t+1})$.
Long-maturity bonds follow from the same logic repeated:
given the price of an $n$-period bond,
use (\ref{eq:bond-recursion}) to find the price of an $n+1$-period bond.

%Optional digression.
We've done this recursively, but you can imagine doing it ``all at once.''
The price of a two-period bond, for example, can be written
\begin{eqnarray*}
    q^{2}_t &=& E_t \left( m_{t+1} q^1_{t+1} \right) \\
            &=& E_t \left( m_{t+1} E_{t+1} [m_{t+2}] \right) \\
            &=& E_t \left( m_{t+1} m_{t+2} \right) .
\end{eqnarray*}
The last line follows from the law of iterated expectations.
Repeating the same logic, we find
\begin{eqnarray*}
    q^{n}_t  &=& E_t \left( m_{t+1} m_{t+2} \cdots m_{t+n} \right) .
\end{eqnarray*}
If this seems overly mysterious, just leave it alone.


Our focus will be on the dynamics of $m$.
To make the point that dynamics are needed to produce realistic
bond prices, consider the case of an iid pricing kernel.
The one-period bond price is, as we've seen,
\begin{eqnarray*}
    q^{1}_t &=& E_t \left( m_{t+1} \right) .
\end{eqnarray*}
Since $m$ is iid, this is just a number $q^1$,
the same in all date-$t$ states.
%
Now let's do the same thing again.
Since $q^1 $ is a number, we have
\begin{eqnarray*}
    q^{2}_t &=& E_t \left( m_{t+1} q^1 \right)
            \;\;=\;\;  q^1 E_t \left( m_{t+1}  \right)
            \;\;=\;\;  (q^1)^2 .
\end{eqnarray*}
Using the same logic over and over again gives us
\begin{eqnarray*}
    q^{n} &=& (q^1)^n .
\end{eqnarray*}
If we apply their definitions,
we would see that bond yields and forward rates are the same across maturities
and constant over time.
Any way you look at it, there's no interest rate risk if $m$ is iid.

The conclusion, evidently, is that we need dynamics in $m$
to generate reasonable bond prices.
But what kind of dynamics?
We'll start simple and add complexity when called for.


\section{Markov models}

Let's start with a Markov model.
A state variable $z_t$ evolves according to
some well-behaved stochastic process.
The pricing kernel between $t$ and $t+1$ is a function of both
states, $m (z_t, z_{t+1})$.

What about bond prices?
The one-period bond price is evidently a function of the current state:
\begin{eqnarray*}
    q^1 (z_t) &=& E_t \big[ m(z_t, z_{t+1}) \big] .
\end{eqnarray*}
Why doesn't it depend on $z_{t+1}$?  Because we've taken the expectation
over $z_{t+1}$.
Similar logic applies to bonds of longer maturity:
\begin{eqnarray*}
    q^{n+1} (z_t) &=& E_t \big[ m(z_t, z_{t+1}) q^{n} (z_{t+1}) \big] .
\end{eqnarray*}
We're looking, therefore, for the functions $q^n(z_t)$.
Once we have them, we can compute yields and forward rates, which are also
functions of the state.
The iid case corresponds to a pricing kernel that depends on the future state only
and bond prices that are constant.


\section{The Vasicek model}

\begin{comment}
In the previous section, we found bond prices and forward rates
from the pricing kernel.
Here we do that, but we also do the reverse:
We use what we know about forward rates
to estimate the parameters of the pricing kernel.
So what do we know?
We have both time-series and cross-section information.
Time-series information includes the autocorrelations
of interest rates.
Cross-section information includes the mean forward rate curve:
that long rates are on average greater than short rates.
These seem like basic features of interest rates that
we would like any reasonable model to reproduce.
\end{comment}


The Vasicek model has a convenient loglinear structure,
but it's a good illustration of how bond-pricing models work.
We describe the model and go on to connect its
parameters to what we know about the behavior of interest rates.
The model delivers bond prices whose logs are linear in the state variable $z_t$,
which means yields and forward rates are linear in the state as well.


{\it Model.\/}
The Vasicek model has two components.
One is an autoregressive state,
\begin{eqnarray*}
    z_{t+1} &=& \varphi z_t + \sigma w_{t+1} ,
\end{eqnarray*}
where $\{ w_t \}$ is the usual collection of independent
standard normals.
The other is a loglinear pricing kernel,
\begin{eqnarray*}
    \log m(z_t, z_{t+1})  &=& \delta + a z_t + b z_{t+1} .
\end{eqnarray*}
The iid case corresponds to $a = 0$.
%
In practice, we often substitute to get
\begin{eqnarray*}
    \log m(z_t, z_{t+1})  &=& \delta + (a+b\varphi) z_t + b\sigma w_{t+1} \\
        &=& \delta + z_t + \lambda w_{t+1} ,
\end{eqnarray*}
which includes the normalization $a + b \varphi = 1$ and the definition
$\lambda = b \sigma$.


We'll see later that $w_t$ reflects interest rate risk.
The coefficient $\lambda$ in the pricing kernel equation is often referred to as
the {\it price of risk\/} --- here, evidently, the price of interest rate risk.
It tells us how much weight the pricing kernel gives to the risk $w_t$.
The price of risk is a slippery concept, but we'll use the term anyway.

One more thing:
If we substitute for $x_t$, we see that $\log m_t$ is
\begin{eqnarray}
    \log m_t &=& \delta + \lambda w_t - \sigma \sum_{j=1}^\infty \varphi^{j-1} w_{t-j} .
    \label{eq:vasicek-ma}
\end{eqnarray}
Since the coefficients decline at a geometric rate after the first one,
this is an ARMA(1,1).
[Think about this if it isn't clear.]

{\it Recursions.\/}
We build up bond prices one at a time from
the pricing relation
\begin{eqnarray*}
    q^{n+1}_t &=& E_t \left( m_{t+1} q^n_{t+1} \right) ,
\end{eqnarray*}
starting with $q^0_t = 1$ (a dollar now is worth a dollar).
Let's get warmed up with the short rate $f^0_t$, corresponding
to $n=0$:
\begin{eqnarray*}
    q^1_t   &=&  E_t (m_{t+1}) \\
    - f^0_t &=&  \log q^1_{t} \;\;=\;\; \log E_t (m_{t+1}) .
\end{eqnarray*}
We'll solve this with the usual ``mean plus variance over two''
formula for lognormal random variables.
As of date $t$,
the conditional distribution of $\log m_{t+1}$ is
normal with mean and variance
\begin{eqnarray*}
    E_t \left( \log m_{t+1}\right) &=& \delta + z_t \\
    \mbox{Var}_t \left( \log m_{t+1} \right) &=& \lambda^2 .
\end{eqnarray*}
The short rate is therefore
$ f^0_t =  - (\delta + \lambda^2/2) - z_t$,
so it inherits the variance and autocorrelation of $z_t$.
The disturbance $w_t$ therefore represents interest rate risk:
an increase in $w_t$ corresponds to an increase in the short rate
of $\sigma w_t$.

For bonds of higher maturity, we
guess prices of ``exponential-affine'' form
\begin{eqnarray*}
    \log q^n_t &=& A_n + B_n x_t
\end{eqnarray*}
for some coefficients $\{ A_n, B_n\}$ to be determined.
%(And no, $A_n$ isn't the cumulative sum as it was previously,
%it's a stand-alone coefficient.)
Think about this for some particular maturity $n$.
The pricing relation tells us we
need to find the conditional mean of the product
$m_{t+1} q^n_{t+1}$.
We do this the usual way:  take logs,
then apply the ``mean plus variance over two'' formula for
lognormal random variables.
Taking logs gives us
\begin{eqnarray*}
    \log m_{t+1} + \log q^{n+1}_t &=&
            (\delta + A_n) + (\varphi B_n +1) z_t + (\lambda + B_n \sigma) w_{t+1} .
\end{eqnarray*}
The conditional mean and variance are
\begin{eqnarray*}
   E_t \big( \log m_{t+1} + \log q^{n}_{t+1} \big) &=&
            (\delta + A_n) + (\varphi B_n + 1) z_t \\
   \mbox{Var}_t \big( \log m_{t+1} + \log q^{n+1}_t \big) &=&
            (\lambda + B_n \sigma)^2 .
\end{eqnarray*}
Now apply the formula:
\begin{eqnarray*}
    \log q^{n+1}_t &=&
            (\delta + A_n) + (\varphi B_n + 1) z_t + (\lambda + B_n \sigma)^2/2 \\
                &=& A_{n+1} + B_{n+1} z_t.
\end{eqnarray*}
Lining up terms, we have
\begin{eqnarray*}
    A_{n+1} &=& A_n + \delta + (\lambda + B_n\sigma)^2/2 \\
    B_{n+1} &=& \varphi B_n + 1 .
\end{eqnarray*}
Given values for the parameters $(\delta, \varphi, \sigma, \lambda)$
we compute these one at a time,
starting with $A_0 = B_0 = 0$.
It's easy to do this in (say) Matlab, or even a spreadsheet program.

{\it Parameters.\/}
We now have bond prices and forward rates as functions of the state.
Here we use what we know about forward rates
to estimate the model's parameters.
So what do we know?
We have both time-series and cross-section information.
Time-series information includes the autocorrelations
of interest rates.
Cross-section information includes the mean forward rate curve:
that long rates are on average greater than short rates.
These seem like basic features of interest rates that
we would like any reasonable model to reproduce.

Let's choose parameters that match some of the observed
properties of forward rates.
Some ballpark numbers are reported in Table \ref{tab:forward-moments}.
The numbers are  computed from monthly data for US Treasuries
over the period 1970-1992.
They're reported as annual
percentages, something that will show up in our calculations.
Annual means we multiplied them by 12, and percentage means we multiplied them by 100,
so we multiplied them by 1200 altogether.

%
\begin{table}[h]
%\begin{center}
\centering
\caption{Properties of forward rates}
\begin{tabular}{lrrr}
\toprule
Forward rate $f^n_t$    &  Mean  &  Std Dev  &  Autocorr \\
\midrule
$f^0_t$                 &  6.683 & 2.703     &  0.959   \\
$f^{12}_t$              &  7.921 & 2.495     &  0.969   \\
$f^{120}_t$             &  8.858 & 1.946     &  0.980   \\
\bottomrule
\end{tabular}
\label{tab:forward-moments}
\end{table}
%

Now let's put these numbers to work.
The time interval for our analysis will be one month,
so the short rate is the one-month rate --- $f^0$ in the table.
%
\begin{itemize}
\item Autocorrelation.  Because $f^0_t$ is linear in $z_t$, and $z_t$ is an AR(1),
the autocorrelation of the short rate is $\varphi$.
So we set $\varphi$ equal to the observed autocorrelation of the short rate in the data.
From the table, we have $\varphi = 0.959$.
[You might think to yourself:  What about the autocorrelations of other rates?]

\item Variance.  Similarly, the variance of the short rate is the variance of $z$,
which is $ \sigma^2/(1-\varphi^2)$.
Using data from the same table gives us
\begin{eqnarray*}
    \mbox{Var}(f^0) &=& \left( 2.703/1200 \right)^2
            \;\;=\;\; \sigma^2/(1-\varphi^2)
            \;\;\;\Rightarrow\;\;\; \sigma = 6.38 \times 10^{-4} .
\end{eqnarray*}
Note that we have converted the numbers in the table from annual percentages
to monthly interest.

\item Long forward rates.  Next we have $\lambda$, which controls the pricing of
interest rate risk and therefore the slope of the forward rate curve.
Recall that forward rates are
\begin{eqnarray*}
    f^n_t &=& \log (q^n_t/q^{n+1}_t)
                \;\;=\;\;  (A_n - A_{n+1}) + (B_n - B_{n+1}) z_t .
\end{eqnarray*}
Since $z_t$ has a zero mean,  we have
\begin{eqnarray*}
    E (f^n - f^0) &=&  (A_n - A_{n+1}) - (A_0 - A_{1}) \\
            &=&  \left[ \lambda^2 - (\lambda + B_n\sigma)^2 \right] / 2 .
\end{eqnarray*}
We solve this numerically.
We choose values for $(\varphi, \sigma, \lambda)$,
compute the $A_n$'s and $B_n$'s, and vary $\lambda$ until
we hit the observed mean forward spread:
$(8.858-6.683)/1200$.
The value that works is $\lambda = -0.125$.

\item Mean short rate.  This doesn't really matter, but
the mean short rate is
\begin{eqnarray*}
    E (f^0) &=&  - (\delta + \lambda^2/2) .
\end{eqnarray*}
Given our value of $\lambda$, we choose $\delta$ to match
the observed mean short rate.

\end{itemize}
If any of this seems mysterious, see the Matlab program.
%{\tt affine.m}.

Let's go back to the dynamics of the log pricing kernel, 
specifically the moving average representation (\ref{eq:vasicek-ma}).  
If $\sigma = 0$ the model is iid, which we've seen means a 
constant and flat forward rate curve. 
How does our estimated version differ from this?  
If you plot the moving average coefficients, 
you'll see that there's a big one ($\lambda$) at the start, 
then small ones of the opposite sign afterwards.  
The small ones represent variation in the short rate 
(also other rates, but that's enough for now).  
Their rate of decay represents the short rate's autocorrelation.  
The initial coefficient controls the slope of the mean forward rate
curve.  


{\it Alternative parameterization (optional).\/}
Suppose we change the model (slightly) to
\begin{eqnarray*}
    \log m_{t+1} &=& - \lambda^2/2 - z_t + \lambda w_{t+1} \\
         z_{t+1} &=& (1-\varphi) \delta + \varphi z_t + \sigma w_{t+1} .
\end{eqnarray*}
Show that the short rate is now exactly $z_t$.
The model is equivalent to the previous one,
we've simply moved the mean to a different part of the model.


\section{The Cox-Ingersoll-Ross model}

Another popular model does generate excess returns that vary with the state.
Let's see how it works.

{\it Model.\/}
Here's a popular variant of the Vasicek model
in which risk and risk premiums move around.
More on that shortly.
The model is
\begin{eqnarray}
    \log m_{t+1} &=& - (1+\lambda^2/2 ) z_t + \lambda z_t^{1/2} w_{t+1} \nonumber \\
         z_{t+1} &=& (1-\varphi) \delta + \varphi z_t + \sigma z_t^{1/2} w_{t+1} ,
         \label{eq:square-root}
\end{eqnarray}
with (as usual) independent standard normal innovations $w_t$.
Here $z_t$ plays two roles:  It governs the conditional mean
of the pricing kernel, as it did before,
but it also governs the conditional variance.

The so-called ``square-root'' process (\ref{eq:square-root}) guarantees that
it remains positive --- at least it does in continuous time.
The idea is that the variance shrinks as $z_t$ gets close to zero.
As a result, the change in $z_t$ near zero comes from the mean reversion term,
\begin{eqnarray*}
       z_{t+1} - z_t  &\approx& (1-\varphi) \delta  \;\;>\;\; 0.
\end{eqnarray*}
As long as $\delta>0$, that pushes $z_t$ away from zero and keeps it positive.


Despite the square-root process, $z_t$ is still an AR(1).
Its mean is
\begin{eqnarray*}
    E(z_t) &=& (1-\varphi) \delta + \varphi E(z_{t-1})
            \;\;\;\Rightarrow\;\;\; E(x) \;\;=\;\; \delta .
\end{eqnarray*}
The variance is
\begin{eqnarray*}
    \mbox{Var}(z) &=& E (z_t - \delta)^2
                \;\;=\;\; E \left[ \varphi (z_{t-1} - \delta) + \sigma z_{t-1}^{1/2} w_t \right]^2 \\
                &=& \varphi^2 \mbox{Var}(x) + E (\sigma^2 z_{t-1})
                \;\;=\;\;  \varphi^2 \mbox{Var}(z_{t}) + \sigma^2 \delta ,
\end{eqnarray*}
or
\begin{eqnarray*}
    \mbox{Var}(z) &=&   \sigma^2 \delta / (1-\varphi^2) .
\end{eqnarray*}
What about the autocovariance?
The first autocovariance is
\begin{eqnarray*}
    \mbox{Cov}(z_t,z_{t-1}) &=& E [(z_t - \delta)(z_{t-1} - \delta)]  \\
                &=& E  \left\{ [ \varphi (z_{t-1} - \delta) + \sigma z_{t-1}^{1/2} w_t ]
                (z_{t-1} - \delta) \right\} \\
                &=& \varphi \mbox{Var}(z) .
\end{eqnarray*}
The first autocorrelation is therefore $\varphi$.
[For practice, show that the $k$th autocorrelation is $\varphi^k$, \
the same as an AR(1) with iid disturbances.]

{\it Recursions.\/}
We follow the same logic as in the Vasicek model,
starting with the short rate
\begin{eqnarray*}
    - f^0_t &=&   \log E_t (m_{t+1} ).
\end{eqnarray*}
The conditional distribution of $\log m_{t+1}$ is
normal with mean and variance
\begin{eqnarray*}
    E_t (m_{t+1}) &=& - (1+\lambda^2/2) z_t \\
    \mbox{Var}_t (m_{t+1}) &=& \lambda^2 z_t .
\end{eqnarray*}
The new ingredient is that the variance is now linear in $z_t$.
The short rate is therefore
$ f^0_t = z_t $.
That's why we started with a somewhat odd coefficient of $z_t$
in the pricing kernel --- so this would work out.

For bonds of arbitrary maturity,
we use the guess  $ \log q^n_t = A_n + B_n z_t$.
That gives us
\begin{eqnarray*}
    \log m_{t+1} + \log q^{n+1}_t &=&
            [A_n + (1-\varphi) \delta] + [\varphi B_n -(1+\lambda^2/2)] z_t
                    + (\lambda + B_n \sigma) z_t^{1/2} w_{t+1} .
\end{eqnarray*}
The conditional mean and variance are
\begin{eqnarray*}
   E_t \left( \log m_{t+1} + \log q^{n}_{t+1} \right) &=&
            [A_n + B_n (1-\varphi) \delta] + [\varphi B_n -(1+\lambda^2/2)] z_t \\
   \mbox{Var}_t \left( \log m_{t+1} + \log q^{n+1}_t \right) &=&
            (\lambda + B_n \sigma)^2 z_t .
\end{eqnarray*}
The ``mean plus variance over two'' formula gives us
\begin{eqnarray*}
    \log q^{n+1}_t &=&
            [A_n + B_n (1-\varphi) \delta] + [\varphi B_n -(1+\lambda^2/2)] z_t
                    + [(\lambda + B_n \sigma)^2/2]  z_t  \\
                &=& A_{n+1} + B_{n+1} z_t.
\end{eqnarray*}
Lining up terms, we have
\begin{eqnarray*}
    A_{n+1} &=& A_n + B_n (1-\varphi) \delta \\
    B_{n+1} &=& \varphi B_n -(1+\lambda^2/2) + (\lambda + B_n \sigma)^2/2 .
\end{eqnarray*}
This is a mess, but it's easy to program, which is what we'll do.
We start, as always, with $A_0 = B_0 = 0$.
[Remind yourself why.]

\begin{comment}
\begin{table}[h]
%\begin{center}
\centering
\caption{Properties of forward rates}
\begin{tabular}{lrrr}
\toprule
Forward rate $f^n_t$    &  Mean  &  Std Dev  &  Autocorr \\
\midrule
$f^0_t$                 &  6.683 & 2.703     &  0.959   \\
$f^{12}_t$              &  7.921 & 2.495     &  0.969   \\
$f^{120}_t$             &  8.858 & 1.946     &  0.980   \\
\bottomrule
\end{tabular}
\label{tab:forward-moments}
\end{table}
\end{comment}

{\it Estimation.\/}
Now let's estimate parameters from the properties of forward rates
reported in Table \ref{tab:forward-moments}.
%
\begin{itemize}
\item Autocorrelation.
We set $\varphi$ equal to the observed autocorrelation of the short rate in the data,
just as we did before.
From the table, that gives us $\varphi = 0.959$.

\item Mean.  We set $\delta$ equal to the mean short rate:
$\delta = 6.683/1200$.

\item Variance.  Similarly, the variance of the short rate is the variance of $z_t$,
which is $ \sigma^2 \delta/(1-\varphi^2)$.
Given values for $\varphi$ and $\delta$,
the variance implies $\sigma = 8.6\times 10^{-3}$.

\item Long forward rates.  Again,
$\lambda$ controls the mean forward rate curve.
Recall that forward rates are
\begin{eqnarray*}
    f^n_t &=& \log (q^n_t/q^{n+1}_t)
                \;\;=\;\;  (A_n - A_{n+1}) + (B_n - B_{n+1}) z_t .
\end{eqnarray*}
Since $z_t$ has a  mean of $\delta$,  we have
\begin{eqnarray*}
    E (f^n - f^0) &=&  [(A_n - A_{n+1}) - (A_0 - A_{1})] +
        [(B_n - B_{n+1}) - (B_0 - B_{1})] \delta.
\end{eqnarray*}
We set $\lambda=1.32$, which gives us the right mean forward spread.

\end{itemize}
See the Matlab program for details.



\section{Interest rate risk and duration}

We can use the Vasicek model as a laboratory for thinking about interest rate risk.
If we were in a world where the model generated interest rates,
how should we think about interest rate risk?
Risk in this model stems entirely from variation in the state variable $z_t$.
The log bond price is a linear function of $z_t$:
\begin{eqnarray*}
    \log q^n_t &=& A_n + B_n z_t .
\end{eqnarray*}
Variation in bond prices is driven by the state $z_t$, period.
Since $z_t$ is (except for a constant) the short rate,
the coefficient $B_n$ summarizes sensitivity to interest rate risk.
We'll use the alternative parameterization so that movements in $z$
correspond to movements in the short rate.

We can be more specific if we look at $B_n$.
The recursion $B_{n+1} = \varphi B_n + 1$ starts with $B_0 = 0$,
which gives us
\begin{eqnarray*}
    B_n &=& (1 + \varphi + \varphi^2 + \cdots + \varphi^{n-1})
            \;\;=\;\; (1-\varphi^n)/(1-\varphi) .
\end{eqnarray*}
This increases with $n$, so long bonds are more sensitive to interest rate risk.
The yield is $y^n_t = -n^{-1} \log q^n_t = - A_n/n - (B_n/n) z_t$.
If $0 < \varphi < 1$,
$B_n/n$ will decline with maturity.

Let's compare this to {\it duration\/}.
Prices of zeros are connected to yields by
\begin{eqnarray*}
    q^n_t &=& \exp( - n y^n_t ).
\end{eqnarray*}
Duration is defined by
\begin{eqnarray*}
    D &=& - \frac{\partial \log q^n_t}{\partial y^n_t} \;\;=\;\; n.
\end{eqnarray*}
This is particularly clean with continuous compounding,
but you get something similar with other kinds of compounding.

All of this is fine, but think about how duration is used.
We typically ignore the superscript $n$ in the yield and consider
an increase in yields at all maturities.
This is clear when we compute duration for coupon bonds from the
relation between the price and the yield to maturity.
In principle we should discount coupons with the rate appropriate
to the dates they're paid, but we use the yield to maturity
of the bond to discount all payments.
We then interpret $D$ as sensitivity to generalized interest rate risk.

In the real world, interest rates aren't locked together.
Ten-year interest rate risk isn't the same as one-year interest rate risk.
In the Vasicek model, there's only one kind of risk --- risk to $z_t$ ---
but sensitivity varies by maturity.
But note that sensitivity $B_n$ isn't proportional to maturity,
as it is with duration.
With mean reversion ($0 < \varphi < 1$), sensitivity $B_n$ increases more slowly than $n$.
We get $B_n = n$ only in the limiting case $\varphi = 1$.
Otherwise, mean reversion moderates the sensitivity of long rates to movements in
the short rate.
Where did we go wrong?
We really need sensitivity to $z_t$,
\begin{eqnarray*}
     - \frac{\partial \log q^n_t}{\partial y^n_t}
      \cdot \frac{\partial \log y^n_t}{\partial z_t} &=& n \cdot (B_n/n) \;\;=\;\; B_n.
\end{eqnarray*}
This is an indication that duration isn't really the right tool for the job.
It's a shortcut that has been used for decades,
but it's not really a measure of interest rate sensitivity.

A problem to think about:  Do the same analysis for the Cox-Ingersoll-Ross model.
The coefficient $B_n$ doesn't have a simple solution, but you can do the same thing
numerically.

\begin{comment}
\section{The theory of long rates}
****** ???? *****
Perron-Frobenius stuff ???
\end{comment}


\section*{Bottom line}

Bond pricing in Markov environments is a classic application of dynamic asset pricing theory.
We get to use what we learned about stochastic processes to characterize
bond prices of all maturities as functions of the current state.
These prices follow, as always, from the no-arbitrage theorem.
The Vasicek model has a particularly user-friendly structure,
in which bond prices are loglinear functions of the state.
The latter is special, but illustrates how the dynamics of the state,
typically captured by the autoregressive parameter $\varphi$,
are reflected in long-maturity bond prices.
Mean reversion, for example, tends to reduce the sensitivity
of long bond prices to the state.
In this sense, the cross section of bond prices
incorporates the time series behavior of interest rates.

We can use similar methods to price other assets.
The new ingredient with other assets is cash flows,
which themselves might be modeled as functions of the state.


\section*{Practice problems}


\begin{enumerate}
\item {\it Bond basics.\/}
Consider the bond prices at some date $t$:
%
\begin{center}
\tabcolsep=0.1in
\begin{tabular}{lrrr}
\toprule
Maturity $n$   & \phantom{xx} Price $q^n_t$ \\ %& \phantom{xx} Yield $y^n$ & Forward $f^{n-1}$ \\
\midrule
0           &    1.0000   \\
1 year      &    0.9512   \\
2 years     &    0.8958   \\
3 years     &    0.8353   \\
4 years     &    0.7788   \\
5 years     &    0.7261   \\
\bottomrule
\end{tabular}
\end{center}
%
\begin{enumerate}
\item What are the yields $y^n_t$?
\item What are the forward rates $f^{n-1}_t$?
\item Suppose $f^1_t$ rises by 1 percent.
How does $y^1_t$ change?
How does $y^2_t$ change?
\end{enumerate}
%
\needspace{2\baselineskip}
Answer.
\begin{center}
\tabcolsep=0.1in
\begin{tabular}{lrrr}
\toprule
Maturity $n$   & \phantom{xx} Price $q^n_t$ & \phantom{xx} Yield $y^n_t$ & Forward $f^{n-1}_t$ \\
\midrule
0           &    1.0000  \\
1 year      &    0.9512  & 0.0500 & 0.0500 \\
2 years     &    0.8958  & 0.0550 & 0.0600 \\
3 years     &    0.8353  & 0.0600 & 0.0700 \\
4 years     &    0.7788  & 0.0625 & 0.0700 \\
5 years     &    0.7261  & 0.0640 & 0.0700 \\
\bottomrule
\end{tabular}
\end{center}
\medskip

\begin{enumerate}
\item See above.
\item See above.
\item If $f^1_t$ rises by 1\%, then $y^1_t= f^0_t$ doesn't change
and $y^2_t = (f^0_t+f^1_t)/2 $ rises by 0.5\%.
\end{enumerate}


%-----------------------------------------------------------------------
\item {\it Moving average bond pricing.\/}
Consider the bond pricing model
\begin{eqnarray*}
    \log m_{t+1} &=& - \lambda^2/2 - x_t + \lambda w_{t+1} \\
    x_t &=& \delta + \sigma (w_t + \theta w_{t-1}) .
\end{eqnarray*}
Here $x_t$ need not correspond to the state $z_t$.

\begin{enumerate}
\item What is the short rate $f^0_t$?
\item Suppose bond prices take the form
\begin{eqnarray*}
    \log q^n_{t} &=& A_n + B_n w_t + C_n w_{t-1} .
\end{eqnarray*}
Use the pricing relation (\ref{eq:E(mr)=1}) to derive recursions connecting
$(A_{n+1}, B_{n+1}, C_{n+1})$ to $(A_{n}, B_{n}, C_{n})$.
What are $(A_{n}, B_{n}, C_{n})$ for $n=0,1,2,3$?
\item Express forward rates as functions of the state $(w_t,w_{t-1}$).
What are $f^1_t$ and $f^2_t$?
\item What is $E(f^1- f^0)$?  What parameters govern its sign?
\end{enumerate}
%
\needspace{2\baselineskip}
Answer.
\begin{enumerate}
\item The short rate is
\begin{eqnarray*}
    f^0_t &=& - \log E_t m_{t+1} \;\;=\;\;  \lambda^2/2 + x_t - \lambda^2/2
            \;\;=\;\; x_t .
\end{eqnarray*}
The second equality is the usual ``mean plus variance over two'' with the sign flipped
(as indicated by the first equality).
In other words:  the usual setup.
In what follows, we'll kill off $x_t$ by substituting.
\item
Bond prices follow from the pricing relation,
\begin{eqnarray*}
    q_t^{n+1} &=& E_t (m_{t+1} q_{t+1}^n) ,
\end{eqnarray*}
starting with $n=0$ and  $q^0_t = 1$.
The state in this case is $(w_t, w_{t-1})$, a simple example of a
two-dimensional model, hence the extra term in the form of the bond price.
We need
\begin{eqnarray*}
    \log (m_{t+1} q_{t+1}^n) &=&
            A_n - (\lambda^2/2 + \delta) + (\lambda+B_n) w_{t+1}
            + (C_n-\sigma) w_t - \sigma \theta w_{t-1} .
\end{eqnarray*}
The (conditional) mean and variance are
\begin{eqnarray*}
    E_t [ \log (m_{t+1} q_{t+1}^n)] &=&
            A_n - (\lambda^2/2 + \delta)
            + (C_n-\sigma) w_t - \sigma \theta w_{t-1} \\
    \mbox{Var}_t [\log (m_{t+1} q_{t+1}^n)] &=&
            (\lambda+B_n)^2 .
\end{eqnarray*}
Using ``mean plus variance over two'' and lining up terms gives us
\begin{eqnarray*}
        A_{n+1} &=& A_n - (\lambda^2/2 + \delta) + (\lambda+B_n)^2/2 \\
                &=& A_n - \delta + \lambda B_n + (B_n)^2/2 \\
        B_{n+1} &=& C_n - \sigma \\
        C_{n+1} &=& - \sigma \theta
\end{eqnarray*}
for $n=0,1,2,\ldots$.
That gives us
\begin{center}
\begin{tabular}{cccc}
        $n$  & $A_n$ & $B_n$ & $C_n$ \\
        \midrule
        0   &  0 & 0 & 0 \\
        1   &  $-\delta$ & $-\sigma$ & $-\sigma\theta$  \\
        2   &  $-2 \delta - \lambda\sigma + \sigma^2/2$   &  $-\sigma(1+\theta)$ & $-\sigma\theta$ \\
        3   &  X  &  $-\sigma(1+\theta)$ & $-\sigma\theta$
\end{tabular}
\end{center}
with
$X = - 3 \delta - \lambda (2+\theta)+ [1 + (1+\theta)^2] \sigma^2/2 $.

\item In general, forward rates are
\begin{eqnarray*}
    f^n_t &=& (A_n - A_{n+1}) + (B_n - B_{n+1}) w_t + (C_n - C_{n+1}) w_{t-1} .
\end{eqnarray*}
That gives us
\begin{eqnarray*}
    f^0_t &=& \delta + \sigma w_t + \sigma \theta  w_{t-1} \\
    f^1_t &=& \delta + \lambda\sigma - \sigma^2/2 + \sigma \theta w_t  \\
    f^2_t &=& \delta - (1+\theta)^2 \sigma^2/2 + \lambda \sigma (1+\theta) .
\end{eqnarray*}

\item The means are the same with $w_t = w_{t-1} = 0$, their mean.
Therefore
\begin{eqnarray*}
   E (f^1 - f^0) &=& \lambda \sigma - \sigma^2/2 .
\end{eqnarray*}
Therefore we need $\lambda\sigma > \sigma^2/2 $,
so a necessary condition is that $\lambda$ and $\sigma$ have the same sign.

\end{enumerate}

\begin{comment}
%-----------------------------------------------------------------------
\item  {\it Expected returns on bonds.\/}
Consider the bond pricing model
\begin{eqnarray*}
    \log m_{t+1} &=& - (\lambda_0 + \lambda_1 w_t)^2/2 + \delta - x_t
            + (\lambda_0 + \lambda_1 w_t) w_{t+1} \\
            x_{t} &=& \delta + \sigma (w_t + \theta w_{t-1}) ,
\end{eqnarray*}
where the $w_t$'s are  independent standard normal random variables.
%
\begin{enumerate}
\item Suppose bond prices take the form
\begin{eqnarray*}
    \log q^n_t &=& A_n + B_n w_t + C_n w_{t-1} .
\end{eqnarray*}
How is $(A_{n+1},B_{n+1},C_{n+1})$ connected to $(A_{n},B_{n},C_{n})$?
What are the values of $(A_{n},B_{n},C_{n})$ for $n=0,1,2$?
\item Express the expected log excess return on a two-period bond as a function
of the coefficients $(A_{n},B_{n},C_{n})$ for $n=1,2$.
Use your solution for the coefficients to describe how expected log excess returns vary with the
interest rate innovation $w_t$.
\end{enumerate}
%
\needspace{2\baselineskip}
Answer.
\begin{enumerate}
\item We use the usual formula,
$ q^{n+1}_t = E_t (m_{t+1} q^n_{t+1})$.
Then with substitutions,
\begin{eqnarray*}
    \log (m_{t+1} q^n_{t+1}) &=& A_n - \delta + (C_n-\sigma) w_t - \sigma\theta w_{t-1}
            - (\lambda_0 + \lambda_1 w_t)^2/2 \\
            && + \; [B_n + (\lambda_0 + \lambda_1 w_t)] w_{t+1} .
\end{eqnarray*}
The first line on the rhs gives us the conditional mean, the second line gives us the variance.
Some intensive algebra gives us
\begin{eqnarray*}
    A_{n+1} &=& A_n - \delta + B_n^2/2 + B_n \lambda_0 \\
    B_{n+1} &=& C_n - \sigma + B_n \lambda_1 \\
    C_{n+1} &=& - \sigma \theta .
\end{eqnarray*}
That gives us
\begin{center}
\begin{tabular}{cccc}
$n$ & $A_n$ & $B_n$ & $C_n$ \\
\midrule
0 & 0 & 0 & 0 \\
1 & $-\delta$ & $-\sigma$ & $-\sigma \theta$ \\
2 & $- 2\delta -\lambda_0 \sigma + \sigma^2/2$ & $- \sigma (1+\theta+\lambda_1)$ & $-\sigma\theta$
\end{tabular}
\end{center}

\item The log excess return is
\begin{eqnarray*}
    \log r^2_{t+1} - \log r^1_{t+1} &=& \log q^1_{t+1} - \log q^2_t + \log q^1_t \\
            &=& (2 A_1 - A_2) + (C_1 + B_1 - B_2) w_t + (C_1 - C_2) w_{t-1} \\
            &&        + \; B_1 w_{t+1} .
\end{eqnarray*}
The expected excess return knocks out the last term.

If we substitute our solutions, we have
\begin{eqnarray*}
   E_t \left( \log r^2_{t+1} - \log r^1_{t+1}\right)
            &=& (\sigma \lambda_0 + \sigma^2/2)  + (\sigma\lambda_1) w_t  .
\end{eqnarray*}
So the key parameter is $\lambda_1$:  the sensitivity of the ``price of risk''
to $w_t$.
\end{enumerate}
\end{comment}

%\item {\it Bond pricing with nonnormal innovations.\/}
%????

\end{enumerate}


{\vfill
{\bigskip \centerline{\it \copyright \ \number\year \
David Backus $|$ NYU Stern School of Business}%
}}


\end{document}


