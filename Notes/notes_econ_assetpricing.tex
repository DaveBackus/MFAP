\documentclass[11pt]{article}

\oddsidemargin=0.25truein \evensidemargin=0.25truein
\topmargin=-0.5truein \textwidth=6.0truein \textheight=8.75truein

\usepackage{comment}
\usepackage{booktabs}
\usepackage{hyperref}
\urlstyle{rm}   % change fonts for url's (from Chad Jones)
\hypersetup{
    colorlinks=true,        % kills boxes
    allcolors=blue,
    pdfsubject={ECON-UB233, Macroeconomic foundations for asset pricing},
    pdfauthor={Dave Backus @ NYU},
    pdfstartview={FitH},
    pdfpagemode={UseNone},
%    pdfnewwindow=true,      % links in new window
%    linkcolor=blue,         % color of internal links
%    citecolor=blue,         % color of links to bibliography
%    filecolor=blue,         % color of file links
%    urlcolor=blue           % color of external links
% see:  http://www.tug.org/applications/hyperref/manual.html
}

\usepackage{verbatim}
%\usepackage{booktabs}
\usepackage[small, compact]{titlesec}

% list spacing
\usepackage{enumitem}
\setitemize{leftmargin=*, topsep=0pt}
\setenumerate{leftmargin=*, topsep=0pt}

\usepackage{needspace}
% example:  \needspace{4\baselineskip} makes sure we have four lines available before pagebreak

\renewcommand{\thefootnote}{\fnsymbol{footnote}}

% document starts here
\begin{document}
\parskip=\bigskipamount
\parindent=0.0in
\thispagestyle{empty}
{\large ECON-UB 233 \hfill Dave Backus @ NYU}

\bigskip\bigskip
\centerline{\Large \bf Approaches to Asset Pricing}
\centerline{Revised: \today}


*** work in digression:  change of measure and Girsanov formula ***
\url{http://www.math.nyu.edu/faculty/goodman/teaching/StochCalc2012/notes/Week10.pdf}


*** also mention CAPM \\
Check Duffie re projections...  

\bigskip
Darrell Duffie notes that the 1970s were a ``golden age''
for asset pricing theory,
but suggests that the period since has been ``a mopping-up operation''
(Duffie, {\it Dynamic Asset Pricing Theory\/}, preface).
That takes some of the glamor out of the subject,
but he's right, the basic theory has been worked out.
The same is true of calculus, but that doesn't make it any less useful.
Our goal here is to summarize the concepts and results,
including the remarkable {\it no-arbitrage theorem\/} and
the mysterious {\it risk-neutral probabilities\/}.

One way to state the theorem is that there is an $m$ that makes
$E(mr^j) = 1$ for every asset $j$.
In finance, it's common to find a statistical $m$ that works reasonably well
for the assets of interest.
The source of this $m$ is left unresolved.
In macroeconomics, it's common to link $m$ to the marginal rate of substitution
of a representative agent.
It's not a perfect theory, but it gives us some insight into asset returns,
particularly the tendency for equity to pay higher returns, on average,
than bonds.
Why?
Assets that pay off mostly in good states (states in which consumption is high)
tend to have lower prices and (therefore) higher returns than those that pay off mostly in bad states.


\section{Overview}

The setting is our usual two-period event tree:
there are two dates, 0 and 1,
and at date 1 some state $z$ occurs with probability $p(z)$.
A particular asset $j$ is a claim to a date-1 ``dividend'' $d^j(z$),
a function of the state $z$ (a random variable, in other words).
If the date-0 price of this asset is $q^j$,
the gross return between dates 0 and 1 is $r^j(z) = d^j(z) / q^j$.

[Draw the event tree, note where prices and dividends are paid.]

In this context, we can summarize modern asset pricing theory.
The idea is to derive prices of assets from state prices
--- prices $Q(z)$ of Arrow securities.
Why we can do this is a subtle issue that we'll address later,
but for now note that if we know the dividends and state prices,
the asset's price is the sum
\begin{eqnarray}
    q^j   &=& \sum_z Q(z) d^j(z) .
    \label{eq:state-prices}
\end{eqnarray}
We'll refer to (\ref{eq:state-prices}) and its successors
as the {\it pricing relation\/}.
Once we know the state prices $Q(z)$,
we use the pricing relation to compute $q^j$.
When we do this, we say we ``price'' the asset.


This is a good example of one of the clever tricks of modern finance:
we decompose an asset into pieces that we value separately.
Here the pieces are Arrow securities,
which we might have mistakenly thought of as a purely theoretical concept.
The theorem says we can always decompose an asset into its component Arrow securities
and value the security as the sum of the values of its parts.
In this respect, the basic theory of asset pricing is very simple.
%Everything else is just decoration.
%That's what Duffie had in mind when he called it a ``mopping up operation.''
What the theorem doesn't tell us is what the state prices are or how
we might compute them.

We'll look at two other versions of the same equation ---
same theory, just different notation.
One is based on a {\it pricing kernel\/}
(or {\it stochastic discount factor\/}) $m$,
defined implicitly  by $ Q(z) = p(z) m(z) $.
[That is:  $m(z) = Q(z)/p(z)$.]
After substitution,
equation (\ref{eq:state-prices}) becomes
\begin{eqnarray}
    q^j   &=& \sum_z p(z) m(z) d^j(z)  \;\;=\;\; E (m d^j) .
    \label{eq:pricing-kernel}
\end{eqnarray}
Dividing by $q^j$ gives us the familiar $E(mr^j) = 1$.
Here $m$ plays the same role the market plays in the CAPM:
the return on an asset depends on the relation between the return and the pricing kernel.
More on that shortly.

Another version is based on so-called {\it risk-neutral probabilities\/} $p^*$,
defined implicitly by $ Q(z) = p(z) m(z) = q^1 p^*(z) $,
where $q^1 = \sum_z p(z) m(z) = E(m)$ is the price of a one-period riskfree bond.
These objects,
\begin{eqnarray*}
    p^*(z) &=& p(z) m(z) /q^1 ,
\end{eqnarray*}
are positive if the $m$'s are, and they sum to one,
so they are legitimate probabilities.
The pricing relation (\ref{eq:pricing-kernel}) becomes
\begin{eqnarray}
    q^j   &=& q^1 \sum_z p^*(z) d^j(z)  \;\;=\;\; q^1 E^* (d^j) ,
    \label{eq:risk-neutral-probs}
\end{eqnarray}
where $E^*$ means the expectation based on the $p^*$s.
In (\ref{eq:pricing-kernel}), the pricing kernel performed two roles:
discounting and risk adjustment.
Here the same roles are divided between $q^1$ (discounting)
and $p^*$ (risk adjustment).
If it were up to me, I'd call them {\it risk-adjusted probabilities\/},
but I haven't had much success selling that to others.


The idea in each of these cases is to go from
state prices (or pricing kernel or risk-neutral probabilities)
to prices of specific assets.
We say we ``price'' these assets.
%The perspective is the reverse of the portfolio choice problem,
%where the prices of assets are given and we choose consumption,
%but the connection between the two is the same.


\section{The no-arbitrage theorem}

We start with the remarkable no-arbitrage theorem
connecting arbitrage and state prices.
Versions were developed by
Steve Ross (``Return, risk, and arbitrage,'' 1977)
and Michael Harrison and David Kreps (``Martingales and arbitrage,'' 1979)
in the late 1970s.
The focus on arbitrage wasn't completely new ---
Fischer Black and Myron Scholes used similar methods to value options ---
but it was a sharp break from the mean-variance approach to asset pricing
that dominated finance at the time ---
and that still plays a central role in textbooks.

We won't prove the theorem, but we'll show how it works with some examples.
The first step is to define arbitrage opportunities
and what it means for an economy to be free of them.
Suppose we have a finite number of assets and states.
Let $q$ be a column vector of asset prices,
with each element corresponding to a different asset.
And let $D$ be a matrix whose $j$th column is the dividend
vector $d^j$ for the $j$th asset,
with each row corresponding to a specific state.
If $J$ is the number of assets and $Z$ the number of states,
then $D$ is $Z$ by $J$.
The asset structure of an economy is thus summarized by $D$.

A portfolio is a vector of quantities $a$,
where element $a_j$ is the number of shares of asset $j$.
An arbitrage is a portfolio that generates positive dividends
at no cost.
Mathematically, we might write positive dividends as
\begin{eqnarray*}
    \sum_j d^j(z) a_j &\geq& 0
\end{eqnarray*}
for all states $z$ and strictly positive for at least one state.
And ``no cost'' might be generalized to
\begin{eqnarray*}
    \sum_j a_j q^j &\leq& 0  .
\end{eqnarray*}
%The first condition means here that all the elements
%of $Da$ are greater than or equal to zero and at least one is strictly
%positive.
We say $(q,D)$ is {\it arbitrage free\/} if there are no portfolios $a$
that allow arbitrage.

Now what about state prices?
We're looking for prices $Q(z)$ %, with one price for each state,
such that the price of every asset equals the sum of the values
of its state-specific dividends ---
that is, equation (\ref{eq:state-prices}).
In matrix terms
\begin{eqnarray}
    q &=&  D^\top Q,
    \label{eq:state-prices-matrix}
\end{eqnarray}
which is equation (\ref{eq:state-prices}) rewritten in matrix form
for all assets at once.

Here's the result (draw a box around it, it's the central result in modern
asset pricing):

{\bf Theorem.\/}  {\it There exist positive state prices $Q(z)$
consistent with (\ref{eq:state-prices})
if and only if the economy is arbitrage free.\/}

What's remarkable about this is that it's so general.
Absence of arbitrage is ruled out in general equilibrium models,
because it's inconsistent with equilibrium,
specifically the combination of agent maximization and market clearing.
But here we get state prices with a lot less structure than that.
The theorem tells us, under the no-arbitrage condition,
that we can always find state prices that value assets
correctly, but doesn't tell us much more.
Note, too, that the theorem goes both ways:
if we value assets with some set of state prices,
the result is free of arbitrage opportunities.

We won't prove the theorem, but some examples should give you the idea.

{\it Examples.\/}
Which of these price-dividend combinations are arbitrage free?
What are the state prices?
\begin{eqnarray*}
  (a) \;\;  q &=& \left[
            \begin{array}{c}
             1 \\ 2
            \end{array}
          \right], \;\;\;
        D \;\;=\;\;
            \left[
            \begin{array}{cc}
             1 & 0 \\ 0 & 1
            \end{array}
          \right]    \hspace*{0.5in}  \\
  (b) \;\;  q &=& \left[
            \begin{array}{c}
             2 \\ 2
            \end{array}
          \right], \;\;\;
        D \;\;=\;\;
            \left[
            \begin{array}{cc}
             1 & 1 \\ 1 & 2
            \end{array}
          \right]      \\
  (c) \;\;    q &=& \left[
            \begin{array}{c}
             2 \\ 3
            \end{array}
          \right], \;\;\;
        D \;\;=\;\;
            \left[
            \begin{array}{cc}
             1 & 1 \\ 1 & 2
            \end{array}
          \right]      \\
  (d) \;\;    q &=& \left[
            \begin{array}{c}
             3 \\ 6
            \end{array}
          \right], \;\;\;
        D \;\;=\;\;
            \left[
            \begin{array}{cc}
             1 & 1 \\ 1 & 2 \\ 1 & 3
            \end{array}
          \right]    .
\end{eqnarray*}
You should stare at this and think about what you see.
In (a) we have Arrow securities,
so their prices are the state prices.
In (b) you might notice that the second asset dominates the first:
same price, but the second has greater dividends in state 2
(and the same in state 1).
This can't work.  If you shorted asset 1 and used the proceeds to invest
in asset 2, you get something for nothing.
More  formally, let $a^\top = (-1,1)$ and see what you get.
What about the state prices?
Since $D$ is square, we can solve (\ref{eq:state-prices-matrix})
for the state prices.
We get $q(z)^\top = (1, 0)$, so they're not strictly positive.
In (c) we have the same dividends, but a higher price of asset 2,
which kills off the arbitrage opportunity.

In (d) we have fewer assets than states, but the same logic applies.
You won't be able to find an arbitrage.
But here, because you have fewer equations than unknowns,
the choice of $Q$ isn't unique:
there are lots of $Q$'s that price the assets the same way.
That's a classic feature of models with more states and assets ---
what we call {\it incomplete markets\/}.
It's an illustration of how the theory is more general than our complete
markets economies.

%[Write out the equations.]


\section{Pricing kernels and risk premiums}

Here we shift from state prices to a pricing kernel
and describe the theoretical foundation of risk premiums.
We'll see that risk premiums are tied to the covariance of returns with the pricing kernel.
This has some of the same flavor as the CAPM, but a different logical foundation.


When we define the pricing kernel $m(z)$ from state prices by $Q(z) = p(z) m(z)$,
we introduce probabilities into the pricing equation.
That's helpful, because we often want to know about things like the mean
return, which depends (obviously) on the probabilities of various outcomes.
Consider the pricing of the one-period riskfree asset.
As we've seen, the price is $q^1 = E(m)$ and the (gross) return is $r^1 = 1/E(m)$.

Now that we have probabilities, we can define {\it risk premiums\/}.
We define the risk premium on asset $j$ as its expected excess return
over the one-period riskfree rate:
$ E(r^j - r^1)$.
It can be positive, negative, or zero, depending on the asset.

Now consider the pricing of an arbitrary asset $j$ with equation (\ref{eq:pricing-kernel}).
The mean return now depends on the relation between the dividend and the pricing kernel.
If $m$ is constant, the price of the asset depends only on its expected dividend:
\begin{eqnarray*}
    q^j &=& E (m d^j) \;\;=\;\; m E(d^j) \;\;=\;\; q^1 E(d^j) .
\end{eqnarray*}
The return is therefore
$r^j = d^j/q^j$ and the expected return $E(r^j) = E(d^j)/[q^1 E(d^j)] = r^1$.
Since every asset has the same expected return, there are no risk premiums.
Evidently variation in $m$ is central to having nonzero risk premiums.

In general, the risk premium depends on the relation between
the two random variables, $m$ and $d$:
whether $E(md)$ is larger or smaller than $E(m) E(d)$.
If it's greater, the price is higher and the expected return lower
than the riskfree rate.
We can elaborate using
\begin{eqnarray*}
    q^j \;\;=\;\;    E(md^j) &=& E(m) E(d^j) + \mbox{Cov}(m,d^j).
\end{eqnarray*}
[Can you show this?  Use the definition of the covariance and expand.]
So if $\mbox{Cov}(m,d) =0$, we're back to the zero risk premium case.
This is analogous to the infamous ``zero beta'' asset of intro finance courses.
If the covariance is negative, the price is lower and the mean return higher.

We can also look directly at excess returns.  Since $ E(mr^j) = 1$ for all assets $j$,
we can subtract asset 1 and express the pricing relation in terms
of the excess return $x^j = r^j - r^1 $:
\begin{eqnarray*}
    E( m x^j) &=& 0.
\end{eqnarray*}
Using the relation for the expectation of a product, we have
\begin{eqnarray*}
    E(x^j) &=& - \mbox{Cov}(m,x^j)/E(m) .
\end{eqnarray*}
%\end{document}
Since $m$ and $E(m)$ are positive, the expected excess return
(the risk premium) has the opposite sign as the covariance with $m$.

The next question is where $m$ comes from.
%People take two different approaches.
In finance, it's common to think of $m$ as an arbitrary random variable
whose properties are chosen to reproduce observed asset prices.
We'll see examples of this sort when we look at models to value options and bonds.
In macroeconomics, it's common to think of $m$ as connected to
the state of the economy, whether it's growing rapidly or slowly.
We give an example of that next.


\section{The Capital Asset Pricing Model reconsidered}

{\bf ***************************************************************}

Describe model, explain how it compares to what we've done...  ??

Or put at end??


\section{Macrofoundations of the pricing kernel}

What would make the covariance between the pricing kernel and dividends negative --- or positive?
At the level of generality of the theorem, we have little basis for an answer.
That's where macroeconomic foundations come in, they supply an economic basis
for saying which states have high prices
and which states have low prices.
The basic idea is that prices are cheaper when goods are plenty ---
namely, in booms.
And prices are high when goods are scarce --- in recessions.

The cleanest version of this is a representative agent model.
In this case, $m$ is the marginal rate of substitution of a representative agent.
Since marginal utility is decreasing, $m$ is lower in
good states --- states where consumption is high ---
than in bad states.
An asset, like an equity index, that pays off most in good times
will therefore have a lower price [$q^j = E(md^j) < E(m) E(d^j)$]
and a higher mean return [$E(r^j) = E(d^j)/q^j$].


%{\it Example.\/}
In practice, we typically specify a (realistic) distribution for consumption
and use it to derive prices of assets.
We illustrate the approach in a lognormal setting, where the math is unusually transparent.
Suppose the state $z \sim \mathcal{N}(\kappa_1,\kappa_2)$.
%(I know, you're getting tired of this one, but first things first.)
Let $y_1(z)/y_0 = e^z$, so that $z = \log (y_1/y_0) = \log y_1 - \log y_0$.
Assume also that we have power utility:
$u'(c) = c^{-\alpha}$ for some $\alpha > 0$.
Now some questions, with answers:
\begin{itemize}
\item What is the pricing kernel?  Its distribution?
The pricing kernel is $m = \beta (c_1/c_0)^{-\alpha}$, so
\begin{eqnarray*}
    \log m(z) &=& \log \beta - \alpha (\log y_1(z) - \log y_0)
            \;\;\sim \;\; \mathcal{N}(\log \beta - \alpha \kappa_1, \alpha^2 \kappa_2) .
\end{eqnarray*}
\item How is the pricing kernel related to output?  They move in opposite directions,
as you can see above.
It's our old marginal utility result:  consumption has less value in states where there's a lot of it.
The strength of this effect depends on the risk aversion parameter $\alpha$.
If we set $\alpha = 0$ the effect goes away and the pricing kernel is constant.

\item What are the price and return of a one-period (riskfree) bond?
They solve
\begin{eqnarray*}
    q^1 &=& E(m) \;\;=\;\; \beta e^{ - \alpha \kappa_1 + \alpha^2 \kappa_2/2 } \\
    r^1 &=& 1/q^1 \;\;=\;\; \beta^{-1} e^{\alpha \kappa_1 - \alpha^2 \kappa_2/2 } .
\end{eqnarray*}
[This kind of calculation should be close to automatic by now.]
\item What is the price of an asset (``equity'') with dividend
$d^e(z) = [y_1(z)/y_0]^\lambda$ for some
arbitrary value of $\lambda$?
This is a useful device, since it allows us to vary the sensitivity of the dividend
to the endowment without killing off our convenient loglinear structure.
As long as $\lambda > 0$, this is an asset that pays off most in good states.
The price of this ``equity'' is
\begin{eqnarray*}
    q^e &=& E(md) \;\;=\;\; \beta e^{ (\lambda- \alpha) \kappa_1
    + (\lambda-\alpha)^2 \kappa_2/2 } .
\end{eqnarray*}
Note that when $\lambda = 0$ we get the one-period bond's dividend and price.
\item What is the return on equity?  The mean return?
The return is $r^e(z) = d(z)/q^e$, which satisfies
\begin{eqnarray*}
    \log r^e(z) &=& \lambda \log y_1(z) - \log q^e
            \;\;\sim \;\; \mathcal{N}
            (-\log \beta + \alpha \kappa_1 - (\lambda-\alpha)^2 \kappa_2/2, \lambda^2 \kappa_2) .
\end{eqnarray*}
(This takes patience, just stick with it.  Or use Matlab for the substitutions.)
The mean return is therefore
\begin{eqnarray*}
    E (r^e) &=& \beta^{-1} e^{\alpha \kappa_1 + [\lambda^2 - (\lambda-\alpha)^2] \kappa_2/2 } .
\end{eqnarray*}

\item Is there a risk premium on equity?
The standard definition of a risk premium is a difference in expected return:  $E(r^e-r^1)$.
The $\beta$ and $\kappa_1$ terms are the same, so we have (thankfully)
no risk premium when $\kappa_2 = 0$.
If we compare the $\kappa_2$ terms, we see that the risk premium is positive if
\begin{eqnarray*}
    0 &<& [\lambda^2 - (\lambda-\alpha)^2] + \alpha^2 \;\;=\;\; 2 \alpha \lambda .
\end{eqnarray*}
So the risk premium is positive if $\lambda > 0$.
It's larger if risk aversion is larger (larger $\alpha$)
or the dividend is more sensitive to the endowment (larger $\lambda$).
\end{itemize}
This delivers on our hope for insight:
assets that pay off more in good times ($\lambda>0$) have positive risk premiums as a result.


\section{Risk-neutral probabilities with power utility 1}

This is just a change in notation, but it's a common one in finance.
As we noted, risk-neutral probabilities $p^*$ are defined implicitly by
\begin{eqnarray*}
    q^1 p^*(z) &=& p(z) m(z) .
\end{eqnarray*}
If you prefer, you can write $  p^*(z) = p(z) m(z)/q^1 $.
They're probabilities in the sense that they're positive and sum to one,
but they incorporate risk pricing via $m$.
If $m$ is constant, then $p^*(z) = p(z)$
and all assets have the same expected return:  there's no adjustment for risk.

With risk-neutral probabilities,
the pricing relation becomes, as we've seen,
\begin{eqnarray*}
    q^j &=& q^1 E^* (d^j).
\end{eqnarray*}
Dividing by $q^j$ and substituting $r^1 = 1/q^1$, we have
\begin{eqnarray*}
    r^1 &=& E^* (r^j).
\end{eqnarray*}
In words:  under the risk-neutral probabilities, all assets have the same expected return.
We saw that a pricing kernel puts less weight on some states than others ---
perhaps the good states, as in macro-based models.
Here we get the same thing by placing lower (risk-neutral) probabilities
on some states.
%As a result,
%under the risk-neutral probabilities all assets have the same expected return.
%How does this work for assets with higher expected returns and positive risk premiums?
%Evidently $p^*$ puts more weight on bad outcomes and less weight on good outcomes than $p$.

{\it Example (Bernoulli risks).\/}
Let $ z = \log y_1 - \log y_0 = \log g$ (log consumption growth)
take on two values,
\begin{eqnarray*}
    z &=&   \left\{
            \begin{array}{ll}
            \gamma_1 & \mbox{with probability } 1-\omega \\
            \gamma_2 > \gamma_1 & \mbox{with probability } \omega .
            \end{array}
            \right.
\end{eqnarray*}
The pricing kernel is $\beta g^{-\alpha} = \beta e^{-\alpha z}$.
When we value assets with the pricing kernel, we put more weight
on the bad state (state 1), because marginal utility is higher.

What happens to the risk-neutral probabilities?
The probability of state 1 is
\begin{eqnarray*}
        p^*(z=\gamma_1)
            &=& \frac{(1-\omega) \beta e^{-\alpha \gamma_1}}
                    {(1-\omega) \beta e^{-\alpha \gamma_1} + \omega \beta e^{-\alpha \gamma_2}}
            \;\;=\;\; \frac{(1-\omega)}
                    {(1-\omega) + \omega e^{-\alpha (\gamma_2-\gamma_1)}}
            \;\;>\;\; 1-\omega .
\end{eqnarray*}
The inequality follows because $\gamma_2 > \gamma_1$,
which means the denominator is less than one.
The risk neutral probabilities put more weight
on the bad outcome, just as the pricing kernel does.
We knew, of course, that the two approaches had to give us the same answer,
but it's useful see it in action.
It's also clear, here and in general, that the discount factor $\beta$
drops out of the risk-neutral probabilities.
It shows up, instead, in the bond price $q^1$ in the pricing relation
(\ref{eq:risk-neutral-probs}).

{\it Example (normal risks).\/}
We return to the example of the previous section.
We have $ z = \log y_1 - \log y_0 \sim \mathcal{N}(\kappa_1, \kappa_2) $.
The pdf is therefore
\begin{eqnarray*}
    p(z) &=& (2 \pi \kappa_2)^{-1/2} \exp[ - (z-\kappa_1)^2/(2\kappa_2)]
\end{eqnarray*}
and the pricing kernel is $m(z) = \beta \exp(-\alpha z)$.
The one-period riskfree bond price is therefore
$ q^1 = E(m) = \beta \exp(-\alpha \kappa_1 + \alpha^2 \kappa_2/2)$,
as we saw earlier.
That gives us the risk-neutral pdf
\begin{eqnarray*}
    p^*(z)  &=& p(z) m (z) /q^1 \\
            &=& (2 \pi \kappa_2)^{-1/2} \exp[ - (z-\kappa_1)^2/(2\kappa_2)]
                        \beta \exp(-\alpha z)/q^1 \\
            &=& (2 \pi \kappa_2)^{-1/2} \exp[ - (z-\kappa_1+\alpha \kappa_2)^2/(2\kappa_2)] .
\end{eqnarray*}
That is, $z$'s true distribution is $\mathcal{N}(\kappa_1,\kappa_2)$,
but its risk-neutral distribution is $\mathcal{N}(\kappa_1-\alpha \kappa_2,\kappa_2)$:
we shift the distribution to the left (more pessimistic) to account
for risk.  How much depends on risk ($\kappa_2$) and risk aversion ($\alpha$).


\section{Risk-neutral probabilities with power utility 2}


We've seen two examples connecting true and risk-neutral probabilities
in representative agent economies with power utility.
In the lognormal case, the only change is in the mean:  risk-neutral
probabilities build in risk aversion by reducing the mean.
That's not true in general.
In general, all of the cumulants of the distribution change.
We'll see that shortly with numerical examples.
In the next section, we derive an analytic expressions connecting
their cumulants.

Let us say that we have an arbitrary continuous probability distribution over states $z$.
We can approximate most such distributions with a grid over a finite number
of states.
If we make the grid fine enough, we can get as close to a continuous distribution
as we like.
Here's an example of such a grid in Matlab:
\begin{verbatim}
zmax = 4;
dz = 0.1;
z = [-zmax:dz:zmax]';
\end{verbatim}
The next step is to define probabilities over the grid points.
If we used a standard normal density, we could generate probabilities
from
\begin{verbatim}
p = exp(-z.^2/2)*dz/sqrt(2*pi);
\end{verbatim}
Here we'll do something a little different.
Since we know how this works for the normal distribution
(we shift the mean left, keep the variance the same),
we'd like to consider alternatives.
Gram-Charlier distributions are a useful class.
The idea is (roughly) to approximate a distribution
with nonzero skewness and excess kurtosis from the normal.
If skewness and excess kurtosis are $\gamma_1$ and $\gamma_2$,
then the density is
\begin{eqnarray*}
    p(z) &=& p = (2 \pi)^{-1/2} \exp(-z.^2/2)[1 + \gamma_1(z^3-3z)/3! + \gamma_2(z^4-6z^2+3)/4!] .
\end{eqnarray*}
Our discrete approximation is therefore
\begin{verbatim}
p = exp(-z.^2/2).*(1 + gamma1*(z.^3-3*z)/6 + gamma2*(z.^4-6*z.^2+3)/24);
p = p*dz/sqrt(2*pi);
\end{verbatim}
If $\gamma_1 = \gamma_2 = 0$ we get the standard normal.
Otherwise, we get distributions with various levels of skewness
and excess kurtosis.

Given a distribution for the state, we then generate consumption
growth, the pricing kernel, and risk-neutral probabilities,
each of them as functions of the same state.
Let us say that log consumption growth is
\begin{eqnarray*}
    \log g(z) &=& \mu + \sigma z .
\end{eqnarray*}
The pricing kernel is then
\begin{eqnarray*}
    m(z) &=& \beta g(z)^{-\alpha}  .
\end{eqnarray*}
The price of a one-period riskfree bond is therefore
\begin{eqnarray*}
    q^1 &=& E(m) \;\;=\;\; \sum_z p(z) m(z) ,
\end{eqnarray*}
giving us the risk-neutral probabilities
\begin{eqnarray*}
    p^*(z) &=& p(z) m(z) / q^1 .
\end{eqnarray*}
In Matlab, these steps are
\begin{verbatim}
logg = mug + sigmag*z;
g = exp(logg);
m = beta*g.^(-alpha);
q1 = p'*m
pstar = p.*m/q1;
\end{verbatim}

So what does this give us?
This is easier to see than describe, but if you vary
$\gamma_1$ and $\gamma_2$ you can generate a lot of different shapes
for $p$.
And you'll notice that $p^*$ can have a much different shape.
It really is true that the normal result (shift the mean) is special.


\section{More fun with generating functions}

We can also approach the difference between true and risk-neutral
probabilities analytically --- again, for the case of power utility.
(The reason for power utility is that its form matches up nicely
with the exponential in the moment generating function.)
The idea comes from Ian Martin.

Suppose, as in our examples,
that the state is $ z = \log y_1 - \log y_0$.
The cumulant generating function  of $z$ is
\begin{eqnarray*}
    k(s) &=& \log E (e^{sz}) .
\end{eqnarray*}
In this notation, the log of the one-period bond price is
\begin{eqnarray*}
    \log q^1  &=& \log E (\beta e^{-\alpha z})
            \;\;=\;\; \log \beta +  k(-\alpha).
\end{eqnarray*}
The cgf of the risk-neutral distribution is
\begin{eqnarray}
    k^*(s) &=& \log [E (m e^{sz})/q^1]
        \;\;=\;\;   \log E \left( \beta e^{-\alpha z}e^{sz} \right) - \log q^1
        \;\;=\;\;  k(s-\alpha) - k(-\alpha)  .
        \label{eq:risk-neutral-cgf}
\end{eqnarray}
This is a thing of beauty, a wonderfully compact
summary of how true and risk-neutral distributions are connected.

{\it Example.\/}
We'll redo the calculation of the previous section:
the risk-neutral distribution of $z = \log y_1 - \log y_0$
when $z \sim \mathcal{N}(\kappa_1,\kappa_2)$.
We know that $k(s) = s \kappa_1 + s^2 \kappa_2/2$.
Therefore
\begin{eqnarray*}
    k^*(s) &=& [(s-\alpha) \kappa_1 + (s-\alpha)^2 \kappa_2/2]
            - [-\alpha \kappa_1 + \alpha^2 \kappa_2/2]  \\
           &=& s(\kappa_1 - \alpha \kappa_2) + s^2 \kappa_2/2 ,
\end{eqnarray*}
the same answer we had before.
Even better, you can have Matlab do the substitution.

For extra credit, show how cumulants are related.
In the example, the only change is the mean.
What happens in general?


\section*{Bottom line}

The no-arbitrage theorem tells us that if arbitrage is ruled out,
we can value assets with state prices.
State prices, pricing kernels, and risk-neutral probabilities all represent
the same idea.


\section*{Practice problems}

\begin{enumerate}
\item {\it State prices.\/}
Consider the following collections of asset prices $q$ and dividends $D$:
\begin{eqnarray*}
  (a) \;\;  q &=& \left[
            \begin{array}{c}
             1 \\ 1.5
            \end{array}
          \right], \;\;\;
        D \;\;=\;\;
            \left[
            \begin{array}{cc}
             1 & 1 \\ 1 & 2
            \end{array}
          \right]    \hspace*{0.5in}  \\
  (b) \;\;  q &=& \left[
            \begin{array}{c}
             1
            \end{array}
          \right], \;\;\;
        D \;\;=\;\;
            \left[
            \begin{array}{cc}
             1 & 2
            \end{array}
          \right]  .
\end{eqnarray*}
Are they arbitrage free?  What are the implied state prices?
%
\needspace{2\baselineskip}
Answer.
\begin{enumerate}
\item If we solve for state prices, we get $Q(1) = Q(2) = 1/2$.
The existence of positive state prices tells us it's arbitrage free.
\item There's only one asset, and it has a positive price,
so it's arbitrage free.
State prices aren't unique: any positive solution to $ 1 = Q(1) + 2 Q(2)$ works.
That means we need $ 0 < Q(2) < 1/2$ and $Q(1) = 1 - 2 Q(2)$.
\end{enumerate}

\item {\it Pricing kernel.\/} For the same problem, example (a), suppose
the (true) probabilities are $p(1) = p(2) = 1/2$.
\begin{enumerate}
\item What is the pricing kernel?
\item What are the risk-neutral probabilities?
\end{enumerate}
%
\needspace{2\baselineskip}
Answer.
\begin{enumerate}
\item State prices are connected to the pricing kernel by $Q(z) = p(z) m(z)$.
Here we have $m(1) = m(2) = 1$.
\item Risk-neutral probabilities are connected to the pricing kernel
by $p^*(z) = p(z) m(z) / q^1$.
Here $q^1 = E(m) = 1$ and $p^*(1) = p^*(2) = 1/2$.
\end{enumerate}

\item {\it Returns and risk premiums.\/}
Consider the asset prices and dividends
\begin{eqnarray*}
    \mbox{Asset 1:}&&  q^1 = 3/4, \;\; d^1(1) = 1, \;\; d^1(2) = 1 \\
    \mbox{Asset 2:}&&  q^2 = 1, \;\; d^2(1) = 1, \;\; d^2(2) = 2.
\end{eqnarray*}
\begin{enumerate}
\item What are the state prices? Is $(q,D)$ arbitrage free?
\item What are the returns on the two assets?
\item If $p(1) = p(2) = 1/2$, what are the expected returns?
\item What is the pricing kernel?
Why does the second asset have a higher excess return?
\item What are the risk-neutral probabilities?
Why does the second asset have a higher excess return?
\end{enumerate}
%
\needspace{2\baselineskip}
Answer.
\begin{enumerate}
\item The state prices are $Q(1) = 1/2$ and $Q(2) = 1/4$.
Since both are positive, the system is arbitrage free.
\item Returns are $r^j(z) = d^j(z)/q^j $.
That gives us returns

\begin{center}
\begin{tabular}{ccc}
\toprule
Asset &  State 1 & State 2 \\
\midrule
1     &  4/3  &   4/3  \\
2     &  1    &   2  \\
\bottomrule
\end{tabular}
\end{center}

\item Expected returns are
$ E(r^1) = 4/3$ and $E(r^2) = 3/2$.
We would say that the second asset has a risk premium of 1/6.

\item The pricing kernel is connected to probabilities and state
prices by $ m(z) = Q(z)/p(z)$.
That gives us $m(1) = 1$ and $m(2) = 1/2$.
The second asset has a higher expected return because it pays off more in state 2,
where $m$ is lower.

\item Risk-neutral probabilities
are connected to the pricing kernel and probabilities by
$ p^*(z) = p(z) m(z) /q^1 $.
That gives us $p^*(1) = 2/3$ and $p^*(2) = 1/3$.
The second asset has a higher expected return because it pays off more in state 2,
where $p^*$ is lower.

\end{enumerate}


\item {\it Risk-neutral probabilities with exponential risk.\/}
In our usual two-period setup, suppose the representative agent
has power utility and
$x(z) = \log c_1(z) - \log c_0 $ has an exponential distribution:
\begin{eqnarray*}
    p(x) &=& \lambda e^{-\lambda x} ,
\end{eqnarray*}
for $x \geq 0$ and $\lambda > 0$.
What is the pricing kernel?
What is the risk-neutral distribution of $x$?
%
\needspace{2\baselineskip}
Answer.
This is a little sloppy, but the following question nails down the details.
The risk-neutral probability is
the product of the true probability and the pricing kernel divided by $q^1$.
The pricing kernel here is $  m(x) = \beta e^{-\alpha x} $.
Since $q^1$ is constant, the risk-neutral probabilities are
\begin{eqnarray*}
    p^*(x) &\propto& p(x) m(x)
        \;\;=\;\; \lambda e^{-\lambda x} \beta e^{-\alpha x}
                \;\;=\;\; \mbox{constant} \times e^{-(\lambda + \alpha) x} .
\end{eqnarray*}
The constant must be $\lambda + \alpha$ if the probabilities are
to integrate to one.
This tell us the risk-neutral distribution is also exponential,
but with parameter $\lambda + \alpha$.
The mean and standard deviation are both $1/(\lambda+\alpha)$,
which decline with risk aversion $\alpha$.
So the risk-neutral distribution has a smaller mean
and also a smaller standard deviation.
The former is like the normal distribution, the latter is not.

\item {\it Exponential risk, cgf version.\/}
Do the same problem using the cgf and equation (\ref{eq:risk-neutral-cgf}).
%
\needspace{2\baselineskip}
Answer.  You may show, or recall,
that an exponential random variable with parameter $\lambda>0$
has cgf
\begin{eqnarray*}
    k(s) &=& - \log (1-s/\lambda) \;\;=\;\; \log \left( \frac{\lambda}{\lambda - s}\right).
\end{eqnarray*}
The cgf formula (\ref{eq:risk-neutral-cgf}) gives us
\begin{eqnarray*}
    k^*(s) &=& k(s-\alpha) - k(-\alpha)
            \;\;=\;\; \log \left( \frac{\lambda+\alpha}{\lambda+\alpha - s}\right),
\end{eqnarray*}
which is the cgf for an exponential random variable with parameter $\lambda + \alpha$.

\item {\it Asset pricing with exponential risk.\/}
We consider asset prices in the same setting:
exponential risk, power utility, and so on.
We assume $\lambda > 1$ throughout.
%For all of these calculations, we set $\lambda = 2$, $\alpha = 3$, and $\beta = 0.9$.
\begin{enumerate}
\item What is the price of a bond paying a dividend of one in all states?
What is its return?
\item What is the price of ``equity,'' a claim to the aggregate growth rate $e^x$?
What is its return?  Its expected return?
\item What is the risk premium on equity?  How does it depend on $\alpha$?
\end{enumerate}
%
\needspace{2\baselineskip}
Answer.
\begin{enumerate}
\item The price of the bond is
\begin{eqnarray*}
    q^1 &=& \int_{0}^\infty  m(x) p(x) dx
            \;\;=\;\; \int_{0}^\infty \beta \lambda e^{-(\lambda + \alpha) x} dx
            \;\;=\;\; \left. - \left( \frac{\lambda \beta}{\lambda + \alpha} \right) e^{-(\lambda + \alpha) x} \right|_0^\infty  \\
        &=& \beta \lambda /(\lambda + \alpha) . \phantom{sum^K}
\end{eqnarray*}
Its return is $r^1 = 1/q^1 = (\lambda + \alpha)/(\beta \lambda )$.

\item The price of ``equity'' is
\begin{eqnarray*}
    q^e &=& \int_{0}^\infty \beta \lambda e^{(1-\lambda -\alpha) x} dx
            \;\;=\;\; \beta \lambda /(\lambda + \alpha-1) . \phantom{sum^K}
\end{eqnarray*}
The return is the dividend divided by the price:  $r^e(x) = e^x/q^e$.
The expected return is the expected dividend divided by the price.
The expected dividend is
\begin{eqnarray*}
    E(d^e) &=& E(e^x) \;\;=\;\; \int_{0}^\infty \lambda e^{(1-\lambda) x} dx
            \;\;=\;\; \lambda/(\lambda-1) .
\end{eqnarray*}
The expected return is $ E(r^e) = [\lambda/(\lambda-1)] [(\lambda+\alpha)/(\beta \lambda) ]$.

\item If $\alpha = 0$, both assets have expected return $1/\beta$.
But if $\alpha > 0$, the return on equity is greater.

\end{enumerate}
\end{enumerate}


\section*{More}

The material is standard.
The best textbook reference is Duffie, {\it Dynamic Asset Pricing Theory\/}, any edition.
(The earlier ones are cheaper, and just as good for our purposes.)
See esp Chapters 1 and 2.
Ross's classic paper is also a good read, esp Section 9.3.
I scanned and posted it
\href{http://pages.stern.nyu.edu/~dbackus/233/SteveRoss_risk-arb_chapter_1977.pdf}{here}.

\vfill
{\bigskip \centerline{\it \copyright \ \number\year \
NYU Stern School of Business}
}

\end{document}

\pagebreak
\verbatiminput{../Matlab/risk_neutral_probs.m}



\end{document}
