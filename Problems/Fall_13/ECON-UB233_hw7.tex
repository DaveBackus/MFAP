\documentclass[11pt]{exam}

\oddsidemargin=0.25truein \evensidemargin=0.25truein
\topmargin=-0.5truein \textwidth=6.0truein \textheight=8.75truein

%\RequirePackage{graphicx}
\usepackage{comment}
\usepackage{verbatim}
\usepackage{booktabs}
\usepackage{graphicx}
\usepackage{hyperref}
\urlstyle{rm}   % change fonts for url's (from Chad Jones)
\hypersetup{
    colorlinks=true,        % kills boxes
    allcolors=blue,
    pdfsubject={ECON-UB233, Macroeconomic foundations for asset pricing},
    pdfauthor={Dave Backus @ NYU},
    pdfstartview={FitH},
    pdfpagemode={UseNone},
%    pdfnewwindow=true,      % links in new window
%    linkcolor=blue,         % color of internal links
%    citecolor=blue,         % color of links to bibliography
%    filecolor=blue,         % color of file links
%    urlcolor=blue           % color of external links
% see:  http://www.tug.org/applications/hyperref/manual.html
}

\renewcommand{\thefootnote}{\fnsymbol{footnote}}
\newcommand{\var}{\mbox{\it Var\/}}

\printanswers

% document starts here
\begin{document}
\parskip=\bigskipamount
\parindent=0.0in
\thispagestyle{empty}
{\large ECON-UB 233 \hfill Dave Backus @ NYU}

\bigskip\bigskip
%\centerline{\Large \bf Lab Report \#7:  Stochastic Processes}
\centerline{\Large \bf Lab Report \#7:  Dynamics in Theory and Data}
\centerline{Revised: \today}

\bigskip
{\it Due at the start of class.
You may speak to others, but whatever you hand in should be your own work.
Please include your Matlab code.}

%\begin{solution}
%Brief answers follow,
%but see also the attached Matlab program;
%download the pdf, open, click on pushpin:
%%\attachfile{../Matlab/hw/hw7_f13.m}
%\end{solution}

\begin{questions}
%-----------------------------------------------------------------------
\question {\it Combination models.\/}
Consider the two-equation model
\begin{eqnarray*}
    y_{t+1}   &=&  x_t + \lambda w_{t+1} \\
    x_{t}     &=& \varphi x_{t-1} + \sigma w_{t}  ,
\end{eqnarray*}
where $w_t$ is our usual iid normal disturbance.
The parameters are $\sigma>0$, $| \varphi| < 1$, and $\lambda$ (any sign).

\begin{parts}
\item Express $x$ as an infinite moving average; that is, an equation of the form
\begin{eqnarray*}
    x_t = \theta_0 w_t + \theta_1 w_{t-1} + \theta_2 w_{t-2} + \cdots .
\end{eqnarray*}
What are the moving average coefficients $\theta_j$?
\item What is the acf of $x$?
\item Now express $y$ as an infinite moving average.
What are its moving average coefficients?
\item What is the acf of $y$?  How does it depend on $\lambda$?
\item {\it Extra credit.\/} 
What kind of ARMA model is $y$?
\end{parts}

\begin{solution}
\begin{parts}
\item Repeated substitution gives us
\begin{eqnarray*}
    x_t &=& \sigma (w_t + \varphi w_{t-1} + \varphi^2 w_{t-2} + \cdots) .
\end{eqnarray*}
The coefficients are evidently $\theta_j = \sigma \varphi^j$.
\item The variance of $x_t$ (the variance of its equilibrium distribution)
is the sum of the squared MA coefficients:
\begin{eqnarray*}
    \mbox{Var}(x_t) &=& \sigma^2 /(1-\varphi^2) .
\end{eqnarray*}
The autocovariances are
\begin{eqnarray*}
    \mbox{Cov}(x_t,x_{t-k}) &=& \sigma^2 \varphi^k /(1-\varphi^2) ,
\end{eqnarray*}
which makes the autocorrelation function
\begin{eqnarray*}
    \rho(k) &=& \varphi^k
\end{eqnarray*}
for integers $k \geq 0$.

\item Here we have
\begin{eqnarray*}
    y_{t+1} &=& \lambda w_{t+1} + \sigma (w_t + \varphi w_{t-1} + \varphi^2 w_{t-2} + \cdots) .
\end{eqnarray*}

\item [(d,e)] This is another ARMA(1,1), see the previous question.

\end{parts}
\end{solution}

%-----------------------------------------------------------------------
\question {\it Dynamics of US interest rates.\/}
We'll look at the autocorrelations of interest rates to get a sense
of their dynamics.
The first step is to download some data from the Fed.
Go to

\url{http://www.federalreserve.gov/releases/h15/data.htm}

and download monthly data for Treasury constant maturities,
specifically the 3-month and 10-year maturities,
for the period 1985 to present.
%Once you have the numbers, read them into Matlab.
The second step is to download the function
{\tt acf.m} posted on the course website:

\url{http://pages.stern.nyu.edu/~dbackus/233/acf.m}.

It's my program, not part of Matlab.
You need to save it in whatever Matlab thinks is the current working directory.
The command {\tt acf(x,n)} computes the first {\tt n} terms of
the sample acf of the vector of data {\tt x}.
To see how it works, look at

\url{http://pages.stern.nyu.edu/~dbackus/233/business_cycles.m}.

Once you've done all this:
%
\begin{parts}
\item Compute the mean, standard deviation,
and autocorrelation function (acf) for the 3-month
interest rate for lags $k$ from 0 to 24 months.

\item Graph the acf.

\item Describe the acf for the AR(1):
\begin{eqnarray*}
    x_t &=&  \varphi x_{t-1} + \sigma w_t ,
\end{eqnarray*}
where $ \{ w_t \}$ is a sequence of standard normal random variables.
How does the sample acf compare to the acf of a suitably chosen AR(1)?

\item Compute the mean, standard deviation,
autocorrelation function for the 10-year interest rate.
How do they compare to those of the 3-month rate?
\end{parts}

\begin{solution}
\begin{enumerate}
\item [(a,c,d)]
I didn't get a chance to update this.  Last spring, the relevant statistics were

\begin{center}
\begin{tabular}{lccc}
\toprule
        &  Mean & Std Dev & Autocorr \\
\midrule
3-month  &   4.00 & 2.52 & 0.989  \\
10-year  &   5.79 & 2.17 & 0.978  \\
\bottomrule
\end{tabular}
\end{center}

They reflect some standard features of interest rates, including:
(i)~rates increase with maturity, on average;
(ii)~standard deviations decline with maturity;
and (iii)~all of them are very persistent.
Which is the point of the exercise.

%We can see more of the autocorrelation functions in the Matlab figure
%(run the code to see it).

\item[(b)] With an AR(1), the autocorrelation has the form
$\rho(k) = \varphi^k$:  it declines geometrically.
If we set $\varphi = \rho(1)$, we can compute
estimated acf's corresponding to AR(1)s.
They're plotted as the dashed lines above.
You can see they're somewhat different from the AR(1).
Whether this represents sampling variability of some more
basic difference between the data and the AR(1) model isn't clear.
\end{enumerate}
\end{solution}

\end{questions}

\vfill \centerline{\it \copyright \ \number\year \
NYU Stern School of Business}

\end{document}

