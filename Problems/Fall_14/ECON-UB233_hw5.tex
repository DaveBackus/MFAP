\documentclass[11pt]{exam}

% spacing on page
\oddsidemargin=0.25truein \evensidemargin=0.25truein
\topmargin=-0.5truein \textwidth=6.0truein \textheight=8.75truein

\usepackage{comment}
\usepackage{graphicx}
\usepackage{amssymb}
\usepackage{amsmath}

\usepackage[margin=0pt, labelsep=period, font=large, labelfont=bf]{caption}
%\usepackage{float}

\usepackage{hyperref}
\urlstyle{rm}   % change fonts for url's (from Chad Jones)
\hypersetup{
    colorlinks=true,        % kills boxes
%    allcolors=blue,
    pdfsubject={ECON-UB233, Macroeconomic foundations for asset pricing},
    pdfauthor={Dave Backus @ NYU},
    pdfstartview={FitH},
    pdfpagemode={UseNone},
%    pdfnewwindow=true,      % links in new window
%    linkcolor=blue,         % color of internal links
%    citecolor=blue,         % color of links to bibliography
%    filecolor=blue,         % color of file links
%    urlcolor=blue           % color of external links
% see:  http://www.tug.org/applications/hyperref/manual.html
}

% for listing code in tt font
\usepackage{verbatim}

% for table spacing
\usepackage{booktabs}

% section headers and spacing
\usepackage[small, compact]{titlesec}

% list spacing
\usepackage{enumitem}
\setitemize{leftmargin=*, topsep=0pt}
\setenumerate{leftmargin=*, topsep=0pt}

% attach files to the pdf
\usepackage{attachfile}
    \attachfilesetup{color=0.75 0 0.75}

\usepackage{needspace}
% \needspace{4\baselineskip} makes sure we have four lines available before a pagebreak

%\renewcommand{\thefootnote}{\fnsymbol{footnote}}
%\newcommand{\var}{\mbox{\it Var\/}}

\newcommand{\cbar}{\bar{c}}
\newcommand{\rp}{\mbox{\em rp\/}}
\newcommand{\tp}{\mbox{\em tp\/}}
\newcommand{\bp}{\mbox{\em bp\/}}


\printanswers

% document starts here
\begin{document}
\parskip=\bigskipamount
\parindent=0.0in
\thispagestyle{empty}
{\large ECON-UB 233 \hfill Dave Backus @ NYU}

\bigskip\bigskip
\centerline{\Large \bf Lab Report \#5: Returns \& Risk Premiums}
\centerline{Revised: \today}

\bigskip
{\it Due at the start of class.
You may speak to others, but whatever you hand in should be your own work.
Please include your Matlab code.}

\begin{solution}
Brief answers follow,
but see also the attached Matlab program.
Download the pdf, open, and click on the pushpin:
\attachfile{hw5_f14.m} \\
\end{solution}

\begin{questions}
%-----------------------------------------------------------------------
\question {\it Disaster risk and the equity premium.\/}
We add a ``disaster'' state to our analysis of the equity premium
and see how it changes our perspective.
We know power utility agents dislike negative skewness,
so we conjecture that an asset with a disaster outcome will
have a large risk premium.
The key input is the distribution of log consumption growth,
\begin{eqnarray*}
    \log g &=& \left\{
                \begin{array}{ll}
                \mu - \sigma & \mbox{with probability } (1-\omega)/2 \\
                \mu + \sigma & \mbox{with probability } (1-\omega)/2 \\
                \mu - \theta & \mbox{with probability } \omega .
                \end{array}
                \right.
\end{eqnarray*}
If $\omega = 0$, we're back to our symmetric two-state distribution.
But if we choose a small positive value of $\omega$ and a ``largish'' $\theta>0$,
we have a ``disaster'' state that changes the distribution dramatically.

We'll define equity as a claim to consumption growth $g$
and explore the impact of a disaster state on the equity premium.
Suppose, as usual, that we have a representative agent with power utility.
Preference parameters are $\beta = 0.99$ and $\alpha = 10$ throughout.

\begin{parts}
\item Verify that the mean and variance of log consumption growth are
\begin{eqnarray*}
    E (\log g)   &=&  \mu - \omega \theta \\
    \mbox{Var}(\log g) &=& (1-\omega) (\sigma^2 + \omega \theta^2) .
\end{eqnarray*}

\item If $\omega = 0$, what values of $\mu$ and $\sigma$ deliver
the observed mean and variance of log consumption growth, namely
0.0200 and $0.0350^2$?

\item With these parameter values, compute the equity premium and its inputs.

\item Now consider $\omega = 0.01$ and $\theta = 0.30$.
(These numbers are based on a series of studies by Robert Barro and his coauthors.)
With these numbers, what values of $\mu$ and $\sigma$ reproduce
the observed mean and variance of log consumption growth?

\item Compute the equity premium and its components
with these parameter values.
How does it compare to your previous calculation?

\item
What are the maximum Sharpe ratios when $\omega = 0$ and when $\omega = 0.01$?
How do they compare to the Sharpe ratio we observe for equity in US data?

\item
What is entropy when $\omega = 0$ and when $\omega = 0.01$?

\item Summarize what you've found.
How does a disaster state affect the magnitude of risk premiums
in this economy?
\end{parts}

\begin{solution}
The idea is to show how changing the distribution
in ways that produce negative skewness can increase risk
premiums even if we keep the standard deviation of log consumption
growth the same.

\begin{parts}
\item I did this using the cgf; see the Matlab code.

\item The expressions for the mean and variance give us
\begin{eqnarray*}
    0.0200  &=&  \mu - \omega \theta  \\
    0.0350^2 &=& (1-\omega) \sigma^2 + \omega (1-\omega) \theta^2 ,
\end{eqnarray*}
When $\omega = 0$, the mean is $\mu = 0.0200$ and the standard
deviation is $\sigma = 0.0350$.

\item With these values, we have
$q^1 = 0.8607$, $r^1 = 1.1618$,
$ E(r^e) = 1.1757$,
and the equity premium $ E(r^e-r^1) = 0.0138$.
As we've come to expect, the equity premium is smaller than we see in the data
(0.0571 was our estimate).

\item With $\omega = 0.01$, we set
$ \mu = 0.0230$ and $\sigma = 0.0183$.


\item The disaster calculations give us
$r^1 = 1.0529$,
$ E(r^e) = 1.1007$, and $ E(r^e-r^1) = 0.0478$.
This is much closer to what we see in the data.
In this respect at least, the disaster state is pushing us in the right direction.

\item
The maximum Sharpe ratio $ \mbox{Std}(m)/E(m)$ rises from 0.3364 to 1.3717.
What we see in the data is 0.3049, so even the first one would have done it.
Here even the first calculation would have worked.
That's largely because risk aversion is so large.

\item Entropy is an upper bound on the equity premium computed from log returns.
Here it rises from 0.0600 to 0.1585.
Even the first one is above what we need.

\item Evidently a disaster state gives us a better chance to
explain observed risk premiums.
It's not part of the question, but this probably overstates the case:
risk aversion is pretty high here, and the disaster state is pretty bad.
\end{parts}
\end{solution}


%-----------------------------------------------------------------------
\item {\it Valuing put options .\/}
We can value options the same way we value any other asset,
it's simply a matter of getting the cash flows right.
With future topics in mind,
we use a discrete approximation to a normal mixture,
which we use to produce departures from normality: skewness, excess kurtosis, and so on.
(No, that's not an oxymoron; see the notes on random variables.)

To be specific, we'll use two normal distributions as inputs.
The first one has mean zero and variance one,
the second has mean $\theta$ and variance $\delta$.
We denote their pdf's by $p_1$ and $p_2$, respectively.
The pdf of our state variable $z$ is
\begin{eqnarray*}
    p(z) &=& (1-\omega) p_1(z) + \omega p_2(z)
\end{eqnarray*}
for some mixing parameter $\omega$ between zero and one.
Roughly speaking, $\theta$ governs skewness and $\delta$ governs excess kurtosis.
We'll set $\omega = 0.2$, $\theta = -1$, and $\delta = 2$.
%
We construct a discrete approximation with the Matlab commands
\begin{verbatim}
% grid
zmax = 6;
dz = 0.1;
z = [-zmax:dz:zmax]';
% mixture
omega = 0.2; theta = -1; delta = 2;
p1 = exp(-z.^2/2)*dz/sqrt(2*pi);
p2 = exp(-(z-theta).^2/(2*delta))*dz/sqrt(2*pi*delta);
p  = (1-omega)*p1 + omega*p2;
\end{verbatim}


We'll use the usual representative agent setup.
Log consumption growth is
\begin{eqnarray*}
    \log g(z) &=& \mu + \sigma z .
\end{eqnarray*}
The mean and variance are then
\begin{eqnarray*}
    E (\log g) &=& \mu + \omega \theta  \\
    \mbox{Var}(\log g) &=& \sigma^2 [(1-\omega) + \omega (\theta^2 + \delta)] .
\end{eqnarray*}
With power utility, the pricing kernel is
\begin{eqnarray*}
    m(z) &=& \beta \exp(-\alpha \log g)
            \;\;=\;\; \beta \exp[-\alpha (\mu + \sigma z) ] .
\end{eqnarray*}
We'll set $\beta = 0.99$ and $\alpha = 1$.


\begin{parts}
\item Plot $p$ (the mixture) and $p_1$ (the standard normal).
How do they differ?

\item Use the commands
\begin{verbatim}
meanz = sum(p.*z)
stdz  = sqrt(sum(p.*(z-meanz).^2))
skewz = sum(p.*(z-meanz).^3)/stdz^3
xkurz = sum(p.*(z-meanz).^4)/stdz^4 - 3
\end{verbatim}
to compute properties of the mixture distribution.
What indications do they give us that the distribution is not normal?
What are the skewness and excess kurtosis of $\log g$?

\item Choose values of $\mu$ and $\sigma$ to reproduce
the mean (0.0200) and standard deviation (0.0350) of $\log g$.

\item Define equity as a claim to the dividend $d(z) = 100 \ g(z) $.
What is its price?
Its expected return?


\item A put option with  ``strike price'' $k$ gives the owner the right to sell the
dividend $d(z)$ at date 1 for price $k$.
The owner of the option will only do this if $ d(z) < k$,
which gives rise to the cash flow
\begin{eqnarray*}
    f(z) &=& \max \{ 0, k-d(z) \} .
\end{eqnarray*}
You can compute this in Matlab with the commands
\begin{verbatim}
k = 125;
f = (k-d).*(k>d);
\end{verbatim}
(Ask yourself what {\tt (k>d)} does.  Try it if you're not sure.)
Graph the cash flow $f(z)$ versus the dividend $d(z)$ for $k = 125$.


\item What is the price (at date 0) of the option?
(Note:  This is a claim to the cash flow $f$ and priced
as we would any other cash flow.)
What is its expected return?
\end{parts}

\begin{solution}
This is a warmup for our next topic:  options.
\begin{parts}
\item Run the Matlab code.
You'll see that the mixture is skewed left.

\item The mixture has negative skewness (evident in the figure)
and positive excess kurtosis.

\item We need $\mu = 0.2200$ and $\sigma = 0.0296$.

\item Its price is $q^e = 98.9965$ and its expected return is $ E(r^e) = 1.2520 $.

\item The cash flow has a slope of minus one until $d=k = 125$,
and is zero after that.

\item The option's price is 1.8611.
Its expected return is 1.1998, which is below equity.

\end{parts}
\end{solution}

\end{questions}

{\vfill
{\bigskip \centerline{\it \copyright \ \number\year \
David Backus $|$ NYU Stern School of Business}%
}}

\end{document}
