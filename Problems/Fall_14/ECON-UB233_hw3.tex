\documentclass[11pt]{exam}

% spacing on page
\oddsidemargin=0.25truein \evensidemargin=0.25truein
\topmargin=-0.5truein \textwidth=6.0truein \textheight=8.75truein

\usepackage{comment}
\usepackage{graphicx}
\usepackage{amssymb}
\usepackage{amsmath}

\usepackage[margin=0pt, labelsep=period, font=large, labelfont=bf]{caption}
%\usepackage{float}

\usepackage{hyperref}
\urlstyle{rm}   % change fonts for url's (from Chad Jones)
\hypersetup{
    colorlinks=true,        % kills boxes
%    allcolors=blue,
    pdfsubject={ECON-UB233, Macroeconomic foundations for asset pricing},
    pdfauthor={Dave Backus @ NYU},
    pdfstartview={FitH},
    pdfpagemode={UseNone},
%    pdfnewwindow=true,      % links in new window
%    linkcolor=blue,         % color of internal links
%    citecolor=blue,         % color of links to bibliography
%    filecolor=blue,         % color of file links
%    urlcolor=blue           % color of external links
% see:  http://www.tug.org/applications/hyperref/manual.html
}

% for listing code in tt font
\usepackage{verbatim}

% for table spacing
\usepackage{booktabs}

% section headers and spacing
\usepackage[small, compact]{titlesec}

% list spacing
\usepackage{enumitem}
\setitemize{leftmargin=*, topsep=0pt}
\setenumerate{leftmargin=*, topsep=0pt}

% attach files to the pdf
\usepackage{attachfile}
    \attachfilesetup{color=0.75 0 0.75}

\usepackage{needspace}
% \needspace{4\baselineskip} makes sure we have four lines available before a pagebreak

%\renewcommand{\thefootnote}{\fnsymbol{footnote}}
%\newcommand{\var}{\mbox{\it Var\/}}

\newcommand{\cbar}{\bar{c}}
\newcommand{\rp}{\mbox{\em rp\/}}
\newcommand{\tp}{\mbox{\em tp\/}}
\newcommand{\bp}{\mbox{\em bp\/}}


\printanswers

% document starts here
\begin{document}
\parskip=\bigskipamount
\parindent=0.0in
\thispagestyle{empty}
{\large ECON-UB 233 \hfill Dave Backus @ NYU}

\bigskip\bigskip
\centerline{\Large \bf Lab Report \#3: Consumption, Risk, \& Portfolio Choice}
\centerline{Revised: \today}

\bigskip
{\it Due at the start of class.
You may speak to others, but whatever you hand in should be your own work.
Please include your Matlab code.}

\begin{solution}
Brief answers follow,
but see also the attached Matlab program for calculations
related to Questions 3 and 4;
download the pdf, open, click on pushpin:
\attachfile{hw3_f14.m} \\
If this doesn't work for you, let me know and I'll explain how it works.
\end{solution}

\begin{questions}
%-----------------------------------------------------------------------
\question {\it Static risk and return.}
Consider an agent with utility
\begin{eqnarray*}
    U &=&  E [u(c)] ,   %\sum_z p(z) u[c(z)] ,
\end{eqnarray*}
where $u(c) = c^{1-\alpha}/(1-\alpha)$ for some risk aversion parameter $\alpha > 0$.
She invests one and consumes the gross return $r$.

\begin{parts}
\item What is her expected utility if she invests everything in a riskfree asset
whose (gross) return is $r = e^x$ for some constant $x>0$?
(Her consumption is therefore $e^x$ in every state.)
What is the certainty equivalent of this outcome?
\item What is her expected utility if she invests in an asset whose return is lognormal:
$ \log r \sim \mathcal{N}(\kappa_1,\kappa_2)$?
What is her certainty equivalent?
\item For what values of $\kappa_1$ and $\kappa_2$ is the risky asset preferred?
How does your answer depend on $\alpha$?
\end{parts}

\begin{solution}
\begin{parts}
\item Expected utility is $E [u(e^x)] = u(e^x)$.
The certainty equivalent of a sure thing is the sure thing,
which here is $e^x$.
More formally, $\mu$ solves
\begin{eqnarray*}
    U(\mu, \mu, \ldots, \mu) &=& u(\mu) \;\;=\;\; u(e^x),
\end{eqnarray*}
so the certainty equivalent is $e^x$.

\item We're using properties of lognormal random variables here.
We know $ E(r^{1-\alpha})= \exp[ (1-\alpha)\kappa_1 + (1-\alpha)^2\kappa_2/2]$.
The certainty equivalent is therefore
$ \mu^{1-\alpha} = E(r^{1-\alpha})$
or $\mu = E(r^{1-\alpha})^{1/(1-\alpha)} = \exp[\kappa_1 + (1-\alpha)\kappa_2/2]$.

\item Evidently we need
$ \kappa_1 + (1-\alpha)\kappa_2/2 > x $.
So large $\kappa_1 $ helps.  If $\alpha > 1$, small $\kappa_2$ helps, too,
otherwise the reverse.
[There's a small technical issue here:  the reason we need $\alpha > 1$,
rather than just $\alpha > 0$, is that increasing $\kappa_2$ here
increases the mean of $r$.  That's where the one comes from.
When $\alpha>1$ this effect is dominated by the impact on risk.]
\end{parts}
\end{solution}


%-----------------------------------------------------------------------
\item {\it Constrained optimization.\/}
Consider the problem:  choose $x$ and $y$ to maximize
\begin{eqnarray*}
    f(x,y) &=& 2 x^{1/2} + 4 y^{1/2}
\end{eqnarray*}
subject to $ x + y \leq 9$.
%
\begin{parts}
\item What is the Lagrangian associated with this problem?
\item What are the first-order conditions?
\item What values of $x$ and $y$ solve the problem?
What is the Lagrange multiplier?
\end{parts}
%
\begin{solution}
\begin{parts}
\item
The Lagrangian is
\begin{eqnarray*}
    \mathcal{L} &=& 2 x^{1/2} + 4 y^{1/2}   + \lambda (9 - x - y ).
\end{eqnarray*}
\item The first-order conditions are
\begin{eqnarray*}
    1/x^{1/2} &=& \lambda \\
    2/y^{1/2} &=& \lambda .
\end{eqnarray*}
\item The solution includes $ y = 4x$,  $x = 9/5$, $y = 36/5$, and $\lambda = {5}^{1/2}/3$.
% ?? cleaner if x + y leg 5 
\end{parts}
\end{solution}

%-----------------------------------------------------------------------
\question {\it Securities and returns.\/}
Consider our usual two-period event tree.
At date 0, we purchase one unit of security or asset $j$ (an arbitrary label) for price $q^j$.
At date 1, we get dividend $d^j(z)$, which depends on the state $z$.
Let us say, specifically,
that there are two securities and two states, with dividends

\medskip
\begin{center}
\begin{tabular}{lcc}
\toprule
Security  &  State 1 & State 2 \\
\midrule
1 (``bond'')    &  1    &   1  \\
2 (``equity'')  &  1    &   2  \\
\bottomrule
\end{tabular}
\end{center}

\medskip
The prices of the securities are $q^1 = 3/4$ (bond) and $q^e = 1$ (equity).

\begin{parts}
\item Why does it make sense for the bond to pay one in each state?
\item What are the (gross) returns on the assets?
\item An Arrow security pays one in a specific state, nothing in other states.
Here we have two states, hence two Arrow securities, whose dividends are

\bigskip
\begin{center}
\begin{tabular}{lcc}
\toprule
Security  &  State 1 & State 2 \\
\midrule
Arrow 1     &  1    &   0  \\
Arrow 2     &  0    &   1  \\
\bottomrule
\end{tabular}
\end{center}

\medskip
What quantities of bonds and equity reproduce the dividend of the second Arrow
security?

\item Securities can be thought of as collections of Arrow securities.
If we know the prices of Arrow securities, we can find the prices
of other securities by adding up the values of their dividends.
Here we do the reverse:  use the prices of bonds and equity to
find the prices $Q(z)$ of Arrow securities.
What are $Q(1)$ and $Q(2)$ here?
\end{parts}

\begin{solution}
\begin{parts}
\item A riskfree bond is riskfree because its dividend is the same
in all states.
Making the dividend one is simply a convention.

\item The bond return is $ r^1 = 1/(3/4) = 4/3 $ in all states.
The equity return is $r^e(z) = d^e(z)/q^e $ or
\begin{eqnarray*}
    r^e(z) &=&
        \left\{
        \begin{array}{ll}
        1/1 = 1 &  \mbox{for } z=1 \\
        2/1 = 2 &  \mbox{for } z=2
        \end{array}
        \right.
\end{eqnarray*}

\item If we buy one unit of equity and sell one unit of the bond,
we are left with a net dividend of zero in state 1 and one in state 2.
So we've replicated the second Arrow security.
The costs of this transaction is $ q^e - q^1 = 1 - 3/4 = 1/4 $,
so that should be its price:  $Q(2) = 1/4$.

\item We can also do the reverse:  combine Arrow securities to replicate
the dividends of the two assets.
If the assets and their replications sell for the same price,
we have
\begin{eqnarray*}
    3/4 &=&  Q(1) + Q(2) \\
     1   &=& Q(1) + 2 Q(2) .
\end{eqnarray*}
That gives us $Q(1) = 1/2$ and $Q(2) = 1/4$.
Lurking behind the scenes here is an arbitrage argument.
Why do the assets and their replications sell for the same price?
Because otherwise people would buy the cheaper one and sell the more expensive one,
giving them a riskless profit.
Markets are unlikely to let that happen:  they should eliminate pure arbitrage opportunities like this.

\end{parts}
\end{solution}


%-----------------------------------------------------------------------
\question {\it Portfolio choice.\/}
An investor must decide how to allocate his saving
between a riskfree bond and equity.
We approximate the world with three states,
each of which occurs with probability $1/3$.
The returns by state are

%\medskip
\begin{center}
\begin{tabular}{lccc}
\toprule
Security  &  State 1 & State 2 & State 3\\
\midrule
1 (``bond'')    &  1.1    &   1.1  &  1.1 \\
2 (``equity'')  &  0.6    &   1.2  &  1.6 \\
\bottomrule
\end{tabular}
\end{center}
%\medskip


The investor's problem is to choose current consumption $c_0$
and the fraction of saving $a$ to invest in equity to solve
\begin{eqnarray*}
   \max_{c_0, a} &&  u(c_0) + \beta \sum_z p(z) u[c_1(z)] \\
        s.t. &&  c_1(z)\;=\; (y_0-c_0)[(1-a) r^1 + a r^e(z)] .
\end{eqnarray*}
If $a>1$, the agent has a levered position, borrowing to fund investments in equity greater
than saving.
As usual, $u(c) = c^{1-\alpha}/(1-\alpha)$.
Where the problem calls for numbers, we'll use $\beta = 0.9$ and $\alpha = 2$.

\begin{parts}
\item What is the upper bound on $a$ consistent with positive consumption
$c_1(z)$ in all states $z$?
Lower bound?
\item What are the first-order conditions for $c_0$ and $a$?
{\it Comment:\/}  I recommend you substituting the expression for $c_1(z)$ into the utility function.
\item Use Matlab to solve the first-order condition for $a$ numerically.
What value of $a$ maximizes utility?
{\it Comment:\/}  I did this
by varying $a$ manually until its first-order condition was satisfied.
You could also compute the first-order condition for a grid of values for $a$,
and choose the one that comes closest to satisfying the first-order condition.
\end{parts}

\begin{solution}
\begin{parts}
\item We need $(1-a)r^1 + a r^e(z) > 0 $.
Only the extremes matter here, so in the bad state (state 1)
we need $ a < 2.2$ and in the good state (state 2)
we need $a > -2.2$.
If you find a solution to the first-order conditions outside this range,
it's not really a solution.

\item The first-order conditions are
\begin{eqnarray*}
    c_0:  &&  c_0^{-\alpha} \;\;=\;\; \beta \sum_z p(z)
%            u' \left\{  (y_0-c_0) [ (1-a)r^1 + a r^e(z)] \right\}
            c_1(z)^{-\alpha} [ (1-a)r^1 + a r^e(z)] \\
    a:    && 0 \;\;=\;\; \beta \sum_z p(z) \left\{(y_0-c_0)[(1-a) r^1 + a r^e(z)]\right\}^{-\alpha}
            [r^e(z) - r^1 ]  .
\end{eqnarray*}
The term in curly brackets is $c_1(z)$.
The second equation simplifies to
\begin{eqnarray*}
    0 &=& \sum_z p(z) [ (1-a)r^1 + a r^e(z)]^{-\alpha} [r^e(z) - r^1 ] ,
\end{eqnarray*}
which depends on $a$ but not $c_0$.

\item This problem doens't have a simple closed-form solution,
but we can crack it easily with Matlab.
We can automate this, and will later in the course,
but for now consider this iterative procedure:
\begin{itemize}
\item Pick a value of $a$.
\item Check the foc.  If it equals zero, we're done. If not, pick another $a$.
\end{itemize}
The solution is $a=0.107$.
\end{parts}
\end{solution}

\end{questions}

{\vfill
{\bigskip \centerline{\it \copyright \ \number\year \
David Backus $|$ NYU Stern School of Business}%
}}


\end{document}

