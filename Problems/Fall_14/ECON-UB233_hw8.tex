\documentclass[11pt]{exam}

\oddsidemargin=0.25truein \evensidemargin=0.25truein
\topmargin=-0.5truein \textwidth=6.0truein \textheight=8.75truein

\usepackage{comment}
\usepackage{verbatim}
\usepackage{booktabs}
\usepackage{graphicx}
\usepackage{hyperref}
\urlstyle{rm}   % change fonts for url's (from Chad Jones)
\hypersetup{
    colorlinks=true,        % kills boxes
    allcolors=blue,
    pdfsubject={ECON-UB233, Macroeconomic foundations for asset pricing},
    pdfauthor={Dave Backus @ NYU},
    pdfstartview={FitH},
    pdfpagemode={UseNone},
%    pdfnewwindow=true,      % links in new window
%    linkcolor=blue,         % color of internal links
%    citecolor=blue,         % color of links to bibliography
%    filecolor=blue,         % color of file links
%    urlcolor=blue           % color of external links
% see:  http://www.tug.org/applications/hyperref/manual.html
}

\usepackage{attachfile}
    \attachfilesetup{color=0.5 0 0.5}

\printanswers

% document starts here
\begin{document}
\parskip=\bigskipamount
\parindent=0.0in
\thispagestyle{empty}
{\large ECON-UB 233 \hfill Dave Backus @ NYU}

\bigskip\bigskip
\centerline{\Large \bf Lab Report \#8:  Bond Prices \& Predictable Returns}
\centerline{Revised: \today}

\bigskip
{\it Due at the start of class.
You may speak to others, but whatever you hand in should be your own work.
Please include your Matlab code.}


\begin{questions}
%-----------------------------------------------------------------------
\question *** Do long horizon results...  

\question {\it Equity prices and dividends.\/}
Suppose the ex-dividend price of equity is
\begin{eqnarray}
    q_t &=& \delta E_t \left( d_{t+1} + q_{t+1} \right)
    \label{eq:forward}
\end{eqnarray}
with discount factor $ 0 < \delta < 1 $.
\begin{parts}
\item Express the price as a function of expected future dividends.
\item Suppose dividends follow
\begin{eqnarray*}
    d_{t+1} &=& (1-\varphi) \mu + \varphi d_t + \sigma( w_{t+1} + \theta w_t ) ,
\end{eqnarray*}
where $\{ w_t \} $ is a sequence of independent standard normal random variables.
What definition of the state is enough to describe the conditional distribution of $d_{t+1}$ at date $t$?
\item How is the price $q_t$ related to the state?
\item Optional, extra credit.  What are the variances of $q$ and $d$?
How do they relate to Shiller's observation that prices are more variable
than dividends?
\end{parts}

\begin{solution}
\begin{parts}
\item Repeated substitution gives us
\begin{eqnarray*}
    q_t &=& \sum_{j=1}^\infty \delta^j E(d_{t+j}) .
\end{eqnarray*}
\item The date-$t$ state for an ARMA(1,1) can be expressed by $z_t = (d_t, w_t)$.
The conditional distribution of $d_{t+1}$ is normal with mean and variance
\begin{eqnarray*}
    E_t (d_{t+1}) &=& (1-\varphi) \mu + \varphi d_t + \sigma \theta w_t \\
    \mbox{Var}_t (d_{t+1}) &=& \sigma^2 .
\end{eqnarray*}
\item We'll use the method of undetermined coefficients.
If we guess $q_t = a + b d_t + c w_t $ for coefficients $(a,b,c)$ to be determined,
then the elements of (\ref{eq:forward}) are
\begin{eqnarray*}
    q_t &=& a + b d_t + c w_t \\
    E_t (q_{t+1}) &=& a + b [(1-\varphi) \mu + \varphi d_t + \sigma \theta w_t ] \\
    E_t (d_{t+1}) &=& (1-\varphi) \mu + \varphi d_t + \sigma \theta w_t .
\end{eqnarray*}
Substituting into (\ref{eq:forward}) and lining up coefficients gives us
\begin{eqnarray*}
    a &=& \delta \big[ a + (1+b)(1-\varphi) \mu \big] \\
    b &=& \delta \big( b \varphi + \varphi \big) \\
    c &=& \delta (b \sigma \theta + \sigma \theta) .
\end{eqnarray*}
The second equation gives us $b= \delta\varphi / (1-\delta\varphi) $,
and therefore $1+b = 1/(1-\delta\varphi)$.
The first and third then give us
\begin{eqnarray*}
    a &=& \frac{(1-\varphi)\mu\delta}{(1-\delta)(1-\delta\varphi)} ,
    \;\;\;\;\; c \;\;=\;\; \frac{\sigma\theta\delta}{1-\delta\varphi} .
\end{eqnarray*}

\end{parts}
\end{solution}


%-----------------------------------------------------------------------
\item {\it Bond basics.\/}
Consider the following bond prices at some date $t$:

\medskip
\begin{center}
\tabcolsep=0.1in
\begin{tabular}{lr}
\toprule
Maturity $n$   & \phantom{xx} Price $q^n$ \\ %& \phantom{xx} Yield $y^n$ & Forward $f^{n-1}$ \\
\midrule
0           &    1.0000   \\
1 year      &    0.9704   \\
2 years     &    0.9324   \\
3 years     &    0.8914   \\
4 years     &    0.8479   \\
5 years     &    0.8065   \\
\bottomrule
\end{tabular}
\end{center}
\medskip

\begin{parts}
\item What are the yields $y^n$?
\item What are the forward rates $f^{n-1}$?
\item How are the yields and forward rates related?  Verify for $y^3$.
\end{parts}

\begin{solution}
\begin{center}
\tabcolsep=0.1in
\begin{tabular}{lrrr}
\toprule
Maturity $n$   & \phantom{xx} Price $q^n$ & \phantom{xx} Yield $y^n$ & Forward $f^{n-1}$ \\
\midrule
0           &    1.0000  \\
1 year      &    0.9704  & 0.0300 & 0.0300 \\
2 years     &    0.9324  & 0.0350 & 0.0400 \\
3 years     &    0.8914  & 0.0383 & 0.0450 \\
4 years     &    0.8479  & 0.0413 & 0.0500 \\
5 years     &    0.8065  & 0.0430 & 0.0500 \\
\bottomrule
\end{tabular}
\end{center}
%\medskip
See the attached Matlab program;
download the pdf, open, click on pushpin:
%\attachfile{../Matlab/hw/hw8_f13.m}

\begin{parts}
\item See above.
\item See above.
\item Yields are averages of forward rates:
\begin{eqnarray*}
    y^n_t &=& n^{-1} \sum_{j=1}^n f^{j-1}_t .
\end{eqnarray*}
Thus $y^3 = (0.0300 + 0.0400 +  0.0450)/3 = 0.0383$.
\end{parts}
\end{solution}

%-----------------------------------------------------------------------
\item {\it Moving average bond pricing.\/}
Consider the bond pricing model
\begin{eqnarray*}
    \log m_{t+1} &=& - \lambda^2/2 - x_t + \lambda w_{t+1} \\
    x_t &=& \delta + \sigma (w_t + \theta w_{t-1}) .
\end{eqnarray*}

\begin{parts}
\item What is the short rate $f^0_t$?
\item Suppose bond prices take the form
\begin{eqnarray*}
    \log q^n_{t} &=& A_n + B_n w_t + C_n w_{t-1} .
\end{eqnarray*}
Use the pricing relation to derive recursions connecting
$(A_{n+1}, B_{n+1}, C_{n+1})$ to $(A_{n}, B_{n}, C_{n})$.
What are $(A_{n}, B_{n}, C_{n})$ for $n=0,1,2,3$?
\item Express forward rates as functions of the state $(w_t,w_{t-1}$).
What are $f^1_t$ and $f^2_t$?
\item What is $E(f^1- f^0)$?  What parameters govern its sign?
\end{parts}

\begin{solution}
\begin{parts}
\item The short rate is
\begin{eqnarray*}
    f^0_t &=& - \log E_t m_{t+1} \;\;=\;\;  \lambda^2/2 + x_t - \lambda^2/2
            \;\;=\;\; x_t .
\end{eqnarray*}
The second equality is the usual ``mean plus variance over two'' with the sign flipped
(as indicated by the first equality).
In other words:  the usual setup.
In what follows, we'll kill off $x_t$ by substituting.
\item
Bond prices follow from the pricing relation,
\begin{eqnarray*}
    q_t^{n+1} &=& E_t (m_{t+1} q_{t+1}^n) ,
\end{eqnarray*}
starting with $n=0$ and  $q^0_t = 1$.
The state in this case is $(w_t, w_{t-1})$, a simple example of a
two-dimensional model, hence the extra term in the form of the bond price.
We need
\begin{eqnarray*}
    \log (m_{t+1} q_{t+1}^n) &=&
            A_n - (\lambda^2/2 + \delta) + (\lambda+B_n) w_{t+1}
            + (C_n-\sigma) w_t - \sigma \theta w_{t-1} .
\end{eqnarray*}
The (conditional) mean and variance are
\begin{eqnarray*}
    E_t [ \log (m_{t+1} q_{t+1}^n)] &=&
            A_n - (\lambda^2/2 + \delta)
            + (C_n-\sigma) w_t - \sigma \theta w_{t-1} \\
    \mbox{Var}_t [\log (m_{t+1} q_{t+1}^n)] &=&
            (\lambda+B_n)^2 .
\end{eqnarray*}
Using ``mean plus variance over two'' and lining up terms gives us
\begin{eqnarray*}
        A_{n+1} &=& A_n - (\lambda^2/2 + \delta) + (\lambda+B_n)^2/2 \\
                &=& A_n - \delta + \lambda B_n + (B_n)^2/2 \\
        B_{n+1} &=& C_n - \sigma \\
        C_{n+1} &=& - \sigma \theta
\end{eqnarray*}
for $n=0,1,2,\ldots$.
That gives us
\begin{center}
\begin{tabular}{cccc}
        $n$  & $A_n$ & $B_n$ & $C_n$ \\
        \midrule
        0   &  0 & 0 & 0 \\
        1   &  $-\delta$ & $-\sigma$ & $-\sigma\theta$  \\
        2   &  $-2 \delta - \lambda\sigma + \sigma^2/2$   &  $-\sigma(1+\theta)$ & $-\sigma\theta$ \\
        3   &  X  &  $-\sigma(1+\theta)$ & $-\sigma\theta$
\end{tabular}
\end{center}
with
$X = - 3 \delta - \lambda (2+\theta)+ [1 + (1+\theta)^2] \sigma^2/2 $.

\item In general, forward rates are
\begin{eqnarray*}
    f^n_t &=& (A_n - A_{n+1}) + (B_n - B_{n+1}) w_t + (C_n - C_{n+1}) w_{t-1} .
\end{eqnarray*}
That gives us
\begin{eqnarray*}
    f^0_t &=& \delta + \sigma w_t + \sigma \theta  w_{t-1} \\
    f^1_t &=& \delta + \lambda\sigma - \sigma^2/2 + \sigma \theta w_t  \\
    f^2_t &=& \delta - (1+\theta)^2 \sigma^2/2 + \lambda \sigma (1+\theta) .
\end{eqnarray*}

\item The means are the same with $w_t = w_{t-1} = 0$, their mean.
Therefore
\begin{eqnarray*}
   E (f^1 - f^0) &=& \lambda \sigma - \sigma^2/2 .
\end{eqnarray*}
Therefore we need $\lambda\sigma > \sigma^2/2 $,
so a necessary condition is that $\lambda$ and $\sigma$ have the same sign.

\end{parts}
\end{solution}




\end{questions}

\vfill \centerline{\it \copyright \ \number\year \ NYU Stern School of Business}

\end{document}


