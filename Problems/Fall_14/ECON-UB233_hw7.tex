\documentclass[11pt]{exam}

% spacing on page
\oddsidemargin=0.25truein \evensidemargin=0.25truein
\topmargin=-0.5truein \textwidth=6.0truein \textheight=8.75truein

\usepackage{comment}
\usepackage{graphicx}
\usepackage{amssymb}
\usepackage{amsmath}

\usepackage[margin=0pt, labelsep=period, font=large, labelfont=bf]{caption}
%\usepackage{float}

\usepackage{hyperref}
\urlstyle{rm}   % change fonts for url's (from Chad Jones)
\hypersetup{
    colorlinks=true,        % kills boxes
%    allcolors=blue,
    pdfsubject={ECON-UB233, Macroeconomic foundations for asset pricing},
    pdfauthor={Dave Backus @ NYU},
    pdfstartview={FitH},
    pdfpagemode={UseNone},
%    pdfnewwindow=true,      % links in new window
%    linkcolor=blue,         % color of internal links
%    citecolor=blue,         % color of links to bibliography
%    filecolor=blue,         % color of file links
%    urlcolor=blue           % color of external links
% see:  http://www.tug.org/applications/hyperref/manual.html
}

% for listing code in tt font
\usepackage{verbatim}

% for table spacing
\usepackage{booktabs}

% section headers and spacing
\usepackage[small, compact]{titlesec}

% list spacing
\usepackage{enumitem}
\setitemize{leftmargin=*, topsep=0pt}
\setenumerate{leftmargin=*, topsep=0pt}

% attach files to the pdf
\usepackage{attachfile}
    \attachfilesetup{color=0.75 0 0.75}

\usepackage{needspace}
% \needspace{4\baselineskip} makes sure we have four lines available before a pagebreak

%\renewcommand{\thefootnote}{\fnsymbol{footnote}}
%\newcommand{\var}{\mbox{\it Var\/}}

\newcommand{\cbar}{\bar{c}}
\newcommand{\rp}{\mbox{\em rp\/}}
\newcommand{\tp}{\mbox{\em tp\/}}
\newcommand{\bp}{\mbox{\em bp\/}}


\printanswers

% document starts here
\begin{document}
\parskip=\bigskipamount
\parindent=0.0in
\thispagestyle{empty}
{\large ECON-UB 233 \hfill Dave Backus @ NYU}

\bigskip\bigskip
%\centerline{\Large \bf Lab Report \#7:  Stochastic Processes}
\centerline{\Large \bf Lab Report \#7:  Dynamics in Theory and Data}
\centerline{Revised: \today}

\bigskip
{\it Due at the start of class.
You may speak to others, but whatever you hand in should be your own work.
%Please include your Matlab code.
}

%\begin{solution}
%Brief answers follow,
%but see also the attached Matlab program;
%download the pdf, open, click on pushpin:
%%\attachfile{../Matlab/hw/hw7_f14.m}
%\end{solution}

\begin{questions}
%-----------------------------------------------------------------------
\question {\it ARMA(1,1) models.\/}
Consider the models
\begin{eqnarray*}
    x_t &=& \sum_{j=0}^\infty a_j w_{t-j}
\end{eqnarray*}
with iid standard normal innovations $w_t$
and coefficients $a_0$, $a_1$, and $a_{j+1} = \varphi a_j $
for $j \geq 1$ and parameter $0 < \varphi < 1$.
\begin{parts}
\item What is the variance of $x$?
\item What is the covariance of $x_t$ and $x_{t-1}$?
\item What is the autocovariance function?
The autocorrelation function?
\item What configuration of parameter values gives us negative autocorrelations?
\item {\it Extra credit.\/}
Show that the model can be expressed in traditional ARMA(1,1)
form,
\begin{eqnarray*}
    x_t &=& \varphi x_{t-1} + \sigma (w_t + \theta w_{t-1} ) .
\end{eqnarray*}

\end{parts}


\begin{solution}
\begin{parts}
\item This follows most naturally from the moving average formula.
Because the $w$'s are independent, the cross terms are zero and we're left with the squares:  
\begin{eqnarray*}
    \mbox{Var}(x) &=& E \Big[ \Big( \sum_{j=0}^\infty a_j w_{t-j}\Big)^2 \Big] 
                \;\;=\;\; \sum_{j=0}^\infty  E \big(  a_j w_{t-j}\big)^2  \\
                &=& \sum_{j=0}^\infty  a_j^2 
                \;\;=\;\; a_0^2 + a_1^2 /(1-\varphi^2) .
\end{eqnarray*}
\item The covariance is similar
\begin{eqnarray*}
    \mbox{Cov}(x_t, x_{t-1}) &=& E \Big[ \Big( \sum_{j=0}^\infty a_j w_{t-j}\Big)
                    \Big( \sum_{j=0}^\infty a_j w_{t-j-1}\Big) \Big] \\
                &=& \sum_{j=0}^\infty  a_j a_{j+1} 
                \;\;=\;\; a_0 a_1 + a_1 \varphi/(1-\varphi^2) .
\end{eqnarray*}
\item The autocovariance function is 
\begin{eqnarray*}
    \gamma(0) &=& a_0^2 + a_1^2 /(1-\varphi^2)  \\
    \gamma(1) &=& a_0 a_1 + a_1 \varphi/(1-\varphi^2)  \\
    \gamma(k+1) &=& \varphi \gamma(k) \mbox{ for } j\geq 1.  
\end{eqnarray*}
The autocorrelation function is $ \rho(k) = \gamma(k)/\gamma(0)$.  
Like the moving average coefficients, the first two are arbitrary, then they decay at rate $\varphi$. 

\item Autocorrelations and autocovariances have the same sign.
Looking at $\gamma(0) = \mbox{Cov}(x_t,x_{t-1})$, we see that 
it's sufficient to have $a_0 > 0$ and $a_1 < 0$.  

\item If we subtract $\varphi x_{t-1}$ from $x_t$ we get
\begin{eqnarray*}
    x_t - \varphi x_{t-1}  &=& \sum_{j=0}^\infty a_j w_{t-j} - \varphi \sum_{j=0}^\infty a_j w_{t-j-1} \\
            &=& a_0 w_t + (a_1 - \varphi a_0) w_{t-1} + (a_2 - \varphi a_1) w_{t-2} + \cdots \\
            &=& a_0 w_t + (a_1 - \varphi a_0) w_{t-1} ,
\end{eqnarray*}
which is an ARMA(1,1).  

\end{parts}
\end{solution}

%-----------------------------------------------------------------------
\question {\it Forward-looking equity prices.\/}
Suppose equity prices are given by the forward-looking difference equation
\begin{eqnarray*}
    q_{t}   &=&  d_t + \delta E_t (q_{t+1})
\end{eqnarray*}
where $E_t$ means the expectation conditional on the state at date $t$.
%
\begin{parts}
\item How is the price related to future dividends?
(Ignore bubbles here.)
\item Take the AR(1) state variable
\begin{eqnarray*}
    x_t &=& \varphi x_{t-1} + \sigma w_t ,
\end{eqnarray*}
with the usual iid standard normal $w's$.
What is the price of equity if $d_t = x_t$?

\item What is the price of equity if $d_t = x_{t-1}$?
\end{parts}

\begin{solution}
\begin{parts}
\item Repeated substitution gives us 
\begin{eqnarray*}
    q_{t}   &=&  d_t + \delta E_t (d_{t+1}) + \delta E_t (d_{t+2}) + \cdots 
            \;\;=\;\; \sum_{j=0}^\infty \delta^j E_t (d_{t+j}) .
\end{eqnarray*}
In words:  the price is the discounted value of expected future dividends.  

\item Here expected future dividends are 
\begin{eqnarray*} 
    E_t (d_{t+j}) &=& E_t (x_{t+j}) \;\;=\;\; \varphi^j x_t , 
\end{eqnarray*} 
which implies the equity price $q_t = x_t /(1-\delta \varphi) $.

\item The one-period lag is a rough approximation to observed corporate dividend policy, 
which tends to lag performance.  
In that case, we have 
\begin{eqnarray*}
    q_t  &=& x_{t-1} + \delta x_t + \delta \varphi x_t + \cdots 
            \;\;=\;\; x_{t-1} + \delta x_t /(1-\delta \varphi) .
\end{eqnarray*}

\end{parts}
\end{solution}

%-----------------------------------------------------------------------
\begin{table}[h]
\centering
\tabcolsep=0.1in
\begin{tabular}{lr}
\toprule
Maturity $n$   & \phantom{xx} Price $q^n$ \\ %& \phantom{xx} Yield $y^n$ & Forward $f^{n-1}$ \\
\midrule
%0           &    1.0000   \\
1 year      &    0.9800   \\
2 years     &    0.9600   \\
3 years     &    0.9400   \\
4 years     &    0.9200   \\
5 years     &    0.9000   \\
\bottomrule
\end{tabular}
\caption{Bond prices.}
\label{tab:bonds}
\end{table}

%\pagebreak
\item {\it Bond basics.\/}
Consider the bond prices in Table \ref{tab:bonds}.
%
\begin{parts}
\item What are the yields $y^n$?
\item What are the forward rates $f^{n-1}$?
\item How are the yields and forward rates related?  Verify for $y^3$.
\end{parts}

\begin{solution}
\begin{center}
\tabcolsep=0.1in
\begin{tabular}{lrrr}
\toprule
Maturity $n$   & \phantom{xx} Price $q^n$ & Forward $f^{n-1}$ & \phantom{xx} Yield $y^n$ \\
\midrule
0           &    1.0000  \\
1 year      &    0.9800  & 0.0202 & 0.0202 \\
2 years     &    0.9600  & 0.0206 & 0.0204 \\
3 years     &    0.9400  & 0.0211 & 0.0206 \\
4 years     &    0.9200  & 0.0215 & 0.0208 \\
5 years     &    0.9000  & 0.0220 & 0.0211 \\
\bottomrule
\end{tabular}
\end{center}
%\medskip
See the attached Matlab program;
download the pdf, open, click on pushpin:
\attachfile{hw7_f14.m}

\begin{parts}
\item [(a,b)] Yields and forward rates are connected to bond prices by 
\begin{eqnarray*}
    y^n_t &=& - n^{-1} \log q^n_t \\
    f^n_t &=& \log (q^{n}_t/q^{n+1}_t) .
\end{eqnarray*} 
See the table for the numbers.  
\item [(c)] Yields are averages of forward rates:
\begin{eqnarray*}
    y^n_t &=& n^{-1} \sum_{j=1}^n f^{j-1}_t .
\end{eqnarray*}
Thus $y^3 = (0.0202 + 0.0206 +  0.0211)/3 = 0.0206$.
\end{parts}
\end{solution}


\end{questions}

{\vfill
{\bigskip \centerline{\it \copyright \ \number\year \
David Backus $|$ NYU Stern School of Business}%
}}


\end{document}

