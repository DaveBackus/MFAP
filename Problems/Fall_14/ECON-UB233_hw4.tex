\documentclass[11pt]{exam}

% spacing on page
\oddsidemargin=0.25truein \evensidemargin=0.25truein
\topmargin=-0.5truein \textwidth=6.0truein \textheight=8.75truein

\usepackage{comment}
\usepackage{graphicx}
\usepackage{amssymb}
\usepackage{amsmath}

\usepackage[margin=0pt, labelsep=period, font=large, labelfont=bf]{caption}
%\usepackage{float}

\usepackage{hyperref}
\urlstyle{rm}   % change fonts for url's (from Chad Jones)
\hypersetup{
    colorlinks=true,        % kills boxes
%    allcolors=blue,
    pdfsubject={ECON-UB233, Macroeconomic foundations for asset pricing},
    pdfauthor={Dave Backus @ NYU},
    pdfstartview={FitH},
    pdfpagemode={UseNone},
%    pdfnewwindow=true,      % links in new window
%    linkcolor=blue,         % color of internal links
%    citecolor=blue,         % color of links to bibliography
%    filecolor=blue,         % color of file links
%    urlcolor=blue           % color of external links
% see:  http://www.tug.org/applications/hyperref/manual.html
}

% for listing code in tt font
\usepackage{verbatim}

% for table spacing
\usepackage{booktabs}

% section headers and spacing
\usepackage[small, compact]{titlesec}

% list spacing
\usepackage{enumitem}
\setitemize{leftmargin=*, topsep=0pt}
\setenumerate{leftmargin=*, topsep=0pt}

% attach files to the pdf
\usepackage{attachfile}
    \attachfilesetup{color=0.75 0 0.75}

\usepackage{needspace}
% \needspace{4\baselineskip} makes sure we have four lines available before a pagebreak

%\renewcommand{\thefootnote}{\fnsymbol{footnote}}
%\newcommand{\var}{\mbox{\it Var\/}}

\newcommand{\cbar}{\bar{c}}
\newcommand{\rp}{\mbox{\em rp\/}}
\newcommand{\tp}{\mbox{\em tp\/}}
\newcommand{\bp}{\mbox{\em bp\/}}


\printanswers

% document starts here
\begin{document}
\parskip=\bigskipamount
\parindent=0.0in
\thispagestyle{empty}
{\large ECON-UB 233 \hfill Dave Backus @ NYU}

\bigskip\bigskip
\centerline{\Large \bf Lab Report \#4: Asset Pricing Fundamentals}
\centerline{Revised: \today}

\bigskip
{\it Due at the start of class.
You may speak to others, but whatever you hand in should be your own work.
Please include your Matlab code.}

\begin{solution}
Brief answers follow,
but see also the attached Matlab program.
Download this document as a pdf, open it, and click on the pushpin:
\attachfile{hw4_f14.m}
\end{solution}


\begin{questions}
% -----------------------------------------------------------------------------
\question {\it State prices and related objects.\/}
Consider an economy with three states.
State prices and probabilities are

\begin{center}
\begin{tabular}{cccc}
\toprule
        & State Price   & Probability &  Dividend  \\
State $z$  &  $Q(z)$    & $p(z)$      &   $d(z)$   \\
\midrule
1       & 1/2 & 1/3 & 1 \\
2       & 1/3 & 1/3 & 2 \\
3       & 1/4 & 1/3 & 3 \\
\bottomrule
\end{tabular}
\end{center}

\smallskip
\begin{parts}
\item What is the pricing kernel in each state?
\item What is the price of a one-period bond?  What is its return?
\item What are the risk-neutral probabilities?
Why are they different from the true probabilities?
\item Suppose equity is a claim to the dividend in the last column.
What is its price? What is the return on equity in each state?
\item What is the expected return on equity?
The risk premium?
%What accounts for the sign of the risk premium?
\end{parts}

\begin{solution}
\vspace*{-0.2in}
\begin{center}
\begin{tabular}{ccccccc}
\toprule
        & State Price   & Probability &  Dividend & Pr Kernel & R-n probs & Return \\
State $z$  &  $Q(z)$    & $p(z)$      &   $d(z)$  &  $m(z)$   &  $p^*(z)$ & $r^e(z)$ \\
\midrule
1       & 1/2 & 1/3 & 1 & 3/2 & 0.4615 & 0.5217 \\
2       & 1/3 & 1/3 & 2 &  1  & 0.3077 & 1.0435 \\
3       & 1/4 & 1/3 & 3 &  3/4& 0.2308 & 1.5652 \\
\bottomrule
\end{tabular}
\end{center}

\begin{parts}
\item See the table.
\item $q^1 = 1.0833$, $r^1 = 0.9231$.
\item See table.  They are a combination of the true probabilities $p$
and the pricing kernel $m$, scaled to sum to one.
\item The price is $q^e = 1.9167$.  The returns are in the table.
\item The expected return is $ E(r^e) = 1.0435 $ and the risk
premium is $ E(r^e - r^1) = 0.1204$.
Why positive?  You'll note that the asset's dividends are high when the pricing kernel is low,
which makes the price low.
That raises returns.
\end{parts}
\end{solution}


% -----------------------------------------------------------------------------
\question {\it Pricing kernels and risk-neutral probabilities with Poisson risk.\/}
Consider a representative agent economy with a power utility agent facing ``Poisson risk.''
Utility is $u(c) = c^{1-\alpha}/(1-\alpha) $ with risk aversion
parameter $\alpha > 0$.
Log consumption growth is $ z = \log g = \log c_1 - \log c_0 $
for $ z = 0,1,2, \ldots$ with probabilities
$ p(z) = e^{-\omega} \omega^z/z! $ for some parameter $\omega>0$.

You may want to use the power series expansion
\begin{eqnarray*}
    e^x &=& 1 + x + x^2/2! + x^3/3! + \cdots
            \;\;=\;\; \sum_{j=0}^\infty x^j/j! ,
\end{eqnarray*}
which is sometimes used to define the exponential function $e^x$.
%
\begin{parts}
\item What is the pricing kernel $m(z)$ in each state $z$?
\item What are the state prices $Q(z)$?
\item What are the risk-neutral probabilities $p^*(z)$?
\item Show that the risk-neutral distribution is Poisson.
\item How do the risk-neutral probabilities $p^*(z)$
differ from the true probabilities $p(z)$?
\item Set $\omega = 4$ and $ \alpha = 1$
and plot $p(z)$ and $p^*(z)$ for $z$ between zero and 10.
How do they differ?  Why?
\end{parts}
{\it Matlab mini-tutorial on bar charts.\/}
Suppose we have vectors {\tt z}, {\tt p}, and {\tt pstar}.
We can plot probabilities against {\tt z} with the commands
\begin{verbatim}
bar(z, p)           % just p
bar(z, [p pstar])   % p and pstar together
\end{verbatim}
The order of inputs in Matlab plot commands is
x variable first (horizontal axis), then the y variable (vertical axis):
{\tt plot(x,y)}, {\tt bar(x,y)}, etc.

\begin{solution}
\begin{parts}
\item The pricing kernel is $ m(z) = \beta e^{-\alpha x} $.
\item State prices are
\begin{eqnarray*}
    Q(z) &=& p(z) m(z) \;\;=\;\; e^{-\omega} (\omega^z/z!) \beta e^{-\alpha z}
            \;\;=\;\; e^{-\omega} \beta (\omega e^{-\alpha})^z /z!.
\end{eqnarray*}
\item Risk-neutral probabilities are $ p^*(z) = Q(z)/q^1 = p(z) m(z) / q^1 $.
Here we have
\begin{eqnarray*}
    q^1 &=&  \sum_{z=0}^\infty Q(z)
            \;\;=\;\; e^{-\omega} \beta \; \sum_{z=0}^\infty (\omega e^{-\alpha})^z /z!
            \;\;=\;\; e^{-\omega} \beta e^{\omega e^{-\alpha}} .
\end{eqnarray*}
%$q^1 = E(m) = (1-\omega) \beta /(1-\omega e^{-\alpha}) $.
Risk-neutral probabilities are therefore
\begin{eqnarray*}
    p^*(z) &=&  Q(z) / q^1
            \;\;=\;\;  e^{-\omega e^{-\alpha}} (\omega e^{-\alpha})^z /z!.
\end{eqnarray*}

\item We can write this as
$ p^*(z) = e^{-\omega^*} (\omega^*)^z /z! $
with $ \omega^* = \omega e^{-\alpha} $, so it has the form of a Poisson distribution.

\item The difference lies in a comparison of $\omega$ and $\omega^*$.
If $\alpha = 0$ (zero risk aversion), the two are the same.
Otherwise, this has the usual flavor of representative agent economies:
the risk-neutral probabilities place more weight on bad states.

\item We see exactly this in the figure generated by the Matlab program.
\end{parts}
\end{solution}


% -----------------------------------------------------------------------------
\question {\it Option prices.\/}
We're going to explore the risk-neutral approach to valuing options.
A call option gives the owner the right to purchase an asset one period from now
at a price $k$ --- the so-called {\it strike price\/} --- determined now.
If the future price is $s(z)$, the owner will {exercise}
the option and purchase the stock only if $s$ is greater than (or equal to?) $k$.
That gives rise to the option cash flow
\begin{eqnarray*}
        d(z) &=&  \max \{0, s(z) - k \} .
\end{eqnarray*}
Given this cash flow, we value the option as we would any other asset.
We'll use specifically the risk-neutral valuation equation
\begin{eqnarray*}
        q^c &=&  q^1 \sum_z p^*(z) d(z)
                \;\;=\;\; q^1 \sum_z p^*(z)  \max \{0, s(z) - k \} ,
\end{eqnarray*}
where $q^c$ is the price of the call option and $q^1$ is the price
of a one-period riskfree bond.

The risk-neutral environment we'll work with is lognormal.
A state $z$ has a standard normal risk-neutral distribution.
The log of the future price is connected to the state (again, in risk-neutral terms)
by $ \log s(z) = \mu + \sigma z $.

One last thing:
we're going to use a discrete approximation to the distribution of $z$,
which is easier to work with than the real thing
(numerical integration is neither pretty nor efficient).
In Matlab terms, set up a grid of points for $z$ and assign probabilities
to them from the standard normal pdf:
\begin{verbatim}
zmax = 4;
dz = 0.5;
z = [-zmax:dz:zmax]';
pstar = exp(-z.^2/2)*dz/sqrt(2*pi);
\end{verbatim}
We can make this approximation as close to the original as we want
by shrinking {\tt dz}.

\begin{parts}
\item One check on the approximation is the sum of the probabilities.
Do they sum to one here?

\item Set up a similar grid of values for $s(z)$.
Use $q^1 = 0.95$, $\sigma = 0.1$, and
\begin{eqnarray*}
    \mu &=&  \log(100/q1) - \sigma^2/2 .
\end{eqnarray*}
More on this later.
What value of $\mu$ do you get?

\item Compute the cash flows $d(z)$ for an option
with strike price $k=110$.
Use the risk-neutral pricing equation to compute its value.

You may find these Matlab commands helpful:
\begin{verbatim}
d_positive = s >= k
d = d_positive.*(s-k);
\end{verbatim}
The first line generates a vector that equals one if $ s \geq k$ and zero otherwise.

\item We usually require the risk-neutral probabilities to value the underlying
asset correctly, a requirement we call the {\it no-arbitrage condition\/}.
With this in mind, apply the risk-neutral pricing equation to the asset itself,
a claim (so to speak) to the value of the asset one period from now:
$d(z) = s(z)$.
What is the value of this asset?
Can you connect it to the condition we used to derive $\mu$?
\end{parts}

\begin{solution}
\begin{parts}
\item To four digits:  1.0000. 

\item $\mu = 4.6515$. 

\item The cash flows are $ d(z) = \max \{ 0, s(z) - k \} $.
Its value is the usual:  
\begin{eqnarray*}
    q &=& q^1 \sum_z p^*(z) d(z) .
\end{eqnarray*} 
In this case we have $ q = 2.1417 $. 

\item If the future price is $s(z)$, 
then the current price follows from the usual pricing formula.
Here we find the price is 99.9981.  
If we were to do this exactly, we'd get 100.
The next part is a little obscure, but we've seen it before and will see it again.
The idea is to choose the risk-neutral distribution so that it's consistent
with the (known) price of the underlying asset.  
If the price is 100, the formula in part (b) sets $\mu$ to reproduce
this value:  what we have called the no-arbitrage condition.
See the discussion in the notes.  
\end{parts} 
\end{solution}

\end{questions}

{\vfill
{\bigskip \centerline{\it \copyright \ \number\year \
David Backus $|$ NYU Stern School of Business}%
}}


\end{document}


