\documentclass[11pt]{exam}

% spacing on page
\oddsidemargin=0.25truein \evensidemargin=0.25truein
\topmargin=-0.5truein \textwidth=6.0truein \textheight=8.75truein

\usepackage{comment}
\usepackage{graphicx}
\usepackage{amssymb}
\usepackage{amsmath}

\usepackage[margin=0pt, labelsep=period, font=large, labelfont=bf]{caption}
%\usepackage{float}

\usepackage{hyperref}
\urlstyle{rm}   % change fonts for url's (from Chad Jones)
\hypersetup{
    colorlinks=true,        % kills boxes
%    allcolors=blue,
    pdfsubject={ECON-UB233, Macroeconomic foundations for asset pricing},
    pdfauthor={Dave Backus @ NYU},
    pdfstartview={FitH},
    pdfpagemode={UseNone},
%    pdfnewwindow=true,      % links in new window
%    linkcolor=blue,         % color of internal links
%    citecolor=blue,         % color of links to bibliography
%    filecolor=blue,         % color of file links
%    urlcolor=blue           % color of external links
% see:  http://www.tug.org/applications/hyperref/manual.html
}

% for listing code in tt font
\usepackage{verbatim}

% for table spacing
\usepackage{booktabs}

% section headers and spacing
\usepackage[small, compact]{titlesec}

% list spacing
\usepackage{enumitem}
\setitemize{leftmargin=*, topsep=0pt}
\setenumerate{leftmargin=*, topsep=0pt}

% attach files to the pdf
\usepackage{attachfile}
    \attachfilesetup{color=0.75 0 0.75}

\usepackage{needspace}
% \needspace{4\baselineskip} makes sure we have four lines available before a pagebreak

%\renewcommand{\thefootnote}{\fnsymbol{footnote}}
%\newcommand{\var}{\mbox{\it Var\/}}

\newcommand{\cbar}{\bar{c}}
\newcommand{\rp}{\mbox{\em rp\/}}
\newcommand{\tp}{\mbox{\em tp\/}}
\newcommand{\bp}{\mbox{\em bp\/}}


\printanswers

% document starts here
\begin{document}
\parskip=\bigskipamount
\parindent=0.0in
\thispagestyle{empty}
{\large ECON-UB 233 \hfill Dave Backus @ NYU}

\bigskip\bigskip
\centerline{\Large \bf Lab Report \#6: Options \& Volatility}
\centerline{Revised: \today}

\bigskip
{\it Due at the start of class.
You may speak to others, but whatever you hand in should be your own work.
Please include your Matlab code.
}

\begin{solution}
Brief answers follow,
but see also the attached Matlab program;
download the pdf, open, click on pushpin:
\attachfile{hw6_f14.m}
\end{solution}


\begin{questions}
%-----------------------------------------------------------------------
\question {\it Root-finding.\/}
% http://www.math.colostate.edu/~gerhard/classes/331/lab/newton.html
Consider the function $f(x) = \sin(x) + x \cos(x) $.
Our mission is to find values of $x$ in the interval $[0.5, 2.5]$ for which $f(x) = 0$.
%
\begin{parts}
\item Plot $f$ against  a grid of points $x$ between 0.5 and 2.5.
How many solutions do you see?
\item Write or adapt a bisection or Newton's method program to find a solution.
What is it?
\item Change the function to $f(x) = sin(x) + x \cos(x) - c$.
Modify your program to find solutions in the same interval
for $ c = \{ 1/2, 0, -1/2 \} $.
What are they?
Can you get your program to find them all at once?
\end{parts}
{\it Comment: See the
Matlab guide to
\href{http://www.mathworks.com/help/matlab/matlab_prog/anonymous-functions.html}{anonymous functions}
or the
\href{http://pages.stern.nyu.edu/~dbackus/233/root_finding_pjb.m}{root-finding code}
posted on the course outline for examples.
}

\begin{solution}
\begin{parts}
\item The function is monotonic in this interval
and crosses the axis once.
\item The solution is 2.0288.
\item For the vector of values for $c$,
the solutions are $( 1.8289, 2.0288, 2.2047)$.
\end{parts}
\end{solution}

%-----------------------------------------------------------------------
\question {\it Black-Scholes-Merton formula.\/}
Our mission here is to examine the role in the BSM formula of the
mysterious volatility parameter  $\sigma$.
The calculations refer to call options on the S\&P 500 exchange-traded fund,
ticker symbol {\tt SPY}.
On October 27, 20014 (today), the price of the underlying is $ s_t = 197.07$.
We'll consider prices of call options with maturities of
5 months or $ \tau = 5/12 $ years.
On the same date, the price of a bond of similar maturity was
$ q_t^\tau = 1.0000$ (yes, the interest rate is essentially zero).
\begin{parts}
\item  Consider a call option with strike price $k = 197$
(``at the money'').
If volatility $\sigma = 0.10$,
what is the price of the option?  If $\sigma = 0.20$?
\item Plot the call option price against volatility for  $\sigma$ between 0.01 and 0.30.
\item On the same graph, do the same for options with strikes of
180 and 220.
\item Consider call prices of
19.68 ($k=180$),
7.28 ($k=197$),
and 0.32 ($k=220$).
Use your graph to estimate the values of $\sigma$ that deliver these prices.
How do they compare?
If $\sigma$ is a ``parameter,'' why are they different?
\end{parts}

\begin{solution}
\begin{parts}
\item Call prices are 5.11 and 10.18.  
\item [(b,c)] See Matlab program.
\item [d] The vols are 0.1728, 0.1428, and 0.1057. 
\end{parts}
\end{solution}


%-----------------------------------------------------------------------
\question {\it Volatilities for S\&P 500 options.\/}
We continue our examination of option prices and volatility:
same date and underlying as the previous question.

Yahoo finance reports the following quotes:
%
\begin{center}
\tabcolsep=0.15in
\begin{tabular}{lrrrr}
\toprule
      &  \multicolumn{2}{c}{\phantom{x}Call Price} &  \multicolumn{2}{c}{\phantom{xx}Put Price}  \\
Strike    &  Bid & Ask &  Bid & Ask  \\
\midrule
%140 &  57.12  &  57.45 &  0.43  &  0.48 \\
150 &  47.25  &  47.62 &  0.73  &  0.79 \\
160 &  37.61  &  37.95 &  1.26  &  1.32 \\
170 &  28.28  &  28.60 &  2.15  &  2.22 \\
180 &  19.56  &  19.80 &  3.63  &  3.69 \\
190 &  11.83  &  11.93 &  5.96  &  6.07 \\
200 &   5.57  &   5.66 &  9.87  &  9.99 \\
210 &   1.69  &   1.76 & 16.03  & 16.37 \\
220 &   0.29  &   0.34 & 24.70  & 25.11 \\
230 &   0.04  &   0.08 & 34.46  & 34.93 \\
\bottomrule
\end{tabular}
\end{center}

\begin{parts}
\item Compute ``mid'' quotes as averages of bid and ask.
Plot these mid-market put and call prices against the strike price.

\item Use put-call parity to compute call prices from mid puts.
Plot call prices --- bid, ask, and implied by puts ---
against the strike.
How do they compare?

\item Write a bisection or Newton's method program
to compute implied volatilities for mid quotes of call options.
Graph them against the strike.
What shape does the resulting ``smile'' have?
\end{parts}
{\it Comments:
\begin{itemize}
\item I use this two-step definition of the BSM formula:
\begin{verbatim}
d = @(sigma,k) (log(s./(q_tau.*k))+tau*sigma.^2/2)./(sqrt(tau)*sigma);
f = @(sigma,k) s*normcdf(d(sigma,k)) - ...
        q_tau.*k.*normcdf(d(sigma,k)-sqrt(tau)*sigma) - call_mid;
\end{verbatim}
Here I've subtracted the call price from the formula to give us a
function {\tt f} whose value is zero when we find the right volatility.

\item If you're lazy like me, you can look up the derivative in Wikipedia,
where they refer to it as the ``vega,'' a nonexistent Greek letter.
Or you could use bisection and skip the whole thing.
We can calculate the derivative with an anonymous function, too:
\begin{verbatim}
fp = @(d) s*sqrt(tau)*exp(-d.^2/2)/sqrt(2*pi);
\end{verbatim}

\end{itemize}
}

\begin{solution}
\begin{parts}
\item See Matlab program. 
\item Ditto.  The call prices implied by puts are above the bid and ask of the calls.  
\item See Matlab program. The ``smile'' is downward sloping, with a little curvature
to the right.  
\end{parts}
\end{solution}

\end{questions}

\vfill \centerline{\it \copyright \ \number\year \ NYU Stern School of Business}

\end{document}


%-----------------------------------------------------------------------
\question {\it Sums and mixtures (review).\/}
Let us say that the log-price of the underlying has two components:
\begin{eqnarray*}
    \log s_{t+1} &=& y_{t+1} \;\;=\;\; x_{1t+1} + x_{2t+1},
\end{eqnarray*}
with $( x_{1t+1}, x_{2t+1})$ independent.
The first component is normal:
$ x_{1t+1} \sim \mathcal{N}(\mu,\sigma^2)$.
The second component, the ``jump,'' is a mixture:
with probability $1-\omega$, $ x_{2t+1} = 0$,
and with probability $\omega$,
$ x_{2t+1} \sim \mathcal{N}(\theta,\delta^2)$.

With these inputs, the pdf for $y$ is a weighted average of normals:
\begin{eqnarray}
    p(y) &=& (1-\omega) \cdot ( 2 \pi \sigma^2)^{-1/2} \exp[ - (y-\mu)^2/2\sigma^2]
            \nonumber \\
    && + \; \omega \cdot [ 2 \pi (\sigma^2+\delta^2)]^{-1/2}
            \exp[ - [y-(\mu+\theta)]^2/2(\sigma^2 + \delta^2)] .
            \label{eq:bernoulli-mixture}
\end{eqnarray}
If $\omega = 0$, the second component drops out.
Otherwise, we have a weighted average of two normal densities.

\begin{parts}
\item  Show that the cumulant generating function for $x_{1t+1}$ is
\begin{eqnarray*}
    k (s; x_1) &=&  \mu s + \sigma^2 s^2 / 2 .
\end{eqnarray*}

\item  Show that the cumulant generating function for $x_{2t+1}$ is
\begin{eqnarray*}
    k (s; x_2) &=& \log \left[ (1-\omega) + \omega e^{ \theta s + \delta^2 s^2 / 2 }
            \right] .
\end{eqnarray*}

\item  What is the cgf for $y_{t+1}$?
What are its mean, variance, skewness, and excess kurtosis?
What parameters determine the sign of skewness?
\end{parts}

\begin{solution}
The overall idea here is that this construction
can give us useful departures from a normal distribution.
Details follow from applying the usual methods.
\begin{parts}
\item   The usual normal cgf.
\item  This is a Bernoulli mixture with one twist:
the ``$\omega$ branch'' has a normal mgf in it.
If $\delta = 0$, it's just like the Bernoulli we looked at in
Lab Report \#1.
Otherwise, we get some additional terms.

\item  The cgf of the sum is the sum of the cgfs:
\begin{eqnarray*}
        k (s; y) &=& k (s; x_1) + k (s; x_2) \\
             &=& ( \mu s + \sigma^2 s^2 / 2 ) +
              \log \left[ (1-\omega) + \omega e^{ \theta s + \delta^2 s^2 / 2 }
            \right] .
\end{eqnarray*}
We find it cumulants from derivatives, which are conveniently
computed by Matlab.
The mean and variance are
\begin{eqnarray*}
    \kappa_1 &=& \mu + \omega \theta \\
    \kappa_2 &=& \sigma^2 + \omega (1-\omega) \theta^2 + \omega \delta^2 .
\end{eqnarray*}
Each has terms from each component.
The third and fourth cumulants come only from $x_2$, because the normal
component has zero cumulants beyond the first two.
The third one is
\begin{eqnarray*}
    \kappa_3 &=& \omega (1-\omega) \theta [ 3 \delta^2 + (1-2\omega) \theta^2]  \\
    \kappa_4 &=& \omega (1-\omega)
        \{  \theta^4 [1 - 6 \omega(1-\omega)] + 3 \delta^4 + (1-2 \omega) 6 \delta^2 \theta^2
        \}
\end{eqnarray*}
This is a bit of a mess, but for $\omega$ small,
$\kappa_3$ depends on the sign of $\theta$
and $\kappa_4 >0 $,
so the mixture introduces skewness and excess kurtosis,
both of which are absent from normal random variables.

\item  We know normal pfd's integrate to one.
The is a weighted average of normal pdf's, so it must
integrate to one as well.
\end{parts}
\end{solution}


%-----------------------------------------------------------------------
\question {\it Merton-like option pricing.\/}
With the same setup, we can illustrate the value of mixtures in
generating nonnormal distributions and
their impact on option prices and volatility smiles.

\begin{parts}
\item  Risk-neutral asset pricing tells us, in general, that
\begin{eqnarray*}
    s_t &=& q^1_t E^* (s_{t+1} )
            \;\;=\;\; q^1_t E^* \left( e^{y_{t+1}}\right)
            \;\;=\;\; q^1_t  e^{k(1; y)}  .
\end{eqnarray*}
We refer to this as the no-arbitrage condition.
What is the no-arbitrage condition for our example?

We'll use this condition to set $\mu$:
given values for everything else, we'll choose $\mu$ to satisfy this condition.

\item  Recall that if the risk-neutral distribution is
 $\log s_{t+1} = y_{t+1} \sim \mathcal{N}(\kappa_1,\kappa_2)$,
 then the put price at strike $k$ is
\begin{eqnarray*}
    f(k; \kappa_1, \kappa_2) &=& q^1_t k N(d) - q^1_t e^{\kappa_1 + \kappa_2/2}
            N(d-\kappa_2^{1/2}) \\
            d&=& (\log k - \kappa_1)/\kappa_2^{1/2} .
\end{eqnarray*}
(Note:  this isn't quite the usual $d$.)
What is the call price?

\item
Use (b) to show that the put price in the mixture model is a weighted average:
\begin{eqnarray*}
    q^p_t &=& (1-\omega) \cdot f(k; \mu, \sigma^2) +
        \omega \cdot f(k; \mu+\theta, \sigma^2 + \delta^2) .
\end{eqnarray*}

\item  Consider these inputs:
$\sigma = 0.04$, $\omega = 0.01$,
$\theta = -0.3$, $\delta = 0.15$,
$s_t = 100$, and $q^1_t = 1$.
What is $\mu$?
What are the prices of put options with strikes
$ k = (80, 90, 100, 110, 120)$?
(Use a finer grid if you have this automated.)
What are the implied volatilities?

\item  What happens to the volatility smile when you set
\begin{itemize}
\item $\theta = 0$?
\item $\theta = + 0.3$?
\item $\delta = 0.25$?
\end{itemize}
Make sure you adjust $\mu$ in each case.
\end{parts}

\begin{solution}
\begin{parts}
\item  The condition is
\begin{eqnarray*}
    s_t &=& q^1_t \exp( \mu + \sigma^2 / 2 )
            \left[ (1-\omega) + \omega e^{ \theta + \delta^2 / 2 } \right] .
\end{eqnarray*}
It's not pretty, but given values of the other inputs we can use it to set $\mu$.

\item  This is a question of integrating over the distribution,
as we did in class.
We find the call price from put-call parity.

\item  If the pdf is a sum, then when we integrate we can integrate
over each term separately:  the integral of a sum is the sum of the integrals.
Each integral gives us a BSM-like formula, but with different mean and variance.
The put price is the weighted average of the two formulas, as stated.

\item  Now we can put all this to work.
With these inputs, the arbitrage condition gives us $ \mu = 4.6069$.
The put prices are (I chose a slightly different range):

\begin{center}
\tabcolsep=0.15in
\begin{tabular}{rr}
\toprule
Strike    &  Put Price  \\
\midrule
  90   & 0.1606   \\
   94  &  0.2806  \\
   98  &  0.9285  \\
  102  &  2.8801  \\
  106  &  6.1523  \\
  110  & 10.0147 \\
\bottomrule
\end{tabular}
\end{center}

It may be asking too much to compute the implied vols, too,
but they give us a clearer picture.
The Matlab code is a little quirky, you need to play with the
starting values to get it to work.
But you'll find that there's a distinct downward slope to the smile.
That disappears if we set $\theta = 0$, since $\theta$ is the source
of skewness in the model.
Instead we get a u-shape.
When we increase $\delta$, the u is more pronounced.

In short, the Bernoulli mixture is capable of giving us a wide range of shapes
for the smile.

\end{parts}
\end{solution}

\end{questions}

{\vfill
{\bigskip \centerline{\it \copyright \ \number\year \
David Backus $|$ NYU Stern School of Business}%
}}


\end{document}


