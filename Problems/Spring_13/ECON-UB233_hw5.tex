\documentclass[11pt]{exam}

\oddsidemargin=0.25truein \evensidemargin=0.25truein
\topmargin=-0.5truein \textwidth=6.0truein \textheight=8.75truein

%\RequirePackage{graphicx}
\usepackage{comment}
\usepackage{verbatim}
\usepackage{booktabs}
\usepackage{hyperref}
\urlstyle{rm}   % change fonts for url's (from Chad Jones)
\hypersetup{
    colorlinks=true,        % kills boxes
    allcolors=blue,
    pdfsubject={ECON-UB233, Macroeconomic foundations for asset pricing},
    pdfauthor={Dave Backus @ NYU},
    pdfstartview={FitH},
    pdfpagemode={UseNone},
%    pdfnewwindow=true,      % links in new window
%    linkcolor=blue,         % color of internal links
%    citecolor=blue,         % color of links to bibliography
%    filecolor=blue,         % color of file links
%    urlcolor=blue           % color of external links
% see:  http://www.tug.org/applications/hyperref/manual.html
}

\renewcommand{\thefootnote}{\fnsymbol{footnote}}
\newcommand{\var}{\mbox{\it Var\/}}

\printanswers

% document starts here
\begin{document}
\parskip=\bigskipamount
\parindent=0.0in
\thispagestyle{empty}
{\large ECON-UB 233 \hfill Dave Backus @ NYU}

\bigskip\bigskip
\centerline{\Large \bf Lab Report \#5: Excess Returns}
\centerline{Revised: \today}

\bigskip
{\it Due at the start of class.
You may speak to others, but whatever you hand in should be your own work.
Please include your Matlab code.}

\begin{questions}
%-----------------------------------------------------------------------
\question {\it Risk and return in US equity portfolios.\/}
Modern economies issue a wide range of assets,
whose returns can be wildly different.
Here we summarize the properties of
returns on some common equity portfolios.

We'll start with data input.  Go to Ken French's data site,
a standard source for data on financial returns for a broad range
of equity-related portfolios:

{\small
\url{http://mba.tuck.dartmouth.edu/pages/faculty/ken.french/data_library.html}}.

Download the files associated with, respectively, the ``Fama-French Factors''
and ``Portfolios Formed on Size'' at the links

{\small
\url{http://mba.tuck.dartmouth.edu/pages/faculty/ken.french/ftp/F-F_Research_Data_Factors.zip} \\
\url{http://mba.tuck.dartmouth.edu/pages/faculty/ken.french/ftp/Portfolios_Formed_on_ME.zip}.
}

In both cases, you'll find a txt file inside a zip file.
Copy the first table in each txt file into a spreadsheet with the dates aligned.
In the first file (``factors'') you want the first column (the date),
the second (the excess return on the market),
and the fifth (the short-term or riskfree interest rate RF).
In the second file (``size'') you want the first column (the date again)
and columns three to five (returns on portfolios of small, medium, and large firms).

When you're done, read the data into Matlab.
At this point you should have the riskfree rate and excess returns
on the market and three size portfolios (small, medium, and large) ---
five variables all together.
The numbers are monthly, July 1926 to (last I looked) January 2013.
I believe they are percentages, but you should check.
%
\begin{parts}
\part
Compute the mean, standard deviation, skewness, and excess kurtosis
for each excess return series.
Which portfolio has the highest excess return?  Lowest?

\part The Sharpe ratio for any asset or portfolio is
the mean of its excess return over its standard deviation.
Which portfolio has the highest Sharpe ratio?  Lowest?

\part Do the same for log excess returns;
that is, compute the mean, standard deviation, skewness, and excess kurtosis
for each excess return series.
(To get this right, you should
add the riskfree rate to each excess return,
divide by 100,
add one, then take logs.
Think about why all this is called for.)

\part Is there a clear link here between risk, measured here by
the standard deviation of the excess return, and return,
measured by the mean excess return?
Should there be?
\end{parts}

\begin{solution}
The idea here is to look at some popular equity portfolios
and see how their returns differ from broad-based equity indexes.  
We see (i)~mean excess returns in some cases are larger than the equity premium
and (ii)~skewness and kurtosis are a standard feature. 
Skewness differs between returns and log returns, as you might expect:
the log is a concave function, so it moderates large positive returns in levels.
Finally, we remind ourselves that $ E(mr) = 1$ does not imply 
any particular relation between mean and standard deviation of excess returns.
In that respect, the model is similar to the CAPM, where the mean is connected to 
covariance with the market, not the standard deviation or Sharpe ratio
or other simple risk measure.  

\begin{parts}
\part Last year's numbers below.  
You can get the current ones by running the posted Matlab program.  

\medskip
\begin{center}
\begin{tabular}{lcccc}
\multicolumn{4}{l}{Properties of monthly excess returns} \\
\toprule
            & \multicolumn{4}{c}{Portfolio} \\
                         \cmidrule{2-5}
Statistic   & market & small  & medium & big \\
\midrule
Mean &  0.6172 &  0.9738  &  0.8532  & 0.5987 \\
Std dev & 5.4571  &  8.5584  &  6.8520 &  5.2757 \\
Skewness &  0.1685 &   2.1813 &  0.9816  &  0.1860 \\
Excess kurtosis &  7.3997 &  21.8628 &  11.7830 &  7.1738 \\
Sharpe ratio  &  0.1131  &  0.1138  &  0.1245 &   0.1135 \\
\bottomrule
\end{tabular}
\end{center}

\medskip
The highest mean excess return is small firms at about 1\% a month.

\part Sharpe ratios also reported above.  
You can compute them from the ratio of the first row to the second.
Here medium-size firms are highest. 
The larger mean return of small firms is undone by a large standard deviation.  

\part Similar table for log excess returns below, also last year's.
Note that we need to need to the riskfree rate to excess returns and take logs.

\medskip
\begin{center}
\begin{tabular}{lrrrr}
\multicolumn{4}{l}{Properties of monthly log excess returns} \\
\toprule
            & \multicolumn{4}{c}{Portfolio} \\
             \cmidrule{2-5}
Statistic   & market & small  & medium & big \\
\midrule
Mean            & 0.4669 & 0.6329 & 0.6224 & 0.4583 \\
Std dev         & 5.4556 & 8.1132 & 6.7205 & 5.2682 \\
Skewness        & $-0.5298$ & 0.4937 & $-0.0850$ & $-0.4779$  \\
Excess kurtosis & 6.5282 & 9.6576 & 7.5655 & 6.3824\\
\bottomrule
\end{tabular}
\end{center}

\part There's some connection between the mean and the standard deviation
in levels, not so much in logs.
Neither is all that informative:  we know risk
has to do with the connection to the pricing kernel,
anything else would be purely accidental.
\end{parts}
\end{solution}

%-----------------------------------------------------------------------
\question {\it Disaster risk and the equity premium.\/}
We'll add a third ``disaster'' state to our analysis of the equity premium
and see how it changes our view of it.
The key input is the distribution of log consumption growth,
\begin{eqnarray*}
    \log g &=& \left\{
                \begin{array}{ll}
                \mu + \sigma & \mbox{with probability } (1-\omega)/2 \\
                \mu - \sigma & \mbox{with probability } (1-\omega)/2 \\
                \mu - \delta & \mbox{with probability } \omega
                \end{array}
                \right.
\end{eqnarray*}
What's the idea?
If $\omega = 0$, we're back to our symmetric two-state distribution.
But if we introduce a small positive value of $\omega$ and a ``largish'' $\delta>0$,
we have a ``disaster'' state that changes the distribution dramatically.

The question is what this does to the equity premium.
We'll define equity as a claim to consumption growth $g$.
We define the equity premium in logs,
\begin{eqnarray*}
    E ( \log r^e - \log r^1 ) ,
\end{eqnarray*}
and aim at a target value of 0.0400 (4\%).


\begin{parts}
\part If $\omega = 0$, what values of $\mu$ and $\sigma$ deliver
the observed mean and variance of log consumption growth, namely
0.0200 and $0.0350^2$?

\part Suppose $\beta = 0.99$ and $\alpha = 10$.
What are $\log r^1$ and
the equity premium,  $E \log r^e - \log r^1 $?

\part What is entropy?
How does it relate to the equity premium you computed above?

\part Now consider $\omega = 0.01$ and $\delta = 0.30$.
(These numbers are based on a series of studies by Robert Barro and his coauthors.)
With these numbers, what values of $\mu$ and $\sigma$ reproduce
the observed mean and variance of log consumption growth?

\part With (again) $\beta = 0.99$ and $\alpha = 10$,
what are $\log r^1$ and
the equity premium,  $E \log r^e - \log r^1 $?
How does it compare to your previous calculation?

\part How does entropy differ between the
disaster and no-disaster cases?

\part How does entropy change if $\delta = - 0.30$,
so that the extreme state is good news rather than bad?
Can you guess why?
\end{parts}

\begin{solution}
The idea is to show how changing the distribution
in ways that produce negative skewness can increase risk 
premiums even if we keep the standard deviation of log consumption
growth the same.
It's like a partial derivative result:  
vary skewness while holding the standard deviation constant.  
The question in practice is how much of this is reasonable.  

\begin{parts}
\part The expressions for the mean and variance of $\log g$ are
\begin{eqnarray*}
    E (\log g)   &=&  \mu + \omega \delta \;\;=\;\; 0.0200 \\
    \mbox{Var}(\log g) &=& (1-\omega) \sigma^2 + \omega (1-\omega) \delta^2
                \;\;=\;\; 0.0350^2 .
\end{eqnarray*}
When $\omega = 0$, the mean is $\mu = 0.0200$ and the standard
deviation is $\sigma = 0.0350$.

\part With these values, we have
$r^1 = 1.1618$, $\log r^1 = 0.1500$, and
$E \log r^e - \log r^1 = 0.0112 $.

\part Entropy here is 0.0600.
This is an upper bound on expected excess returns, 
so the model is evidently able to generate risk premiums greater than the equity premium.

\part When $\omega = 0.01$,
we need to set $\mu = 0.0230$ and $\sigma = 0.0183$ to maintain the mean and
variance at their sample values.
See (a).

\part With these values, we have
$r^1 = 1.0529$, $\log r^1 = 0.0516$, and
$E \log r^e - \log r^1 = 0.0438 $.
We have a success!  
The equity premium goes up and is now above our target. 
In that respect, the disaster state is a useful innovation,
but we still need a large risk aversion parameter for it to work.  

\part Entropy rises, too, to 0.1585.  Yaron's bazooka!

\part If we switch the sign of $\delta$, 
entropy falls to 0.0371.  
Evidently positive skewness in consumption and dividend growth
isn't helpful.
This is connected to our discussion of skewness and entropy:
positive skewness in $\log m$ increases entropy.
With power utility, that requires negative skewness in $\log g$.
Changing the sign of $\delta$ gives us positive skewness
and reduces entropy. 
\end{parts}
\end{solution}

\end{questions}

\vfill \centerline{\it \copyright \ \number\year \ NYU Stern School of Business}
%\end{document}


\pagebreak
{\bf Matlab program:}
\verbatiminput{../Matlab/hw5_s13.m}
\end{document}

