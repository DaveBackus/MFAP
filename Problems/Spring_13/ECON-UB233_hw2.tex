\documentclass[11pt]{exam}

\oddsidemargin=0.25truein \evensidemargin=0.25truein
\topmargin=-0.5truein \textwidth=6.0truein \textheight=8.75truein

%\RequirePackage{graphicx}
\usepackage{comment}
\usepackage{verbatim}
\usepackage{booktabs}
\usepackage{hyperref}
\urlstyle{rm}   % change fonts for url's (from Chad Jones)
\hypersetup{
    colorlinks=true,        % kills boxes
    allcolors=blue,
    pdfsubject={ECON-UB233, Macroeconomic foundations for asset pricing},
    pdfauthor={Dave Backus @ NYU},
    pdfstartview={FitH},
    pdfpagemode={UseNone},
%    pdfnewwindow=true,      % links in new window
%    linkcolor=blue,         % color of internal links
%    citecolor=blue,         % color of links to bibliography
%    filecolor=blue,         % color of file links
%    urlcolor=blue           % color of external links
% see:  http://www.tug.org/applications/hyperref/manual.html
}

\renewcommand{\thefootnote}{\fnsymbol{footnote}}
\newcommand{\var}{\mbox{\it Var\/}}

\printanswers

% document starts here
\begin{document}
\parskip=\bigskipamount
\parindent=0.0in
\thispagestyle{empty}
{\large ECON-UB 233 \hfill Dave Backus @ NYU}

\bigskip\bigskip
\centerline{\Large \bf Lab Report \#2: Sums, Mixtures, \& Certainty Equivalents}
\centerline{Revised: \today}

\bigskip
{\it Due at the start of class.
You may speak to others, but whatever you hand in should be your own work.
Please include your Matlab code.}

\begin{questions}

\begin{solution}
Most of the answers are computed in the Matlab
program listed at the end.
\end{solution}

%-----------------------------------------------------------------------
\question {\it The sum of normals is normal.\/}
The idea here is to use cumulant generating functions (cgfs)
to show that the sum of independent normal random variables is also normal.
It's helpful to break the problem into manageable pieces, like this:
\begin{parts}
\part Consider two independent random variables $x_1$ and $x_2$,
    not necessarily normal.
    Show that the cgf of the sum $y = x_1 + x_2$ is the sum of their cgfs:
\begin{eqnarray*}
	k(s; y) &=& k(s; x_1) + k(s;x_2) .
\end{eqnarray*}
Hint:  note the form of the pdf and apply the definition of the cgf.

\part Suppose $x_i \sim \mathcal{N}(\kappa_{i1}, \kappa_{i2})$.
[This bit of notation means:   $x_i$ is normally distributed with
mean $\kappa_{i1}$ and variance $\kappa_{i2}$.]
What is $x_i$'s cgf?
Hint:  we did this in class.

\part Use (a) to find the cgf of $y = x_1 + x_2$,
with $(x_1,x_2)$ as described in (b) (namely, normal with given means and variances).
How do you know that $y$ is also normal?  What are its mean and variance?

\part Extend this result to $y = a x_1 + b x_2$ for any real numbers $(a,b)$.
\end{parts}

\begin{solution}
\begin{parts}
\part If $x_1$ and $x_2$ are independent, their pdf factors:
$ p_{12}(x_1,x_2) = p_1(x_1) p_2(x_2) $.
That means the mgf of $x_1+x_2$ is the product of their individual mgf's:  
$ h_y(s) = h_1(s) h_2(s) $. 
We take the log to get the cgf's:  
$ k_y(s) = \log h_y(s) = \log h_1(s) + \log h_2(s) = k_1(s) + k_2(s) $. 

\part $  k(s; x_i) = s \kappa_{1i} + s^2 \kappa_{2i}/2 $.
(If this isn't burned into your memory, please burn it in now.)
\part Sum the cgf's:
\begin{eqnarray*}
    k(s; y) &=& k(s,x_1) + k(s,x_2) \\
            &=& (s \kappa_{11} + s^2 \kappa_{21}/2) + (s\kappa_{12} + s^2 \kappa_{22}/2) \\
            &=& s (\kappa_{11} + \kappa_{12} )
                + s^2 (\kappa_{21} + \kappa_{22})/2 .
\end{eqnarray*}
It's normal because it's mgf has the form of a normal random variable.
In fact, we can pick the mean and variance right out of the formula.  
\part Still normal, but with a slight change in mean and variance:
\begin{eqnarray*}
    k(s; y) &=& s (a \kappa_{11} + b \kappa_{12} )
                + s^2 (a^2 \kappa_{21} + b^2 \kappa_{22})/2  .
\end{eqnarray*}
\end{parts}
\end{solution}

%-----------------------------------------------------------------------
\question {\it Sums and mixtures.\/}
More fun with cumulant generating functions (cgfs).
The ingredients include two independent normal random variables:
$ x_1 \sim \mathcal{N}(0,\sigma^2)$ and $ x_2 \sim \mathcal{N}(\theta,\delta^2)$.
Suggestion:  Have Matlab do the calculations.
%
\begin{parts}
\part {\it Sum.\/}  Consider the random variable $y = x_1 + x_2$.
Use the results of the previous question to derive the first four cumulants of $y$.
What are the measures of skewness and excess kurtosis, $\gamma_1$ and $\gamma_2$?

\part {\it Mixture.\/}  Now consider the mixture $x_3$:
\begin{eqnarray*}
    x_3 &=& \left\{
            \begin{array}{ll}
            0   & \mbox{with probability } 1-\omega \\
            x_2 & \mbox{with probability } \omega .
            \end{array}
            \right.
\end{eqnarray*}
What is its cgf?
Hint:  apply the definition.

\part Now consider the sum $y = x_1 + x_3$.
What is its cgf?
What are the measures of skewness and excess kurtosis, $\gamma_1$ and $\gamma_2$?
What parameter configuration do you need to generate negative skewness?
\end{parts}

\begin{solution}
\begin{parts}
\part Cumulants follow from differentiating.  
The cgf is 
\begin{eqnarray*}
     k_y(s) &=& k_1(s) + k_1(s) 
        \;\;=\;\; s (0 + \theta) + s^2 (\sigma^2 + \delta^2)/2 .
\end{eqnarray*}     
The cumulants are $\kappa_1 = \theta $ (the mean) 
and $\kappa_2 = \sigma^2 + \delta^2 $ (the variance). 
Since the cgf is quadratic (in $s$!), 
all the derivatives after the second one are zero.  
Therefore $\gamma_1 = \gamma_2 = 0$ as well.  
\part Mixtures are a little different.  
The mgf is 
\begin{eqnarray*}
    h(s) &=& (1-\omega) + \omega e^{s\theta + s^2 \delta^2/2} .
\end{eqnarray*}
The cgf is the log of $h$. 

\part The cgf of a sum is the sum of the cgf's:  
\begin{eqnarray*}
    h_y(s) &=& s^2 \sigma^2 /2 + \log \left[(1-\omega) + \omega e^{s\theta + s^2 \delta^2/2} \right].
\end{eqnarray*}
We find the cumulants by differentiating.
The expressions are a mess, but include:  
\begin{eqnarray*}
    \kappa_1 &=& \omega \theta \\
    \kappa_2 &=& \sigma^2 + \omega \delta^2 + \omega(1-\omega) \theta^2 \\
    \kappa_3 &=& \omega (1-\omega) \theta \left[ 3 \delta^2 + \theta^2 (1-2\omega) \right] .
\end{eqnarray*}
See the Matlab output for details.  
It looks like $\theta $ is the key.
If $\omega < 1/2$, then the sign of $\kappa_3$ is the same as the sign of $\theta$.  

You might picture this:  start with a normal pdf for $x_1$, now add a smaller pdf on top of it
for $x_2$. If $\theta<0$ this is to the left of the mean of the first one, 
which gives you negative skewness. 

\end{parts}
\end{solution}


%-----------------------------------------------------------------------
\question {\it Lognormal risks.\/}
Our mission is to compute the {\it risk penalty\/}
associated with lognormal risks and power utility.
What risks matter?  What role does the risk aversion parameter play?

To be concrete, let $x = \log c \sim \mathcal{N}(\kappa_1, \kappa_2)$
and $u(c) = c^{1-\alpha}/(1-\alpha)$ for $\alpha> 0$.
The risk penalty is
\begin{eqnarray*}
    \mbox{\it rp\/} &=& \log (\bar{c}/\mu)
                \;\;=\;\; \log \bar{c} - \log \mu ,
\end{eqnarray*}
where $\bar{c} = E (c) = E( e^{x})$ and $\mu$ is the certainty equivalent.
We'll compute the terms one by one.

% ?? change to mgf -- also this one's a little hard 
\begin{parts}
\part What is the cumulant generating function (cgf) of $x = \log c$?
\part Use the cgf to compute $\bar{c}$.
\part Use the cgf to compute the certainty equivalent $\mu$.
\part What is the risk penalty?  How does it depend on risk?  Risk aversion?
\part Suppose $\kappa_2 = 0.02^2$.  What is the risk penalty if $\alpha = 2$?
$\alpha = 10$?  $\alpha = 20$?
\end{parts}

\begin{solution}
The idea is to explore risk and risk aversion in a lognormal setting, 
a setting we'll see over and over again.
We're using the idea that if $x = \log c$, then $c = e^x$.  
\begin{parts}
\part The mgf of $x$  is (get used to this, you'll see it a lot) 
\begin{eqnarray*}
    h(s) &=& E(e^{sx}) \;\;=\;\; \exp [s \kappa_1 + s^2 \kappa_2/2 ].
\end{eqnarray*}
The cgf is the log of this:  
\begin{eqnarray*}
    k(s) &=& \log h(s)  \;\;=\;\; s \kappa_1 + s^2 \kappa_2/2 .
\end{eqnarray*}

\part We find the mean by setting $s=1$:  
$ \bar{c} = E(c) = E(e^x) = h(1) = \exp[k(1)] = e^{\kappa_1+\kappa_2/2}$.

\part We find expected utility by setting $s=1-\alpha$:  
$ E[u(c)] = E (c^{1-\alpha})/(1-\alpha) = E (e^{(1-\alpha) x})/(1-\alpha) $.
The numerator is the mgf evaluated at $s = 1-\alpha$. 
So we have  
\begin{eqnarray*}
    E[u(c)] &=& \exp [ k(1-\alpha)/(1-\alpha) ] \\
            &=& \exp[ (1-\alpha) \kappa_1 + (1-\alpha)^2 \kappa_2 /2 ] /(1-\alpha) \\
            &=& \exp[ (1-\alpha) \{ \kappa_1 + (1-\alpha) \kappa_2 /2 \} ] /(1-\alpha) .  
\end{eqnarray*}
What level of consumption gives us the same utility?  
Well, a constant level of consumption $\mu$ gives utility 
$ E[u(\mu)] = u(\mu) = \mu^{1-\alpha}/(1-\alpha)$. 
Equating $u(\mu)$ and $ E[u(c)] $ gives us 
\begin{eqnarray*}
    \mu &=& \exp[ \kappa_1 + (1-\alpha) \kappa_2 /2 ] .  
\end{eqnarray*}

\part The risk penalty is 
\begin{eqnarray*}
    \log (\bar{c}/\mu) &=& \log \bar{c} - \log \mu) \;\;=\;\; \alpha\kappa_2/2 .
\end{eqnarray*}
So it increases with risk $\kappa_2$ and risk aversion $\alpha$. 
\end{parts}
\end{solution}


\end{questions}

\vfill \centerline{\it \copyright \ \number\year \
NYU Stern School of Business}

%\end{document}


\pagebreak
{\bf Matlab program:}
\verbatiminput{../Matlab/hw2_s13.m}

\end{document}



%-----------------------------------------------------------------------
\question {\it Exponential risks.\/}
Now consider the same problem when $x = - \log c $ is exponential,
a tractable nonnormal distribution.
The connection between $c$ and $x$ implies $c = e^{\log c} = e^{-x}$.
The pdf is
\begin{eqnarray*}
    p(x)  &=& \lambda e^{-\lambda x}
\end{eqnarray*}
for $x \geq 0$ and $\lambda > 0$.
The cgf is
\begin{eqnarray*}
    k(s)  &=& - \log \left( 1 - s/\lambda \right) .
\end{eqnarray*}
In terms of $x$, the mean of $c$ is $\bar{c} = E(c) = E (e^{-x})$.

\begin{parts}
\part Use the cgf to compute the variance, skewness, and excess kurtosis
of $x$.  How do they compare to those of the normal distribution?
\part Use the cgf to compute $\bar{c}$.
\part Choose the value of $\lambda$ that sets
the variance equal to $0.02^2$, the number we used earlier.
\part
What is the risk penalty if $\alpha = 2$? $\alpha = 10$?
$\alpha = 20$?
How does your answer differ from the previous one?  Why?
\end{parts}

\begin{solution}
Same approach as the previous problem.
\begin{parts}
\part The first four cumulants are
$\kappa_1 = 1/\lambda$,
$\kappa_2 = 1/\lambda^2$,
$\kappa_3 = 2/\lambda^3$,
and $\kappa_4 = 6/\lambda^4$.
Skewness and excess kurtosis are therefore
$\gamma_1 = 2$ and $\gamma_2 6$.

\part If $k(s)$ is the cgf of $x$,
then $\log \bar{c} = k(-1)$ ($k$ evaluated at $s=-1$):
$k(-1) = - \log (1 + 1/\lambda)$.

\part We solve $0.02^2 = 1/\lambda^2$ or $\lambda = 1/0.02 = 50$.


\part The risk penalties are
%
\begin{center}
\begin{tabular}{lccc}
        & $\alpha=2$  & $\alpha=10$  & $\alpha=20$ \\
\midrule
lognormal    &  0.0004 & 0.0020 & 0.0040 \\
exponential  &  0.0004 & 0.0022 & 0.0054
\end{tabular}
\end{center}
%
The point is that departures from normality matter, although they're not
especially large here.
The numbers suggest that the impact increases with risk aversion $\alpha$.
We could elaborate on this using the power series expansion of the cgf,
which would show you that the $j$th cumulant is multiplied
by $(-\alpha)^j/j!$, so as we increase $\alpha$ some of these terms can increase
dramatically.
Amir Yaron (Wharton prof) refers to this as a ``bazooka,''
since it allows you to take a little nonnormality
and generate enormous risk premiums.
More on this kind of thing later in the course.
\end{parts}
\end{solution}
