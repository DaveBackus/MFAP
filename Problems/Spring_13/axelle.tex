\documentclass[11pt]{exam}

\oddsidemargin=0.25truein \evensidemargin=0.25truein
\topmargin=-0.5truein \textwidth=6.0truein \textheight=8.75truein

\usepackage{comment}
\usepackage{hyperref}
\urlstyle{rm}   % change fonts for url's (from Chad Jones)
\hypersetup{
    colorlinks=true,        % kills boxes
    allcolors=blue,
    pdfsubject={ECON-UB233, Macroeconomic foundations for asset pricing},
    pdfauthor={Dave Backus @ NYU},
    pdfstartview={FitH},
    pdfpagemode={UseNone},
%    pdfnewwindow=true,      % links in new window
%    linkcolor=blue,         % color of internal links
%    citecolor=blue,         % color of links to bibliography
%    filecolor=blue,         % color of file links
%    urlcolor=blue           % color of external links
% see:  http://www.tug.org/applications/hyperref/manual.html
}

%\usepackage{amssymb}
%\usepackage{amsfonts}
%\usepackage{booktabs}

\renewcommand{\thefootnote}{\fnsymbol{footnote}}
%\newcommand{\bv}{\begin{verbatim}}
%\newcommand{\ev}{\end{verbatim}}

% document starts here
\begin{document}
\parskip=\bigskipamount
\parindent=0.0in
\thispagestyle{empty}
{\large ECON-UB 233 \hfill Dave Backus @ NYU}

\bigskip\bigskip
\centerline{\Large \bf Lab Report \#0:  Risk and Return}
\centerline{Revised: \today}

\bigskip
{\it I used the old template, ignore and go to the problems.}

\begin{questions}
\question {\it Asset returns.\/}
The canonical asset pricing relation can be expressed, variously, as
\begin{eqnarray*}
    q^j &=& E ( m d^j)  \phantom{\sum_s} \\
        &=& \sum_s p(s) m(s) d^j(s)\\
        &=& \sum_s \theta(s) d^j(s) .
\end{eqnarray*}
Here $q^j$ is the price ``today'' of asset $j$, a claim to the dividend $d^j$ ``tomorrow.''
States $s$ occur with probability $p(s)$.
In each state $s$, the pricing kernel is $m(s)$
and the state price is $\theta(s)$.

Suppose we have two states, $s=1$ and $s=2$,
which occur with equal probabilities $\theta(1) = 2/3$ and $\theta(2) = 1/3$.
State prices are $\theta(1) = 1$ and $\theta(2) = 2$.
\begin{parts}
\part Draw the event tree that corresponds to this environment.
\part What are the pricing kernel's values in the two states?
\part Consider an asset with $d(1) = 1$ and $d(2) = 2$.
Compute its price $q$ using state prices and the pricing kernel.
Verify that you get the same answer.
\part How is the (gross) return $r$ related, in general, to $q$ and $d$?
The mean return?
\part What are the returns in each state on the asset in part (c)?
What is the mean return?
\part Consider a second asset with dividends $d(1) = 2$ and $d(2) = 1$.
What are its price and mean return?
Why is its mean return different from the asset in (b)?
\end{parts}


\question {\it Risk and risk aversion.\/}
In the same setting as the previous equation, we'll
consider the pricing implied by a representative agent
with power utility.
Here the pricing kernel is related to consumption growth $g$ by
$m = \beta g^{-\alpha}$.  Here $m$ is the intertemporal marginal rate of substitution,
$\beta$ is the discount factor, and $\alpha \geq 0$ is the coefficient of relative
risk aversion.

\begin{parts}
\part Suppose $\beta = 0.95 $ and $\alpha = 2$.
Using the pricing kernel from the previous question,
what is consumption growth in the two states?
\part In addition to the other assets, compute the riskfree return:
the return on an asset with dividends $d(1) = d(2) = 1$.
\part What is the risk premium on the asset in part (c) above?
\part Now increase risk aversion to $\alpha = 5$.
What is the expected return on the asset on part (c)?
Its risk premium?
What do you think is happening here?
\end{parts}

\end{questions}

\vfill \centerline{\it \copyright \ \number\year \
NYU Stern School of Business}

\end{document}
