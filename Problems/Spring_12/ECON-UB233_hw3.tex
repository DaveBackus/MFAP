\documentclass[11pt]{exam}

\oddsidemargin=0.25truein \evensidemargin=0.25truein
\topmargin=-0.5truein \textwidth=6.0truein \textheight=8.75truein

%\RequirePackage{graphicx}
\usepackage{comment}
\usepackage{verbatim}
\usepackage{booktabs}
\usepackage[dvipdfm]{hyperref}
\urlstyle{rm}   % change fonts for url's (from Chad Jones)
\hypersetup{
    colorlinks=true,        % kills boxes
    allcolors=blue,
    pdfsubject={ECON-UB233, Macroeconomic foundations for asset pricing},
    pdfauthor={Dave Backus @ NYU},
    pdfstartview={FitH},
    pdfpagemode={UseNone},
%    pdfnewwindow=true,      % links in new window
%    linkcolor=blue,         % color of internal links
%    citecolor=blue,         % color of links to bibliography
%    filecolor=blue,         % color of file links
%    urlcolor=blue           % color of external links
% see:  http://www.tug.org/applications/hyperref/manual.html
}

\renewcommand{\thefootnote}{\fnsymbol{footnote}}
\newcommand{\var}{\mbox{\it Var\/}}

\printanswers

% document starts here
\begin{document}
\parskip=\bigskipamount
\parindent=0.0in
\thispagestyle{empty}
{\large ECON-UB 233 \hfill Dave Backus @ NYU}

\bigskip\bigskip
\centerline{\Large \bf Lab Report \#3: Two-Period Macro Models}
\centerline{(Started: February 2, 2012; Revised: \today)}

\bigskip
{\it Due at the start of class.
You may speak to others, but whatever you hand in should be your own work.}

\begin{questions}

%\begin{solution}
%Most of the answers are computed in the Matlab
%program listed at the end and posted
%\href{http://pages.stern.nyu.edu/~dbackus/233/hw2_s12.m}{here}.
%\end{solution}

%-----------------------------------------------------------------------
\question (consumption, investment, and state prices)
Consider a two-period economy with a linear technology.
What is equilibrium consumption growth?
What are the state prices?
What is a claim to one unit of capital worth?

To address these questions,
we use a variant of our two-period economy,
with dates 0 and 1 and states $z$ at date 1 that occur with probability $p(z)$.
The representative agent has utility function
\begin{eqnarray*}
    u(c_0) + \beta \sum_z p(z) u[c_1(z)] ,
\end{eqnarray*}
with $u(c) = c^{1-\alpha}/(1-\alpha)$ for $\alpha > 0$
(power utility).
She is endowed with $y_0$ units of the date-0 good, nothing at date 1.
The technology is linear:  $k$ units of the date-0 good invested in capital
generate $z k$ units of the good in state $z$ at date 1.
The resource constraints are therefore
\begin{eqnarray*}
    c_0 + k &=& y_0  \\
    c_1(z) &=& z k ,
\end{eqnarray*}
with one of the latter for each state $z$.
Think of the productivity factor $z$ as $a(z)$ with $a(z) = z$.

\begin{parts}
\part What are the classical ``ingredients'' of this economy?
\part What is the associated Pareto problem?
What are its first-order conditions?
\part Suppose $z$ is lognormal:  that is,
$\log z \sim \mathcal{N}(\kappa_1,\kappa_2)$.
Use the properties of lognormal random variables to show
that $E(z^a) = \exp( a \kappa_1 + a^2 \kappa_2/2)$
for any real number $a$.
\part Use this result to find the optimal values of $c_0$ and $k$.
Given these values, what is saving?
\part  (extra credit) What is $c_1(z)$?
% ?? next part a little tough
What are the state prices?
\part (extra credit) What is the value of one unit of capital,
that is, a claim to $z$ units of output in each state $z$?
%What are the effects of the parameters $(\alpha, \kappa_1,\kappa_2)$?
\end{parts}

\begin{solution}
\begin{parts}
\part Commodities:  the good at date 0 plus the good in all states at date 1 \\
Agents:  one \\
Preferences, endowment, and technologies:  given above \\
Resource constraints:  ditto
\part Pareto problem based on the Lagrangean:
\begin{eqnarray*}
    \mathcal{L}  &=&  \log c_0  + \beta \sum_z p(z) \log c_1(z) \\
         && + \; q_0 (y_0 - c_0 - k)
            + \sum_z q_1(z) [z k - c_1(z)] .
\end{eqnarray*}
We choose $c_0$, $k$, and $c_1(z)$ to maximize this.
The first-order conditions are
\begin{eqnarray*}
    c_0: &&  c_0^{-\alpha} \;\;=\;\; q_0 \\
    k:   &&  q_0 \;\;=\;\; \sum_z q_1(z) z \\
    c_1(z): &&  \beta p(z) c_1(z)^{-\alpha} \;\;=\;\; q_1(z)
\end{eqnarray*}

\part Define $x=\log z$.
Its mgf is $e^{sx} = E(e^{s\log z}) = E(z^s) = e^{s\kappa_1 + s^2 \kappa_2/2}$.
Setting $s=a$ gives us the answer.

\part This is moderately demanding, but here's how it works.
With apologies for mixing sums and integrals,
we use the first and third first-order conditions to substitute
for $q_0$ and $q_1(z)$ in the second:
\begin{eqnarray*}
        1 &=& \sum_z \frac{q_1(z)}{q_0} z
            \;\;=\;\; \sum_z \frac{\beta p(z) c_1(z)^{-\alpha}}{c_0^{-\alpha}} \; z
            \;\;=\;\; \sum_z \beta p(z) \frac{(zk)^{-\alpha}}{(y_0-k)^{-\alpha}} \; z \\
          &=& \beta (y_0-k)^\alpha k^{-\alpha} \sum_z  p(z)  z^{1-\alpha}
             \;\;=\;\;  \beta (y_0-k)^\alpha k^{-\alpha} E (z^{1-\alpha} ) .
\end{eqnarray*}
To simplify, denote $ E(z^{1-\alpha}) = Z$.
Then consumption and capital are
\begin{eqnarray*}
    k &=& \frac{(\beta Z)^{1/\alpha}}{1+ (\beta Z)^{1/\alpha}} \; y_0,
        \;\;\;  c_0 \;\;=\;\; \frac{1}{1+ (\beta Z)^{1/\alpha}} \; y_0
\end{eqnarray*}
Using the lognormal result, we have
\begin{eqnarray*}
    Z &=& E(z^{1-\alpha}) \;\;=\;\; e^{(1-\alpha)\kappa_1 + (1-\alpha)^2 \kappa_2/2}
\end{eqnarray*}

\part Evidently
\begin{eqnarray*}
    c_1(z) &=& z k \;\;=\;\; z y_0 \frac{(\beta Z)^{1/\alpha}}{1+ (\beta Z)^{1/\alpha}} .
\end{eqnarray*}
The big part at the end is a constant, so it's ugly but innocuous.
The state prices are (more tedious substitution)
\begin{eqnarray*}
    q(z) &=& p(z) \beta [c_1(z)/c_0]^{-\alpha}
            \;\;=\;\; p(z) z^{-\alpha}/Z .
\end{eqnarray*}
As usual, the state price is the product of
a pricing kernel [here $m(z) = z^{-\alpha}/Z$]
and a probability [the normal density for $\log z$,
which we haven't bothered to write out].

\part
This is claim to $z$ next period,
with value at date 0 of
\begin{eqnarray*}
    q^e &=& E (z^{1-\alpha}/Z \;\;=\;\; 1.
\end{eqnarray*}
Hmmmm...
Why does this make sense?
Because one unit of capital is one unit of the good
at date 0, whose price is one since we're valuing
assets in units of the date-0 good.


\end{parts}
\end{solution}


%-----------------------------------------------------------------------
\question (pricing state-contingent claims)
Consider a two-period exchange economy with power utility
and endowment growth
and $\log c_1 - \log c_0 = \log y_1 - \log y_0 = z \sim \mathcal{N}(\kappa_1,\kappa_2)$
and $y_0 = 1$.
We'll compute the prices of equity and two equity derivatives
using a representative agent's marginal rate of substitution
as a pricing kernel.

\begin{parts}
\part What is the marginal rate of substitution for this economy?
\part What is the price $q^1$ of a bond
that pays one in each state?
Express it as a function of $\alpha$, $\beta$, $\kappa_1$, and $\kappa_2$.
\part What is the price $q^e$ of ``equity,''
a claim to the date-1 endowment $y_1(z) = e^z$?
What is its return? Expected return?
\part Consider an ``upside'' derivative that pays
$y_1 $ if $y_1 \geq b$
and a ``downside'' derivative that pays
$y_1 $ if $y_1 \leq b$.
Show that the prices (call them $q^u$ and $q^d$)
sum to the price of equity.

\part (extra credit) What is the price of the downside derivative
for the lognormal case?
Hint:  this involves integrating the normal distribution
as we did when we computed the normal mgf.
The difference is that we only integrate over the region in
which the dividend is positive.
\end{parts}

\begin{solution}
\begin{parts}
\part The mrs is
\begin{eqnarray*}
    m(z) &=& \beta [c_1(z)/c_0]^{-\alpha}
        \;\;=\;\; \beta e^{-\alpha z} ,
\end{eqnarray*}
so $\log m \sim \mathcal{N}(\log \beta - \alpha \kappa_1, \alpha^2 \kappa_2) $.
\part The lognormal result again:
\begin{eqnarray*}
    q^1 &=& E(m) \;\;=\;\; \beta e^{-\alpha \kappa_1 + \alpha^2 \kappa_2/2} .
\end{eqnarray*}
The return is $r^1 = 1/q^1 $.

\part With our distributional assumptions, we have
\begin{eqnarray*}
    q^e  &=& E(m y_1) \;\;=\;\; \beta E ( e^{-\alpha z} e^z )
        \;\;=\;\; \beta E( e^{(1-\alpha) z}) \\
             &=&
             \beta e^{(1-\alpha)\kappa_1 + (1-\alpha)^2 \kappa_2/2 }.
\end{eqnarray*}
This should look familiar.

\part We're splitting the dividends in two --- upside and downside ---
so a claim to one unit of each is equivalent to a claim to equity.

\part This is moderately demanding, too,
but it's exactly the kind of math we'll use later to price options.
The price of the downside derivative is
\begin{eqnarray*}
    q^d &=&  \int_{-\infty}^{z^*} m(z) y_1(z) p(z) dz
\end{eqnarray*}
with $z^* = \log b$ the upper bound on $z$ corresponding to
$y_1 = b$.
Now we're back to the same math we used to compute the mgf for a normal
random variable:
\begin{eqnarray*}
     m(z) y_1(z) p(z)  &=& \beta e^{-\alpha z} e^z (2 \pi \kappa_2)^{-1/2}
        e^{-(z-\kappa_1)^2/2\kappa_2} \\
            &=& \beta e^{ (1-\alpha) \kappa_1 + (1-\alpha)^2 \kappa_2/2}
                (2 \pi \kappa_2)^{-1/2}
        e^{-(z + (1-\alpha) \kappa_2-\kappa_1)^2/2\kappa_2}  \\
        &=& q^e  (2 \pi \kappa_2)^{-1/2}
        \exp[-(w-\kappa_1)^2/2\kappa_2] ,
\end{eqnarray*}
where $ w = z + (1-\alpha)\kappa_2$.
The integral is therefore
\begin{eqnarray*}
    (2 \pi \kappa_2)^{-1/2} \int_{-\infty}^{\log b + (1-\alpha) \kappa_2}
            \exp[-(w-\kappa_1)^2/2\kappa_2] dw .
\end{eqnarray*}
More on this another time.

\end{parts}
\end{solution}


%-----------------------------------------------------------------------
\question (risk-neutral probabilities)
Consider yet another two-period exchange economy with power utility
and Bernoulli endowment growth:
\begin{eqnarray*}
    \log (c_1/c_0)  &=&  \left\{
                    \begin{array}{lll}
                    0   & \mbox{with probability } 1-\omega & \mbox{(state } z=1)\\
                    g>0 & \mbox{with probability } \omega   & \mbox{(state } z=2)
                    \end{array}
                    \right.
\end{eqnarray*}

\begin{parts}
\part What is the pricing kernel for this economy?
Is the pricing kernel higher in state 1 or state 2?  Why?
\part What are the state prices?
Which is more valuable, a claim to one in state 1
or a claim to one in state 2?  Why?
\part What is the price $q^1$ of a bond paying one in each state?
\part What are the risk-neutral probabilities?
Why are they different from the true probabilities?
\end{parts}

\begin{solution}
\begin{parts}
\part Same pricing kernel as before, but in a two-state setting.
In logs:
\begin{eqnarray*}
    \log m   &=&  \left\{
                    \begin{array}{ll}
                    \log \beta              & \mbox{in state } z=1\\
                    \log \beta - \alpha g   & \mbox{in state } z=2 .
                    \end{array}
                    \right.
\end{eqnarray*}
Evidently $m$ is lower in state 2.  Why?
There's more consumption in state 2, so its
marginal utility is lower.
\part State prices are connected to the pricing kernel by $ q(z) = p(z) m(z) $:
\begin{eqnarray*}
    q(z) \;\;=\;\; p(z) m(z)   &=&  \left\{
                    \begin{array}{ll}
                    (1-\omega) \beta               & \mbox{in state } z=1 \\
                    \omega \beta e^{- \alpha g}   & \mbox{in state } z=2 .
                    \end{array}
                    \right.
\end{eqnarray*}
\part The sum of state prices:
\begin{eqnarray*}
    q^1 \;\;=\;\; q(1) + q(2)  &=&
                    \beta \left[ (1-\omega) + \omega e^{- \alpha g} \right].
\end{eqnarray*}
\part Risk-neutral probabilities are defined by
$ q^1 p^*(z) = p(z) m(z)$.
Here we have
\begin{eqnarray*}
    p^*(2)    &=&  \frac{ \omega e^{-\alpha g}}{ 1 -\omega + \omega e^{-\alpha g} }
\end{eqnarray*}
and $p^*(1) = 1-p^*(2)$.
These ``probabilities'' are adjusted for risk.
If there's no risk aversion ($\alpha = 0$), $p^*(z) = p(z)$.
Otherwise, $p^*(2) < p(2)$:  we put less weight
on the good state (and more on the bad state),
which is a way of taking into account the risk.
Note, too, that $\beta$ drops out:  it affects
discounting but not risk adjustment.



\end{parts}
\end{solution}

\end{questions}


\vfill \centerline{\it \copyright \ \number\year \
NYU Stern School of Business}

\end{document}



%  EXTRA STUFF
%-----------------------------------------------------------------------
\question (labor supply and demand)
One of the central variables in macroeconomics is employment, so it's helpful
to have something like that in our models.
We do that here with a modest extension of our two-period economy,
with dates 0 and 1 and date-1 states $z$ that occur with probability $p(z)$.

Suppose the representative agent has one unit of labor each period,
which she can use to work or enjoy as leisure (everything but work).
Her utility function might be written
\begin{eqnarray*}
    u(c_0,1-n_0) + \beta \sum_z p(z) u[c_1(z),1-n_1(z)] ,
\end{eqnarray*}
with $n$ being the quantity of time spent working and $1-n$ the quantity of leisure.
There are no endowments other than the one unit of labor in each period.

Production uses labor as follows.  At date-0, each unit of labor used in production
generates on unit of consumption.
The resource constraint for date-0 consumption is therefore
\begin{eqnarray*}
    c_0 + k &=& n_0 .
\end{eqnarray*}
Production at date 1 in state $z$ uses both capital and labor,
giving us the date-1 resource constraints
\begin{eqnarray*}
    c_1(z) &=& z f[k,n(z)] ,
\end{eqnarray*}
one for each state $z$.
There are a finite number of states $z$,
each occurring with probability $p(z)$.
Productivity is also $z$.

\begin{parts}
\part What are the classical ``ingredients'' of this economy?
\part What is the associated Pareto problem?
\part Solve the Pareto problem for the special case
$u(c,1-n) = \log c + \lambda \log (1-n)$ and $f(k,n) = k^\theta n^{1-\theta} $.
Comment:  If you get stuck, try the deterministic version.
\part How does $c_1$ vary with $z$?  Why?  Does that seem realistic to you?
\part How does $n$ vary with $z$?  Why?  Does that seem realistic to you?
\part Wage = mpn ...  How does that vary with $z$?
%\part Equity:  claim to mpk times k.
\end{parts}

\begin{solution}
\begin{parts}
\part Commodities:  the good at date 0, labor/leisure at date 0,
and the same in all states at date 1  \\
Agents:  one \\
Preferences, endowment, and technologies:  given above \\
Resource constraints:  ditto
\part Pareto problem based on the Lagrangean:
\begin{eqnarray*}
    \mathcal{L}  &=&  \left[ \log c_0 + \lambda \log (1-n_0) \right]
            + \beta \sum_z p(z) \left[ \log c_1(z) + \lambda \log [1-n_1(z)] \right]  \\
         && + \; q_0 (n_0 - c_0 - k)
            + \sum_z q_1(z) [z k^\theta n_1(z)^{1-\theta} - c_1(z)] .
\end{eqnarray*}
We choose consumption, leisure, and capital to maximize this.
\part The first-order conditions are

\end{parts}
\end{solution}

