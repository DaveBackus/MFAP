\documentclass[11pt]{exam}

\oddsidemargin=0.25truein \evensidemargin=0.25truein
\topmargin=-0.5truein \textwidth=6.0truein \textheight=8.75truein

%\RequirePackage{graphicx}
\usepackage{comment}
\usepackage[dvipdfm]{hyperref}
\urlstyle{rm}   % change fonts for url's (from Chad Jones)
\hypersetup{
    colorlinks=true,        % kills boxes
    allcolors=blue,
    pdfsubject={ECON-UB233, Macroeconomic foundations for asset pricing},
    pdfauthor={Dave Backus @ NYU},
    pdfstartview={FitH},
    pdfpagemode={UseNone},
%    pdfnewwindow=true,      % links in new window
%    linkcolor=blue,         % color of internal links
%    citecolor=blue,         % color of links to bibliography
%    filecolor=blue,         % color of file links
%    urlcolor=blue           % color of external links
% see:  http://www.tug.org/applications/hyperref/manual.html
}
%\usepackage{booktabs}

\renewcommand{\thefootnote}{\fnsymbol{footnote}}
\newcommand{\var}{\mbox{\it Var\/}}

% document starts here
\begin{document}
\parskip=\bigskipamount
\parindent=0.0in
\thispagestyle{empty}
{\large ECON-UB 233 \hfill Dave Backus @ NYU}

\bigskip\bigskip
\centerline{\Large \bf Lab Report \#??}
\centerline{(Started: July 19, 2011; Revised: \today)}

\bigskip
{\it Due at the start of class.
You may speak to others, but whatever you hand in should be your own work.}

\begin{questions}
\question (two-state Markov chain)
Consider a two-state chain with transition matrix
\begin{eqnarray*}
    P &=& \left[
            \begin{array}{cc}
            (1-\rho) p + \rho & (1-\rho)(1-p) \\
            (1-\rho) p & (1-\rho)(1-p) + \rho
            \end{array}
          \right]
\end{eqnarray*}
based on parameters $ 0 < p < 1$ and $-1 < \rho < 1$.
\begin{parts}
\part Verify that this is a legitimate transition matrix?  (What are the requirements for this?)
\part What are the two-period transition probabilities?  Three-period probabilities?
\part What is the invariant distribution?
\part What are the eigenvalues of $P$?
\part ??
\end{parts}

\question (exponential)
Sometimes you need a positive random variable.  
A good example is an exponential, which has pdf ...
\begin{parts}
\part Verify pdf integrates to one.
\part What is cgf?  
\part Cumulants??
\part Central limit theorem?
\end{parts}


\question (discounting dividends)
We'll use the same Markov chain as in question 1
with $\rho = 2/3$ and $p = 1/3$.
Suppose the dividend on an asset is one in state 1 and two in state 2.
The discount factor is $q = 2/3$.
\begin{parts}
\part What is the one-period riskfree interest rate?
\part What is the invariant distribution?
What is the (unconditional) mean dividend?
\part What is the discounted value of expected dividends?
\part Price asset recursively.
Let the price be $q^e(1)$ in state 1 and $q^e(2)$ in state 2.
[??] What are these prices?

\part Suppose returns are $r' = (d'+q')/q $.  What are the returns?  [??]
\end{parts}



\end{questions}


\end{document}
