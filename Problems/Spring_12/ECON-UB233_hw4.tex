\documentclass[11pt]{exam}

\oddsidemargin=0.25truein \evensidemargin=0.25truein
\topmargin=-0.5truein \textwidth=6.0truein \textheight=8.75truein

%\RequirePackage{graphicx}
\usepackage{comment}
\usepackage{verbatim}
\usepackage{booktabs}
\usepackage[dvipdfm]{hyperref}
\urlstyle{rm}   % change fonts for url's (from Chad Jones)
\hypersetup{
    colorlinks=true,        % kills boxes
    allcolors=blue,
    pdfsubject={ECON-UB233, Macroeconomic foundations for asset pricing},
    pdfauthor={Dave Backus @ NYU},
    pdfstartview={FitH},
    pdfpagemode={UseNone},
%    pdfnewwindow=true,      % links in new window
%    linkcolor=blue,         % color of internal links
%    citecolor=blue,         % color of links to bibliography
%    filecolor=blue,         % color of file links
%    urlcolor=blue           % color of external links
% see:  http://www.tug.org/applications/hyperref/manual.html
}

\renewcommand{\thefootnote}{\fnsymbol{footnote}}
\newcommand{\var}{\mbox{\it Var\/}}

\printanswers

% document starts here
\begin{document}
\parskip=\bigskipamount
\parindent=0.0in
\thispagestyle{empty}
{\large ECON-UB 233 \hfill Dave Backus @ NYU}

\bigskip\bigskip
\centerline{\Large \bf Lab Report \#4: Excess Returns}
\centerline{(Started: July 19, 2011; Revised: \today)}

\bigskip
{\it Due at the start of class.
You may speak to others, but whatever you hand in should be your own work.}

\begin{questions}

\begin{solution}
Short answers follow.  See Matlab code at the end for complete answers.
\end{solution}


%-----------------------------------------------------------------------
\question (disaster risk and the equity premium)
We'll add a third ``disaster'' state to our analysis of the equity premium
and see how it changes our view of it.
The key input is the distribution of log consumption growth,
\begin{eqnarray*}
    \log g &=& \left\{
                \begin{array}{ll}
                \mu + \sigma & \mbox{with probability } (1-\omega)/2 \\
                \mu - \sigma & \mbox{with probability } (1-\omega)/2 \\
                \mu - \delta & \mbox{with probability } \omega
                \end{array}
                \right.
\end{eqnarray*}
What's the idea?
If $\omega = 0$, we're back to our symmetric two-state distribution.
But if we introduce a small positive value of $\omega$ and a ``largish'' $\delta>0$,
we have a ``disaster'' state that changes the distribution dramatically.

The question is what this does to the equity premium.
For reasons that should be clear from class,
we define the equity premium in logs,
\begin{eqnarray*}
    E ( \log r^e - \log r^1 ) ,
\end{eqnarray*}
and aim at a target value of 0.0400 (4\%).
As before, we'll define equity in this model as a claim to consumption growth $g$.

The structure is roughly similar to that in the Matlab program we used in class,
but I encourage you to write your own.
(Why?   I usually find it harder to adapt someone else's code than write my own.
You're welcome to do either, but don't say I didn't warn you.)

\begin{parts}
\part If $\omega = 0$, what values of $\mu$ and $\sigma$ deliver
the observed mean and variance of log consumption growth, namely
0.0200 and $0.0350^2$?
\part Suppose $\beta = 0.99$ and $\alpha = 5$.
What are $r^1$, $\log r^1$, and
the equity premium,  $E \log r^e - \log r^1 $?

\part What is the equity premium if $\alpha$
equals 10?  20?

\part Now consider $\omega = 0.01$ and $\delta = 0.30$.
(These numbers are based on a series of studies by Robert Barro and his coauthors.)
With these numbers, what values of $\mu$ and $\sigma$ reproduce
the observed mean and variance of log consumption growth?

\part With (again) $\beta = 0.99$ and $\alpha = 5$,
what are $r^1$, $\log r^1$, $E \log r^e$, and
the equity premium,  $E \log r^e - \log r^1 $?

\part What is the equity premium if $\alpha$
equals 10?  20?
How does it compare to your previous calculations?

\part (extra credit)
How does the equity premium change if $\delta = - 0.30$,
so that the extreme state is good news rather than bad?
Why?

\part (extra credit) How does entropy differ between the
disaster and no-disaster cases?
\end{parts}

\begin{solution}
\begin{parts}
\part The expressions for the mean and variance of $\log g$ are
\begin{eqnarray*}
    E (\log g)   &=&  \mu + \omega \delta \;\;=\;\; 0.0200 \\
    \mbox{Var}(\log g) &=& (1-\omega) \sigma^2 + \omega (1-\omega) \delta^2
                \;\;=\;\; 0.0350^2 .
\end{eqnarray*}
When $\omega = 0$, the mean is $\mu = 0.0200$ and the standard
deviation is $\sigma = 0.0350$.

\part With these values, we have
$r^1 = 1.0995$, $\log r^1 = 0.0948$, and
$E \log r^e - \log r^1 = 0.0055 $.

\part The equity premium is 0.0112 when $\alpha = 10$ and 0.0208 when $\alpha = 20$.
Both are well below our target value of 0.0400.

\part When $\omega = 0.01$,
we need to set $\mu = 0.0230$ and $\sigma = 0.0183$ to maintain the mean and
variance at their sample values.
See (a).

\part With these values, we have
$r^1 = 1.0907$, $\log r^1 = 0.0868$, and
$E \log r^e - \log r^1 = 0.0097 $.
That's partial success: the equity premium went up,
even though we held the mean and variance of
log consumption growth constant.
In that respect, the disaster state is a useful innovation,
but we still need risk aversion above 5 to hit the equity premium.

\part The equity premium is 0.0438 when $\alpha = 10$
and 0.2285 when $\alpha = 20$.
Evidently a much smaller value of $\alpha$ suffices when
we have a disaster state.

\part If we switch the sign of $\delta$, the equity premium
is below even the two-state version:  0.0037.
Apparently positive skewness in consumption and dividend growth
isn't helpful.
It's not part of the question,
but this is connected to the discussion of skewness and entropy:
positive skewness in $\log m$ increases entropy.
With power utility, that requires negative skewness in $\log g$.

\part We compute entropy directly from $m$:
\begin{eqnarray*}
    H(m) &=& \log E (m) - E \log m .
\end{eqnarray*}
With $\alpha = 5$, entropy is 0.0232 with the disaster state,
0.0152 without.  So the disaster increases entropy.
The difference between the two increases with risk aversion.
When $\alpha = 20$,
entropy is 1.5675 with the disaster state,
0.2273 without.
Yaron's bazooka!

It's not part of the question, but changing the sign of
$\delta$ reverses all this:
we need disasters, not booms.

\end{parts}
\end{solution}


%-----------------------------------------------------------------------
\question (risk and return)
Modern economies issue a wide range of assets,
whose returns can be wildly different.
We typically attribute this to ``risk,''
somehow defined,
but before we get to that point,
it's helpful to have some idea of the properties of
returns on some common assets:  their sample moments, for example.

We'll start with data input.  Go to Ken French's data site,
a standard source for data on financial returns for a broad range
of equity-related portfolios:

{\small
\url{http://mba.tuck.dartmouth.edu/pages/faculty/ken.french/data_library.html}}.

Download the files associated with, respectively, the ``Fama-French Factors''
and ``Portfolios Formed on Size'' at the links

{\small
\url{http://mba.tuck.dartmouth.edu/pages/faculty/ken.french/ftp/F-F_Research_Data_Factors.zip} \\
\url{http://mba.tuck.dartmouth.edu/pages/faculty/ken.french/ftp/Portfolios_Formed_on_ME.zip}.
}

In both cases, you'll find a txt file inside a zip file.
Copy the first table in each txt file into a spreadsheet with the dates aligned.
In the first file (``factors'') you want the first column (the date),
the second (the excess return on the market),
and the fifth (the short-term or riskfree interest rate RF).
In the second file (``size'') you want the first column (the date again)
and columns three to five (returns on portfolios of small, medium, and large firms).

When you're done, read the data into Matlab and convert returns
to excess returns, where necessary, by subtracting the one-period rate.
At this point you should have excess returns
on the market and three size portfolios (small, medium, and large) ---
four variables all together.
%
\begin{parts}
\part
Compute the mean, standard deviation, skewness, and excess kurtosis
for each excess return series.
Which portfolio has the highest excess return?  Lowest?

\part Do the same for log excess returns.
(Make sure you add one to them before taking logs.)

\item Is there a clear link here between risk, measured here by
the standard deviation of the excess return, and return,
measured by the mean excess return?
Does it matter whether we use levels or logs?

\item The Sharpe ratio for any asset or portfolio is
the mean of its excess return over its standard deviation.
Is there a clear link between Sharpe ratios and
mean excess returns?
\end{parts}
%
Comment:
Returns here are monthly.

\begin{solution}
The essential step here is to understand the units.
The data are monthly returns, expressed as percentages
(that is, multiplied by 100).
That's not a problem with returns and excess returns,
the change in units carries through the calculations.
But with log returns, it's essential that we divide by 100
before adding one and taking logs.
I multiplied by 100 after taking logs to make the numbers comparable
to those in levels, but that's a subtle issue.

\begin{parts}
\part I added skewness and kurtosis to maintain our theme,
but it's obviously not expected of you.
The results are included in the table below.

\medskip
\begin{center}
\begin{tabular}{lcccc}
\multicolumn{4}{l}{Properties of monthly excess returns} \\
\toprule
            & \multicolumn{4}{c}{Portfolio} \\
                         \cmidrule{2-5} 
Statistic   & market & small  & medium & big \\
\midrule
Mean &  0.6172 &  0.9738  &  0.8532  & 0.5987 \\
Std dev & 5.4571  &  8.5584  &  6.8520 &  5.2757 \\
Skewness &  0.1685 &   2.1813 &  0.9816  &  0.1860 \\
Excess kurtosis &  7.3997 &  21.8628 &  11.7830 &  7.1738 \\
Sharpe ratio  &  0.1131  &  0.1138  &  0.1245 &   0.1135 \\
\bottomrule
\end{tabular}
\end{center}

\medskip
The highest mean excess return is small firms.

\part Similar table for log excess returns below.
We need to add one to returns and take logs.

\medskip
\begin{center}
\begin{tabular}{lrrrr}
\multicolumn{4}{l}{Properties of monthly log excess returns} \\
\toprule
            & \multicolumn{4}{c}{Portfolio} \\
             \cmidrule{2-5} 
Statistic   & market & small  & medium & big \\
\midrule
Mean            & 0.4669 & 0.6329 & 0.6224 & 0.4583 \\
Std dev         & 5.4556 & 8.1132 & 6.7205 & 5.2682 \\
Skewness        & $-0.5298$ & 0.4937 & $-0.0850$ & $-0.4779$  \\
Excess kurtosis & 6.5282 & 9.6576 & 7.5655 & 6.3824\\
\bottomrule
\end{tabular}
\end{center}


\part There's some connection between the mean and the standard deviation
in levels, not so much in logs.
Neither is all that informative:  we know risk
has to do with the connection to the pricing kernel,
anything else would be purely accidental.

\part There's less variation in Sharpe ratios than mean returns,
because the high return assets also have high standard deviations.

\end{parts}
\end{solution}

\end{questions}

\pagebreak
Matlab program:
\verbatiminput{../Matlab/hw4_s12.m}
\end{document}


%  EXTRA STUFF


